\documentclass[a4paper, 11pt]{article}
\newcommand\hmmax{0}
\newcommand\bmmax{0}
%\usepackage[utf8]{inputenc}
\tracinglostchars=3
\usepackage{amsmath, amsfonts, amssymb, bm}

\usepackage[dvipsnames]{xcolor}
\usepackage{listings}
\usepackage{verbatim,mathtools,graphicx}

\usepackage[T1]{fontenc}
\usepackage[scaled=.97,helvratio=.93,p,theoremfont]{newpxtext}
\usepackage[varbb,smallerops,bigdelims]{newpxmath}
\usepackage[scaled=.85]{beramono}
\usepackage[scr=rsfso,cal=euler]{mathalfa}

\usepackage{tikz, tikz-cd}
\usepackage{enumitem,xparse,url}
\usepackage[most]{tcolorbox}
\usepackage{multirow,array}
\usepackage{hyperref,bookmark}
\usepackage{stmaryrd}
\usepackage{fontspec}
\usepackage{microtype}

% Renewed commands
\renewcommand{\injlim}{\varinjlim}
\renewcommand{\projlim}{\varprojlim}

% Column stuff
\newcolumntype{x}[1]{>{\centering\let\newline\\\arraybackslash\hspace{0pt}}p{#1}}

% TikZ stuff
\tikzset{
    labl/.style={anchor=north, rotate=90},
    pullback/.style={commutative diagrams/phantom, "\usebox\pullback" , very near start},
  pushout/.style={commutative diagrams/phantom, "\usebox\pushout" , very near start},
  symbol/.style={%
    draw=none,
    every to/.append style={edge node={node [sloped, allow upside down, auto=false]{$#1$}}}
  }
}

\newsavebox{\pullback}
\sbox\pullback{%
\begin{tikzpicture}%
\draw (0,0) -- (1ex,0ex);%
\draw (1ex,0ex) -- (1ex,1ex);%
\end{tikzpicture}}

\newsavebox{\pushout}
\sbox\pushout{%
\begin{tikzpicture}%
\draw (0,0) -- (0ex,1ex);%
\draw (0ex,1ex) -- (1ex,1ex);%
\end{tikzpicture}}

% Environments
% warning box
\newenvironment{warningbox}{\begin{tcolorbox}[colback=red!5!white, colframe=red!80!black, title=\textbf{Warning!}]}{\end{tcolorbox}}
% note box
\newenvironment{notebox}{\begin{tcolorbox}[colback=blue!5!white, colframe=blue!80!black, title=\textbf{Note}]}{\end{tcolorbox}}
% todo box
\newenvironment{todobox}{\begin{tcolorbox}[colback=green!5!white, colframe=green!80!black, title=\textbf{TODO}]}{\end{tcolorbox}}
% tikzcd diagram
\newenvironment{diagram*}{\begin{center}\begin{tikzcd}}{\end{tikzcd}\end{center}}
\newenvironment{diagram}{\begin{equation}\begin{tikzcd}}{\end{tikzcd}\end{equation}\ignorespaces}

\newenvironment{graph*}{\vspace{0.1cm}\begin{center}\begin{tikzcd}[every arrow/.append style={dash,thick}]}{\end{tikzcd}\end{center}\vspace{0.1cm}}
\newenvironment{graph}{\begin{equation}\begin{tikzcd}[every arrow/.append style={dash,thick}]}{\end{tikzcd}\end{equation}\ignorespaces}


% Fancy theorem, proof, example, etc. environments.
\tcbuselibrary{theorems}
\tcbuselibrary{skins}
\tcbuselibrary{breakable}
\tcbuselibrary{vignette}

% styling
\tcbset{
  plategeneric/.style={
    enhanced, breakable,
    borderline west={2pt}{0pt}{white!75!black},
    colback=white,
    top=1.3pt,
    bottom=1.3pt,
    left=7pt,
    right=7pt,
    grow to left by=10pt,
    grow to right by=10pt,
    boxrule=0pt,
    sharp corners
  },
  plate/.style={
    plategeneric,
    frame hidden,
    beforeafter skip balanced=0.22\baselineskip plus 1pt,
    height fixed for=first and middle,
    lines before break=1,
    pad at break*=1.1mm,
    detach title,
    coltitle=black
  },
  fancythm/.style={
    plate,
    top=4pt,
    bottom=4pt,
    borderline west={2pt}{0pt}{Goldenrod!70!black},
    colback=Goldenrod!5!white,
    before upper={\parindent15pt\noindent\textup{\textbf{\tcbtitle.}}\,\,},
    fontupper=\itshape
  },
  fancyconj/.style={
    plate,
    borderline west={2pt}{0pt}{Goldenrod!80!black},
    %colback=Goldenrod!10!white,
    before upper={\parindent15pt\noindent\textup{\textbf{\tcbtitle.}}\,\,},
    fontupper=\itshape
  },
  fancyproof/.style={
    plate,
    borderline west={2pt}{0pt}{Peach!75!black},
    colback=Peach!1!white,
    fonttitle=\itshape,
    before upper={\parindent15pt\noindent\tcbtitle\textit{.}\,\,\,},
    fontupper=\upshape
  },
  fancydef/.style={
    plate,
    borderline west={2pt}{0pt}{NavyBlue!85!black},
    colback=white,
    fonttitle=\bfseries,
    before upper={\parindent15pt\noindent\tcbtitle\textbf{.}\,\,\,}
  },
  fancynotation/.style={
    plate,
    borderline west={2pt}{0pt}{Cyan!65!black},
    colback=white,
    fonttitle=\bfseries,
    before upper={\parindent15pt\noindent\tcbtitle\textbf{.}\,\,\,}
  },
  fancyexample/.style={
    plate,
    borderline west={2pt}{0pt}{Plum!75!black},
    colback=white,
    fonttitle=\bfseries,
    before upper={\parindent15pt\noindent\tcbtitle\textbf{.}\,\,\,}
  },
  fancyremark/.style={
    plate,
    borderline west={2pt}{0pt}{Emerald!75!black},
    colback=white,
    fonttitle=\itshape,
    before upper={\parindent15pt\noindent\tcbtitle\textit{.}\,\,\,}
  },
  fancyexercise/.style={
    plate,
    borderline west={2pt}{0pt}{WildStrawberry!85!black},
    colback=WildStrawberry!10!white,
    fonttitle=\bfseries,
    before upper={\parindent15pt\noindent\tcbtitle\textbf{.}\,\,\,}
  },
  fancysoln/.style={
    plate,
    borderline west={2pt}{0pt}{YellowGreen!85!black},
    colback=white,
    fonttitle=\bfseries,
    before upper={\parindent15pt\noindent\tcbtitle\textbf{.}\,\,\,}
  },
  fancywarning/.style={
    plate,
    borderline west={2pt}{0pt}{Red!75!black},
    colback=Red!15!white,
    before upper={\parindent15pt\noindent\tcbtitle\textbf{.}\,\,\,},
    fonttitle=\bfseries,
    fontupper=\upshape
  },
  fancynote/.style={
    plategeneric,
    grow to right by=4pt,
    borderline west={2pt}{0pt}{ProcessBlue!75!black},
    colback=ProcessBlue!15!white,
    colbacktitle=ProcessBlue!50!white,
    coltitle={ProcessBlue!30!black},
    title={Note},
    fonttitle=\bfseries,
    fontupper=\upshape,
    before upper={\parindent15pt\noindent}
  },
  fancytodo/.style={
    plategeneric,
    grow to right by=4pt,
    borderline west={2pt}{0pt}{Green!75!black},
    colback=Green!15!white,
    colbacktitle=Green!50!white,
    coltitle={Green!40!black},
    title={TO-DO},
    fonttitle=\bfseries,
    fontupper=\upshape,
    before upper={\parindent15pt\noindent}
  }
}

% fix the newtcbtheorem command to be more useful (broken, don't know how to fix this)
\ExplSyntaxOn

\NewDocumentCommand{\betternewtcbtheorem}{O{}mmmm}
 {
  \newtcbtheorem[#1]{#2inner}{#3}{#4}{#5}
  \NewDocumentEnvironment{#2}{O{}}
   {
    \keys_set:nn { hushus/tcb } { ##1 }
    \hushus_tcb_begin:nVV {#2inner} \l__hushus_tcb_title_tl \l__hushus_tcb_label_tl
   }
   {
    \end{#2inner}
   }
  \cs_if_exist:cF { c@#5} { \newcounter{#5} }
 }

\cs_new_protected:Nn \hushus_tcb_begin:nnn
 {
  \begin{#1}{#2}{#3}
 }
\cs_generate_variant:Nn \hushus_tcb_begin:nnn { nVV }
\keys_define:nn { hushus/tcb }
 {
  title .tl_set:N = \l__hushus_tcb_title_tl,
  label .tl_set:N = \l__hushus_tcb_label_tl,
 }

\ExplSyntaxOff

% Command for creating fancy theorem-like environments
% Creates the title format for some parts.
% args: display name, thesection, number (& added text for optional)
\newcommand{\ThmTitleNoOptional}[3]{#1 #2.#3}
\newcommand{\ThmTitleOptional}[4]{#1 #2.#3: #4}
% args: counter prefix (optional), name, display name, style, counter    \ThmTitleOptional{#3}{\thesection}{\arabic{#5}}{##1}
\NewDocumentCommand\goodnewtcbtheorem{ommmm}{%
  \IfNoValueTF{#1}{%
  \newenvironment{#2}[1][]{\refstepcounter{#5}\begin{tcolorbox}[title={\if\relax\detokenize{##1}\relax{\ThmTitleNoOptional{#3}{\thesection}{\arabic{#5}}}\else{\ThmTitleOptional{#3}{\thesection}{\arabic{#5}}{##1}}\fi},#4]}{\end{tcolorbox}}%
  \newenvironment{#2*}[1][]{\begin{tcolorbox}[title={\if\relax\detokenize{##1}\relax{#3}\else{#3: ##1}\fi},#4]}{\end{tcolorbox}}%
  }{%
  \newenvironment{#2}[1][]{\refstepcounter{#5}\begin{tcolorbox}[title={\if\relax\detokenize{##1}\relax{#3 #1\arabic{#5}}\else{#3 #1\arabic{#5}: ##1}\fi},#4]}{\end{tcolorbox}}%
  \newenvironment{#2*}[1][]{\begin{tcolorbox}[title={\if\relax\detokenize{##1}\relax{#3}\else{#3: ##1}\fi},#4]}{\end{tcolorbox}}%
  }
}
\NewDocumentCommand\goodrenewtcbtheorem{ommmm}{%
  \IfNoValueTF{#1}{%
  \renewenvironment{#2}[1][]{\refstepcounter{#5}\begin{tcolorbox}[title={\if\relax\detokenize{##1}\relax{#3 \thesection.\arabic{#5}}\else{#3 \thesection.\arabic{#5}: ##1}\fi},#4]}{\end{tcolorbox}}%
  \renewenvironment{#2*}[1][]{\begin{tcolorbox}[title={\if{#3}\else{#3: ##1}\fi},#4]}{\end{tcolorbox}}%
  }{%
  \renewenvironment{#2}[1][]{\refstepcounter{#5}\begin{tcolorbox}[title={\if\relax\detokenize{##1}\relax{#3 #1\arabic{#5}}\else{#3 #1\arabic{#5}: ##1}\fi},#4]}{\end{tcolorbox}}%
  \renewenvironment{#2*}[1][]{\begin{tcolorbox}[title={\if{#3}\else{#3: ##1}\fi},#4]}{\end{tcolorbox}}%
  }
}

% Counters
\newcounter{thmcounter}
\numberwithin{thmcounter}{section}

\newcounter{excounter}
\newcounter{probcounter}
\newcounter{solcounter}
\newcounter{taskcounter}

% Theorems, Lemmas, etc.
% Theorem style
\goodnewtcbtheorem{theorem}{Theorem}{fancythm}{thmcounter}
\goodnewtcbtheorem{theoremdef}{Theorem-Definition}{fancythm}{thmcounter}
\goodnewtcbtheorem{proposition}{Proposition}{fancythm}{thmcounter}
\goodnewtcbtheorem{propositiondef}{Proposition-Definition}{fancythm}{thmcounter}
\goodnewtcbtheorem{corollary}{Corollary}{fancythm}{thmcounter}
\goodnewtcbtheorem{lemma}{Lemma}{fancythm}{thmcounter}
\goodnewtcbtheorem{fact}{Fact}{fancythm}{thmcounter}

\goodnewtcbtheorem{conjecture}{Conjecture}{fancyconj, fontupper=\upshape}{thmcounter}

\goodnewtcbtheorem{goal}{Goal}{fancyconj, fontupper=\upshape}{thmcounter}

% Definition style
\goodnewtcbtheorem{definition}{Definition}{fancydef}{thmcounter}

% Notation style
\goodnewtcbtheorem{notation}{Notation}{fancynotation}{thmcounter}
\goodnewtcbtheorem{terminology}{Terminology}{fancynotation}{thmcounter}
\goodnewtcbtheorem{convention}{Convention}{fancynotation}{thmcounter}
\goodnewtcbtheorem{question}{Question}{fancynotation}{thmcounter}
\goodnewtcbtheorem{axiom}{Axiom}{fancynotation}{thmcounter}

% Example style
\goodnewtcbtheorem{example}{Example}{fancyexample}{thmcounter}
\goodnewtcbtheorem{construction}{Construction}{fancyexample}{thmcounter}
\goodnewtcbtheorem{sketch}{Sketch}{fancyexample}{thmcounter}

% Remark style
\goodnewtcbtheorem{remark}{Remark}{fancyremark}{thmcounter}
\goodnewtcbtheorem{discussion}{Discussion}{fancyremark}{thmcounter}

% Exercise style, empty [] necessary.
\goodnewtcbtheorem[]{exercise}{Exercise}{fancyexercise}{excounter}
\goodnewtcbtheorem[]{task}{Task}{fancyexercise}{taskcounter}
\goodnewtcbtheorem[]{problem}{Problem}{fancyexercise}{probcounter}

% Solution style, empty [] necessary.
\goodnewtcbtheorem[]{solution}{Solution}{fancysoln}{solcounter}

% Warning style
\goodnewtcbtheorem{warning}{Warning}{fancywarning}{thmcounter}

% fancy note and todo
\newtcolorbox{note}{fancynote}
\newtcolorbox{todo}{fancytodo}

% Proof environment.
\newcommand{\qedsymbol}{$\blacksquare$}
\newtcolorbox{proof}[1][Proof]{fancyproof, title={#1}}
\AtEndEnvironment{proof}{\null\hfill\qedsymbol}
% "Proof Sketch" environment
\newenvironment{proofsketch}{\begin{proof}[Proof sketch]\renewcommand\qedsymbol{$\square$}}{\end{proof}\renewcommand\qedsymbol{$\blacksquare$}}

% QED symbol should be black tombstone.
\renewcommand\qedsymbol{$\blacksquare$}
\makeatletter
\newcommand{\oset}[3][0ex]{%
  \mathrel{\mathop{#3}\limits^{
    \vbox to#1{\kern-2\ex@
    \hbox{$\scriptstyle#2$}\vss}}}}
\makeatother

% utility "such that" command.
\newcommand{\st}{
  \text{ s.t. }
}

\newcommand{\bigmid}{\biggm|}
\newcommand{\ldsq}{\llbracket}
\newcommand{\rdsq}{\rrbracket}
\newcommand{\dashto}{\dashrightarrow}

\newcommand{\vecb}{\mathbf}
\newcommand{\lcm}{\operatorname{lcm}}
\newcommand{\divides}{\mid}
\newcommand{\longto}{\longrightarrow}
\newcommand{\ot}{\leftarrow}
\newcommand{\xto}{\xrightarrow}
\newcommand{\xot}{\xleftarrow}
\newcommand{\longot}{\longleftarrow}
\newcommand{\Longto}{\Longrightarrow}
\newcommand{\oT}{\Leftarrow}
\newcommand{\To}{\Rightarrow}
\newcommand{\longoT}{\Longleftarrow}

\newcommand{\fd}{\textup{f.d.}}
\newcommand{\ev}{\operatorname{ev}}

% mathbf
\newcommand{\bfA}{\ensuremath{\mathbf{A}}}
\newcommand{\bfB}{\ensuremath{\mathbf{B}}}
\newcommand{\bfC}{\ensuremath{\mathbf{C}}}
\newcommand{\bfD}{\ensuremath{\mathbf{D}}}
\newcommand{\bfE}{\ensuremath{\mathbf{E}}}
\newcommand{\bfF}{\ensuremath{\mathbf{F}}}
\newcommand{\bfG}{\ensuremath{\mathbf{G}}}
\newcommand{\bfH}{\ensuremath{\mathbf{H}}}
\newcommand{\bfI}{\ensuremath{\mathbf{I}}}
\newcommand{\bfJ}{\ensuremath{\mathbf{J}}}
\newcommand{\bfK}{\ensuremath{\mathbf{K}}}
\newcommand{\bfL}{\ensuremath{\mathbf{L}}}
\newcommand{\bfM}{\ensuremath{\mathbf{M}}}
\newcommand{\bfN}{\ensuremath{\mathbf{N}}}
\newcommand{\bfO}{\ensuremath{\mathbf{O}}}
\newcommand{\bfP}{\ensuremath{\mathbf{P}}}
\newcommand{\bfQ}{\ensuremath{\mathbf{Q}}}
\newcommand{\bfR}{\ensuremath{\mathbf{R}}}
\newcommand{\bfS}{\ensuremath{\mathbf{S}}}
\newcommand{\bfT}{\ensuremath{\mathbf{T}}}
\newcommand{\bfU}{\ensuremath{\mathbf{U}}}
\newcommand{\bfV}{\ensuremath{\mathbf{V}}}
\newcommand{\bfW}{\ensuremath{\mathbf{W}}}
\newcommand{\bfX}{\ensuremath{\mathbf{X}}}
\newcommand{\bfY}{\ensuremath{\mathbf{Y}}}
\newcommand{\bfZ}{\ensuremath{\mathbf{Z}}}
\newcommand{\bfDelta}{\ensuremath{\mathbf{\Delta}}}

% mathsf
\newcommand{\sfA}{\ensuremath{\mathsf{A}}}
\newcommand{\sfB}{\ensuremath{\mathsf{B}}}
\newcommand{\sfC}{\ensuremath{\mathsf{C}}}
\newcommand{\sfD}{\ensuremath{\mathsf{D}}}
\newcommand{\sfE}{\ensuremath{\mathsf{E}}}
\newcommand{\sfF}{\ensuremath{\mathsf{F}}}
\newcommand{\sfG}{\ensuremath{\mathsf{G}}}
\newcommand{\sfH}{\ensuremath{\mathsf{H}}}
\newcommand{\sfI}{\ensuremath{\mathsf{I}}}
\newcommand{\sfJ}{\ensuremath{\mathsf{J}}}
\newcommand{\sfK}{\ensuremath{\mathsf{K}}}
\newcommand{\sfL}{\ensuremath{\mathsf{L}}}
\newcommand{\sfM}{\ensuremath{\mathsf{M}}}
\newcommand{\sfN}{\ensuremath{\mathsf{N}}}
\newcommand{\sfO}{\ensuremath{\mathsf{O}}}
\newcommand{\sfP}{\ensuremath{\mathsf{P}}}
\newcommand{\sfQ}{\ensuremath{\mathsf{Q}}}
\newcommand{\sfR}{\ensuremath{\mathsf{R}}}
\newcommand{\sfS}{\ensuremath{\mathsf{S}}}
\newcommand{\sfT}{\ensuremath{\mathsf{T}}}
\newcommand{\sfU}{\ensuremath{\mathsf{U}}}
\newcommand{\sfV}{\ensuremath{\mathsf{V}}}
\newcommand{\sfW}{\ensuremath{\mathsf{W}}}
\newcommand{\sfX}{\ensuremath{\mathsf{X}}}
\newcommand{\sfY}{\ensuremath{\mathsf{Y}}}
\newcommand{\sfZ}{\ensuremath{\mathsf{Z}}}

% mathbb
\newcommand{\bbA}{\ensuremath{\mathbb{A}}}
\newcommand{\bbB}{\ensuremath{\mathbb{B}}}
\newcommand{\bbC}{\ensuremath{\mathbb{C}}}
\newcommand{\bbD}{\ensuremath{\mathbb{D}}}
\newcommand{\bbE}{\ensuremath{\mathbb{E}}}
\newcommand{\bbF}{\ensuremath{\mathbb{F}}}
\newcommand{\bbG}{\ensuremath{\mathbb{G}}}
\newcommand{\bbH}{\ensuremath{\mathbb{H}}}
\newcommand{\bbI}{\ensuremath{\mathbb{I}}}
\newcommand{\bbJ}{\ensuremath{\mathbb{J}}}
\newcommand{\bbK}{\ensuremath{\mathbb{K}}}
\newcommand{\bbL}{\ensuremath{\mathbb{L}}}
\newcommand{\bbM}{\ensuremath{\mathbb{M}}}
\newcommand{\bbN}{\ensuremath{\mathbb{N}}}
\newcommand{\bbO}{\ensuremath{\mathbb{O}}}
\newcommand{\bbP}{\ensuremath{\mathbb{P}}}
\newcommand{\bbQ}{\ensuremath{\mathbb{Q}}}
\newcommand{\bbR}{\ensuremath{\mathbb{R}}}
\newcommand{\bbS}{\ensuremath{\mathbb{S}}}
\newcommand{\bbT}{\ensuremath{\mathbb{T}}}
\newcommand{\bbU}{\ensuremath{\mathbb{U}}}
\newcommand{\bbV}{\ensuremath{\mathbb{V}}}
\newcommand{\bbW}{\ensuremath{\mathbb{W}}}
\newcommand{\bbX}{\ensuremath{\mathbb{X}}}
\newcommand{\bbY}{\ensuremath{\mathbb{Y}}}
\newcommand{\bbZ}{\ensuremath{\mathbb{Z}}}

% mathcal
\newcommand{\calA}{\ensuremath{\mathcal{A}}}
\newcommand{\calB}{\ensuremath{\mathcal{B}}}
\newcommand{\calC}{\ensuremath{\mathcal{C}}}
\newcommand{\calD}{\ensuremath{\mathcal{D}}}
\newcommand{\calE}{\ensuremath{\mathcal{E}}}
\newcommand{\calF}{\ensuremath{\mathcal{F}}}
\newcommand{\calG}{\ensuremath{\mathcal{G}}}
\newcommand{\calH}{\ensuremath{\mathcal{H}}}
\newcommand{\calI}{\ensuremath{\mathcal{I}}}
\newcommand{\calJ}{\ensuremath{\mathcal{J}}}
\newcommand{\calK}{\ensuremath{\mathcal{K}}}
\newcommand{\calL}{\ensuremath{\mathcal{L}}}
\newcommand{\calM}{\ensuremath{\mathcal{M}}}
\newcommand{\calN}{\ensuremath{\mathcal{N}}}
\newcommand{\calO}{\ensuremath{\mathcal{O}}}
\newcommand{\calP}{\ensuremath{\mathcal{P}}}
\newcommand{\calQ}{\ensuremath{\mathcal{Q}}}
\newcommand{\calR}{\ensuremath{\mathcal{R}}}
\newcommand{\calS}{\ensuremath{\mathcal{S}}}
\newcommand{\calT}{\ensuremath{\mathcal{T}}}
\newcommand{\calU}{\ensuremath{\mathcal{U}}}
\newcommand{\calV}{\ensuremath{\mathcal{V}}}
\newcommand{\calW}{\ensuremath{\mathcal{W}}}
\newcommand{\calX}{\ensuremath{\mathcal{X}}}
\newcommand{\calY}{\ensuremath{\mathcal{Y}}}
\newcommand{\calZ}{\ensuremath{\mathcal{Z}}}

% mathscr
\newcommand{\scrA}{\ensuremath{\mathscr{A}}}
\newcommand{\scrB}{\ensuremath{\mathscr{B}}}
\newcommand{\scrC}{\ensuremath{\mathscr{C}}}
\newcommand{\scrD}{\ensuremath{\mathscr{D}}}
\newcommand{\scrE}{\ensuremath{\mathscr{E}}}
\newcommand{\scrF}{\ensuremath{\mathscr{F}}}
\newcommand{\scrG}{\ensuremath{\mathscr{G}}}
\newcommand{\scrH}{\ensuremath{\mathscr{H}}}
\newcommand{\scrI}{\ensuremath{\mathscr{I}}}
\newcommand{\scrJ}{\ensuremath{\mathscr{J}}}
\newcommand{\scrK}{\ensuremath{\mathscr{K}}}
\newcommand{\scrL}{\ensuremath{\mathscr{L}}}
\newcommand{\scrM}{\ensuremath{\mathscr{M}}}
\newcommand{\scrN}{\ensuremath{\mathscr{N}}}
\newcommand{\scrO}{\ensuremath{\mathscr{O}}}
\newcommand{\scrP}{\ensuremath{\mathscr{P}}}
\newcommand{\scrQ}{\ensuremath{\mathscr{Q}}}
\newcommand{\scrR}{\ensuremath{\mathscr{R}}}
\newcommand{\scrS}{\ensuremath{\mathscr{S}}}
\newcommand{\scrT}{\ensuremath{\mathscr{T}}}
\newcommand{\scrU}{\ensuremath{\mathscr{U}}}
\newcommand{\scrV}{\ensuremath{\mathscr{V}}}
\newcommand{\scrW}{\ensuremath{\mathscr{W}}}
\newcommand{\scrX}{\ensuremath{\mathscr{X}}}
\newcommand{\scrY}{\ensuremath{\mathscr{Y}}}
\newcommand{\scrZ}{\ensuremath{\mathscr{Z}}}

% mathfrak
\newcommand{\frakA}{\ensuremath{\mathfrak{A}}}
\newcommand{\frakB}{\ensuremath{\mathfrak{B}}}
\newcommand{\frakC}{\ensuremath{\mathfrak{C}}}
\newcommand{\frakD}{\ensuremath{\mathfrak{D}}}
\newcommand{\frakE}{\ensuremath{\mathfrak{E}}}
\newcommand{\frakF}{\ensuremath{\mathfrak{F}}}
\newcommand{\frakG}{\ensuremath{\mathfrak{G}}}
\newcommand{\frakH}{\ensuremath{\mathfrak{H}}}
\newcommand{\frakI}{\ensuremath{\mathfrak{I}}}
\newcommand{\frakJ}{\ensuremath{\mathfrak{J}}}
\newcommand{\frakK}{\ensuremath{\mathfrak{K}}}
\newcommand{\frakL}{\ensuremath{\mathfrak{L}}}
\newcommand{\frakM}{\ensuremath{\mathfrak{M}}}
\newcommand{\frakN}{\ensuremath{\mathfrak{N}}}
\newcommand{\frakO}{\ensuremath{\mathfrak{O}}}
\newcommand{\frakP}{\ensuremath{\mathfrak{P}}}
\newcommand{\frakQ}{\ensuremath{\mathfrak{Q}}}
\newcommand{\frakR}{\ensuremath{\mathfrak{R}}}
\newcommand{\frakS}{\ensuremath{\mathfrak{S}}}
\newcommand{\frakT}{\ensuremath{\mathfrak{T}}}
\newcommand{\frakU}{\ensuremath{\mathfrak{U}}}
\newcommand{\frakV}{\ensuremath{\mathfrak{V}}}
\newcommand{\frakW}{\ensuremath{\mathfrak{W}}}
\newcommand{\frakX}{\ensuremath{\mathfrak{X}}}
\newcommand{\frakY}{\ensuremath{\mathfrak{Y}}}
\newcommand{\frakZ}{\ensuremath{\mathfrak{Z}}}


% Calculus, transforms, and other stuff
\newcommand*{\diff}{\mathop{}\!\mathrm{d}}
\newcommand{\Res}{\operatorname{Res}}
\newcommand{\p}{\partial}
\newcommand{\diam}{\mathrm{diam}\,}
\renewcommand{\Im}{\operatorname{Im}}
\renewcommand{\Re}{\operatorname{Re}}
\newcommand{\sinc}{\mathrm{sinc}}

% Common Sets
\newcommand{\Z}{\mathbb{Z}}
\newcommand{\Q}{\mathbb{Q}}
\newcommand{\R}{\mathbb{R}}
\newcommand{\C}{\mathbb{C}}
\newcommand{\D}{\mathbb{D}}
\newcommand{\N}{\mathbb{N}}
\newcommand{\E}{\mathbb{E}}
\newcommand{\F}{\mathbb{F}}
\newcommand{\G}{\mathbb{G}}
\newcommand{\A}{\mathbb{A}}
\newcommand{\TT}{\mathbb{T}}
\newcommand{\PP}{\mathbb{P}}
\newcommand{\pow}{
  \mathcal{P}
}

% Algebra
\newcommand{\adj}{\operatorname{adj}}
\newcommand{\ind}{\operatorname{ind}}
\newcommand{\height}{\operatorname{ht}}
\newcommand{\dual}{\vee}
\newcommand{\sep}{\text{sep}}
\newcommand{\nor}{\text{nor}}
\newcommand{\red}{\mathrm{red}}
\newcommand{\gal}{\text{gal}}
\newcommand{\ideal}{\mathfrak}
\newcommand{\GL}{\operatorname{GL}}
\newcommand{\SL}{\operatorname{SL}}
\newcommand{\M}{\operatorname{M}}
\newcommand{\nrmleq}{\trianglelefteq}
\newcommand{\nrmgeq}{\trianglerighteq}
\newcommand{\cyclic}[1]{\langle #1 \rangle}
\newcommand{\coimg}{\operatorname{coim}}
\newcommand{\img}{\operatorname{im}}
\newcommand{\Frac}{\operatorname{Frac}}
\newcommand{\aut}{\operatorname{Aut}}
\newcommand{\Aut}{\operatorname{Aut}}
\newcommand{\End}{\operatorname{End}}
\newcommand{\Ann}{\operatorname{Ann}}
\newcommand{\Gal}{\operatorname{Gal}}
\newcommand{\Tor}{\operatorname{Tor}}
\newcommand{\coker}{\operatorname{coker}}
\newcommand{\eq}{\operatorname{eq}}
\newcommand{\coeq}{\operatorname{coeq}}
\newcommand{\spec}{\operatorname{Spec}}
\newcommand{\Spec}{\operatorname{Spec}}
\newcommand{\Proj}{\operatorname{Proj}}
\newcommand{\Ox}{\mathcal{O}}
\newcommand{\Open}{\operatorname{Open}}
\newcommand{\Char}{\operatorname{char}}
\newcommand{\Span}{\operatorname{Span}}
\newcommand{\intO}{\mathscr{O}}
\newcommand{\Pic}{\operatorname{Pic}}
\newcommand{\Ext}{\operatorname{Ext}}
\newcommand{\KProj}{\operatorname{KProj}}


\newcommand{\hooktwoheadrightarrow}{\hookrightarrow\mathrel{\mspace{-15mu}}\rightarrow}
\makeatletter
\newcommand{\xhooktwoheadrightarrow}[2][]{%
  \lhook\joinrel
  \ext@arrow 0359\rightarrowfill@ {#1}{#2}%
  \mathrel{\mspace{-15mu}}\rightarrow
}
\makeatother
\newcommand{\iso}{\xrightarrow{\sim}}
\newcommand{\isom}{\hooktwoheadrightarrow}
\newcommand{\inj}{\hookrightarrow}
\newcommand{\sur}{\twoheadrightarrow}
\newcommand{\orbits}{\mathrm{O}}
\newcommand{\rank}{\operatorname{rank}}
\newcommand{\nullity}{\operatorname{nullity}}
\newcommand{\tr}{\operatorname{tr}}

% Geometry
\newcommand{\simh}{\sim_\textup{h}}
\newcommand{\Supp}{\operatorname{Supp}}
\newcommand{\spa}{\operatorname{sp}}
\newcommand{\HH}{\operatorname{H}}
\newcommand{\hh}{\operatorname{h}}
\newcommand{\mult}{\operatorname{mult}}
\newcommand{\codim}{\operatorname{codim}}
\newcommand{\model}{\mathcal}
\newcommand{\Gm}[1][]{\mathbf{G}_{\mathrm{m} #1}}
\newcommand{\Ga}[1][]{\mathbf{G}_{\mathrm{a} #1}}
\newcommand{\Sp}{\mathrm{Sp}}
\newcommand{\Sk}{\operatorname{Sk}}
\newcommand{\Sing}{\operatorname{Sing}}
\newcommand{\ho}{\operatorname{ho}}

% Category Theory
\newcommand{\Qis}{\operatorname{Qis}}
\newcommand{\thick}{\operatorname{thick}}
\newcommand{\Lan}{\operatorname{Lan}}
\newcommand{\Ran}{\operatorname{Ran}}
\newcommand{\Com}{\operatorname{Com}}
\newcommand{\Ind}{\operatorname{Ind}}
\newcommand{\Pro}{\operatorname{Pro}}
\newcommand{\pre}{\mathrm{pre}}
\newcommand{\op}{\mathrm{op}}
\newcommand{\Hom}{\operatorname{Hom}}
\newcommand{\Map}{\operatorname{Map}}
\newcommand{\iHom}{\underline{\Hom}}
\newcommand{\SHom}{\mathcal{H}\hspace{-0.0833em}om}
\newcommand{\Mor}{\operatorname{Mor}}
\newcommand{\Fun}{\operatorname{Fun}}
\newcommand{\Nat}{\operatorname{Nat}}
\newcommand{\Der}{\operatorname{Der}}
\newcommand{\hocolim}{\operatorname{hocolim}}
\newcommand{\sheaf}{\mathscr}
\newcommand{\sh}{\text{sh}}
\newcommand{\cat}{\mathbf}
\newcommand{\natcat}[1]{ \mathbf{[#1]} }
\newcommand{\dom}{\operatorname{dom}}
\newcommand{\cod}{\operatorname{cod}}
\newcommand{\id}{\mathrm{id}}
\newcommand{\Cones}{\operatorname{Cones}}
\newcommand{\ob}{\operatorname{Ob}}
\newcommand{\Ob}{\ob}
\newcommand{\Ar}{\operatorname{Ar}}
\newcommand{\natto}{%
  \mathrel{\vbox{\offinterlineskip
    \mathsurround=0pt
    \ialign{\hfil##\hfil\cr
      \normalfont\scalebox{1.2}{.}\cr
%      \noalign{\kern-.05ex}
      $\rightarrow$\cr}
  }}%
}

% Some Categories
\newcommand{\Cat}{\cat{Cat}}
\newcommand{\inftyCat}{\infty\text{-}\cat{Cat}}
\newcommand{\Set}{\cat{Set}}
\newcommand{\PSet}{\cat{Set_*}}
\newcommand{\Ens}[1]{\cat{Ens}_{#1}}
\newcommand{\Mon}{\cat{Mon}}
\newcommand{\CMon}{\cat{CMon}}
\newcommand{\Grp}{\cat{Grp}}
\newcommand{\Gpd}{\cat{Gpd}}
\newcommand{\Ab}{\cat{Ab}}
\newcommand{\Rng}{\cat{Rng}}
\newcommand{\CRng}{\cat{CRng}}
\newcommand{\Mod}{\cat{Mod}}
\newcommand{\Alg}{\cat{Alg}}
\newcommand{\Rep}{\cat{Rep}}
\newcommand{\RMod}[1][R]{#1\text{-}\cat{Mod}}
\newcommand{\ModR}[1][R]{\cat{Mod}\text{-}#1}
\newcommand{\RAlg}[1][R]{#1\text{-}\cat{Alg}}
\renewcommand{\Top}{\cat{Top}}
%\@ifundefined{Top}{\newcommand{\Top}{\cat{Top}}}{\renewcommand{\Top}{\cat{Top}}}
\newcommand{\SMan}{\cat{SMan}}
\newcommand{\SmMan}{\cat{SmMan}}
\newcommand{\Sm}{\cat{Sm}}
\newcommand{\PTop}{\cat{Top_*}}
\newcommand{\hTop}{\cat{hTop}}
\newcommand{\Vect}{\cat{Vect}}
\newcommand{\PShf}{\operatorname{\cat{PShf}}}
\newcommand{\Shf}{\operatorname{\cat{Shf}}}
\newcommand{\PSh}{\operatorname{\cat{PSh}}}
\newcommand{\Sh}{\operatorname{\cat{Sh}}}
\newcommand{\RSp}{\cat{RSp}}
\newcommand{\Sch}{\cat{Sch}}
\newcommand{\AffSch}{\cat{AffSch}}
\newcommand{\sSet}{\cat{sSet}}
\newcommand{\Ch}{\cat{Ch}}

% Python Code Highlighting Style
\definecolor{codebg}{rgb}{0.9,0.9,0.9}
\lstdefinestyle{pythoncode}{
  backgroundcolor=\color{codebg},
  commentstyle=\color{Green},
  keywordstyle=\color{Magenta},
  numberstyle=\tiny\color{Gray},
  stringstyle=\color{ForestGreen},
  basicstyle=\footnotesize,
  breakatwhitespace=false,
  breaklines=true,
  captionpos=b,
  keepspaces=true,
  numbers=left,
  numbersep=5pt,
  showspaces=false,
  showstringspaces=false,
  showtabs=false,
  tabsize=3
}
\lstset{style=pythoncode}

%\usepackage[utf8]{inputenc}
%\tracinglostchars=3
\usepackage{amsmath, amsfonts, amssymb, bm}
\usepackage[dvipsnames]{xcolor}
\usepackage{verbatim,mathtools,graphicx}
\usepackage{tikz, tikz-cd}
\usepackage{enumitem,xparse,url}
\usepackage[most]{tcolorbox}
\usepackage{multirow,array}
\usepackage{hyperref,bookmark}
\usepackage{stmaryrd}
\usepackage{microtype}

% font stuff, taken from Kaobook, but with a different caligraphic font
\usepackage{unicode-math}
\setromanfont[
  Scale=1.04
]{Libertinus Serif}
\setsansfont[
  Scale=1
]{Libertinus Sans}
\setmonofont[
  Scale=.89
]{Liberation Mono}
\setmathfont{Libertinus Math}
\setmathfont{Euler Math}[range=cal]

% Renewed commands
\renewcommand{\injlim}{\varinjlim}
\renewcommand{\projlim}{\varprojlim}

% Column stuff
\newcolumntype{x}[1]{>{\centering\let\newline\\\arraybackslash\hspace{0pt}}p{#1}}


% TikZ stuff
\newsavebox{\pullbacksq}
\sbox\pullbacksq{%
\begin{tikzpicture}%
\draw ({0ex},{0ex}) -- ({1ex},{0ex});%
\draw ({1ex},{0ex}) -- ({1ex},{1ex});%
\end{tikzpicture}}

\newsavebox{\pushoutsq}
\sbox\pushoutsq{%
\begin{tikzpicture}%
\draw (0,0) -- (0ex,1ex);%
\draw (0ex,1ex) -- (1ex,1ex);%
\end{tikzpicture}}

\tikzset{
    labl/.style={anchor=north, rotate=90},
    pullback/.style={commutative diagrams/phantom, "\usebox\pullbacksq" , very near start},
    pushout/.style={commutative diagrams/phantom, "\usebox\pushoutsq" , very near start},
    symbol/.style={%
      draw=none,
      every to/.append style={edge node={node [sloped, allow upside down, auto=false]{$#1$}}}
    }
}


% Environments
% warning box
\newenvironment{warningbox}{\begin{tcolorbox}[colback=red!5!white, colframe=red!80!black, title=\textbf{Warning!}]}{\end{tcolorbox}}
% note box
\newenvironment{notebox}{\begin{tcolorbox}[colback=blue!5!white, colframe=blue!80!black, title=\textbf{Note}]}{\end{tcolorbox}}
% todo box
\newenvironment{todobox}{\begin{tcolorbox}[colback=green!5!white, colframe=green!80!black, title=\textbf{TODO}]}{\end{tcolorbox}}
% tikzcd diagram
\newenvironment{diagram*}{\begin{center}\begin{tikzcd}}{\end{tikzcd}\end{center}}
\newenvironment{diagram}{\begin{equation}\begin{tikzcd}}{\end{tikzcd}\end{equation}\ignorespaces}

\newenvironment{graph*}{\vspace{0.1cm}\begin{center}\begin{tikzcd}[every arrow/.append style={dash,thick}]}{\end{tikzcd}\end{center}\vspace{0.1cm}}
\newenvironment{graph}{\begin{equation}\begin{tikzcd}[every arrow/.append style={dash,thick}]}{\end{tikzcd}\end{equation}\ignorespaces}


% Fancy theorem, proof, example, etc. environments.
\tcbuselibrary{theorems}
\tcbuselibrary{skins}
\tcbuselibrary{breakable}
\tcbuselibrary{vignette}

% styling
\tcbset{
  plategeneric/.style={
    enhanced, breakable,
    borderline west={2pt}{0pt}{white!75!black},
    colback=white,
    top=1.3pt,
    bottom=1.3pt,
    left=7pt,
    right=7pt,
    grow to left by=10pt,
    grow to right by=10pt,
    boxrule=0pt,
    sharp corners
  },
  plate/.style={
    plategeneric,
    frame hidden,
    beforeafter skip balanced=0.22\baselineskip plus 1pt,
    height fixed for=first and middle,
    lines before break=1,
    pad at break*=1.1mm,
    detach title,
    coltitle=black
  },
  fancythm/.style={
    plate,
    top=4pt,
    bottom=4pt,
    borderline west={2pt}{0pt}{Goldenrod!70!black},
    colback=Goldenrod!5!white,
    before upper={\parindent15pt\noindent\textup{\textbf{\tcbtitle.}}\,\,},
    fontupper=\itshape
  },
  fancyconj/.style={
    plate,
    borderline west={2pt}{0pt}{Goldenrod!80!black},
    %colback=Goldenrod!10!white,
    before upper={\parindent15pt\noindent\textup{\textbf{\tcbtitle.}}\,\,},
    fontupper=\itshape
  },
  fancyproof/.style={
    plate,
    borderline west={2pt}{0pt}{Peach!75!black},
    colback=Peach!1!white,
    fonttitle=\itshape,
    before upper={\parindent15pt\noindent\tcbtitle\textit{.}\,\,\,},
    fontupper=\upshape
  },
  fancydef/.style={
    plate,
    borderline west={2pt}{0pt}{NavyBlue!85!black},
    colback=white,
    fonttitle=\bfseries,
    before upper={\parindent15pt\noindent\tcbtitle\textbf{.}\,\,\,}
  },
  fancynotation/.style={
    plate,
    borderline west={2pt}{0pt}{Cyan!65!black},
    colback=white,
    fonttitle=\bfseries,
    before upper={\parindent15pt\noindent\tcbtitle\textbf{.}\,\,\,}
  },
  fancyexample/.style={
    plate,
    borderline west={2pt}{0pt}{Plum!75!black},
    colback=white,
    fonttitle=\bfseries,
    before upper={\parindent15pt\noindent\tcbtitle\textbf{.}\,\,\,}
  },
  fancyremark/.style={
    plate,
    borderline west={2pt}{0pt}{Emerald!75!black},
    colback=white,
    fonttitle=\itshape,
    before upper={\parindent15pt\noindent\tcbtitle\textit{.}\,\,\,}
  },
  fancyexercise/.style={
    plate,
    borderline west={2pt}{0pt}{WildStrawberry!85!black},
    colback=WildStrawberry!10!white,
    fonttitle=\bfseries,
    before upper={\parindent15pt\noindent\tcbtitle\textbf{.}\,\,\,}
  },
  fancysoln/.style={
    plate,
    borderline west={2pt}{0pt}{YellowGreen!85!black},
    colback=white,
    fonttitle=\bfseries,
    before upper={\parindent15pt\noindent\tcbtitle\textbf{.}\,\,\,}
  },
  fancywarning/.style={
    plate,
    borderline west={2pt}{0pt}{Red!75!black},
    colback=Red!15!white,
    before upper={\parindent15pt\noindent\tcbtitle\textbf{.}\,\,\,},
    fonttitle=\bfseries,
    fontupper=\upshape
  },
  fancynote/.style={
    plategeneric,
    grow to right by=4pt,
    borderline west={2pt}{0pt}{ProcessBlue!75!black},
    colback=ProcessBlue!15!white,
    colbacktitle=ProcessBlue!50!white,
    coltitle={ProcessBlue!30!black},
    title={Note},
    fonttitle=\bfseries,
    fontupper=\upshape,
    before upper={\parindent15pt\noindent}
  },
  fancytodo/.style={
    plategeneric,
    grow to right by=4pt,
    borderline west={2pt}{0pt}{Green!75!black},
    colback=Green!15!white,
    colbacktitle=Green!50!white,
    coltitle={Green!40!black},
    title={TO-DO},
    fonttitle=\bfseries,
    fontupper=\upshape,
    before upper={\parindent15pt\noindent}
  }
}

% Command for creating fancy theorem-like environments
% Creates the title format for some parts.
% args: display name, thesection, number (& added text for optional)
\newcommand{\ThmTitleNoOptional}[3]{#1 #2.#3}
\newcommand{\ThmTitleOptional}[4]{#1 #2.#3: #4}
% args: counter prefix (optional), name, display name, style, counter    \ThmTitleOptional{#3}{\thesection}{\arabic{#5}}{##1}
\NewDocumentCommand\goodnewtcbtheorem{ommmm}{%
  \IfNoValueTF{#1}{%
  \newenvironment{#2}[1][]{\refstepcounter{#5}\begin{tcolorbox}[title={\if\relax\detokenize{##1}\relax{\ThmTitleNoOptional{#3}{\thesection}{\arabic{#5}}}\else{\ThmTitleOptional{#3}{\thesection}{\arabic{#5}}{##1}}\fi},#4]}{\end{tcolorbox}}%
  \newenvironment{#2*}[1][]{\begin{tcolorbox}[title={\if\relax\detokenize{##1}\relax{#3}\else{#3: ##1}\fi},#4]}{\end{tcolorbox}}%
  }{%
  \newenvironment{#2}[1][]{\refstepcounter{#5}\begin{tcolorbox}[title={\if\relax\detokenize{##1}\relax{#3 #1\arabic{#5}}\else{#3 #1\arabic{#5}: ##1}\fi},#4]}{\end{tcolorbox}}%
  \newenvironment{#2*}[1][]{\begin{tcolorbox}[title={\if\relax\detokenize{##1}\relax{#3}\else{#3: ##1}\fi},#4]}{\end{tcolorbox}}%
  }
}
\NewDocumentCommand\goodrenewtcbtheorem{ommmm}{%
  \IfNoValueTF{#1}{%
  \renewenvironment{#2}[1][]{\refstepcounter{#5}\begin{tcolorbox}[title={\if\relax\detokenize{##1}\relax{#3 \thesection.\arabic{#5}}\else{#3 \thesection.\arabic{#5}: ##1}\fi},#4]}{\end{tcolorbox}}%
  \renewenvironment{#2*}[1][]{\begin{tcolorbox}[title={\if{#3}\else{#3: ##1}\fi},#4]}{\end{tcolorbox}}%
  }{%
  \renewenvironment{#2}[1][]{\refstepcounter{#5}\begin{tcolorbox}[title={\if\relax\detokenize{##1}\relax{#3 #1\arabic{#5}}\else{#3 #1\arabic{#5}: ##1}\fi},#4]}{\end{tcolorbox}}%
  \renewenvironment{#2*}[1][]{\begin{tcolorbox}[title={\if{#3}\else{#3: ##1}\fi},#4]}{\end{tcolorbox}}%
  }
}

% Counters
\newcounter{thmcounter}
\numberwithin{thmcounter}{section}

\newcounter{excounter}
\newcounter{probcounter}
\newcounter{solcounter}
\newcounter{taskcounter}

% Theorems, Lemmas, etc.
% Theorem style
\goodnewtcbtheorem{theorem}{Theorem}{fancythm}{thmcounter}
\goodnewtcbtheorem{theoremdef}{Theorem-Definition}{fancythm}{thmcounter}
\goodnewtcbtheorem{proposition}{Proposition}{fancythm}{thmcounter}
\goodnewtcbtheorem{propositiondef}{Proposition-Definition}{fancythm}{thmcounter}
\goodnewtcbtheorem{corollary}{Corollary}{fancythm}{thmcounter}
\goodnewtcbtheorem{lemma}{Lemma}{fancythm}{thmcounter}
\goodnewtcbtheorem{fact}{Fact}{fancythm}{thmcounter}

\goodnewtcbtheorem{conjecture}{Conjecture}{fancyconj, fontupper=\upshape}{thmcounter}

\goodnewtcbtheorem{goal}{Goal}{fancyconj, fontupper=\upshape}{thmcounter}

% Definition style
\goodnewtcbtheorem{definition}{Definition}{fancydef}{thmcounter}

% Notation style
\goodnewtcbtheorem{notation}{Notation}{fancynotation}{thmcounter}
\goodnewtcbtheorem{terminology}{Terminology}{fancynotation}{thmcounter}
\goodnewtcbtheorem{convention}{Convention}{fancynotation}{thmcounter}
\goodnewtcbtheorem{question}{Question}{fancynotation}{thmcounter}
\goodnewtcbtheorem{axiom}{Axiom}{fancynotation}{thmcounter}

% Example style
\goodnewtcbtheorem{example}{Example}{fancyexample}{thmcounter}
\goodnewtcbtheorem{construction}{Construction}{fancyexample}{thmcounter}
\goodnewtcbtheorem{sketch}{Sketch}{fancyexample}{thmcounter}

% Remark style
\goodnewtcbtheorem{remark}{Remark}{fancyremark}{thmcounter}
\goodnewtcbtheorem{discussion}{Discussion}{fancyremark}{thmcounter}

% Exercise style, empty [] necessary.
\goodnewtcbtheorem[]{exercise}{Exercise}{fancyexercise}{excounter}
\goodnewtcbtheorem[]{task}{Task}{fancyexercise}{taskcounter}
\goodnewtcbtheorem[]{problem}{Problem}{fancyexercise}{probcounter}

% Solution style, empty [] necessary.
\goodnewtcbtheorem[]{solution}{Solution}{fancysoln}{solcounter}

% Warning style
\goodnewtcbtheorem{warning}{Warning}{fancywarning}{thmcounter}

% fancy note and todo
\newtcolorbox{note}{fancynote}
\newtcolorbox{todo}{fancytodo}

% Proof environment.
\newcommand{\qedsymbol}{$\blacksquare$}
\newtcolorbox{proof}[1][Proof]{fancyproof, title={#1}}
\AtEndEnvironment{proof}{\null\hfill\qedsymbol}
% "Proof Sketch" environment
\newenvironment{proofsketch}{\begin{proof}[Proof sketch]\renewcommand\qedsymbol{$\square$}}{\end{proof}\renewcommand\qedsymbol{$\blacksquare$}}

% QED symbol should be black tombstone.
\renewcommand\qedsymbol{$\blacksquare$}
\makeatletter
\newcommand{\oset}[3][0ex]{%
  \mathrel{\mathop{#3}\limits^{
    \vbox to#1{\kern-2\ex@
    \hbox{$\scriptstyle#2$}\vss}}}}
\makeatother

% utility "such that" command.
\newcommand{\st}{
  \text{ s.t. }
}

\newcommand{\bigmid}{\biggm|}
\newcommand{\ldsq}{\llbracket}
\newcommand{\rdsq}{\rrbracket}
\newcommand{\dashto}{\dashrightarrow}

\newcommand{\lcm}{\operatorname{lcm}}

% arrows
\newcommand{\longto}{\longrightarrow}
\newcommand{\ot}{\leftarrow}
\newcommand{\xto}{\xrightarrow}
\newcommand{\xot}{\xleftarrow}
\newcommand{\longot}{\longleftarrow}
\newcommand{\Longto}{\Longrightarrow}
\newcommand{\oT}{\Leftarrow}
\newcommand{\To}{\Rightarrow}
\newcommand{\longoT}{\Longleftarrow}

% useful
\newcommand{\fd}{\textup{f.d.}}
\newcommand{\ev}{\operatorname{ev}}

% mathbf
\newcommand{\bfA}{\ensuremath{\symbf{A}}}
\newcommand{\bfB}{\ensuremath{\symbf{B}}}
\newcommand{\bfC}{\ensuremath{\symbf{C}}}
\newcommand{\bfD}{\ensuremath{\symbf{D}}}
\newcommand{\bfE}{\ensuremath{\symbf{E}}}
\newcommand{\bfF}{\ensuremath{\symbf{F}}}
\newcommand{\bfG}{\ensuremath{\symbf{G}}}
\newcommand{\bfH}{\ensuremath{\symbf{H}}}
\newcommand{\bfI}{\ensuremath{\symbf{I}}}
\newcommand{\bfJ}{\ensuremath{\symbf{J}}}
\newcommand{\bfK}{\ensuremath{\symbf{K}}}
\newcommand{\bfL}{\ensuremath{\symbf{L}}}
\newcommand{\bfM}{\ensuremath{\symbf{M}}}
\newcommand{\bfN}{\ensuremath{\symbf{N}}}
\newcommand{\bfO}{\ensuremath{\symbf{O}}}
\newcommand{\bfP}{\ensuremath{\symbf{P}}}
\newcommand{\bfQ}{\ensuremath{\symbf{Q}}}
\newcommand{\bfR}{\ensuremath{\symbf{R}}}
\newcommand{\bfS}{\ensuremath{\symbf{S}}}
\newcommand{\bfT}{\ensuremath{\symbf{T}}}
\newcommand{\bfU}{\ensuremath{\symbf{U}}}
\newcommand{\bfV}{\ensuremath{\symbf{V}}}
\newcommand{\bfW}{\ensuremath{\symbf{W}}}
\newcommand{\bfX}{\ensuremath{\symbf{X}}}
\newcommand{\bfY}{\ensuremath{\symbf{Y}}}
\newcommand{\bfZ}{\ensuremath{\symbf{Z}}}
\newcommand{\bfDelta}{\ensuremath{\symbf{\Delta}}}

% mathsf
\newcommand{\sfA}{\ensuremath{\symsf{A}}}
\newcommand{\sfB}{\ensuremath{\symsf{B}}}
\newcommand{\sfC}{\ensuremath{\symsf{C}}}
\newcommand{\sfD}{\ensuremath{\symsf{D}}}
\newcommand{\sfE}{\ensuremath{\symsf{E}}}
\newcommand{\sfF}{\ensuremath{\symsf{F}}}
\newcommand{\sfG}{\ensuremath{\symsf{G}}}
\newcommand{\sfH}{\ensuremath{\symsf{H}}}
\newcommand{\sfI}{\ensuremath{\symsf{I}}}
\newcommand{\sfJ}{\ensuremath{\symsf{J}}}
\newcommand{\sfK}{\ensuremath{\symsf{K}}}
\newcommand{\sfL}{\ensuremath{\symsf{L}}}
\newcommand{\sfM}{\ensuremath{\symsf{M}}}
\newcommand{\sfN}{\ensuremath{\symsf{N}}}
\newcommand{\sfO}{\ensuremath{\symsf{O}}}
\newcommand{\sfP}{\ensuremath{\symsf{P}}}
\newcommand{\sfQ}{\ensuremath{\symsf{Q}}}
\newcommand{\sfR}{\ensuremath{\symsf{R}}}
\newcommand{\sfS}{\ensuremath{\symsf{S}}}
\newcommand{\sfT}{\ensuremath{\symsf{T}}}
\newcommand{\sfU}{\ensuremath{\symsf{U}}}
\newcommand{\sfV}{\ensuremath{\symsf{V}}}
\newcommand{\sfW}{\ensuremath{\symsf{W}}}
\newcommand{\sfX}{\ensuremath{\symsf{X}}}
\newcommand{\sfY}{\ensuremath{\symsf{Y}}}
\newcommand{\sfZ}{\ensuremath{\symsf{Z}}}

% mathbb
\newcommand{\bbA}{\ensuremath{\symbb{A}}}
\newcommand{\bbB}{\ensuremath{\symbb{B}}}
\newcommand{\bbC}{\ensuremath{\symbb{C}}}
\newcommand{\bbD}{\ensuremath{\symbb{D}}}
\newcommand{\bbE}{\ensuremath{\symbb{E}}}
\newcommand{\bbF}{\ensuremath{\symbb{F}}}
\newcommand{\bbG}{\ensuremath{\symbb{G}}}
\newcommand{\bbH}{\ensuremath{\symbb{H}}}
\newcommand{\bbI}{\ensuremath{\symbb{I}}}
\newcommand{\bbJ}{\ensuremath{\symbb{J}}}
\newcommand{\bbK}{\ensuremath{\symbb{K}}}
\newcommand{\bbL}{\ensuremath{\symbb{L}}}
\newcommand{\bbM}{\ensuremath{\symbb{M}}}
\newcommand{\bbN}{\ensuremath{\symbb{N}}}
\newcommand{\bbO}{\ensuremath{\symbb{O}}}
\newcommand{\bbP}{\ensuremath{\symbb{P}}}
\newcommand{\bbQ}{\ensuremath{\symbb{Q}}}
\newcommand{\bbR}{\ensuremath{\symbb{R}}}
\newcommand{\bbS}{\ensuremath{\symbb{S}}}
\newcommand{\bbT}{\ensuremath{\symbb{T}}}
\newcommand{\bbU}{\ensuremath{\symbb{U}}}
\newcommand{\bbV}{\ensuremath{\symbb{V}}}
\newcommand{\bbW}{\ensuremath{\symbb{W}}}
\newcommand{\bbX}{\ensuremath{\symbb{X}}}
\newcommand{\bbY}{\ensuremath{\symbb{Y}}}
\newcommand{\bbZ}{\ensuremath{\symbb{Z}}}

% mathcal
\newcommand{\calA}{\ensuremath{\symcal{A}}}
\newcommand{\calB}{\ensuremath{\symcal{B}}}
\newcommand{\calC}{\ensuremath{\symcal{C}}}
\newcommand{\calD}{\ensuremath{\symcal{D}}}
\newcommand{\calE}{\ensuremath{\symcal{E}}}
\newcommand{\calF}{\ensuremath{\symcal{F}}}
\newcommand{\calG}{\ensuremath{\symcal{G}}}
\newcommand{\calH}{\ensuremath{\symcal{H}}}
\newcommand{\calI}{\ensuremath{\symcal{I}}}
\newcommand{\calJ}{\ensuremath{\symcal{J}}}
\newcommand{\calK}{\ensuremath{\symcal{K}}}
\newcommand{\calL}{\ensuremath{\symcal{L}}}
\newcommand{\calM}{\ensuremath{\symcal{M}}}
\newcommand{\calN}{\ensuremath{\symcal{N}}}
\newcommand{\calO}{\ensuremath{\symcal{O}}}
\newcommand{\calP}{\ensuremath{\symcal{P}}}
\newcommand{\calQ}{\ensuremath{\symcal{Q}}}
\newcommand{\calR}{\ensuremath{\symcal{R}}}
\newcommand{\calS}{\ensuremath{\symcal{S}}}
\newcommand{\calT}{\ensuremath{\symcal{T}}}
\newcommand{\calU}{\ensuremath{\symcal{U}}}
\newcommand{\calV}{\ensuremath{\symcal{V}}}
\newcommand{\calW}{\ensuremath{\symcal{W}}}
\newcommand{\calX}{\ensuremath{\symcal{X}}}
\newcommand{\calY}{\ensuremath{\symcal{Y}}}
\newcommand{\calZ}{\ensuremath{\symcal{Z}}}

% mathscr
\newcommand{\scrA}{\ensuremath{\symscr{A}}}
\newcommand{\scrB}{\ensuremath{\symscr{B}}}
\newcommand{\scrC}{\ensuremath{\symscr{C}}}
\newcommand{\scrD}{\ensuremath{\symscr{D}}}
\newcommand{\scrE}{\ensuremath{\symscr{E}}}
\newcommand{\scrF}{\ensuremath{\symscr{F}}}
\newcommand{\scrG}{\ensuremath{\symscr{G}}}
\newcommand{\scrH}{\ensuremath{\symscr{H}}}
\newcommand{\scrI}{\ensuremath{\symscr{I}}}
\newcommand{\scrJ}{\ensuremath{\symscr{J}}}
\newcommand{\scrK}{\ensuremath{\symscr{K}}}
\newcommand{\scrL}{\ensuremath{\symscr{L}}}
\newcommand{\scrM}{\ensuremath{\symscr{M}}}
\newcommand{\scrN}{\ensuremath{\symscr{N}}}
\newcommand{\scrO}{\ensuremath{\symscr{O}}}
\newcommand{\scrP}{\ensuremath{\symscr{P}}}
\newcommand{\scrQ}{\ensuremath{\symscr{Q}}}
\newcommand{\scrR}{\ensuremath{\symscr{R}}}
\newcommand{\scrS}{\ensuremath{\symscr{S}}}
\newcommand{\scrT}{\ensuremath{\symscr{T}}}
\newcommand{\scrU}{\ensuremath{\symscr{U}}}
\newcommand{\scrV}{\ensuremath{\symscr{V}}}
\newcommand{\scrW}{\ensuremath{\symscr{W}}}
\newcommand{\scrX}{\ensuremath{\symscr{X}}}
\newcommand{\scrY}{\ensuremath{\symscr{Y}}}
\newcommand{\scrZ}{\ensuremath{\symscr{Z}}}

% mathfrak
\newcommand{\frakA}{\ensuremath{\symfrak{A}}}
\newcommand{\frakB}{\ensuremath{\symfrak{B}}}
\newcommand{\frakC}{\ensuremath{\symfrak{C}}}
\newcommand{\frakD}{\ensuremath{\symfrak{D}}}
\newcommand{\frakE}{\ensuremath{\symfrak{E}}}
\newcommand{\frakF}{\ensuremath{\symfrak{F}}}
\newcommand{\frakG}{\ensuremath{\symfrak{G}}}
\newcommand{\frakH}{\ensuremath{\symfrak{H}}}
\newcommand{\frakI}{\ensuremath{\symfrak{I}}}
\newcommand{\frakJ}{\ensuremath{\symfrak{J}}}
\newcommand{\frakK}{\ensuremath{\symfrak{K}}}
\newcommand{\frakL}{\ensuremath{\symfrak{L}}}
\newcommand{\frakM}{\ensuremath{\symfrak{M}}}
\newcommand{\frakN}{\ensuremath{\symfrak{N}}}
\newcommand{\frakO}{\ensuremath{\symfrak{O}}}
\newcommand{\frakP}{\ensuremath{\symfrak{P}}}
\newcommand{\frakQ}{\ensuremath{\symfrak{Q}}}
\newcommand{\frakR}{\ensuremath{\symfrak{R}}}
\newcommand{\frakS}{\ensuremath{\symfrak{S}}}
\newcommand{\frakT}{\ensuremath{\symfrak{T}}}
\newcommand{\frakU}{\ensuremath{\symfrak{U}}}
\newcommand{\frakV}{\ensuremath{\symfrak{V}}}
\newcommand{\frakW}{\ensuremath{\symfrak{W}}}
\newcommand{\frakX}{\ensuremath{\symfrak{X}}}
\newcommand{\frakY}{\ensuremath{\symfrak{Y}}}
\newcommand{\frakZ}{\ensuremath{\symfrak{Z}}}


% Calculus, transforms, and other stuff
\newcommand*{\diff}{\mathop{}\!\mathrm{d}}
\newcommand{\Res}{\operatorname{Res}}
\newcommand{\p}{\partial}
\newcommand{\diam}{\mathrm{diam}\,}
\renewcommand{\Im}{\operatorname{Im}}
\renewcommand{\Re}{\operatorname{Re}}
\newcommand{\sinc}{\mathrm{sinc}}

% Common Sets
\newcommand{\Z}{\symbb{Z}}
\newcommand{\Q}{\symbb{Q}}
\newcommand{\R}{\symbb{R}}
\newcommand{\C}{\symbb{C}}
\newcommand{\D}{\symbb{D}}
\newcommand{\N}{\symbb{N}}
\newcommand{\E}{\symbb{E}}
\newcommand{\F}{\symbb{F}}
\newcommand{\G}{\symbb{G}}
\newcommand{\A}{\symbb{A}}
\newcommand{\TT}{\symbb{T}}
\newcommand{\PP}{\symbb{P}}
\newcommand{\pow}{
  \symcal{P}
}

% Algebra
\newcommand{\adj}{\operatorname{adj}}
\newcommand{\ind}{\operatorname{ind}}
\newcommand{\height}{\operatorname{ht}}
\newcommand{\dual}{\vee}
\newcommand{\sep}{\text{sep}}
\newcommand{\nor}{\text{nor}}
\newcommand{\red}{\mathrm{red}}
\newcommand{\gal}{\text{gal}}
\newcommand{\ideal}{\symfrak}
\newcommand{\GL}{\operatorname{GL}}
\newcommand{\SL}{\operatorname{SL}}
\newcommand{\M}{\operatorname{M}}
\newcommand{\nrmleq}{\trianglelefteq}
\newcommand{\nrmgeq}{\trianglerighteq}
\newcommand{\cyclic}[1]{\langle #1 \rangle}
\newcommand{\coimg}{\operatorname{coim}}
\newcommand{\img}{\operatorname{im}}
\newcommand{\Frac}{\operatorname{Frac}}
\newcommand{\aut}{\operatorname{Aut}}
\newcommand{\Aut}{\operatorname{Aut}}
\newcommand{\End}{\operatorname{End}}
\newcommand{\Ann}{\operatorname{Ann}}
\newcommand{\Gal}{\operatorname{Gal}}
\newcommand{\Tor}{\operatorname{Tor}}
\newcommand{\coker}{\operatorname{coker}}
\newcommand{\eq}{\operatorname{eq}}
\newcommand{\coeq}{\operatorname{coeq}}
\newcommand{\spec}{\operatorname{Spec}}
\newcommand{\Spec}{\operatorname{Spec}}
\newcommand{\Proj}{\operatorname{Proj}}
\newcommand{\Ox}{\symcal{O}}
\newcommand{\Open}{\operatorname{Open}}
\newcommand{\Char}{\operatorname{char}}
\newcommand{\Span}{\operatorname{Span}}
\newcommand{\intO}{\symscr{O}}
\newcommand{\Pic}{\operatorname{Pic}}
\newcommand{\Ext}{\operatorname{Ext}}
\newcommand{\KProj}{\operatorname{KProj}}


\newcommand{\hooktwoheadrightarrow}{\hookrightarrow\mathrel{\mspace{-15mu}}\rightarrow}
\makeatletter
\newcommand{\xhooktwoheadrightarrow}[2][]{%
  \lhook\joinrel
  \ext@arrow 0359\rightarrowfill@ {#1}{#2}%
  \mathrel{\mspace{-15mu}}\rightarrow
}
\makeatother
\newcommand{\iso}{\xrightarrow{\sim}}
\newcommand{\isom}{\hooktwoheadrightarrow}
\newcommand{\inj}{\hookrightarrow}
\newcommand{\sur}{\twoheadrightarrow}
\newcommand{\orbits}{\mathrm{O}}
\newcommand{\rank}{\operatorname{rank}}
\newcommand{\nullity}{\operatorname{nullity}}
\newcommand{\tr}{\operatorname{tr}}

% Geometry
\newcommand{\simh}{\sim_\textup{h}}
\newcommand{\Supp}{\operatorname{Supp}}
\newcommand{\spa}{\operatorname{sp}}
\newcommand{\HH}{\operatorname{H}}
\newcommand{\hh}{\operatorname{h}}
\newcommand{\mult}{\operatorname{mult}}
\newcommand{\codim}{\operatorname{codim}}
\newcommand{\Sp}{\mathrm{Sp}}
\newcommand{\Sk}{\operatorname{Sk}}
\newcommand{\Sing}{\operatorname{Sing}}
\newcommand{\ho}{\operatorname{ho}}

% Category Theory
\newcommand{\Qis}{\operatorname{Qis}}
\newcommand{\thick}{\operatorname{thick}}
\newcommand{\Lan}{\operatorname{Lan}}
\newcommand{\Ran}{\operatorname{Ran}}
\newcommand{\Com}{\operatorname{Com}}
\newcommand{\Ind}{\operatorname{Ind}}
\newcommand{\Pro}{\operatorname{Pro}}
\newcommand{\pre}{\mathrm{pre}}
\newcommand{\op}{\mathrm{op}}
\newcommand{\Hom}{\operatorname{Hom}}
\newcommand{\Map}{\operatorname{Map}}
\newcommand{\iHom}{\underline{\Hom}}
\newcommand{\SHom}{\symcal{H}\hspace{-0.0833em}om}
\newcommand{\Mor}{\operatorname{Mor}}
\newcommand{\Fun}{\operatorname{Fun}}
\newcommand{\Nat}{\operatorname{Nat}}
\newcommand{\Der}{\operatorname{Der}}
\newcommand{\hocolim}{\operatorname{hocolim}}
\newcommand{\sheaf}{\symscr}
\newcommand{\sh}{\text{sh}}
\newcommand{\cat}{\symbf}
\newcommand{\natcat}[1]{ \symbf{[#1]} }
\newcommand{\dom}{\operatorname{dom}}
\newcommand{\cod}{\operatorname{cod}}
\newcommand{\id}{\mathrm{id}}
\newcommand{\Cones}{\operatorname{Cones}}
\newcommand{\ob}{\operatorname{Ob}}
\newcommand{\Ob}{\ob}
\newcommand{\Ar}{\operatorname{Ar}}
\newcommand{\natto}{%
  \mathrel{\vbox{\offinterlineskip
    \mathsurround=0pt
    \ialign{\hfil##\hfil\cr
      \normalfont\scalebox{1.2}{.}\cr
%      \noalign{\kern-.05ex}
      $\rightarrow$\cr}
  }}%
}

% Some Categories
\newcommand{\Cat}{\cat{Cat}}
\newcommand{\inftyCat}{\infty\text{-}\cat{Cat}}
\newcommand{\Set}{\cat{Set}}
\newcommand{\PSet}{\cat{Set_*}}
\newcommand{\Ens}[1]{\cat{Ens}_{#1}}
\newcommand{\Mon}{\cat{Mon}}
\newcommand{\CMon}{\cat{CMon}}
\newcommand{\Grp}{\cat{Grp}}
\newcommand{\Gpd}{\cat{Gpd}}
\newcommand{\Ab}{\cat{Ab}}
\newcommand{\Rng}{\cat{Rng}}
\newcommand{\CRng}{\cat{CRng}}
\newcommand{\Mod}{\cat{Mod}}
\newcommand{\Alg}{\cat{Alg}}
\newcommand{\Rep}{\cat{Rep}}
\newcommand{\RMod}[1][R]{#1\text{-}\cat{Mod}}
\newcommand{\ModR}[1][R]{\cat{Mod}\text{-}#1}
\newcommand{\RAlg}[1][R]{#1\text{-}\cat{Alg}}
\newcommand{\Top}{\cat{Top}}
%\@ifundefined{Top}{\newcommand{\Top}{\cat{Top}}}{\renewcommand{\Top}{\cat{Top}}}
\newcommand{\SMan}{\cat{SMan}}
\newcommand{\SmMan}{\cat{SmMan}}
\newcommand{\Sm}{\cat{Sm}}
\newcommand{\PTop}{\cat{Top_*}}
\newcommand{\hTop}{\cat{hTop}}
\newcommand{\Vect}{\cat{Vect}}
\newcommand{\PShf}{\operatorname{\cat{PShf}}}
\newcommand{\Shf}{\operatorname{\cat{Shf}}}
\newcommand{\PSh}{\operatorname{\cat{PSh}}}
\newcommand{\Sh}{\operatorname{\cat{Sh}}}
\newcommand{\RSp}{\cat{RSp}}
\newcommand{\Sch}{\cat{Sch}}
\newcommand{\AffSch}{\cat{AffSch}}
\newcommand{\sSet}{\cat{sSet}}
\newcommand{\Ch}{\cat{Ch}}

%\input{preamble}
\usepackage[textwidth=15.5cm,top=2.5cm,bottom=2.5cm]{geometry}
\usepackage{comment}

\title{Lecture Notes}
\author{on assorted topics}
\date{Carl-Fredrik Lidgren}

\defaultfontfeatures[HaranoAjiMincho]{
    Renderer=HarfBuzz,
    Script=Kana,
    Scale=MatchUppercase,
    UprightFont=*-Light,
    BoldFont=*-Medium,
    Extension=.otf
}

\newfontfamily\kanafont{HaranoAjiMincho}
\DeclareRobustCommand\yo{\textup{\kanafont よ}}

\newcommand{\1}{\ensuremath{\mathbb{1}}}
\newcommand{\2}{\ensuremath{\mathbb{2}}}
\newcommand{\3}{\ensuremath{\mathbb{3}}}
\newcommand{\n}{\ensuremath{\mathbb{n}}}
\newcommand{\I}{\ensuremath{\mathbb{I}}}
\newcommand{\Isofib}{\ensuremath{\mathsf{Isofib}}}

\newcommand{\Prob}{\operatorname{Prob}}
\newcommand{\Aff}{\operatorname{Aff}}

\newcommand{\ladj}{\vdash}
\newcommand{\fib}{\operatorname{fib}}
\newcommand{\cof}{\operatorname{cof}}
\newcommand{\cofib}{\operatorname{cofib}}

\newcommand{\bp}[1]{\prescript{p}{}{#1}}
\newcommand{\bperp}[1]{\prescript{\perp}{}{#1}}

\usepackage[backend=biber, style=alphabetic]{biblatex}
\addbibresource{bibliography.bib}



\begin{document}
\maketitle


\tableofcontents

\clearpage
\setcounter{section}{-1}
\section{Welcome}
\subsection{Introduction}
This document contains an assortment of lecture notes prepared initially for use
in lectures of 2--3 hours, focused at a listener with a decent amount of
mathematical maturity and some experience with category theory, algebra,
topology, and so on. There is no overarching goal, other than to cover
interesting topics in and around category theory, homological algebra, geometry,
higher category theory, and higher algebra.

Very often, the notes have been prepared without too much care, and so one should expect mistakes, occasional hand-waving, and sub-optimal approaches. Hopefully, all the essential ideas
remain correct. On a related note, while there has been some effort put into making the exposition clear, we are sometimes rather terse as it makes it easier to use the notes
for lectures, where the exposition may be expanded upon in any case.

References are not always carefully tracked. If in doubt, assume no originality.

\subsection{Foundations}
In this preliminary section, we spell out the general foundational framework we adopt in these notes. Since foundations will never play a \emph{huge} role in what we do, or at least
it won't affect how we do things generally, we will not be too careful. However, the crux of it is as follows: we adopt your favourite common axiomatics for set theory, such as ZFC, along with
Grothendieck's universe axiom. A (Grothendieck) universe is essentially a model for ZF set theory within ZFC set theory itself, so it is a set wherein one can do all basic expected operations (and contains \(\N\)).
The universe axiom postulates that every set is contained in a universe. In particular, we may pick some universe \(\bbU_1\ni\varnothing\), the elements of which we call the \emph{small sets.} Iterating this, the axiom
also implies the existence of a hierarchy of universes
\[ \bbU_1 \in \bbU_2 \in \bbU_3\in \cdots \in \bbU_n\in\cdots \]
and we call elements of \(\bbU_2\) \emph{large,} elements of \(\bbU_3\) \emph{very large,} and so on. If we have picked some universe \(\bbU\), we may also refer to a set \(X\) as \(\bbU\)-small to mean that \(X\in\bbU\), or
that \(X\) is in bijection with a set in \(\bbU\). Note that we will essentially never need to use this nor think about it in any detail. To be noted as well is that in this framework, the notion of a \emph{class} is subsumed
by the notion of a set, in the sense that we may think of elements of \(\bbU_2\backslash\bbU_1\) as analogous to ``proper classes.'' If one wishes, type theory can play a similar role, where one demands a hierarchy of type universes instead.

\subsection{Planned contents}
Some planned contents already have a rough lecture section assigned to them, but some others are a lot looser and so are hard to pin down enough for that to be reasonable, and may span many lectures.
Here is a list to which I aspire.
\begin{enumerate}[label=(\arabic*)]
\item The theory of \((\infty,1)\)-categories, blending the standard approaches (say, \cite{lurie-htt} and \cite{cisinski-book}) with the synthetic approach of \cite{riehl-verity-elements}.
\item Sheaves on sites, probably following \cite{kashiwara-schapira-book}. Relating sheaves between sites will be a focus. Maybe something from \cite[§6.2.2]{lurie-htt} and \cite[§1.3.1]{lurie-sag}
to include \(\infty\)-categorical aspects.
\item Approximable triangulated categories.
\item Perverse sheaves, as an application of the gluing of t-structures.
\item Brown representability.
\item Topics surrounding model categories: Bousfield localizations, stable model categories, algebraic small object argument?
\item Operads, due to their importance in understanding e.g. \(\E_n\)-rings and their modules.
\item Huber's adic spaces, as a ``reasonably elementary'' formalism for rigid analytic geometry.
\item Condensed sets, with a focus on eventually building up to condensed analytic geometry (via Clausen \& Scholze's analytic stacks).
\item \(K\)-theory, particularly of stable \(\infty\)-categories or Waldhausen \(\infty\)-categories. Perhaps recent research extending the \(K\)-theory machine to certain large categories (dualizable ones), à la Efimov.
\item Waldhausen categories, i.e.\ categories with cofibrations and weak equivalences.
\item ``Lower'' higher category theory, so 2-categories and bicategories. (Virtual) double categories, for their applications to \(\infty\)-cosmoi.
\item Topoi and \(\infty\)-topoi.
\item The theorem of Hoshino--Kato--Miyachi on t-structures generated by silting objects. Also requires the computation of \(\Ext^1\) in the heart of a t-structure
in terms of the Hom-sets in the ambient triangulated category.
\item Witt vectors.
\item Tensor triangular geometry.
\end{enumerate}

%!TEX root = ../lectures.tex

\section{Abelian categories}
\subsection{(Pre-)additive categories \& additive functors}
\begin{definition}
	A \emph{pre-additive category} is a category enriched in Abelian groups. In particular, a pre-additive category \(\calC\) consists of
	\begin{enumerate}[label=(\arabic*)]
	\item a set of objects \(\Ob(\calC)\),
	\item for each pair of objects \((x,y)\in\Ob(\calC)\times\Ob(\calC)\), an Abelian group \(\Hom_{\calC}(x,y) = \calC(x,y)\), and
	\item for each triple of objects \((x,y,z)\in \Ob(\calC)\times \Ob(\calC)\times \Ob(\calC)\) a composition law
	\[ \circ\!:\calC(y,z)\otimes\calC(x,y)\to\calC(x,z) \]
	which is associative and unital.
	\end{enumerate}
	That is, it's just a category where each Hom-set has the structure of an Abelian group and composition is bilinear.
\end{definition}
\begin{remark}
	This implies that for each \(x,y\in\calC\), the set \(\calC(x,y)\) has a distinguished element \(0\) which is absorbative. This is a simple
	computation:
	\[ 0\circ f = (0+0)\circ f = 0\circ f + 0\circ f \implies 0 = 0\circ f. \]
	The other direction for composition is dual. Categories with this property (or more general ones) are sometimes called \emph{pointed,} though this terminology
	is also used for the stricter notion of a category with a zero object.
\end{remark}

\begin{exercise}
	Let \(\calC\) be a pre-additive category, and suppose that there is an initial object \(\varnothing\in\calC\). Show that \(\varnothing\) is terminal.
	Dually, show that any terminal object \(*\in\calC\) is also initial.
\end{exercise}

Our first goal is to show that the above exercise generalizes to arbitrary finite (co)products; note that initial/terminal objects are exactly \emph{empty} (co)products.

\begin{construction}\label{construction:pre-additive-product-inclusions}
	Let \(\calC\) be a pointed category (i.e. one that admits zero morphisms), let \(x_1,x_2\in\calC\), and suppose that the product \(x_1\times x_2\) exists. Denote by
	\[ p_k\!:x_1\times x_2 \to x_k,\quad k=1,2 \]
	the canonical projection maps. We can construct maps
	\[ i_k\!:x_k\to x_1\times x_2,\quad k=1,2 \]
	by applying the universal property of the product to \(\id_{x_k}\) and \(0\).
\end{construction}
\begin{proposition}
	Let \(x_k\), \(p_k\), and \(i_k\) be as above. Then
	\[ \id_{x_1\times x_2} = (i_1\circ p_1) + (i_2\circ p_2). \]
\end{proposition}
\begin{proof}
By universal property, the following computation suffices:
\[ p_k \circ ((i_1\circ p_1) + (i_2\circ p_2)) = p_k\circ i_k\circ p_k = p_k \]
where we use that \(p_k\circ i_k = \id\).
\end{proof}
\begin{remark}
	Intuitively, this is the statement that
	\[ (a,b) = (a,0) + (0,b). \]
\end{remark}
\begin{proposition}\label{prop:pre-additive-direct-sum-data}
	Let \(\calC\) be a pre-additive category, and let \(x_1,x_2,y\in\calC\). Suppose we have maps
	\[ p_k\!:y\to x_k,\quad i_k\!:x_k\to y \]
	such that
	\[ \id_{y} = (i_1\circ p_1) + (i_2\circ p_2),\quad p_j\circ i_k = \begin{cases} \id_{x_k} & \text{if }j=k,\\ 0 & \text{if }j\not=k. \end{cases} \]
	Then
	\begin{enumerate}[label=(\arabic*)]
	\item \((y,p_1,p_2)\) defines a product \(x_1\times x_2\), and
	\item \((y,i_1,i_2)\) defines a coproduct \(x_1\amalg x_2\).
	\end{enumerate}
\end{proposition}
\begin{proofsketch}
To prove (1), we must show that the provided data makes \(y\) represent the functor \(\calC(-,x_1)\times\calC(-,x_2)\). Let \(z\in\calC\). Then we see that
\[  p_{k,*}\!:\calC(z,y)\to \calC(z,x_k),\quad i_{k,*}\!:\calC(z,x_k)\to\calC(z,y) \]
are maps of Abelian groups satisfying the assumptions in the proposition. It is then a simple computation that this induces a functorial isomorphism
\[ \calC(z,y) \iso \calC(z,x_1)\times\calC(z,x_2). \]
Similarly, to prove (2), the same argument applies but after applying \(\calC(-,z)\) instead.
\end{proofsketch}
\begin{remark}
	Objects in pre-additive categories equipped with maps as above are sometimes called \emph{direct sums,} and are denoted \(x_1\oplus x_2\). The proposition
	shows that any a product of two objects and a coproduct of two objects define the direct sum of those objects.
\end{remark}
\begin{corollary}
	Let \(\calC\) be a pre-additive category and let \(x_1,x_2\in\calC\). The product \(x_1\times x_2\) exists if and only if the coproduct \(x_1\amalg x_2\) exists if and only
	if the direct sum \(x_1\oplus x_2\) exists. Furthermore, in the case that these exist, the canonical map
	\begin{diagram*}
		x_1\amalg x_2 \ar[r,dashed] & x_1\times x_2 \\
		x_k\ar[u] \ar[r,equal] & x_k \ar[u,"i_k"']
	\end{diagram*}
	is an isomorphism, where \(i_k\) are the maps in Construction \ref{construction:pre-additive-product-inclusions}.
\end{corollary}
\begin{proof}
Proposition \ref{prop:pre-additive-direct-sum-data} handles the first statement by a clear induction argument. For the second statement,
note that the maps \(i_k\) exhibit \(x_1\times x_2\) as a direct sum by Proposition \ref{prop:pre-additive-direct-sum-data}, and therefore the induced map is an isomorphism.
\end{proof}

We conclude that in an appropriate sense, finite products and finite coproducts agree in any pre-additive category. It turns out, rather profoundly, that the pre-additive
structure on a category is detected by these (co)products whenever they exist.
\begin{proposition}
	Let \(\calC\) be a pre-additive category, and let \(f,g\!:x\to y\) be morphisms in \(\calC\). If \(x\oplus x\) and \(y\oplus y\) exist, then
	the morphism \(f+g\) is equal to the composition
	\[ x\xrightarrow{\Delta_x} x\oplus x \xrightarrow{f\oplus g} y\oplus y\xrightarrow{\nabla_y} y. \]
\end{proposition}
\begin{proofsketch}
One can check by a computation that
\[ i_1^x + i_2^x = \Delta_x,\quad \pi_1^y + \pi_2^y = \nabla_y \]
and deduce that
\[ \nabla_y \circ (f\oplus g)\circ i_1^x = f,\quad \nabla_y \circ (f\oplus g)\circ i_2^x = g. \]
Therefore
\[ \nabla_y \circ (f\oplus g)\circ \Delta_x = f+g \]
as desired.
\end{proofsketch}

\begin{definition}
	An \emph{additive category} is a pre-additive category which admits finite products.
\end{definition}
\begin{remark}\label{remark:additive-internal-characterization}
	An additive category therefore admits all finite direct sums, and the pre-additive structure is entirely determined by the properties of the underlying unenriched category. In fact,
	there is a characterization of additive categories in terms of purely ordinary category theory. Specifically, a category \(\calC\) is additive if and only if
	\begin{enumerate}[label=(\arabic*)]
	\item it admits a zero object \(0\in\calC\),
	\item for any \(x,y\in\calC\), the product \(x\times y\) and coproduct \(x\amalg y\) exist,
	\item the canonical map \(r\!:x\amalg y \to x\times y\) induced by the maps from Construction \ref{construction:pre-additive-product-inclusions} is an isomorphism, and
	\item for all \(x\in\calC\), there is some \(a\in\calC(x,x)\) such that
	\[ x\xrightarrow{\Delta_x} x\times x\xrightarrow{(a,\id_x)}x\times x\xrightarrow{r^{-1}} x\amalg x\xrightarrow{\nabla_x} x \]
	is zero.
	\end{enumerate}
	Here, (4) is what guarantees the existence of additive inverses.
\end{remark}

\begin{definition}
	A functor \(F\!:\calC\to\calD\) between pre-additive categories is \emph{additive} if for all \(x,y\in\calC\), the induced maps
	\[ \calC(x,y)\to\calD(Fx,Fy) \]
	are group homomorphisms.
\end{definition}
\begin{proposition}
	A functor between additive categories is additive if and only if it preserves finite products.
\end{proposition}
\begin{proof}
See \cite[Prop.\ 8.2.15]{kashiwara-schapira-book}.
\end{proof}

\subsection{Abelian categories}
The category \(\Ab\) of Abelian groups has certain special properties. First of all, it is additive; this is clear. However, on top of that, given any morphism of Abelian groups,
we can form a kernel, cokernel, image, etc., and these behave well (e.g.\ in that they detect interesting properties of the morphism). We wish to capture these properties
in a general form.
\begin{definition}
	Let \(\calC\) be a category with a zero object, and let \(f\!:x\to y\) be a morphism in \(\calC\). The \emph{kernel} of \(f\) is the pullback
	\begin{diagram*}
		\ker{f}\ar[r,dashed]\ar[d,dashed]\ar[dr, pullback] & x\ar[d,"f"] \\
		0\ar[r] & y
	\end{diagram*}
	in \(\calC\). Dually, the \emph{cokernel} of \(f\) is the pushout
	\begin{diagram*}
		x\ar[r,"f"]\ar[d] & y \ar[d,dashed] \\
		0\ar[r,dashed] & \coker{f}\ar[ul,pushout]
	\end{diagram*}
	in \(\calC\).
\end{definition}
\begin{exercise}
	Express \(\ker{f}\) and \(\coker{f}\) in terms of an equalizer and coequalizer, respectively.
\end{exercise}
\begin{exercise}\label{exercise:kernel-map-monomorphism}
	Let \(\calC\) be a pre-additive category with a zero object.
	\begin{enumerate}[label=(\arabic*)]
	\item Show that a morphism \(f\!:x\to y\) is a monomorphism if and only if for any \(a\!:z\to x\), \(f\circ a = 0\) implies \(a = 0\).
	\item Show that whenever \(\ker{f}\) exists, the canonical map \(\ker{f}\to x\) is a monomorphism.
	\end{enumerate}
	Furthermore, dualize these statements to obtain the corresponding results for epimorphisms and \(\coker{f}\).
\end{exercise}

\begin{proposition}
	Let \(\calC\) be a pre-additive category with a zero object, and let \(f\!:x\to y\) be a morphism with a kernel and cokernel.
	\begin{enumerate}[label=(\arabic*)]
	\item The morphism \(f\) is monic if and only if \(\ker{f} = 0\).
	\item The morphism \(f\) is epic if and only if \(\coker{f} = 0\).
	\end{enumerate}
\end{proposition}
\begin{proof}
Statement (2) is just the dual of (1). To prove (1), consider the diagram
\begin{diagram*}
	& z\ar[d,"a"']\ar[dl,dashed]\ar[dr,"0"] & \\
	\ker{f} \ar[r,hook] & x\ar[r,"f"] & y
\end{diagram*}
for some \(a\!:z\to x\) satisfying \(f\circ a = 0\). If \(\ker{f}=0\), then \(a\) factors through the zero object so \(a=0\) and \(f\) is monic.
Conversely, if \(f\) is monic, then \(a=0\) and we see that \(0\) satisfies the universal property of \(\ker{f}\).
\end{proof}


\begin{definition}
	Let \(\calC\) be a category with a zero object. The \emph{image} of a morphism \(f\!:x\to y\) is
	\[ \img{f} := \ker(y\sur\coker{f}) \]
	and the \emph{coimage} of \(f\) is
	\[ \coimg{f} := \coker(\ker{f}\inj x). \]
\end{definition}
\begin{remark}
	By Exercise \ref{exercise:kernel-map-monomorphism}, the canonical maps
	\[ \img{f}\to y \quad\text{and}\quad x\to\coimg{f} \]
	are a monomorphism and epimorphism, respectively.
\end{remark}
\begin{remark}
	The intuition for the definition comes from viewing the cokernel operation as taking the quotient.
\end{remark}
\begin{remark}
	Concretely, the universal property of the image of \(f\!:x\to y\) is given by the diagram
	\begin{diagram*}
		& z\ar[d]\ar[dr,"0"]\ar[dl,dashed,"\exists!"'] & \\
		\img{f}\ar[r,hook] & y \ar[r,two heads] & \coker{f}
	\end{diagram*}
	while, dually, the universal property of the coimage is given by the diagram
	\begin{diagram*}
		\ker{f}\ar[r,hook]\ar[dr,"0"'] & x\ar[r,two heads]\ar[d] & \coimg{f}\ar[dl,dashed,"\exists!"] \\
		& z
	\end{diagram*}
\end{remark}
\begin{construction}\label{construction:coimg-to-img}
	Observe that since the composition \(\ker{f}\to x\to y\) is zero (and by the dual statement), we can build the diagram
	\begin{diagram*}
		\ker{f} \ar[r,hook] \ar[dr,"0"'] & x\ar[r,"f"]\ar[d,two heads]\ar[dr] & y\ar[from=dl,crossing over] \ar[r,two heads] & \coker{f} \\
		& \coimg{f} & \img{f}\ar[u,hook]\ar[ur,"0"'] &
	\end{diagram*}
	and easily see that the compositions
	\[ \coimg{f} \to y \sur\coker{f}\quad \ker{f}\to x\to\img{f} \]
	are zero. In particular, we have an induced map
	\[ \coimg{f} \to \img{f} \]
	fitting in the square
	\begin{diagram*}
		x\ar[r]\ar[d,two heads] & y\\ 
		\coimg{f}\ar[r] & \img{f}\ar[u,hook]
	\end{diagram*}
	and is unique.
\end{construction}
\begin{definition}
	Let \(\calA\) be an additive category. We say \(\calA\) is \emph{Abelian} if for every \(x,y\in\calA\) and \(f\in\calA(x,y)\),
	\begin{enumerate}[label=(\arabic*)]
	\item the kernel and cokernel of \(f\) exist, and
	\item the canonical map
	\[ \coimg{f}\to\img{f} \]
	from Construction \ref{construction:coimg-to-img} is an isomorphism.
	\end{enumerate}
\end{definition}
\begin{remark}
	Condition (2) above is effectively demanding that the first isomorphism theorem holds in \(\calA\). Indeed, we can think of \(\coimg{f}\) as the quotient \(x/\ker{f}\),
	and (2) thus says we have a canonical isomorphism \(x/\ker{f}\iso\img{f}\).
\end{remark}
\begin{remark}
	Limits can be built from products and equalizers. Furthermore, in an additive category, any equalizer can be phrased as a kernel, namely \(\eq(f,g)\) will be \(\ker(f-g)\).
	In particular, since any Abelian category thus admits both equalizers and finite products, it admits all finite limits. The dual argument shows that Abelian categories also
	admit all finite colimits.
\end{remark}
\begin{exercise}\label{exercise:monic-epic-image}
	Let \(f\!:x\to y\) be a morphism in an Abelian category.
	\begin{enumerate}[label=(\arabic*)]
	\item Show that if \(f\) is monic, then
	\[ x\iso\img{f}. \]
	\item Dually, show that if \(f\) is epic, then
	\[ \img{f}\iso y. \]
	\item Conclude that \(f\) is an isomorphism if and only if it is monic and epic.
	\end{enumerate}
\end{exercise}

\subsection{Appendix: Enriched category theory}
Pre-additive categories are a special case of a more general paradigm, \emph{enriched categories.} There are many different formalisms for this at many different levels
of generality; we present only one, roughly following \cite{riehl-categorical-homotopy-theory}. The general idea is that an enriched category is like a category, but where the set of morphisms between two objects is replaced by
an object in some category, called the \emph{enrichment base.} The first order of business is to specify what kind of category we wish to have as base for enrichments.
\begin{definition}
	A \emph{symmetric monoidal category} is triple \((\calV,\otimes,\1)\) where \(\calV\) is a category, \(\otimes\!:\calV\times\calV\to\calV\) is a functor,
	and \(\1\in\calV\) is an object, together with the data of specified natural isomorphisms
	\[ v\otimes w \cong w\otimes v,\quad u\otimes(v\otimes w)\cong (u\otimes v)\otimes w,\quad \1\otimes v\cong v \cong v\otimes \1. \]
	These natural isomorphisms are required to satisfy various coherences which ensure that any two bracketings of \(\otimes\)-products are naturally isomorphic.
\end{definition}
\begin{example}
	By considering the tensor product over \(\Z\), the category \(\Ab\) can be promoted to a symmetric monoidal category \((\Ab,\otimes_{\Z},\Z)\). More generally, for
	any ring \(R\), the category \(\Mod_R\) forms a symmetric monoidal category \((\Mod_R,\otimes_R,R)\).
\end{example}
\begin{example}
	Any category \(\calC\) admitting finite products yields a symmetric monoidal category \((\calC,\times,*)\) where \(*\in\calC\) is the terminal object. In particular,
	\(\Set\) provides a symmetric monoidal category \((\Set,\times,*)\).
\end{example}
\begin{example}
	Let \(R\) be a commutative ring. The tensor product on \(\Mod_R\) can be lifted to the category \(\Ch(R)\) of chain complexes in \(R\), to be defined in Section \ref{subsection:chain-complexes-and-exact-sequences},
	to give a functor \(\otimes_R\!:\Ch(R)\times\Ch(R)\to\Ch(R)\). This provides a symmetric monoidal category \((\Ch(R),\otimes_R,R)\).
\end{example}
\begin{definition}
	Let \((\calV,\otimes,\1)\) be a symmetric monoidal category. A \(\calV\)\emph{-category} \(\underline\calC\) consists of
	\begin{enumerate}[label=(\arabic*)]
	\item a set of objects \(\Ob(\underline\calC)\),
	\item for every pair of objects \((x,y)\) in \(\underline\calC\), an object \(\underline\calC(x,y)\in\calV\),
	\item for every \(x\in\underline\calC\), a morphism \(1_x\!:\1\to\underline\calC(x,x)\), and
	\item for each triple \((x,y,z)\) of objects in \(\calV\), a morphism
	\[ \circ\!:\underline\calC(y,z)\otimes\underline\calC(x,y)\to\underline\calC(x,z) \]
	\end{enumerate}
	such that for all \(x,y,z,w\in\underline\calC\), the diagrams
	\begin{diagram*}
		\underline\calC(z,w)\otimes\underline\calC(y,z)\otimes\underline\calC(x,y)\ar[r,"1\otimes\circ"]\ar[d,"\circ\otimes 1"'] & \underline\calC(z,w)\otimes\underline\calC(x,z)\ar[d,"\circ"] \\
		\underline\calC(y,w)\otimes\underline\calC(x,y)\ar[r,"\circ"] & \underline\calC(x,w)
	\end{diagram*}
	\begin{center}
	\begin{tikzcd}
		\underline\calC(x,y)\otimes\1\ar[r,"\id\otimes1_x"]\ar[dr,"\cong"'] & \underline\calC(x,y)\otimes\underline\calC(x,x)\ar[d,"\circ"] \\
		& \underline\calC(x,y)
	\end{tikzcd}
	\quad
	\begin{tikzcd}
		\underline\calC(y,y)\otimes\underline\calC(x,y)\ar[d,"\circ"] & \1\otimes\underline\calC(x,y)\ar[l,"1_y\otimes\id"']\ar[dl,"\cong"]\\
		\underline\calC(x,y) &
	\end{tikzcd}
	\end{center}
	expressing associativity and unitality commute. One also says that \(\underline\calC\) is a \emph{category enriched over} \((\calV,\otimes,\1)\).
\end{definition}
\begin{example}
	An ordinary category is a category enriched over \((\Set,\times,*)\).
\end{example}
\begin{example}
	A 2-category is a category enriched over \((\Cat,\times,[0])\).
\end{example}
\begin{example}
	A pre-additive category is a category enriched over \((\Ab,\otimes_\Z,\Z)\).
\end{example}
\begin{example}
	A \emph{dg-category} over a commutative ring \(R\) is a category enriched over \((\Ch(R),\otimes_R,R)\). These play a particularly important role in homological algebra,
	as they are a computationally convenient presentation for \(R\)-linear stable \(\infty\)-categories.
\end{example}
\begin{example}
	A \emph{simplicial category} is a category enriched over \((\sSet,\times,\Delta^0)\).
\end{example}
\begin{remark}
	Any \(\calV\)-category \(\underline\calC\) induces an ordinary category \(\calC\) whose objects are just the objects of \(\underline\calC\), and whose morphisms are given by
	\[ \calC(x,y) := \calV(\1,\underline\calC(x,y)). \]
\end{remark}
\begin{definition}
	Let \((\calV,\otimes,\1)\) be a symmetric monoidal category. A \(\calV\)\emph{-enriched functor,} or \(\calV\)\emph{-functor} \(F\!:\underline\calC\to\underline\calD\) between
	\(\calV\)-categories consists of
	\begin{enumerate}[label=(\arabic*)]
	\item a map \(\Ob(\underline\calC)\to\Ob(\underline\calD)\), \(x\mapsto Fx\),
	\item for each pair of objects \((x,y)\) in \(\underline\calC\), a morphism
	\[ F\!:\underline\calC(x,y)\to\underline\calD(Fx,Fy), \]
	\end{enumerate}
	such that for all triples \((x,y,z)\) of objects in \(\underline\calC\), the diagrams
	\begin{center}
	\begin{tikzcd}
		\underline\calC(y,z)\otimes\underline\calC(x,y)\ar[r,"\circ"]\ar[d,"F\otimes F"'] & \underline\calC(x,z)\ar[d,"F"] \\
		\underline\calD(Fy,Fz)\otimes\underline\calD(Fx,Fy)\ar[r,"\circ"] & \underline\calD(Fx,Fz)
	\end{tikzcd}
	\quad
	\begin{tikzcd}
		\1 \ar[r,"1_x"]\ar[dr,"1_{Fx}"'] & \underline\calC(x,x)\ar[d,"F"] \\
		& \underline\calD(Fx,Fx)
	\end{tikzcd}
	\end{center}
	commute.
\end{definition}

Much of category theory can be developed in the enriched setting, including the Yoneda lemma, (co)limits of various sorts, and so on. This turns out to be very useful
in higher category theory, as many models of higher categories can be ``strictified'' by working in an enriched context. An example of this mentioned before is
given by dg-categories.

Categories enriched in Kan complexes give a model for \((\infty,1)\)-categories, and categories enriched in weak Kan complexes (a.k.a.\ quasicategories, or \(\infty\)-categories) give a model for \((\infty,2)\)-categories.
In practice, this means that doing higher category theory, homotopy theory, etc.\ can be simplified by knowing some enriched category theory. Indeed, this is the topic of
\cite{riehl-categorical-homotopy-theory} and partly \cite{riehl-verity-elements}.

\subsection{Appendix: (Co)limits in terms of (co)products \& (co)equalizers}
Consider a small diagram \(D\!:I\to\Set\) in the category of sets. It is a standard result that one can compute the limit and colimit of \(D\)
as
\[ \projlim{D} \cong \left\{ (d_i)_{i\in I}\in\prod_{i\in I}D(i)\bigmid \forall(\varphi\!:i\to j)\in I,\, D(\varphi)(d_i) = d_j \right\} \]
and
\[ \injlim{D} \cong \left(\coprod_{i\in I}D(i)\right) / \sim \]
where \(\sim\) is the equivalence relation generated by (!) \((i,d) \sim (i',d')\) if there exists \(\varphi\!:i\to i'\) such that \(D(\varphi)(d) = d'\).
Notably, these are compute in terms of the easier-to-understand products and coproducts in \(\Set\).

We wish to generalize this to any category in order to deduce that a category admitting (co)products and (co)equalizers admits all small (co)limits.
\begin{construction}\label{construction:diagram-product-equalizer-diagram}
	Let \(D\!:I\to\calC\) be a small diagram in a category \(\calC\) which admits products, and consider a map \(\varphi\!:i\to j\) in \(I\). From this, we obtain two maps
	\begin{diagram*}
		D(i)\times D(j) \ar[r,shift left,"D(\varphi)\circ\pi_1"]\ar[r,shift right,"\pi_2"'] & D(j)
	\end{diagram*}
	and these allow us to produce two maps
	\begin{diagram*}
		D(\cod\varphi)\ar[r,equal] & D(\cod\varphi) \\
		\displaystyle\prod_{i\in I}D(i)\ar[r,shift left,dashed,"a"]\ar[r,shift right,dashed,"b"']\ar[u,"\pi_{\cod\varphi}"]\ar[d,"\pi_{\dom\varphi}"'] & \displaystyle\prod_{\varphi\in\Ar(I)}D(\cod{\varphi})\ar[u,"\pi_\varphi"']\ar[d,"\pi_\varphi"] \\
		D(\dom\varphi)\ar[r,"D(\varphi)"] & D(\cod\varphi)
	\end{diagram*}
	defined on the \(\varphi\)th component by the indicated morphisms. Furthermore, there is a canonical map
	\[ c\!:\projlim{D} \to \prod_{i\in I}D(i) \]
	induced by the projections \(\projlim{D}\to D(i)\), \(i\in I\). One notes that \(a\circ c = b\circ c\) since this equality comes down to verifying it on components, where it
	holds by the commutativity of
	\begin{diagram*}
		\projlim{D} \ar[r]\ar[dr] & D(\dom{\varphi})\ar[d,"D(\varphi)"] \\
		& D(\cod\varphi).
	\end{diagram*}
	In particular, we have an induced map \(\projlim{D}\to\eq(a,b)\).
\end{construction}
\begin{proposition}\label{prop:limit-from-products-and-equalizers}
	Let \(\calC\) be a category admitting products and equalizers, and let \(D\!:I\to\calC\) be a small diagram. Consider the maps
	\begin{diagram*}
		\displaystyle\prod_{i\in I}D(i)\ar[r,shift left,"a"]\ar[r,shift right,"b"'] & \displaystyle\prod_{\varphi\in\Ar(I)}D(\cod{\varphi})
	\end{diagram*}
	of Construction \ref{construction:diagram-product-equalizer-diagram}. Then \(\projlim{D}\) exists and the canonical map satisfies
	\[ \projlim{D}\iso\eq(a,b). \]
\end{proposition}
\begin{proof}
In \(\calC=\Set\), this is a triviality: indeed, it is just the statement we made at the start. To lift it to an arbitrary category \(\calC\), pick \(x\in\calC\) and apply
\(\calC(x,-)\). We then have, by the \(\Set\) case, that
\begin{diagram*}
	 & \calC(x,\eq(a,b))\ar[r]\ar[d,"\sim" labl] & \calC(x,\prod_{i\in I}D(i))\ar[d, "\sim" labl] \ar[r,shift left,"a_*"]\ar[r,shift right,"b_*"'] & \calC(x,\prod_{\varphi\in\Ar(I)}D(\cod{\varphi}))\ar[d, "\sim" labl] \\
	\projlim\calC(x,D)\ar[r,"\sim"] & \eq(a_*,b_*) \ar[r] & \prod_{i\in I}\calC(x,D(i)) \ar[r,shift left,"a_*"]\ar[r,shift right,"b_*"'] & \prod_{\varphi\in\Ar(I)}\calC(x,D(\cod{\varphi}))
\end{diagram*}
and thus that we have an isomorphism
\[ \projlim\calC(x,D) \cong \calC(x,\eq(a,b)) \]
which is natural in \(x\). We conclude that \(\projlim{D}\) exists and is isomorphic to \(\eq(a,b)\).
\end{proof}
\begin{corollary}
	Let \(\calC\) be a category. Then the following statements are equivalent.
	\begin{enumerate}[label=(\arabic*)]
	\item \(\calC\) admits all small (resp.\ finite) limits.
	\item \(\calC\) admits equalizers and all small (resp.\ finite) products.
	\end{enumerate}
	Let \(F\!:\calC\to\calD\) be a functor. Then the following are equivalent.
	\begin{enumerate}[label=(\arabic*')]
	\item \(F\) preserves all small (resp.\ finite) limits.
	\item \(F\) preserves equalizers and all small (resp.\ finite) products.
	\end{enumerate}
\end{corollary}
\begin{proof}
Clearly, (1) implies (2) and (1') implies (2'). On the other hand, (2) implies (1) by Proposition \ref{prop:limit-from-products-and-equalizers} (noting that the products
are finite whenever \(I\) is finite), so only (2') implies (1') remains. However, if \(F\) preserves products and equalizers then
\begin{diagram*}
	F(\projlim{D}) \ar[d,"\sim" labl] \ar[r] & F(\prod_{i\in I}D(i))\ar[r,shift left,"Fa"]\ar[r,shift right,"Fb"']\ar[d,"\sim" labl] & F(\prod_{\varphi\in\Ar(I)}D(\cod{\varphi}))\ar[d,"\sim" labl] \\
	F(\eq(a,b)) \ar[d,"\sim" labl] \ar[r] & F(\prod_{i\in I}D(i))\ar[r,shift left,"Fa"]\ar[r,shift right,"Fb"']\ar[d,"\sim" labl] & F(\prod_{\varphi\in\Ar(I)}D(\cod{\varphi}))\ar[d,"\sim" labl] \\
	\eq(Fa,Fb) \ar[r] & \prod_{i\in I}F(D(i))\ar[r,shift left,"Fa"]\ar[r,shift right,"Fb"'] & \prod_{\varphi\in\Ar(I)}F(D(\cod{\varphi})) \\
\end{diagram*}
and since \(\projlim(F\circ D)\cong\eq(Fa,Fb)\), we are done.
\end{proof}

%!TEX root = ../lectures.tex

\section{Basics of homological algebra}
\subsection{Chain complexes \& exact sequences}\label{subsection:chain-complexes-and-exact-sequences}
\begin{definition}
	Let \(\calC\) be a pre-additive category. A \emph{chain complex} \(x^\bullet\) in \(\calC\) is a sequence of morphisms
	\[ \cdots \to x^{i-1}\xrightarrow{d^{i-1}} x^{i} \xrightarrow{d^{i}} x^{i+1} \to \cdots  \]
	such that \(d^{i+1}\circ d^i = 0\) for all \(i\in\Z\). We denote this by either \(x^\bullet\) or \((x^\bullet,d_x)\).
\end{definition}
\begin{remark}
	Let \(\calC\) be some category, and regard the set \(\Z\) as a discrete category. A \emph{graded object} in \(\calC\) is a functor
	\[ \Z\to\calC. \]
	Consider the functor \(s\!:\Z\to\Z\) given by sending \(n\) to \(n+1\). This induces a \emph{shift} functor
	\[ (1)\!:\Fun(\Z,\calC)\to\Fun(\Z,\calC),\quad x^\bullet \mapsto x^\bullet\circ s, \]
	which concretely sends a graded object \(x^\bullet\) in \(\calC\) to the graded object given by \(x(1)^i = x^{i+1}\).

	Now, suppose \(\calC\) is pre-additive. A \emph{chain complex} in \(\calC\) can be regarded as a graded object \(x^\bullet\!:\Z\to\calC\) together with a morphism
	\[ d_x\!:x^\bullet \to x(1)^\bullet \]
	satisfying \(d_x(1)\circ d_x = 0\).
\end{remark}

Our first goal is to define the \emph{cohomology} of a chain complex. There are two obvious ways of doing this (accepting that one is aware of the ``classical'' hands-on definition),
and we want to ensure the two are the same in our formalism.
\begin{construction}\label{construction:chain-complex-induced-maps}
	Consider a chain complex \(x^\bullet\), and in particular zoom in around a fixed index \(i\in\Z\),
	\[ x^{i-1}\xrightarrow{d^{i-1}} x^i \xrightarrow{d^i} x^{i+1}. \]
	Forming kernels, cokernels, and images, we can produce the diagram
	\begin{diagram*}
		 & \img{d^{i-1}}\ar[rr,dashed,"\phi^i"]\ar[dr,hook] & & \ker{d^i}\ar[dr,"0"]\ar[dl,hook] & \\
		x^{i-1}\ar[ur,two heads]\ar[rr,"d^{i-1}"]\ar[dr,"0"'] & & x^i\ar[rr,"d^i"]\ar[dr,two heads]\ar[dl,two heads] & & x^{i+1} \\
		 & \coker{d^{i-1}}\ar[rr,dashed,"\psi^i"] & & \img{d^i}\ar[ur,hook] & 
	\end{diagram*}
	where the induced maps \(\img{d^{i-1}}\to\ker{d^i}\) and \(\coker{d^{i-1}}\to\img{d^i}\) come from \(d^2=0\) and various arrows being monic/epic.
\end{construction}
\begin{proposition}\label{prop:chain-complex-induced-maps-monic-epic}
	Let \(\phi^i\) and \(\psi^i\) be as in Construction \ref{construction:chain-complex-induced-maps}. Then
	\begin{enumerate}[label=(\arabic*)]
	\item \(\phi^i\) is monic, and
	\item \(\psi^i\) is epic.
	\end{enumerate}
\end{proposition}
\begin{proof}
Statement (2) is dual to (1). To prove (1), note that
\[ (z\to\img{d^{i-1}}\xrightarrow{\phi^i}\ker{d^i}) = 0 \]
if and only if
\[ (z\to\img{d^{i-1}}\xrightarrow{\phi^i}\ker{d^i} \inj x^i) = 0 \]
if and only if
\[ (z\to\img{d^{i-1}} \inj x^i) = 0 \]
if and only if \(z\to\img{d^{i-1}}\) is zero.
\end{proof}
\begin{exercise}
	Spell out the proof of (2) in Proposition \ref{prop:chain-complex-induced-maps-monic-epic}.
\end{exercise}
\begin{construction}\label{construction:chain-complex-induced-maps-2}
	Let \(u^i\) be the composition \(\ker{d^i}\inj x^i\sur\coker{d^{i-1}}\). Forming the kernel and cokernel of \(u^i\), we can enlarge our previous diagram to
	\begin{diagram*}[row sep=small]
		& & \ker{u^i} \ar[dr,hook]\ar[dl,hook,dashed,shift left] & &\\
		 & \img{d^{i-1}}\ar[rr,hook,"\phi^i"]\ar[ur,hook,dashed,shift left]\ar[dr,hook] & & \ker{d^i}\ar[dr,"0"]\ar[dl,hook] & \\
		x^{i-1}\ar[ur,two heads]\ar[rr,"d^{i-1}"]\ar[dr,"0"'] & & x^i\ar[rr,"d^i"]\ar[dr,two heads]\ar[dl,two heads] & & x^{i+1} \\
		 & \coker{d^{i-1}}\ar[rr,two heads,"\psi^i"]\ar[dr,two heads] & & \img{d^i}\ar[ur,hook]\ar[dl,two heads,dashed, shift right] & \\
		& & \coker{u^i}\ar[ur,two heads,dashed,shift right] & &
	\end{diagram*}
	where the dashed maps are induced by the universal properties of the image, kernel, and cokernel.
\end{construction}
\begin{exercise}
	Show that the dashed arrows in Construction \ref{construction:chain-complex-induced-maps-2} define isomorphsisms
	\[ \ker{u^i}\iso \img{d^{i-1}} \quad\text{and}\quad \img{d^i}\iso\coker{u^i}. \]
\end{exercise}
Thus, in particular, we finally have the diagram
\begin{diagram*}
	& & \ker{u^i} \ar[dr,hook]\ar[dl,"\sim"'] & &\\
	 & \img{d^{i-1}}\ar[rr,hook,"\phi^i"]\ar[dr,hook] & & \ker{d^i}\ar[dr,"0"]\ar[dl,hook] & \\
	x^{i-1}\ar[ur,two heads]\ar[rr,"d^{i-1}"]\ar[dr,"0"'] & & x^i\ar[rr,"d^i"]\ar[dr,two heads]\ar[dl,two heads] & & x^{i+1} \\
	 & \coker{d^{i-1}}\ar[rr,two heads,"\psi^i"]\ar[dr,two heads] & & \img{d^i}\ar[ur,hook]\ar[dl,"\sim"'] & \\
	& & \coker{u^i} & &
\end{diagram*}
and using this, we define cohomology. In particular, we compute the image
\begin{align*}
	\img{u^i} &\cong \ker(\coker{d^{i-1}}\sur\coker{u^i}) \\
	&\cong \ker(\coker{d^{i-1}}\sur\img{d^{i-1}})
\end{align*}
and the coimage
\begin{align*}
	\coimg{u^i} &\cong \coker(\ker{u^i}\inj\ker{d^i}) \\
	&\cong \coker(\img{d^{i-1}}\inj\ker{d^i}) \\
	&\cong \ker(\coker{d^{i-1}}\to x^{i+1}) \\
	&\cong \coker(x^{i-1}\to\ker{d^{i-1}}).
\end{align*}
\begin{definition}
	Let \((x^\bullet,d)\) be a chain complex in an Abelian category. The \emph{cohomology} of \(x^\bullet\) at \(i\in\Z\) is
	\[ \HH^i(x^\bullet) := \coker(\img{d^{i-1}}\inj\ker{d^{i}}) \]
	or any one of the other canonically isomorphic choices from above. We say \(x^\bullet\) is \emph{exact} at \(i\in\Z\) if \(\HH^i(x^\bullet) = 0\).
\end{definition}
\begin{remark}
	Explicitly, \(\HH^i(x^\bullet) = 0\) is the same as
	\[ u^i = 0 \iff \coker{d^{i-1}} \iso \img{d^i} \iff \img{d^{i-1}}\iso\ker{d^i} \]
	which are the conditions one expect from exactness.
\end{remark}
\begin{definition}
	Let \(\calA\) be an Abelian category. A \emph{short exact sequence} is an exact sequence of the form
	\[ 0 \to x \to y \to z \to 0. \]
\end{definition}
\begin{lemma}\label{lemma:left-exact-sequence-implies-monic}
	A sequence \(0 \to x \overset{f}\to y\) is exact if and only if \(f\) is a monomorphism.
\end{lemma}
\begin{proof}
The sequence is exact if and only if \(\ker{f} = 0\), since the image of \(0\to x\) is \(0\). Since \(\ker{f}=0\) if and only if \(f\) is
monic, this completes the proof.
\end{proof}
\begin{remark}
	It follows that in any short exact sequence
	\[ 0\to x\overset{f}\to y\overset{g}\to z\to 0 \]
	the morphism \(f\) is monic and the morphism \(g\) is epic (by the dual of Lemma \ref{lemma:left-exact-sequence-implies-monic}).
\end{remark}
\begin{theorem}[First isomorphism theorem]
	Consider a short exact sequence
	\[ 0\to x\overset{f}\inj y\overset{g}\sur z\to 0 \]
	in an Abelian category. Then
	\[ x\iso\img{f}\quad\text{and}\quad\coker{f}\iso z. \]
\end{theorem}
\begin{proof}
Lemma \ref{lemma:left-exact-sequence-implies-monic} tells us that \(f\) is a monomorphism, in which case Exercise \ref{exercise:monic-epic-image} tells us that
\(x\iso\img{f}\). For the second isomorphism, we use that Exercise \ref{exercise:monic-epic-image} provides \(\img{g}\cong z\), and that exactness provides us a canonical isomorphism \(\ker{g}\cong\img{f}\).
Thus, compute
\begin{align*}
	z &\cong \img{g} \\
	&\cong \coker(\ker{g}\inj y) \\
	&\cong \coker(\img{f}\inj y) \cong \coker{f}
\end{align*}
which completes the proof.
\end{proof}

\subsection{Exactness properties of functors}
\begin{definition}
	Let \(F\!:\calA\to\calB\) be an additive functor between Abelian categories. We say \(F\) is \emph{left exact} if for any short exact sequence
	\[ 0 \to x \to y \to z \to 0 \]
	in \(\calA\), the sequence
	\[ 0\to Fx \to Fy \to Fz \]
	in \(\calB\) is exact. We say \(F\) is \emph{right exact} if \(F^\op\!:\calA^\op\to\calB^\op\) is left exact. We say \(F\) is \emph{exact} if
	it is both left exact and right exact.
\end{definition}
\begin{theorem}
	Let \(\calA\) be an Abelian category, and let \(a\in\calA\). Then
	\[ \calA(a,-)\quad\text{and}\quad\calA(-,a) \]
	are left exact.
\end{theorem}
\begin{proof}
Consider an exact sequence
\[ 0 \to x \overset{i}\to y \overset{p}\to z \to 0 \]
in \(\calA\). Applying \(\calA(a,-)\), we must show that
\[ 0 \to \calA(a,x)\overset{i_*}\to \calA(a,y)\overset{p_*}\to\calA(a,z) \]
is exact. Since \(i\) is a monomorphism, it follows (by definition!) that \(i_*\) is a monomorphism. For exactness in the middle, consider
a morphism \(f\in\ker{p_*}\). Since \(x\cong\img{i}\cong\ker{p}\), we then have the diagram
\begin{diagram*}
	& & a\ar[d,"f"']\ar[dr,"0"]\ar[dl,dashed,"f'"'] & & \\
	0 \ar[r] & x \ar[r,"i"] & y\ar[r,"p"] & z\ar[r] & 0
\end{diagram*}
so that \(f = i\circ f'\) and \(\img{i_*} = \ker{p_*}\). The other functor is dual, since \(\calA(-,a) = \calA^\op(a,-)\).
\end{proof}
\begin{exercise}
	Prove that \(\calA(-,a)\) is left exact explicitly.
\end{exercise}
\begin{theorem}
	A functor \(F\!:\calA\to\calB\) between Abelian categories is left exact if and only if it preserves finite limits.
\end{theorem}
\begin{proof}
If \(F\) is left exact, then it is additive and preserves left exact sequences. In particular, it preserves finite products and kernels, and therefore
equalizers. Since finite limits are built out of these, it preserves finite limits. Conversely, if it preserves finite limits, then it is additive since it
preserves finite products, and left exact because it preserves kernels. In particular, we use that
\[ 0\to Fx \to Fy\to Fz \]
is exact if and only if \(Fx\to Fy\) is the kernel of \(Fy\to Fz\).
\end{proof}
\begin{corollary}
	A functor \(F\!:\calA\to\calB\) is right exact if and only if it preserves finite colimits, and it is exact if and only if it preserves both
	finite limits and finite colimits.
\end{corollary}

\begin{example}
	Let \(X\) be a topological space, and let \(x\in X\). The stalk functor
	\[ (-)_x\!:\Sh(X,\Ab)\to\Ab \]
	is exact. In particular, \((-)_x\) is a left adjoint and hence commutes with all colimits. Furthermore, it is defined by a filtered colimit, and since the
	forgetful functor \(\Ab\to\Set\) preserves filtered colimits, this means \((-)_x\) commutes with finite limits (since filtered colimits commute with finite limits in \(\Set\)).

	In fact, more is true: a sequence of sheaves
	\[ 0 \to\scrF\to\scrG\to\scrH\to 0 \]
	is exact if and only if for all \(x\in X\),
	\[ 0\to\scrF_x\to\scrG_x\to\scrH_x\to0 \]
	is exact.
\end{example}

\begin{example}
	Let \(X\) be a topological space. The global sections functor
	\[ \Gamma(X,-)\!:\Sh(X,\Ab)\to\Ab \]
	is left exact. This is because it is right adjoint to the constant sheaf functor, hence commutes with all limits.

	More generally, given a continuous map \(f\!:X\to Y\), the functor
	\[ f_*\!:\Sh(X,\Ab)\to\Sh(Y,\Ab) \]
	is left exact since it has a right adjoint given by \(f^{-1}\). This specializes to the previous case when one sets \(Y = *\). The observation
	that these functors are only left exact is the start of the cohomology theory of sheaves, wherein one studies e.g.\ the right derived functor
	\[ \sfR f_*\!:\sfD(X)\to\sfD(Y). \]
	Sheaf cohomology is defined by the right derived functor \(\sfR\Gamma(X,-)\).
\end{example}
\begin{example}
	Sheafification is an exact functor. This is because on one hand, it is a left adjoint functor, and on the other hand, it is actually defined by
	a filtered colimit (if handled appropriately).
\end{example}
\begin{remark}
	We saw that \(\calA(x,-)\) and \(\calA(-,x)\) are left exact. One says \(x\in\calA\) is \emph{projective} if \(\calA(x,-)\) is exact, and dually,
	that \(x\) is \emph{injective} if \(\calA(-,x)\) is exact.
\end{remark}
\begin{exercise}
	Let \(X\) be a topological space, let \(x\in X\), and consider the stalk functor
	\[ (-)_x\!:\Sh(X,\calD)\to\calD. \]
	Using these, we will show, by hand, that under certain conditions, the sheafification functor is exact.
	\begin{enumerate}[label=(\arabic*)]
	\item Suppose we have a functor \(F\!:\calD\to\Set\) which preserves finite limits. Show that the functor
	\[ F_*\!:\Sh(X,\calD)\to\Sh(X),\quad \scrF\mapsto F\circ\scrF \]
	preserves finite limits.
	\item Suppose that \(F\) in addition preserves filtered colimits and reflects isomorphisms. Show that the stalk functor commutes with finite limits.
	\item Show that the sheafification functor \((-)^\dagger\!:\PSh(X,\calD)\to\Sh(X,\calD)\) preserves finite limits. \emph{Hint:} it suffices to show
	that it commutes with kernels and binary products; show this by checking it on stalks.
	\item Check that \(\Mod_R\) satisfies the above conditions.
	\end{enumerate}
\end{exercise}


\subsection{Appendix: Quillen exact categories}\label{appendix:quillen-exact-categories}
There are interesting additive categories, particularly in areas like functional analysis, where only some kernels and cokernels make sense, but not all of them. In
situations like this, one would still like access to some kind of theory. One possible approach to situations of this type is given by Quillen's exact categories. Roughly speaking,
they are additive categories wherein certain distinguished maps lie in kernel/cokernel pairs.
\begin{definition}
	Let \(\calA\) be an additive category. A \emph{kernel/cokernel pair} in \(\calA\) is a pair of morphisms
	\[ x\overset{f}\to y\overset{g}\to z \]
	such that \(f\) defines a kernel of \(g\) and \(g\) a cokernel of \(f\).
\end{definition}
\begin{example}
	In an Abelian category \(\calA\), the short exact sequences
	\[ 0\to x\to y \to z \to 0 \]
	define all kernel/cokernel pairs.
\end{example}
\begin{definition}
	An \emph{exact category} is a pair \((\calC,\calE)\) where \(\calC\) is additive and \(\calE\) is a class of \emph{admissible} kernel/cokernel pairs in \(\calC\); we
	call \(f\) (resp.\ \(g\)) in an admissible kernel/cokernel pair
	\[ x\overset{f}\to y\overset{g}\to z \]
	an \emph{admissible monomorphism} (resp. \emph{admissible epimorphism}) or an \emph{inflation} (resp.\ \emph{deflation}). The pair \((\calC,\calE)\) is required
	to satisfy the following conditions:
	\begin{enumerate}[label=(\arabic*)]
	\item The class \(\calE\) is closed under isomorphism, meaning that if we have a diagram
	\begin{diagram*}
		x\ar[r]\ar[d,"\sim" labl] & y\ar[r]\ar[d, "\sim" labl] & z\ar[d, "\sim" labl] \\
		x'\ar[r] & y'\ar[r] & z'
	\end{diagram*}
	where one row is in \(\calE\) then the other row is in \(\calE\).
	\item For each \(x\in X\), the identity map \(\id_x\) is an inflation and a deflation.
	\item The class of inflations is closed under composition. Dually, the class of deflations is closed under composition.
	\item The class of inflations is stable under pushouts. Dually, the class of deflations is closed under pullbacks.
	\end{enumerate}
\end{definition}
We provide a few arguments to give a taste of the kinds of proofs that are involved when working with exact categories.
\begin{remark}
	The axioms imply that all isomorphisms \(f\!:x\to x\) are both inflations and deflations. In particular, we have isomorphisms
	\begin{diagram*}
		x\ar[r,"f"]\ar[d,equal] & x\ar[r]\ar[d,"f^{-1}"] & 0\ar[d] \\
		x\ar[r,equal] & x\ar[r] & 0
	\end{diagram*}
	and
	\begin{diagram*}
		0 \ar[r]\ar[d] & x\ar[r,"f"]\ar[d,equal] & x\ar[d,"f^{-1}"] & \\
		0 \ar[r] & x\ar[r,equal] & x
	\end{diagram*}
	and by the closure of \(\calE\) under isomorphism, we conclude.
\end{remark}
\begin{remark}
	It also follows that the zero map \(0\to x\) is an inflation, and the zero map \(x\to 0\) is a deflation.
\end{remark}
\begin{remark}
	The data of the inflations or the deflations both determine the class \(\calE\). Indeed, supposing
	we have
	\[ x\overset{i}\to y\overset{p}\to z \]
	where \(i\) is an inflation and \(p\) is a cokernel of \(i\), we may find a deflation \(p'\!:y\to z'\) so that we have an isomorphism
	\begin{diagram*}
		x\ar[r,"i"]\ar[d,"\sim" labl] & y\ar[r,"p"]\ar[d,"\sim" labl] & z\ar[d,"\sim" labl] \\
		x\ar[r,"i"] & y\ar[r,"p'"] & z'
	\end{diagram*}
	where the bottom row is in \(\calE\), and so the original pair of maps was already in \(\calE\). Notably, a morphism is a deflation if and only if it appears as the cokernel of an inflation. A
	dual argument shows that a morphism is an inflation if and only if it appears as the kernel of a deflation.
\end{remark}
\begin{proposition}
	Let \((\calC,\calE)\) be an exact category, and let \(x,y\in\calC\). Then
	\begin{diagram*}
		x\ar[r,"\iota"] & x \oplus y\ar[r,"\pi"] & y
	\end{diagram*}
	is an admissible kernel/cokernel pair.
\end{proposition}
\begin{proof}
We have a pushout square
\begin{diagram*}
	0 \ar[r]\ar[d] & y\ar[d] \\
	x \ar[r,"\iota"] & x\oplus y\ar[ul,pushout]
\end{diagram*}
and since \(0\to y\) is an inflation, it follows that \(\iota\) is an inflation. Since \(\pi\) is the cokernel of \(\iota\), it is a deflation and
\begin{diagram*}
	x\ar[r,"\iota"] & x \oplus y\ar[r,"\pi"] & y
\end{diagram*}
is an admissible kernel/cokernel pair, as desired.
\end{proof}
\begin{corollary}
	Let \((\calC,\calE)\) be an exact category. Then \(\calE\) contains all split exact sequences.
\end{corollary}

One can prove various additional results, but since we do not aim to give an actual introduction here, we leave it for the references. A possible resource is \cite{frerick2010exact}.



\subsection{Appendix: An aside on stable \(\infty\)-categories}
Abelian categories form a convenient setting to formalize the properties of categories like \(\Ab\), and with which to develop homological algebra. However,
it has certain deficiencies. The most natural way to formalize homological algebra is through the use of \emph{derived categories} \(\sfD(\calA)\) of Abelian
categories \(\calA\). The way to understand these is as follows: in homological algebra, one studies properties of chain complexes that only depend on the cohomologies
of the chain complexes. In particular, we would like to think of maps between chain complexes which induce isomorphisms on the level of cohomology as being
``genuine'' isomorphisms; we call such maps \emph{quasi-isomorphisms.} The derived category \(\sfD(\calA)\) is obtained by taking the category of chain
complexes \(\Ch(\calA)\) and formally inverting the quasi-isomorphisms; the process of doing this is called \emph{localization.}

Localizations of categories, which we will study later, are very hard to control. For example, given a small category \(\calC\) and some class of morphisms \(\calS\) in \(\calC\),
the localization \(\calC[\calS^{-1}]\) need not be locally small. This is already a sign that localizations are badly behaved in general. The issue that comes up in homological algebra
is that while the localization of an additive category does turn out to be additive, the localization of an \emph{Abelian} category, such as \(\Ch(\calA)\), need not be Abelian.
This is bad! After all, we would like to understand their behaviour in terms of things we already know.

Historically, a proposed solution came in the development of \emph{triangulated categories,} which we will also study. Triangulated categories were the invention of Alexander Grothendieck
together with his PhD student Jean-Louis Verdier, and they capture the notion of ``homotopy'' Abelian categories. They can be thought of as an analogue of Quillen exact categories, in that
one specifies as part of additional structure a collection of \emph{distinguished} sequences of maps to be considered the ``homotopy short exact sequences.''

A more systematic proposal came later in the form of stable model categories. However, the most recent and sophisticated upgrade to the concept is provided by \emph{stable} \(\infty\)\emph{-categories.}
We give a heuristic explanation of these now.
\begin{definition}
	A \emph{pointed} \(\infty\)-category is an \(\infty\)-category which admits a zero element, namely an element which is both initial and terminal.
\end{definition}
\begin{remark}
	As an aside, any pointed \(\infty\)-category \(A\) which admits finite limits and finite colimits also admits two functors \(\Sigma\!:A\to A\) and \(\Omega\!:A\to A\) given on objects by the diagrams
	\begin{center}
	\begin{tikzcd}
		x\ar[r]\ar[d] & *\ar[d] \\
		*\ar[r] & \Sigma x\ar[ul,pushout]
	\end{tikzcd}
	\quad
	\begin{tikzcd}
		\Omega x\ar[r]\ar[d]\ar[dr,pullback] & * \ar[d] \\
		*\ar[r] & x
	\end{tikzcd}
	\end{center}
	and furthermore, it turns out that \(\Sigma\ladj\Omega\). These functors themselves can be used, along with some other things, to define stability. This is how it is handled
	in the context of stable model categories.
\end{remark}

\begin{definition}
	Let \(A\) be a pointed \(\infty\)-category, and let \(f\!:x\to y\) be a morphism in \(A\). The \emph{fiber} and \emph{cofiber} of \(f\) are defined by
	the diagrams
	\begin{center}
	\begin{tikzcd}
		\fib(f)\ar[r]\ar[d]\ar[dr,pullback] & x\ar[d,"f"] \\
		*\ar[r] & y
	\end{tikzcd}
	\quad
	\begin{tikzcd}
		x\ar[r,"f"]\ar[d] & y\ar[d] \\
		*\ar[r] & \cofib(f)\ar[ul,pushout]
	\end{tikzcd}
	\end{center}
	in \(A\) whenever they exist. The square above left is called a \emph{fiber sequence,} and the square above right is called a \emph{cofiber sequence.}
\end{definition}

With the above definition available, we can make a very simple definition.
\begin{definition}
	Let \(A\) be a pointed \(\infty\)-category. We say \(A\) is \emph{stable} if
	\begin{enumerate}[label=(\arabic*)]
	\item every morphism in \(A\) has a fiber and a cofiber, and
	\item every fiber sequence is a cofiber sequence, and vice versa.
	\end{enumerate}
\end{definition}

To any \(\infty\)-category \(X\), one can associate a 1-category \(\ho(X)\), called its \emph{homotopy category.} The functor \(X\mapsto\ho(X)\) is left adjoint to the inclusion of 1-categories
into \(\infty\)-categories. Stable \(\infty\)-categories \(A\) have the property that their homotopy categories \(\ho(A)\) have a canonical triangulated structure, induced by the functor
\(\Sigma\!:A\to A\) and with distinguished triangles
\[ x\to y\to z \to \Sigma x \]
in \(\ho(A)\) defined by being those sequences of morphisms which fit into
\begin{diagram*}
	x \ar[r]\ar[d] & y\ar[d]\ar[r] & *\ar[d] \\
	* \ar[r] & z \ar[r] & \Sigma x
\end{diagram*}
where both squares are (co)fiber sequences in \(A\). In this way, stable \(\infty\)-categories are a natural enrichment of triangulated categories.

Triangulated categories have a deficiency in that their distinguished triangles do not yield any universal properties, so there is no good way to characterize the ``homotopy (co)kernels''
that arise from them. Stable \(\infty\)-categories fix this as the \(\infty\)-categorical framework allows a genuine ``homotopy-theoretic'' universal property. Due to the framework
being so well-behaved, they come along with a number of other theoretical benefits, such as cofibers of exact functors between stable \(\infty\)-categories being stable. Furthermore,
there is a natural way to promote derived categories into stable \(\infty\)-categories in a way which recovers the classical ones after applying \(\ho(-)\).


%!TEX root = ../lectures.tex

\section{Monadicity}
Having built up the formalism of Abelian categories a bit, we would like to produce examples. In order to do this, we will take a detour into the topic of monads,
because they will allow us to easily deduce properties of certain nice subcategories, namely those whose inclusions admit a left adjoint, and this will be of interest
to us when we want to construct new Abelian categories from old ones. For this lecture, we follow \cite{riehl-category-theory-in-context}.

\subsection{Monads}
\begin{definition}
	Let \(\calC\) be a category. A \emph{monad} on \(\calC\) is a monoid in the category of endofunctors on \(\calC\). This means it is a triple \((T,\eta,\mu)\) consisting of a functor \(T\!:\calC\to\calC\),
	a natural transformation \(\eta\!:\1 \To T\) called the \emph{unit,} and a natural transformation \(\mu\!:T^2\to T\) called the \emph{multiplication,} such that the diagrams
	\begin{center}
		\begin{tikzcd}
			T \ar[r,Rightarrow,"\eta T"]\ar[dr,equal] & T^2\ar[d,Rightarrow,"\mu"] & \ar[l,Rightarrow,"T\eta"'] T\ar[dl,equal] \\
			& T
		\end{tikzcd}
		\quad
		\begin{tikzcd}
			T^3\ar[r,Rightarrow,"T\mu"]\ar[d,Rightarrow,"\mu T"'] & T^2\ar[d,Rightarrow,"\mu"] \\
			T^2\ar[r,Rightarrow,"\mu"] & T
		\end{tikzcd}
	\end{center}
	expressing unitality and associativity commute.
\end{definition}

The way to think of monads is as the ``shadow'' an adjunction on the codomain of the right adjoint (i.e.\ the domain of the left adjoint). In particular, we have the following
\begin{proposition}\label{prop:adjunctions-induce-monads}
	Consider an adjunction
	\begin{center}
	\begin{tikzcd}
		\calD\ar[from=r,bend right,"L"',""{name=A,below}] & \calC, \ar[from=l,bend right,"R"',""{name=B,above}]\ar[from=A,to=B,symbol=\dashv]
	\end{tikzcd}
	\quad \(\eta\!:\1\To RL,\quad \varepsilon\!: LR\To\1.\)
	\end{center}
	Let \(\mu = R\varepsilon L\) and \(T = RL\). Then \((T, \eta, \mu)\) is a monad on \(\calC\).
\end{proposition}
\begin{proof}
Expanding the diagrams we need to check, we get
\begin{center}
	\begin{tikzcd}
		RL \ar[r,Rightarrow,"\eta RL"]\ar[dr,equal] & RLRL\ar[d,Rightarrow,"R\varepsilon L"] & \ar[l,Rightarrow,"RL\eta"'] RL\ar[dl,equal] \\
		& RL
	\end{tikzcd}
	\quad
	\begin{tikzcd}
		RLRLRL\ar[r,Rightarrow,"RLR\varepsilon L"]\ar[d,Rightarrow,"R\varepsilon LRL"'] & RLRL\ar[d,Rightarrow,"R\varepsilon L"] \\
		RLRL\ar[r,Rightarrow,"R\varepsilon L"] & RL
	\end{tikzcd}
\end{center}
and one now sees that the diagram on the left commutes by the triangle identities, and the diagram on the right commutes by applying the following lemma to the natural transformation
\(R\varepsilon\!:RLR\to R\).
\end{proof}
% \begin{lemma}
% 	Consider two functors \(L\!:\calD\to\calC\), \(R\!:\calC\to\calD\) together with a natural transformation \(\varepsilon\!:LR\To\1\). Then the natural transformation
% 	\[ R\varepsilon\!:RLR\To R \]
% 	defines a natural transformation \((R\varepsilon)\) as below
% 	\begin{diagram*}[column sep=large]
% 		\Fun(\calC,\calD) \ar[r,bend left,"(RLR)",""{name=A,below}] & \Fun(\calC,\calD)\ar[l,bend left,"(R)",""{name=B,above}]\ar[from=A,to=B,Rightarrow,"(R\varepsilon)"]
% 	\end{diagram*}
% 	where \((RLR)\!:\Fun(\calC,\calD)\) is the functor \(F\mapsto RLRF\), similarly for \((R)\), and the natural transformation \((R\varepsilon)\) is defined by
% 	\[ (R\varepsilon)_F := R\varepsilon F. \]
% \end{lemma}
% \begin{proof}
% Given a natural transformation \(\sigma\!:G\To H\), we have to verify that the diagram
% \begin{diagram*}
% 	RLRG \ar[r,Rightarrow,"R\varepsilon G"]\ar[d,Rightarrow, "RLR\sigma"'] & RG\ar[d,Rightarrow,"R\sigma"] \\
% 	RLRH \ar[r,Rightarrow,"R\varepsilon H"] & RH
% \end{diagram*}
% commutes. For this, we may reduce to checking on the components for some generic \(x\in\calC\), which requires us to check that the diagram
% \begin{diagram*}
% 	RLRGx \ar[r,"R\varepsilon_{Gx}"]\ar[d, "RLR\sigma_x"'] & RGx\ar[d,"R\sigma_x"] \\
% 	RLRHx \ar[r,"R\varepsilon_{Hx}"] & RHx
% \end{diagram*}
% commutes. Here, we use the naturality of \(\varepsilon\!:LR\To\1\) to note that
% \begin{diagram*}
% 	LRGx\ar[d,"LR\sigma_x"']\ar[r,"\varepsilon_{Gx}"] & Gx\ar[d,"\sigma_x"] \\
% 	LRHx\ar[r,"\varepsilon_{Hx}"] & Hx
% \end{diagram*}
% commutes, so that
% \begin{align*}
% 	R\varepsilon_{Hx}\circ RLR\sigma_x &= R(\varepsilon_{Hx}\circ LR\sigma_x) \\
% 	&= R(\sigma_x\circ\varepsilon_{Gx}) = R\sigma_x\circ R\varepsilon_{Gx}
% \end{align*}
% as desired.
% \end{proof}
\begin{lemma}
	Consider a natural transformation \( \tau\!:F\To F' \) between functors \(\calC\to\calD\). This defines a natural transformation \((\tau)\) as below
	\begin{diagram*}
		\Fun(\calD',\calC) \ar[r,bend left,"(F)",""{name=A,below}]\ar[r,bend right,"(F')"',""{name=B,above}] & \Fun(\calD',\calD)\ar[from=A,to=B,Rightarrow,"(\tau)"]
	\end{diagram*}
	where \((F)\!:\Fun(\calC,\calD)\to\Fun(\calD',\calD)\) is the functor \(G\mapsto FG\), similarly for \((F')\), and the natural transformation \((\tau)\) is defined on components by
	\[ (\tau)_G := \tau G. \]
\end{lemma}
\begin{proof}
Given a natural transformation \(\sigma\!:G\To H\), we have to verify that the diagram
\begin{diagram*}
	FG \ar[r,Rightarrow,"\tau G"]\ar[d,Rightarrow, "F\sigma"'] & F'G\ar[d,Rightarrow,"F'\sigma"] \\
	FH \ar[r,Rightarrow,"\tau H"] & F'H
\end{diagram*}
commutes. For this, we may reduce to checking on the components for some generic \(x\in\calC\), which requires us to check that the diagram
\begin{diagram*}
	FGx \ar[r,"\tau_{Gx}"]\ar[d, "F\sigma_x"'] & F'Gx\ar[d,"F'\sigma_x"] \\
	FHx \ar[r,"\tau_{Hx}"] & F'Hx
\end{diagram*}
commutes, but this follows by the naturality of \(\tau\!:F\To F'\).
\end{proof}
\begin{remark}
	The above lemma can be seen as a special case of \emph{pasting diagrams} producing a single, well-defined composite in a 2-category. Indeed, the two composites in the square
	of natural transformations above are the two possible ways of composing the pasting diagram
	\begin{diagram*}[column sep=large]
		\calD'\ar[r,bend left,"G",""{name=A,below}]\ar[r,bend right,"H"',""{name=B,above}]\ar[from=A,to=B,Rightarrow,"\sigma"] & \calC\ar[r,bend left,"F",""{name=C,below}]\ar[r,bend right,"F'"',""{name=D,above}]\ar[from=C,to=D,Rightarrow,"\tau"] & \calD
	\end{diagram*}
	and it is a general theorem that such diagrams always yield a unique composite. In the case of Proposition \ref{prop:adjunctions-induce-monads}, applying this to the diagram
	\begin{diagram*}
		& & \calC\ar[dr,"L"] & & \calC\ar[dr,"L"] & \\
		\calC\ar[r,"L"] & \calD\ar[rr,equal,""{name=A,above}]\ar[ur,"R"]\ar[from=ur,to=A,Rightarrow,"\varepsilon"] & & \calD\ar[rr,equal,""{name=B,above}]\ar[ur,"R"]\ar[from=ur,to=B,Rightarrow,"\varepsilon"] & & \calD\ar[r,"R"] & \calC
	\end{diagram*}
	gives another way to obtain the result.
\end{remark}

\begin{example}
	Let \(\calC\) be a category admitting finite coproducts, and let \(x_0\in\calC\). There is an adjunction
	\begin{diagram*}
		\mathmakebox[\widthof{\calC}][r]{x_0/\calC} \ar[r,bend right,"U"',""{name=B}] & \calC \ar[l,bend right,"F"',""{name=A}]\ar[from=A,to=B,symbol=\vdash]
	\end{diagram*}
	where \(U\) is the forgetful functor, and \(F\) is given by \(x\mapsto (x_0\to x_0\amalg x)\). It is immediate that \(F\ladj U\). The induced monad
	\((-)_+\!:\calC\to\calC\) sends \(x\) to \(x_0\amalg x\); the unit \(\eta_x\!:x\to x_+\) is just the canonical map \(x\to x_0\amalg x\), and the
	multiplication \(\mu_x\!:(x_+)_+\to x\) is the map \(x_0\amalg x_0\amalg x \to x_0 \amalg x\) induced by the identities \(\id_{x_0\amalg x}\) and \(\id_{x_0}\).
\end{example}
\begin{example}
	There is a free-forgetful adjunction
	\begin{diagram*}[cramped,column sep=large]
		\Mon \ar[r,bend right,"U"',""{name=B}] & \Set \ar[l,bend right,"F"',""{name=A}]\ar[from=A,to=B,symbol=\vdash]
	\end{diagram*}
	and the induced monad \(T\!:\Set\to\Set\) is the \emph{free monoid monad,} also known as the \emph{list monad.} For a set \(X\), the unit \(\eta_X\!:X\to TX\)
	is \(x\mapsto (x)\). The multiplication is given by concatenation.
\end{example}
\begin{example}
	Similarly, for a ring \(R\), there is a free-forgetful functor
	\begin{diagram*}[cramped,column sep=large]
		\Mod_R \ar[r,bend right,"U"',""{name=B}] & \Set \ar[l,bend right,"F"',""{name=A}]\ar[from=A,to=B,symbol=\vdash]
	\end{diagram*}
	which induces a monad \(R[-]\!:\Set\to\Set\), the \emph{free} \(R\)\emph{-module monad.} It sends a set \(X\) to the set \(R[X]\) of formal sums of elements of \(X\) with coefficients in \(R\).
	The unit \(\eta_X\!:X\to R[X]\) is given by \(x\mapsto 1\cdot x\), and the multiplication is given by distributing sums.
\end{example}
\begin{example}
	Consider the two adjunctions
	\begin{diagram*}[cramped,column sep=large]
		\cat{CHaus} \ar[r,bend right,hook,"U"',""{name=D}] & \Top \ar[r,bend right,"U"',""{name=B}]\ar[l,bend right,"\beta"',""{name=C}] & \Set. \ar[l,bend right,"\text{disc.}"',""{name=A}]\ar[from=A,to=B,symbol=\vdash]\ar[from=C,to=D,symbol=\vdash]
	\end{diagram*}
	Composing them, we get a monad \(\beta\!:\Set\to\Set\), which sends a set \(X\) to the Stone--\v{C}ech compactification of the discrete space \(X\), i.e.\ to the set of ultrafilters on \(X\).
\end{example}
\begin{example}
	The (covariant) power set functor \(\sfP\!:\Set\to\Set\), sending a set \(X\) to its power set \(\sfP X\) and a map \(f\!:X\to Y\) to the image map \(f_*\!:\sfP X\to\sfP Y\), \(A\mapsto f(A)\),
	can be equipped with the structure of a monad. The unit \(\eta_X\!:X\to\sfP X\) is the map \(x\mapsto \{x\}\), and the multiplication \(\mu_X\!:\sfP^2X\to\sfP X\) sends a collection of subsets \(\calA\)
	to the union \(\cup\calA\).
\end{example}
\begin{example}
	The \emph{contravariant} power set functor \(\sfP\!:\Set\to\Set^\op\), sending a map \(f\!:X\to Y\) to the inverse image map \(f^{-1}\!:\sfP Y\to \sfP X\), is its own adjoint. Indeed,
	\[ \Set(X,\sfP Y) = \Set(X,2^Y) \cong \Set(X\times Y,2)\cong\Set(Y,2^X)\cong\Set(Y,\sfP X). \]
	In particular, we get the \emph{double power set monad} \(\sfP^2\!:\Set\to\Set\).
\end{example}
\begin{example}
	Let \((M,\cdot,e)\in\Mon\) be a monoid. Then there is a monad \((-)\times M\!:\Set\to\Set\), where the monad structure is given by \(\eta_X\!:X\to X\times M\), \(x\mapsto (x,e)\),
	and \(\mu_X\!:X\times M\times M\to X\times M\), \((x,m,m')\mapsto (x,m\cdot m')\). In particular, one gets a monad \((-)\times\N\), which corresponds to considering ``discrete time variables.''
\end{example}
\begin{example}
	Let \(\cat{Meas}\) be the category of measurable spaces, with morphisms measurable functions. There is a monad \(\Prob\!:\cat{Meas}\to\cat{Meas}\), which sends a measurable space
	\(X\) to the measurable space \(\Prob(X)\) of probability measures on \(X\), where the \(\sigma\)-algebra is the smallest one for which the evaluation maps \(\ev_A\!:\Prob(X)\to[0,1]\) are measurable.
	The unit \(X\to\Prob(X)\) is given by \(x\mapsto\delta_x\), where the latter is the Dirac measure at \(x\). The multiplication is given by integration: \(\mu_X\!:\Prob(\Prob(X))\to\Prob(X)\)
	is given by sending a measure \(\nu\in\Prob(\Prob(X))\) to the measure
	\[ \mu_X(\nu)\!: A \mapsto \int_{\Prob(X)} \ev_A\diff\nu. \]
\end{example}
\begin{example}
	Let \(\Bbbk\) be a field. There is an \emph{affine monad,} \(\Aff_\Bbbk\!:\Set\to\Set\), which is very similar to the free \(R\)-module monad, with a small modification.
	It is given by
	\[ \Aff_\Bbbk(X) = \left\{\text{formal sums } \sum_{x\in X}\lambda_x x\text{ such that } \sum_{x\in X}\lambda_x = 1\right\}. \]
	The unit \(\eta_X\!:X\to\Aff_\Bbbk(X)\) is given by \(x\mapsto 1\cdot x\), while the multiplication distributes sums.
\end{example}

\subsection{Algebras over monads}
We saw in Proposition \ref{prop:adjunctions-induce-monads} that an adjunction induces a monad. One may wonder about the converse question: does every monad
come from an adjunction?

\begin{terminology}
	We say a monad \((T,\eta,\mu)\) on a category \(\calC\) is \emph{induced from an adjunction} if there is an adjunction
	\begin{center}
	\begin{tikzcd}
		\calD\ar[from=r,bend right,"L"',""{name=A,below}] & \calC, \ar[from=l,bend right,"R"',""{name=B,above}]\ar[from=A,to=B,symbol=\dashv]
	\end{tikzcd}
	\quad \(\eta\!:\1\To RL,\quad \varepsilon\!: LR\To\1.\)
	\end{center}
	such that \((T,\eta,\mu) = (RL,\eta,R\varepsilon L)\). In this case, we say \(T\) is induced from \(L\ladj R\).
\end{terminology}

It turns out that the answer is yes, and showing this can be done by introducing an independently very interesting category of objects given by the \emph{algebras} over a monad.

\begin{definition}
	Let \((T,\eta,\mu)\) be a monad on a category \(\calC\). A \(T\)\emph{-algebra} is a pair \((x,a)\) of an object \(x\in\calC\) and a morphism \(a\!:Tx\to x\), such that the diagrams
	\begin{center}
	\begin{tikzcd}
		x\ar[r,"\eta_x"]\ar[dr,equal] & Tx \ar[d,"a"] & \\
		& x
	\end{tikzcd}\quad
	\begin{tikzcd}
		T^2x\ar[r,"\mu_x"]\ar[d,"Ta"'] & Tx\ar[d,"a"]\\
		Tx\ar[r,"a"] & x
	\end{tikzcd}
	\end{center}
	commute. A \emph{morphism} of \(T\)-algebras \(f\!:(x,a)\to (y,b)\) is a morphism \(f\!:x\to y\) such that the diagram
	\begin{diagram*}
		Tx\ar[r,"Tf"]\ar[d,"a"'] & Ty\ar[d,"b"] \\
		x\ar[r,"f"] & y
	\end{diagram*}
	commutes. This organizes into a category \(\calC^T\) called the \emph{Eilenberg--Moore category} for \(T\), or the \emph{category of} \(T\)\emph{-algebras.}
\end{definition}
\begin{construction}\label{construction:free-forgetful-algebra-functors}
	Let \((T,\eta,\mu)\) be a monad on \(\calC\). We can construct two functors
	\[ F^T\!:\calC\to\calC^T, \quad U^T\!:\calC^T\to\calC \]
	as follows. The functor \(U^T\) is the obvious forgetful functor which sends \((x,a)\) to \(x\). The \emph{free} functor \(F^T\) sends \(x\in\calC\) to the \emph{free} \(T\)\emph{-algebra}
	\[ F^T(x) := (Tx, \mu_x\!:T^2x\to Tx) \]
	and a morphism \(f\!:x\to y\) in \(\calC\) to the morphism of \(T\)-algebras
	\[ F^Tf := (Tx,\mu_x) \xrightarrow{Tf} (Ty,\mu_y). \]
	This is a morphism of \(T\)-algebras since the diagram
	\begin{diagram*}
		T^2x\ar[r,"T^2f"]\ar[d,"\mu_x"'] & T^2y\ar[d,"\mu_y"] \\
		Tx\ar[r,"Tf"] & Ty
	\end{diagram*}
	commutes by the naturality of \(\mu\). Note that the functors \(F^T\) and \(U^T\) trivially satisfy \(U^T\circ F^T = T\).
\end{construction}

\begin{proposition}
	Let \((T,\eta,\mu)\) be a monad on \(\calC\). Then we have an adjunction
	\begin{diagram*}[column sep=large]
		\calC\ar[r,bend left,"F^T",""{name=A,below}] & \ar[l,bend left,"U^T",""{name=B,above}]\calC^T\ar[from=A,to=B,symbol=\dashv]
	\end{diagram*}
	inducing \(T\) involving the functors from Construction \ref{construction:free-forgetful-algebra-functors}. In particular, every monad is induced from an adjunction.
\end{proposition}
\begin{proof}
We have to produce unit and counit natural transformations
\[ \eta\!:\1\To U^TF^T = T,\quad \varepsilon\!:F^TU^T\To\1. \]
In the first case, we use the existing unit \(\eta\!:\1\To T\). For the counit, we first note that given a \(T\)-algebra \((x,a)\in\calC^T\), the morphism \(a\!:Tx\to x\) defines
a morphism of \(T\)-algebras
\[ F^TU^T(x,a) = (Tx,\mu_x) \to (x,a) \]
since the diagram
\begin{diagram*}
	T^2x\ar[r,"Ta"]\ar[d,"\mu_x"'] & Tx\ar[d,"a"] \\
	Tx \ar[r,"a"] & x
\end{diagram*}
commutes. Thus, we let \(\varepsilon\) be given on components by
\[ \varepsilon_{(x,a)} := a\!: F^TU^T(x,a) \to (x,a). \]
Thus we have the data of a unit and counit; it remains to check that they satisfy the triangle identities, which are
\begin{center}
\begin{tikzcd}[column sep=small]
	 & U^TF^TU^T\ar[dr,Rightarrow,"U^T\varepsilon"] & \\
	U^T\ar[ur,Rightarrow,"\eta U^T"]\ar[rr,equal,"?"] & & U^T
\end{tikzcd}\quad
\begin{tikzcd}[column sep=small]
	 & F^TU^TF^T\ar[dr,Rightarrow,"\varepsilon F^T"] & \\
	F^T\ar[ur,Rightarrow,"F^T\eta"]\ar[rr,equal,"?"] & & F^T
\end{tikzcd}
\end{center}
and so, applying these to some \((x,a)\in\calC^T\) and \(x\in\calC\), we get
\begin{center}
\begin{tikzcd}[column sep=small]
	 & Tx\ar[dr,"a"] & \\
	x\ar[ur,"\eta_x"]\ar[rr,equal] & & x
\end{tikzcd}\quad
\begin{tikzcd}[column sep=tiny]
	 & (T^2x,\mu_{Tx})\ar[dr,"\mu_x"] & \\
	(Tx,\mu_x)\ar[ur,"T\eta_x"]\ar[rr,equal] & & (Tx,\mu_x)
\end{tikzcd}
\end{center}
which trially commute by definition of \(T\)-algebras and monads, respectively. Thus we have our adjunction.
Now note that \(U^T\varepsilon F^T = \mu\), since \(U^T\varepsilon_{F^Tx} = U^T\mu_x = \mu_x\), so we see that
\((T,\eta,\mu) = (U^TF^T,\eta,U^T\varepsilon F^T)\) as desired.
\end{proof}
\begin{remark}
	In particular, we see that monads can, in some sense, be thought of as structures formalizing free-forgetful adjunctions. Indeed, since any adjunction induces a monad,
	and any monad induces a free-forgetful adjunction, we see that we can in some sense identify monads with those adjunctions which are of this form. Of course, a
	``free-forgetful adjunction'' is not really well-defined on its own, so perhaps it is better to see monads as the way in which one makes them precise.
\end{remark}
\begin{remark}
	There is another adjunction one can extract from a monad, formed with respect to a category called the \emph{Kleisli category} of the monad. However, since we will not need it,
	we will neglect looking into it in these notes.
\end{remark}

\begin{example}
	Let \(\Bbbk\) be a field, and consider the affine monad \(\Aff_\Bbbk\). An \emph{affine space} over \(\Bbbk\) (in the sense of linear algebra, not algebraic geometry) is an
	\(\Aff_\Bbbk\)-algebra.
\end{example}
\begin{example}
	Let \(\calC\) be a category admitting finite coproducts, let \(x_0\in\calC\), and let \((-)_+\!:\calC\to\calC\) be the associated monad \(x\mapsto x_0\amalg x\). An algebra
	for this monad is an object \(x\in\calC\) together with a structure map \(a\!:x_0\amalg x \to x\), so that some diagrams commute. The square says nothing, but the triangle tells us that the component
	\(x\to x\) of \(a\) is the identity, so the only new data is a map \(x_0\to x\). One sees by inspection that the category of algebras is given by \(x_0/\calC\) itself,
	and the adjunction inducing the monad is recovered by the category of algebras.
\end{example}
\begin{example}
	Consider the free monoid monad \(T\) on \(\Set\). Then the category of algebras \(\Set^T\) is equivalent to the category \(\Mon\) of monoids.
\end{example}
\begin{example}
	Let \(R\) be a ring, and consider the free \(R\)-module monad \(R[-]\!:\Set\to\Set\). Then \(R[-]\)-algebras are just left \(R\)-modules, i.e.\ \(\Set^{R[-]}\simeq\Mod_{R^\op}\). An \(R[-]\)-module
	consists of a set \(A\) and a morphism \(a\!:R[A]\to A\) such that \(a\circ\eta_A = \id_A\) and \(a\circ\mu_A = a\circ R[a]\). The former condition just means that \(a(1\cdot x) = x\) for all \(x\in A\).
	Now, define
	\begin{align*}
		+\!:A\times A &\to A, & \cdot\!:R\times A&\to A, \\
		(x,y) &\mapsto a(1\cdot x+1\cdot y), & (r,x) &\mapsto a(r\cdot x).
	\end{align*}
	This turns \(A\) into a left \(R\)-module. In particular, the commutativity and associativity of \(+\) come from the commutativity of addition of formal sums, as well as the commutative square in the definition
	of an algebra. The \(R\)-multiplication will give a linear action on \(A\) for similar reasons. To see that \(A\) has a zero element, let \(0\in A\) be the image of the empty formal sum; one
	easily sees that \(0+x = x\) for all \(x\in A\). The existence of additive inverses is not hard either.

	An \(R[-]\)-algebra morphism \((A,a)\to (B,b)\) provides an \(R\)-linear homomorphism \(A\to B\), when \(A\) and \(B\) are equipped with the left \(R\)-module structures described above. In
	particular, we get a functor \(\Set^{R[-]}\to\Mod_{R^\op}\), which is clearly faithful. One can check, by explicitly constructing an \(R[-]\)-algebra from a left \(R\)-module, that it is also essentially surjective.
	In the process, one will also see that the functor is full, hence an equivalence.
\end{example}
\begin{exercise}
	Let \(R\) be a ring (commutative, for simplicity). Consider the adjunction given by the tensor product \(R\otimes_\Z(-)\!:\Ab\to\Mod_R\) and the forgeful functor \(U\!:\Mod_R\to\Set\).
	Abusively, let \(R\otimes_\Z(-)\) also denote the associated monad on \(\Ab\). Show that the category of \((R\otimes_\Z(-))\)-algebras is equivalent to \(\Mod_R\).
\end{exercise}

The category of \(T\)-algebras is unique in that it is, in a suitable way, \emph{terminal} amongst categories with adjunctions that induce \(T\). In order to see this, we will also briefly
need to explain a concept about adjunctions.
\begin{proposition}
	Let \((T,\eta,\mu)\) be a monad on \(\calC\). Then, for any adjunction \(F\ladj U\) inducing \(T\), we have a unique induced functor
	\begin{diagram*}[column sep=huge, row sep=large]
		\calD\ar[r,dashed,"K"]\ar[d,bend left,"U",""{name=D,left}] & \calC^T\ar[dl,shift left, bend left=40, "U^T",""{name=B,above}] \\
		\calC\ar[u,bend left,"F",""{name=C,right}]\ar[ur,bend right=10,"F^T",""{name=A,right}]\ar[to=A,from=B,symbol=\vdash]\ar[to=C,from=D,symbol=\vdash]
	\end{diagram*}
	for which \(KF = F^T\) and \(U^T K = U\).
\end{proposition}
\begin{proof}
Let \(\varepsilon\!:FU\To\1\) be the counit for the adjunction \(F\ladj U\). Since the adjunction induces \(T\), we have \(\mu = U\varepsilon F\). Furthermore, since
the units of \(F\ladj U\) and \(F^T\ladj U^T\) agree, both being \(\eta\!:\1\To T\), Lemma \ref{lemma:morphism-of-adjunctions-equivalence} below tells us that if \(K\) exists,
\(\varepsilon\) is the counit for \(F^T\ladj U^T\) and \(\varepsilon'\) is the counit for \(F\ladj U\), then \(K\varepsilon' = \varepsilon K\).

We need to define \(K\) so that \(U^TK = U\), i.e.\ so that \(U^TKx = Ux\), \(x\in\calD\). In other words, we are forced to design \(K\) by putting a \(T\)-algebra structure on \(Ux\), i.e.\ a map \(TUx \to Ux\)
in such a way that a map \(f\!:x\to y\) should lead to \(Uf\) being a \(T\)-algebra morphism.

On the other hand, for any \(T\)-algebra \((z,a)\in\calC^T\), the morphism \(a\!:Tz\to z\) is given by the component of the counit \(\varepsilon\) at \((z,a)\). Thus, we are forced to endow \(Kx\) with the algebra
structure given by the morphism
\[ \varepsilon_{Kx} = K\varepsilon_x' = U\varepsilon_x'. \]
Therefore, we set \(Kx := (Ux,U\varepsilon)\). One easily checks that given a morphism \(f\!:x\to y\) ion \(\calC\), the morphism \(Uf\) is a morphism of \(T\)-algebras
\[ Kx = (Ux,U\varepsilon_x')\to(Uy,U\varepsilon_y') = Ky \]
so we have our desired functor. Since all choices were forced on us, the choice is unique. Finally, it is clear that \(KF = F^T\)
since \(KFx = (UFx, U\varepsilon_{Fx}) = (Tx, \mu_x) = F^Tx\).
\end{proof}

In the above, we use a property of certain diagrams involving adjunctions, indicating to the below lemma, which explains a few equivalent conditions. In general, a diagram as below satisfying
the listed equivalent properties is called a \emph{morphism of adjunctions.} As witnessed in the proof above, they have convenient properties.

\begin{lemma}\label{lemma:morphism-of-adjunctions-equivalence}
	Consider a diagram
	\begin{diagram*}
		\calC\ar[r,"H"]\ar[d,shift right,bend right=10,"F"',""{name=A,right}]\ar[from=d,shift right,bend right=10,"G"',""{name=B,left}] & \calC'\ar[d,shift right,bend right=10,"F'"',""{name=C,right}]\ar[from=d,shift right,bend right=10,"G'"',""{name=D,left}] \\
		\calD\ar[r,"K"'] & \calD'
		\ar[from=A,to=B,symbol=\vdash]\ar[from=C,to=D,symbol=\vdash]
	\end{diagram*}
	such that \(KF = HF'\) and \(HG = G'K\). Then the following are equivalent.
	\begin{enumerate}[label=(\arabic*)]
	\item \(H\eta = \eta' H\), where \(\eta\) and \(\eta'\) are the units of the adjunctions.
	\item \(K\varepsilon = \varepsilon' K\), where \(\varepsilon\) and \(\varepsilon'\) are the counits of the adjunctions.
	\item The diagram
	\begin{diagram*}
		\calD(Fx,y)\ar[d,"K"']\ar[r,"\sim"] & \calC(x,Gy)\ar[d,"H"] \\
		\calD'(KFx,Ky)\ar[d,equal] & \calC'(Hc,HGd)\ar[d,equal] \\
		\calD'(F'Hx,Ky)\ar[r,"\sim"] & \calC'(Hx,G'Ky)
	\end{diagram*}
	commutes.
	\end{enumerate}
\end{lemma}
\begin{proof} We begin by showing that (1) is equivalent to (3). The two composites in the diagram in (3) are
\[ f\mapsto H(Gf\circ\eta_{x})\quad \text{and} \quad f\mapsto G'Kf\circ\eta'_{Hx}. \]

(1) \(\Rightarrow\) (3). For this, we just compute
\[ G'Kf\circ\eta'_{Hx} = HGf\circ\eta'_{Hx} \overset{(1)}= HGf\circ H\eta_x = H(Gf\circ\eta_x) \]
as desired.

(3) \(\Rightarrow\) (1). Here, we set \(y=Fx\) and track what happens to \(\id_{Fx}\). On one hand, we have
\[ HG\id_{Fx}\circ H\eta_{x} = \id_{HGFx}\circ H\eta_{x} = H\eta_x \]
and on the other
\[ G'K\id_{Fx}\circ\eta'_{Hx} = \id_{G'KFx}\circ\eta'_{Hx} = \eta'_{Hx} = (\eta'H)_x \]
so that
\[ \forall x\in\calC,\quad H\eta_x \overset{(3)}= (\eta'H)_x \quad\implies\quad H\eta = \eta H \]
as desired. Thus, (1) is equivalent to (3).

The equivalence between (2) and (3) is dual. In particular, following the same argument as above but using the inverses of the adjunction isomorphisms yields the result.
\end{proof}


\subsection{Monadic functors}
\begin{definition}
	Monadic functors are those which are determined by algebras over monads.
	\begin{enumerate}[label=(\arabic*)]
	\item Consider an adjunction
	\begin{tikzcd}[cramped]
		\calD\ar[from=r,bend right,"L"',""{name=A,below}] & \calC, \ar[from=l,bend right,"R"',""{name=B,above}]\ar[from=A,to=B,symbol=\dashv]
	\end{tikzcd}
	and let \(T\) be the induced monad on \(\calC\). The adjunction \(L\ladj R\) is \emph{monadic} if the canonical functor \(\calC\to\calC^T\) is an equivalence.
	\item A functor \(F\!:\calC\to\calD\) is \emph{monadic} if it admits a left adjoint \(L\) for which the adjunction \(L\ladj F\) is monadic.
	\end{enumerate}
\end{definition}

Here is a simple way to see that knowing a functor is monadic may be useful.

\begin{proposition}\label{prop:monadic-functors-reflect-isomorphisms}
	Let \(U\!:\calD\to\calC\) be monadic. Then \(U\) reflects isomorphisms.
\end{proposition}
\begin{proof}
Since \(U\) is monadic, we may assume that \(U = U^T\) and \(\calD = \calC^T\). An isomorphism \(f\!:(x,a)\to(y,b)\) in \(\calC^T\) is merely an isomorphism \(f\!:x\to y\)
in \(\calC\), since the commutative diagram for \(f\) provides one for \(f^{-1}\) and vice versa. Since \(Uf = f\), we are done.
\end{proof}

\begin{definition}\label{definition:creates-limits}
	Let \(F\!:\calC\to\calD\) be a functor, and \(D\!: I \to\calC\) be a diagram. We say that \(F\) \emph{creates} the limit of \(D\) in \(\calC\) if
	\begin{enumerate}[label=(\arabic*)]
	\item whenever we have a cone over \(D\) in \(\calC\) for which the image in \(\calD\) is a limit cone for \(F\circ D\), the original cone is a limit cone;
	\item whenever a limit cone for \(F\circ D\) exists in \(\calD\), it lifts to a cone in \(\calC\) for \(D\) (which must be a limit cone by (1)).
	\end{enumerate}
\end{definition}

\begin{theorem}\label{thm:monadic-functors-create-limits}
	Let \(U\!:\calD\to\calC\) be a monadic functor. Then \(U\) creates all limits that \(\calC\) admits.
\end{theorem}
\begin{proof}
Since \(U\) is monadic, there is a monad \(T\) on \(\calC\) for which we have a commutative diagram
\begin{diagram*}[cramped]
	\calD\ar[rr,"\sim"]\ar[dr,"U"'] & & \calC^T\ar[dl,"U^T"] \\
	& \calC &
\end{diagram*}
and furthermore, it is clear that \(U^T\) creates limits if and only if \(U\) does. In particular, we may assume that \(\calD = \calC^T\) and \(U = U^T\).

Let \(D\!:I\to\calC^T\) define some diagram of objects \((Di, a_i)\) for which \(U^TD\!:I\to\calC\) admits a limit \(L\in\calC\), given by some limit cone
\[ \beta\!: L\To U^TD,\quad (\beta_i\!: L \to Di)_{i\in I}. \]
The strategy is simple: we will endow \(L\) with the structure of a \(T\)-algebra in such a way that the \(\beta_i\) define \(T\)-algebra maps to \((Di,a_i)\). Since \(D\)
is a diagram in \(\calC^T\), the \(T\)-algebra structure maps \(a_i\) define a natural transformation \(a\!:TU^TD\To U^TD\); given \(\varphi\!:i\to j\) in \(I\),
\begin{diagram*}
	TDi\ar[d,"TU^TD\varphi"']\ar[r,"a_i"] & Di\ar[d,"U^TD\varphi"] \\
	TDj\ar[r,"a_j"] & Dj
\end{diagram*}
commutes. We can now apply \(T\) to the limit cone \(\beta\) and compose to get a cone
\begin{diagram*}
	TL\ar[r,Rightarrow,"T\beta"] & TU^TD\ar[r,Rightarrow,"a"] & U^TD
\end{diagram*}
which by univeral property of the limit thus induces a unique map \(b\!:TL\to L\) in \(\calC\) for which the diagrams
\begin{diagram*}
	TL \ar[r,dashed,"b"]\ar[d,"T\beta_i"'] & L\ar[d,"\beta_i"] \\
	TDi\ar[r,"a_i"] & Di
\end{diagram*}
commute. This \(b\) is our candidate map for making \(L\) into a \(T\)-algebra. To have \((L,b)\in\calC^T\), we must check that the left-most faces in the diagrams
\begin{center}
	\begin{tikzcd}[column sep=small]
		L \ar[dd,equal]\ar[rr,"\beta_i"]\ar[dr,"\eta_L"] & & Di\ar[dr,"\eta_{Di}"]\ar[dd,equal] \\
		& TL\ar[dl,"b"'] & & TDi\ar[from=ll,crossing over,"T\beta_i" near start]\ar[dl,"a_i"] \\
		L\ar[rr,"\beta_i"] & & Di
	\end{tikzcd}
	\quad
	\begin{tikzcd}
		T^2L\ar[dd,"\mu_L"']\ar[rr,"T^2\beta_i"]\ar[dr,"Tb"] & & T^2Di\ar[dd,"\mu_{Di}" near start]\ar[dr,"Ta_i"] & \\
		& TL & & TDi \ar[from=ll,crossing over,"T\beta_i" near start]\ar[dd,"a_i"] \\
		TL\ar[dr,"b"']\ar[rr,"T\beta_i" near start] & & TDi\ar[dr,"a_i"] & \\
		& L \ar[rr,"\beta_i"]\ar[from=uu,crossing over,"b" near start] & & Di
	\end{tikzcd}
\end{center}
with the knowledge that the rest of the faces already do. By universal property, we can do this by checking that it holds after composing with \(\beta_i\) for all \(i\in I\),
which is the content of the other faces commuting. We conclude that \((L,b\!:TL\to L)\) is a \(T\)-algebra, and the commutativity defining \(b\) from before means we get a cone
\[ \alpha\!:(L,b)\To D. \]
We have therefore proven that \(U^T\) satisfies condition (2) in Definition \ref{definition:creates-limits}. It remains to check condition (1).

Using the same notation as before, suppose we have a cone \((\alpha\!:(L,d)\To D)\in\Fun(I,\calC^T)\) such that \(U^T\alpha\!:L\To U^TD\) is a limit cone, and let
\(\gamma\!:(L',b')\To D\) be some other cone. Then \(U^T\gamma\!:L'\To D\) factors uniquely through \(U^T\alpha\) as
\begin{diagram*}[sep=small]
	L'\ar[rr,Rightarrow,"U^T\gamma"]\ar[dr,dashed,"f"'] & & U^TD \\
	& L\ar[ur,Rightarrow,"U^T\alpha"'] &
\end{diagram*}
and all we need to do is check that \(f\) defines a \(T\)-algebra morphism \((L',b')\to (L,b)\), i.e.\ that the left-most face of the diagram
\begin{diagram*}
	TL'\ar[rr,Rightarrow,"TU^T\gamma"]\ar[dr,"Tf"']\ar[dd,"b'"'] & & TU^TD\ar[dd,Rightarrow,"a"] \\
	& TL\ar[ur,Rightarrow,"TU^T\alpha"' near start] & \\
	L'\ar[rr,Rightarrow,"U^T\gamma" near start]\ar[dr,"f"'] & & U^TD \\
	& L\ar[ur,Rightarrow,"U^T\alpha"']\ar[from=uu,crossing over,"b" near start] &
\end{diagram*}
commutes. This follows by the same argument as before since \(U^T\alpha\) is a limit cone.
\end{proof}
\begin{example}
	Let \(R\) be a ring (commutative, for sake of simplicity). We have seen that the forgetful functor \(\Mod_R\to\Set\) is monadic. In particular, we can deduce that \(\Mod_R\) is complete,
	which we already knew anyway; nonetheless, this gives an ``automatic'' way to prove the result. Using the monadicity of the forgetful functor \(\Mod_R\to\Ab\), one can (using a variant of
	Theorem \ref{thm:monadic-functors-create-limits}) show that \(\Mod_R\) admits all colimits that \(\Ab\) admits. Using some more monadicity results, one can prove quite abstractly that \(\Ab\)
	is cocomplete.
\end{example}
\begin{example}
	From knowing that \(\Set\) is complete, Theorem \ref{thm:monadic-functors-create-limits} can prove that \(\Set\) is also cocomplete: one shows that the contravariant power set
	functor \(\Set^\op\to\Set\) is monadic, and thus deduce that \(\Set^\op\) is complete.
\end{example}
\begin{example}
	The forgetful functor \(\cat{CHaus}\to\Set\) is monadic, so Proposition \ref{prop:monadic-functors-reflect-isomorphisms} tells us that a continuous bijective map
	between compact Hausdorff spaces is a homeomorphism.
\end{example}




%!TEX root = ../lectures.tex

\section{More about Abelian categories}
This lecture is about some ways to obtain Abelian categories from ones we already know of. We begin with an easy method, and follow it up with an application of the results of the last
lecture.

\subsection{Functor categories}
In general, given an object \(x_0\) with some property \(P\), the collection of maps \(x\to x_0\) tends to inherit the property \(P\). For example, for a set \(X\) and a group \(G\),
the collection of maps of sets \(X\to G\) forms a group. If \(R\) is a ring, then the collection of maps of sets \(X\to R\) forms a ring. The same holds true for categories of functors.
\begin{proposition}\label{prop:functor-category-inherits-limits-from-codomain}
	Let \(\calC\) and \(\calD\) be categories, and suppose \(\calD\) admits limits of \(I\)-shaped diagrams. Then \(\Fun(\calC,\calD)\) admits \(I\)-shaped limits.
\end{proposition}
\begin{proof}
Given a diagram \(D\!:I\to\Fun(\calC,\calD)\), we consider the evaluation functors
\[ \forall x\in\calC,\quad \ev_x\!:\Fun(\calC,\calD)\to\calD,\quad F\mapsto F(x) \]
and define the functor \(\projlim{D}\) by
\[ (\projlim{D})(x) := \projlim(\ev_x\circ D). \]
We note that this defines a functor since for any \(f\!:x\to y\) in \(\calC\), we obtain a natural transformation
\[ \ev_x\To \ev_y \]
given, for a natural transformation \(\eta\!:F\To G\), by
\begin{diagram*}
	\ev_x(F)\ar[d,"\ev_x(\eta)"']\ar[r,equal] & F(x)\ar[d,"\eta_x"']\ar[r,"F(f)"] & F(y)\ar[d,"\eta_y"]\ar[r,equal] & \ev_y(F)\ar[d,"\ev_y(\eta)"] \\
	\ev_x(G)\ar[r,equal] & F(x)\ar[r,"G(f)"] & G(y) \ar[r,equal] & \ev_y(G)
\end{diagram*}
and so induces a natural transformation \(\ev_x\circ D\to\ev_y\circ D\). One defines the map
\[ (\projlim{D})(f)\!:(\projlim{D})(x) \to (\projlim{D})(y) \]
as the image of this natural transformation under the functor
\[ \projlim\!:\Fun(I,\calD)\to\calD \]
which exists since \(\calD\) admits \(I\)-shaped limits.
\end{proof}
\begin{exercise}
	Let \(\calC\) and \(\calD\) be categories.
	\begin{enumerate}[label=(\arabic*)]
	\item Show that \((-)^\op\) determines a functor \(\Fun(\calC,\calD)^\op\to\Fun(\calC^\op,\calD^\op)\).
	\item Prove that the functor in (1) is an isomorphism of categories.
	\item Deduce from Proposition \ref{prop:functor-category-inherits-limits-from-codomain} that if \(\calD\) admits \(I\)-shaped colimits, then \(\Fun(\calC,\calD)\) admits \(I\)-shaped colimits.
	\item Prove the result in (3) by hand following the proof of Proposition \ref{prop:functor-category-inherits-limits-from-codomain}.
	\end{enumerate}
\end{exercise}
\begin{theorem}
	Let \(\calA\) be an additive category, and let \(\calC\) be some arbitrary category. Then \(\Fun(\calC,\calA)\) is additive. Furthermore, if \(\calA\) is Abelian,
	then \(\Fun(\calC,\calA)\) is Abelian.
\end{theorem}
\begin{proof}
We promote \(\Fun(\calC,\calA)\) into a pre-additive category by defining \(\eta+\sigma\), for \(\eta,\sigma\!:F\To G\), by
\[ (\eta+\sigma)_x = \eta_x+\sigma_x. \]
This defines a natural transformation, since if \((f\!:x\to y)\in\calC\) then
\[ (\eta+\sigma)_y \circ F(f) = \eta_y\circ F(f) + \sigma_y\circ F(f) = G(f)\circ\eta_x+G(f)\circ\sigma_x = G(f)\circ(\eta+\sigma)_x. \]
It is clear that this makes \(\Fun(\calC,\calA)\) into a pre-additive category. It now follows that \(\Fun(\calC,\calA)\) is additive by applying Proposition \ref{prop:functor-category-inherits-limits-from-codomain}.

If we assume that \(\calA\) is Abelian, then Proposition \ref{prop:functor-category-inherits-limits-from-codomain} and its dual specialize to tell us that \(\Fun(\calC,\calA)\) admits
kernels and cokernels, which for some \(\eta\!:F\To G\) are given by
\[ \ker\eta\!:x\mapsto\ker\eta_x,\quad\coker\eta\!:x\mapsto\coker\eta_x. \]
We see in particular that the image and coimage are given by
\[ \img\eta\!:x\mapsto\img\eta_x,\quad \coimg\eta\!:x\mapsto\coimg\eta_x \]
and therefore the map
\[ \coimg\eta\to\img\eta \]
must be an isomorphism, since it is an isomorphism on its components by \(\calA\) being Abelian.
\end{proof}


\subsection{Reflective \& Giraud subcategories}
Consider two categories \(\calC\) and \(\calD\) together with a fully faithful functor \(P\!:\calC\inj\calD\). One might wonder what properties of \(\calD\) are inherited by \(\calC\) and vice versa.
For example, since \(P\) is fully faithful, it reflects isomorphisms, and so any objects in \(\calC\) which are isomorphic in \(\calD\) are isomorphic in \(\calC\) too. On the other hand,
\(P\) need not preserve epimorphisms nor monomorphisms, except for split ones, and similarly, need not preserve limits nor colimits. So, we need more data in order to deduce
properties about \(\calC\) from properties of \(\calD\).

Last lecture, we saw that monadic functors create limits. We will now apply this to a particularly interesting example, namely that of \emph{reflective subcategories.} We begin
with some general properties of adjoints, as well as of reflective subcategories.
\begin{lemma}\label{lemma:fully-faithful-adjoint}
	Consider an adjunction
	\begin{diagram*}
		\calD\ar[from=r,bend right,"L"',""{name=A,below}] & \calC, \ar[from=l,bend right,"R"',""{name=B,above}]\ar[from=A,to=B,symbol=\dashv]
	\end{diagram*}
	with unit \(\eta\) and counit \(\varepsilon\).
	\begin{enumerate}[label=(\arabic*)]
	\item \(R\) is faithful if and only if every component of \(\varepsilon\) is an epimorphism.
	\item \(R\) is fully faithful if and only if \(\varepsilon\) is a natural isomorphism.
	\end{enumerate}
	In the case of (2), \(L\) is essentially surjective and the natural transformation \(L \eta\!:L\To LRL\) is a natural isomorphism.
\end{lemma}
\begin{proof}
We have natural transformations
\[ \calC(x,-)\to\calD(Rx,R-)\iso\calC(LRx,-) \]
the composite of which corresponds to the unit map \(\varepsilon_x\!:LRx\to x\) under the Yoneda lemma. In particular,
\(\varepsilon_x\) is an epimorphism (isomorphism) if and only if the composite \(\calC(x,-)\to\calC(LRx,-)\) is a monomorphism (isomorphism) if and only
if the morphism \(\calC(x,-)\to\calD(Rx,R-)\) is a monomorphism (isomorphism), i.e.\ \(R\) is faithful (fully faithful).

For the final claim, suppose that \(R\) is fully faithful. We then have \(x\cong L(R(x))\) so that \(L\) is essentially surjective; the triangle identity
\begin{diagram*}
	& LRL\ar[dr,Rightarrow,"\varepsilon L"] & \\
	L\ar[ur,Rightarrow,"L\eta"]\ar[rr,equal] & & L
\end{diagram*}
sandwiches \(L\eta\) between the natural isomorphisms \(\id\) and \(\varepsilon L\), so that \(L\eta\) is a natural isomorphism.
\end{proof}
\begin{exercise}
	Show that \(R\) is full if and only if every component of \(\varepsilon\) is a split monomorphism. Furthermore, dualize the above results.
\end{exercise}
\begin{lemma}\label{lemma:right-adjoints-preserve-monomorphisms}
	Let \(R\!:\calC\to\calD\) be a functor admitting a left adjoint \(L\!:\calD\to\calC\). Then \(R\) preserves monomorphisms.
\end{lemma}
\begin{proof}
Suppose \(f\) is a monomorphism, and that we have \(g,g'\!:z\to Rx\) such that
\begin{diagram*}
	z\ar[r,shift left,"g"]\ar[r,shift right,"g'"'] & Rx\ar[r,"Rf"] & Ry
\end{diagram*}
commutes. Applying \(L\), we transpose this to
\begin{diagram*}
	Lz\ar[r,shift left,"Lg"]\ar[r,shift right,"Lg'"']\ar[d,equal] & LRx\ar[r,"LRf"]\ar[d,"\varepsilon_x"'] & LRy\ar[d,"\varepsilon_y"'] \\
	Lz\ar[r,shift left]\ar[r,shift right] & x\ar[r,hook,"f"] & y
\end{diagram*}
and now, since \(f\) is monic, we have that
\[ \varepsilon_x\circ Lg = \varepsilon_x\circ Lg'. \]
Thus \(g = g'\), since the map
\[ \calD(z,Rx)\iso\calC(Lz,x),\quad h\mapsto \varepsilon_x\circ Lh \]
is a bijection.
\end{proof}
\begin{definition}
	Let \(\calC\inj\calD\) be a fully faithful functor. We say it is \emph{reflective} if it has a left adjoint, called the \emph{reflection.} A \emph{reflective subcategory} of \(\calD\) is a full subcategory
	for which the inclusion is reflective.
\end{definition}
\begin{remark}
	In other words, \(\iota\!:\calC\inj\calD\) is reflective if it has a left adjoint \(\pi\) for which the counit \(\varepsilon\!:\1\To \pi \iota\) is an isomorphism.
	One thinks of \(\iota\) as an inclusion and of \(\pi\) as a projection onto \(\calC\). The counit being an isomorphism then says that including then projecting does nothing.
	When one regards \(\calC\) as a genuine full subcategory of \(\calD\), this is somewhat more literal: the counit is then an isomorphism \(x\iso \pi x\) in \(\calD\) for all \(x\in\calC\).
\end{remark}
\begin{proposition}
	Suppose \(\iota\!:\calC\inj\calD\) is reflective, and let \(f\!:x\to y\) be a morphism in \(\calC\). Then \(f\) is a monomorphism if and only if \(\iota(f)\) is a monomorphism.
\end{proposition}
\begin{proof}
By Lemma \ref{lemma:right-adjoints-preserve-monomorphisms}, \(\iota\) preserves monomorphisms since it is a right adjoint, so one direction is done. Hence, suppose \(\iota(f)\)
is a monomorphism, and that we have morphisms \(g,g'\!:z\to x\) such that
\begin{diagram*}
	z\ar[r,shift left,"g"]\ar[r,shift right,"g'"'] & x\ar[r,"f"] & y
\end{diagram*}
commutes. Applying \(\iota\), we have
\begin{diagram*}
	\iota(z)\ar[r,shift left,"\iota(g)"]\ar[r,shift right,"\iota(g)'"'] & \iota(x)\ar[r,"\iota(f)"] & \iota(y)
\end{diagram*}
and since \(\iota(f)\) is monic, we see that \(\iota(g) = \iota(g')\). Since \(\iota\) is fully faithful, the map
\[ \calC(z,x)\to\calD(\iota(z),\iota(x)) \]
is injective, and therefore \(g = g'\).
\end{proof}
\begin{lemma}\label{lemma:reflective-subcategory-essential-image}
	Let \(\iota\!:\calC\inj\calD\) be a reflective fully faithful functor with reflector \(\pi\!:\calD\to\calC\). An object \(y\in\calD\) is in the essential image
	of \(\iota\) if and only if \(\eta_y\!:y\to \iota\pi(y)\) is an isomorphism.
\end{lemma}
\begin{proof}
By universal property of the unit, if \(x\in\calC\) and we have a morphism \(f\!:y\to\iota(x)\), then we have a diagram
\begin{diagram*}
	y\ar[r,"\eta_y"]\ar[dr,"f"'] & \iota\pi(y)\ar[d,dashed,"g'"] \\
	& \iota(x).
\end{diagram*}
Since \(\iota\) is fully faithful, there is some unique \(g\!:\pi(y)\to x\) such that \(\iota(g) = g'\). Applying \(\pi\) to the above diagram, we have
\begin{diagram*}
	\pi(y)\ar[r,"\pi(\eta_y)"',"\sim"]\ar[dr,bend right,"\pi(f)"'] & \pi\iota\pi(y)\ar[d,"\pi(g')"]\ar[r,"\sim","\varepsilon_{\pi(y)}"'] & \pi(y)\ar[d,"g"] \\
	& \pi\iota(x)\ar[r,"\sim","\varepsilon_x"'] & x
\end{diagram*}
and from this it is clear that
\[ \pi(f)\text{ iso.} \iff \pi(g')\text{ iso.}\iff g\text{ iso.} \iff g'\text{ iso.} \iff f\text{ iso.} \]
Now, if \(\eta_y\) is an isomorphism, there is nothing to do. Conversely, if \(f\) is an isomorphism, then by the above, \(g'\) is also an isomorphism, so \(\eta_y\) is an isomorphism.
\end{proof}
\begin{proposition}\label{proposition:reflective-subcategories-are-monadic}
	The inclusion \(\iota\!:\calC'\inj\calC\) of a reflective subcategory is monadic. That is, leting \(L\!:\calC\to\calC\) be the induced monad, there is a canonical equivalence of categories
	\(\calC'\simeq\calC^L\) given by the functor \(K := F^L\circ\iota\).
\end{proposition}
\begin{proof}
An \(L\)-algebra in \(\calC\) consists of an object \(x\in\calC\) together with a map \(a\!:Lx\to x\). By definition, \(a\) is a retract of the unit \(\eta_x\!:x\to Lx\), i.e.\
\(a\circ\eta_x = \id_x\). We first show that in this case, we also have \(\eta_x\circ a = \id_{Lx}\). Note that \(\mu\) is a natural isomorphism since it is built from the
counit of the adjunction (involving a fully faithful right adjoint) inducing \(L\). Thus, \(L\eta = \eta L\), since both are right inverses to \(\mu\) by definition (the unitality condition
for a monad). Now, by naturality,
\begin{diagram*}
	Lx\ar[d,"a"']\ar[r,"\eta_{Lx}"] & L^2x\ar[d,"La"] \\
	x\ar[r,"\eta_x"] & Lx
\end{diagram*}
so
\[ \eta_x\circ a = La\circ\eta_{Lx} = La\circ L\eta_x = L(a\circ\eta_x) = \id_{Lx} \]
as desired. Therefore, if \(x\in\calC\) admits an \(L\)-algebra structure, the unit component \(\eta_x\) necessarily must be invertible. However,
this is also a sufficient condition: the above naturality diagram shows that if \(\eta_x\) is invertible, then \(\eta_x^{-1}\!:Lx\to x\) gives \(x\) the structure
of an \(L\)-algebra.

Now, an object \(x\) is in the essential image of \(\calC'\inj\calC\) if and only if \(\eta_x\) is an isomorphism. In particular, \(K\) is essentially surjective.
That \(K\) is fully faithful now follows just by the naturality of \(\eta\).
\end{proof}

\begin{theorem}\label{thm:limits-and-colimits-in-reflective-subcategories}
	Let \(P\!:\calC\inj\calD\) be reflective with reflection \(Q\!:\calD\to\calC\), and suppose \(D\!:I\to\calC\) is a diagram.
	\begin{enumerate}[label=(\arabic*)]
	\item If the colimit of \(P\circ D\) exists, then the colimit of \(D\) exists and is given by \(Q\injlim(P\circ D)\).
	\item If the limit of \(P\circ D\) exists, then the limit of \(D\) exists and is given by \(Q\projlim(P\circ D)\).
	\end{enumerate}
\end{theorem}
\begin{proof}
(1) Since \(Q\) is a left adjoint, it preserves colimits. In particular, we have isomorphisms
\[  \calC(Q\injlim(P\circ D),x)\cong \calC(\injlim(P\circ D),Px) \cong \projlim\calC(P\circ D,Px) \cong \projlim\calC(D,x) \]
which are functorial in \(x\in\calC\).

(2) By Proposition \ref{proposition:reflective-subcategories-are-monadic}, the inclusion \(\calC\inj\calD\) is a monadic functor. Thus, by Theorem \ref{thm:monadic-functors-create-limits},
it creates all limits which \(\calD\) admits. In particular, if \(\projlim(P\circ D)\) exists, then \(\projlim{D}\) exists, and so we can use that \(P\) is a right adjoint to see that
\[ Q\projlim(P\circ D) \cong QP\projlim{D} \cong\projlim{D} \]
as desired.
\end{proof}

Reflective subcategories are interesting to us because the above property, with a small modification, allows us to deduce that certain subcategories of Abelian categories are
Abelian, with predictable limits and colimits. This is one of the standard ``abstract'' ways to show that e.g.\ the category of sheaves of Abelian groups \(\Sh(X,\Ab)\) on a space \(X\) is an Abelian category,
or more generally, that \(\Mod_{\Ox_X}\) is an Abelian category for a ringed space \((X,\Ox_X)\).

\begin{definition}
	Let \(\iota\!:\calD\inj\calC\) be a reflective subcategory. We say it is \emph{Giraud} if the left adjoint preserves finite limits.
\end{definition}
\begin{theorem}\label{thm:giraud-subcategory-of-abelian-category-is-abelian}
	Let \(\iota\!:\calB\inj\calA\) be a Giraud subcategory of an Abelian category \(\calA\). Then \(\calB\) is Abelian.
\end{theorem}
\begin{proof}
The pre-additive structure on \(\calA\) induces one on \(\calA\). Since \(\calA\) is Abelian, we see that \(\calB\) is additive and admits all kernels and cokernels.

Let \(\pi\!:\calA\to\calB\) be the reflector, and let \(f\!:x\to y\) be a morphism in \(\calB\). We need to check that
\[ \coimg{f}\iso\img{f}. \]
However, for this, note that we have canonical isomorphisms
\begin{align*}
	\coimg{f} &= \coker(\ker{f}\inj x)\\
	&\cong \pi\coker(\ker\iota(f)\inj\iota(x))\\
	&= \pi\coimg\iota(f) \\
	&\cong \pi\img\iota(f) \\
	&= \pi\ker(\iota(y)\sur \coker\iota(f)) \\
	&\cong \ker(\pi\iota(y)\sur \pi\coker\iota(f)) \\
	&\cong \ker(y\sur \coker{f}) = \img{f}
\end{align*}
as desired.
\end{proof}

\subsection{Generators \& resolutions in Abelian categories}
\begin{definition}
	Let \(\calC\) be a category. A \emph{family of generators} for \(\calC\) is a set of objects \(\calU\subseteq\Ob(\calC)\) which can \emph{separate morphisms} in the sense that
	they satisfy the following condition:
	\begin{itemize}[label=\(\star\)]
	\item Let \(f,g\!:x\to y\) be morphisms in \(\calC\). If for every morphism \(h\!:u\to x\), \(u\in\calU\), we have \(f\circ h = g\circ h\) then \(f = g\).
	\end{itemize}
	In other words, if \(f\not= g\) then there is some morphism \(h_0\!:u\to x\) with \(u\in\calU\) such that \(f\circ h_0\not= g\circ h_0\).
\end{definition}

The following proposition is the motivation for the terminology.

\begin{proposition}\label{prop:generators-equivalent-conditions}
	Let \(\calC\) be a locally small category which admits small coproducts, and let \(\calU\) be a small set of objects in \(\calC\). Then the following are equivalent.
	\begin{enumerate}[label=(\arabic*)]
	\item The set \(\calU\) is a generating family.
	\item For all \(x\in\calC\), the canonical morphism
	\[ \coprod_{u\in\calU}\coprod_{u\to x}u \to x \]
	is an epimorphism.
	\item For all \(x\in\calC\), there is some family of sets \(\{I_u\}_{u\in\calU}\) and an epimorphism
	\[ \coprod_{u\in\calU}\coprod_{i\in I_u}u\sur x. \]
	\end{enumerate}
\end{proposition}
\begin{proof}
(1) \(\Rightarrow\) (2). Just observe that having a commutative diagram
\begin{diagram*}
	\displaystyle \coprod_{u\in\calU}\coprod_{u\to x} u \ar[r] & x\ar[r,shift left,"f"]\ar[r,shift right,"g"'] & y
\end{diagram*}
is the same as saying that for all \(u\in\calU\) and \(h\in\calC(u,x)\) we have \(f\circ h = g\circ h\).

(2) \(\Rightarrow\) (3). This is clear.

(3) \(\Rightarrow\) (1). Suppose we have \(f,g\!:x\to y\) such that \(f\circ r = g\circ r\) for all \(u\in\calU\), \(r\!:u\to x\). Let \(\iota_{u,i}\) be the canonical inclusion of \(u\) into the coproduct,
let \(h\) be the epimorphism, and set \(h_{u,i} := h\circ\iota_{u,i} \), so we in particular have a diagram
\begin{diagram*}
	u\ar[d,"\iota_{i,u}"']\ar[dr,bend left,"h_{i,u}"] & \\
	\coprod_{u\in\calU}\coprod_{i\in I_u}u \ar[r,two heads,"h"] & x\ar[r,shift left,"f"]\ar[r,shift right,"g"'] & y
\end{diagram*}
which commutes. We see that \(f\circ h = g\circ h\), so that \(f = g\) since \(h\) is an epimorphism.
\end{proof}
\begin{definition}\label{definition:generator}
	We say that an object \(u\in\calC\) is a \emph{generator} for \(\calC\) if \(\{u\}\) is a family of generators.
\end{definition}
\begin{example}
	The one-point set \(\{*\}\) is a generator for the category \(\Set\) of (small) sets.
\end{example}
\begin{example}
	Let \(\Bbbk\) be a field. Then \(\Bbbk\in\Vect_\Bbbk\) is a generator. In particular, if \(V\) is a vector space over \(\Bbbk\), then a linear
	map \(\Bbbk\to V\) is just a choice of a vector \(v\in V\), and we clearly have an epimorphism
	\[ \coprod_{v\in V}\Bbbk\cdot v \sur V \]
	so that the above proposition implies \(\Bbbk\) is a generator. The same statement holds more generally when \(\Bbbk\) is replaced by a ring \(R\)
	and \(\Vect_\Bbbk\) by \(\Mod_R\).
\end{example}

Now, generators in the context of an Abelian category let us create \emph{resolutions} of objects. For simplicity, we only handle the case
with one generator, but of course this restriction is artificial. These resolutions can be useful when the generator is nice.
\begin{proposition}
	Consider a locally small Abelian category \(\calA\) admitting small coproducts, and suppose \(u\in\calA\) is a generator. Let \(x\in\calA\). Then there is an exact sequence
	of the form
	\[ \cdots \to \coprod_{j \in J_1}u \to \coprod_{j\in J_0}u \sur x \to 0 \]
	in \(\calA\).
\end{proposition}
\begin{proof}
The idea for this is simple, and essentially follows by induction. First, we use Proposition \ref{prop:generators-equivalent-conditions} to produce an epimorphism and form an
exact sequence
\[ 0 \to \ker{\pi_0} \inj \coprod_{j\in J_0}u \overset{f_0}\sur x \to 0. \]
Now, we pick an epimorphism onto \(\ker{f_0}\), and take its kernel to get an exact sequence
\[ 0 \to \ker{f_1} \inj \coprod_{j\in J_1} u \overset{f_1}\to \coprod_{j\in J_0}u \overset{f_0}\sur x \to 0. \]
Continuing like this indefinitely, one obtains the result.
\end{proof}


\subsection{Appendix: Grothendieck categories}
Some Abelian categories are nicer than others, and the nicest of them are typically the module categories \(\Mod_R\) over a ring \(R\). A general Abelian category can be used
to phrase many results of homological algebra, but it turns out that they are usually not enough in applications. For example, in homological algebra one often needs to take a
resolution of one's objects in order to calculate, but no such thing is guaranteed to exist in an arbitrary Abelian category. Grothendieck isolated a few properties
that can be used to solve this.

\begin{definition}
	Let \(\calA\) be an Abelian category. We say \(\calA\) is a \emph{Grothendieck category} if it has a generator, and satisfies the following exactness condition:
	\begin{itemize}[label=(Ab5)]
	\item \(\calA\) admits all small colimits, and for any filtered category \(I\) the colimit functor
	\[ \Fun(I,\calA)\xrightarrow{\injlim(-)}\calA \]
	preserves finite limits, i.e.\ is exact.
	\end{itemize}
\end{definition}
\begin{remark}
	Concretely, if we are given a filtered category \(I\) and an \(I\)-diagram of exact sequences
	\[ \{ 0 \to x_i \to y_i \to z_i \to 0 \}_{i\in I} \]
	then the sequence
	\[ 0 \to \injlim_{i\in I}x_i \to \injlim_{i\in I}y_i \to \injlim_{i\in I}z_i \to 0 \]
	is exact.
\end{remark}
\begin{remark}
	A category \(I\) is filtered if any finite subcategory has a join. That is, if for any finite category \(J\) with a functor \(J\to I\), there is an extension
	to a functor \(J^\triangleright \to I\). These are the analogues of \emph{directed posets,} and a poset is directed if and only if it is filtered as a category.
\end{remark}

\begin{proposition}
	Let \(\iota\!:\calB\inj\calA\) be a Giraud subcategory of a Grothendieck category. Then \(\calB\) is a Grothendieck category.
\end{proposition}
\begin{proof}
We have seen in Theorem \ref{thm:giraud-subcategory-of-abelian-category-is-abelian} that \(\calB\) is Abelian. It remains to check that it has a generator, and that
filtered colimits are exact, i.e.\ (Ab5). Denote by \(\pi\!:\calA\to\calB\) the left adjoint of \(\iota\), and let \(u\in\calA\) be a generator for \(\calA\). Then
\(\pi(u)\) is a generator for \(\calB\). In particular, by the dual of Lemma \ref{lemma:right-adjoints-preserve-monomorphisms}, the functor \(\pi\) preserves epimorphisms;
therefore, for \(x\in\calB\), we can produce epimorphisms
\[ \coprod_{i\in I}u \sur \iota(x) \text{ in }\calA\quad\leadsto\quad \coprod_{i\in I}\pi(u) \sur \pi\iota(x) \iso x\text{ in }\calB. \]
We now need to check (Ab5). Since colimits commute with colimits, the colimit functor is already right exact, so we need only check that it preserves monomorphisms.
So, let \(I\) be a filtered category, and consider two diagrams \(D,D'\!:I\to\calB\) with a monomorphism \(D\inj D'\) (this means that each morphism \(D(i)\to D'(i)\) is a monomorphism).
Since \(\calA\) is (Ab5), we know that
\[ \projlim(\iota\circ D) \inj \projlim(\iota\circ D'). \]
However, colimits in \(\calB\) are calculated by applying \(\pi\) to the above diagram, and since \(\calB\) is a Giraud subcategory, \(\pi\) preserves finite limits, hence kernels.
Therefore, the induced map
\[ \projlim{D} \cong \pi\projlim(\iota\circ D) \to \pi\projlim(\iota\circ D') \cong \projlim{D'} \]
is a monomorphism.
\end{proof}

The following is the reason to care about Grothendieck categories. We will not prove it for now.

\begin{theorem}
	Let \(\calA\) be a Grothendieck category. Then
	\begin{enumerate}[label=(\arabic*)]
	\item \(\calA\) is locally small,
	\item \(\calA\) admits all small limits, and
	\item \(\calA\) has enough injectives.
	\end{enumerate}
\end{theorem}

\subsection{Appendix: Abelian categories with a compact projective generator}\label{appendix:abelian-categories-with-a-compact-projective-generator}
Let \(\calA\) be an Abelian category. Recall that an object \(p\in\calA\) is \emph{projective} if \(\calA(p,-)\!:\calA\to\Ab\) is an exact functor.
We will prove a theorem which allows us to recognize module categories in terms of the existence of a \emph{compact projective generator.} We defined what
a generator is in Definition \ref{definition:generator}, so we will now say what a \emph{compact object} is in the context of an additive category.
\begin{construction}\label{construction:additive-category-compactness-map}
	Let \(\calC\) be an additive category, let \(x\in\calC\), and let \(\{x_i\}_{i\in I}\) be some collection of objects indexed by a small set \(I\).
	Whenever the coproduct exists, consider the maps
	\[ \iota_i\!: x_i \to \coprod_{i\in I}x_i. \]
	We construct a map
	\[ \coprod_{i\in I}\calC(x,x_i) \to \calC(x,\coprod_{i\in I}x_i) \]
	by letting it be given on components by the maps
	\[ \calC(x,x_i) \to \calC(x,\coprod_{i\in I}x_i),\quad f \mapsto \iota_i\circ f \]
	given by composition with the canonical inclusion \(\iota_i\).
\end{construction}
\begin{definition}
	Let \(\calC\) be an additive category. We say that an object \(x\in\calC\) is \emph{compact} if for any small set \(I\) and objects \(\{x_i\}_{i\in I}\), the map
	\[ \coprod_{i\in I}\calC(x,x_i) \to \calC(x,\coprod_{i\in I}x_i) \]
	from Construction \ref{construction:additive-category-compactness-map} is an isomorphism.
\end{definition}
\begin{remark}
	Another formulation of this is that for any small set \(I\) and objects \(\{x_i\}_{i\in I}\), any map from \(x\) to the coproduct will factor uniquely through a finite subset.
	That is, there is some finite \(J\subseteq I\) such that
	\begin{diagram*}[cramped, sep=small]
		x\ar[rr]\ar[dr,dashed,"\exists!"'] & & \coprod_{i\in I}x_i \\
		& \bigoplus_{j\in J}x_j\ar[ur] & 
	\end{diagram*}
	commutes. That is, any map to a coproduct is defined by a ``finite amount of data.''
\end{remark}
\begin{remark}
	This notion of compactness is unique to additive categories, and is not really the ``right'' notion outside this context. In an arbitrary category \(\calC\),
	one is instead interested in the analogous notion but with coproducts replaced by filtered colimits.
\end{remark}

\begin{theorem}\label{thm:abelian-cat-with-compact-projective-generator}
	Let \(\calA\) be a locally small Abelian category which admits small coproducts, and let \(p\in\calA\). Consider the functor
	\[ \calA(p,-)\!:\calA\to\Mod_{\End(p)} \]
	from \(\calA\) to the category of right modules over \(\End(p) := \calA(p,p)\).
	\begin{enumerate}[label=(\arabic*)]
	\item If \(p\) is a generator, then \(\calA(p,-)\) is faithful.
	\item If \(p\) is a compact projective generator, then \(\calA(p,-)\) is an equivalence.
	\end{enumerate}
\end{theorem}
\begin{proof}
We must investigate, for all \(x,y\in\calA\), the homomorphism
\[ \calA(x,y) \to \Mod_{\End(p)}(\calA(p,x),\calA(p,y)). \]

(1) To see that it is injective, consider a morphism \(f\!:x\to y\) such that \(f_*\!:\calA(p,x)\to\calA(p,y)\) is the zero morphism. We want
to use the Yoneda lemma to deduce that \(f=0\), so pick \(z\in\calA\) and, using the fact that \(p\) is a generator, pick an epimorphism \(\pi\!:\coprod_{i\in I}p\sur z\).
Then
\begin{diagram*}
	\calA(z,x)\ar[r,"f_*"]\ar[d,hook,"\pi^*"'] & \calA(z,y)\ar[d,hook,"\pi^*"'] \\
	\calA(\coprod_{i\in I}p,x)\ar[r,"f_*"]\ar[d,"\sim" labl] & \calA(\coprod_{i\in I}p,y)\ar[d,"\sim" labl] \\
	\prod_{i\in I}\calA(p,x)\ar[r,"(f_*)"] & \prod_{i\in I}\calA(p,y)
\end{diagram*}
commutes, and the bottom horizontal arrow is the zero map by assumption. We conclude that the natural transformation \(f_*\!:\calA(-,x)\to\calA(-,y)\) is zero,
so that \(f=0\) by the Yoneda lemma.

(2)(i) To have fullness, suppose we have an \(\End(p)\)-linear homomorphism
\[ \varphi\!:\calA(p,x)\to\calA(p,y). \]
Using the fact that \(p\) is a generator, we can build an exact sequence for \(x\) in terms of \(p\),
\[ 0\to \ker\pi \overset{\kappa}\inj \coprod_{i\in I}p \overset{\pi}\sur x \to 0. \]
Maps \(x\to y\) can be described in terms of this, through universal property:
\begin{diagram*}
	0 \ar[r] & \ker\pi \ar[r,hook,"\kappa"]\ar[dr,"0"'] & \coprod_{i\in I}p \ar[r,two heads,"\pi"]\ar[d] & x\ar[dl,dashed] \ar[r] & 0. \\
	& & y & &
\end{diagram*}
We build the solid vertical arrow. Let \(f_i\!:p\to x_i\) be the \(i\)th component of \(\pi\). That is, if \(\iota_i\!:p\to\coprod_{i\in I}p\) is the canonical inclusion, we set
\(f_i := \pi\circ\iota_i\). Using \(\varphi\), we get maps \(g_i := \varphi(f_i)\!:p\to y\) which define a map \(g\!:\coprod_{i\in I}p \to y\). We need to see that \(g\circ\kappa\) is zero
as in the above diagram. For this, since \(p\) is a generator, it suffices to check that for all \(a\!:p\to\ker\pi\), we have \(g\circ\kappa\circ a = 0\). Using the compactness of \(p\),
we factorize:
\begin{diagram*}
	& p\ar[d,"a"'] \ar[r,dashed,"h_{a}"] & \bigoplus_{j\in J_a}p\ar[d,hook]\ar[r,"\pi_j"] & p\ar[dl,"\iota_j"]\ar[d,"f_j"] & \\
	0 \ar[r] & \ker\pi \ar[r,hook,"\kappa"] & \coprod_{i\in I}p \ar[r,two heads,"\pi"]\ar[d,"g"] & x \ar[r] & 0. \\
	& & y & &
\end{diagram*}
and setting \(h_{a,j} := \pi_j\circ h_a\), we see that it suffices to check that for all \(j\in J_a\subseteq I\), \(g\circ\iota_j\circ h_{a,j} = 0\).
To check this, we use \(\End(p)\)-linearity to see that
\[ g\circ\iota_j\circ h_{a,j} = g_j \circ h_{a,j} = \varphi(f_j\circ h_{a,j}) = 0 \]
as desired. We conclude that \(\calA(p,-)\) is fully faithful, since the above induces a map \(f\!:x\to y\) and one easily checks that it recovers \(\varphi\) using the projectivity
and compactness of \(p\).

(2)(ii) We show that \(\calA(p,-)\) is essentially surjective. Let \(M\in\Mod_{\End(p)}\), and choose a free resolution
\[ \cdots \to \coprod_{j\in J_1}\End(p) \to \coprod_{j\in J_0}\End(p) \sur M \to 0. \]
Since \(p\) is compact, we have natural isomorphisms
\begin{diagram*}
	\cdots \ar[r] & \coprod_{j\in J_1}\End(p) \ar[r] \ar[d,"\sim" labl] & \coprod_{j\in J_0}\End(p) \ar[r,two heads]\ar[d,"\sim" labl] & M \ar[r] \ar[d,equal] & 0 \ar[d,equal] \\
	\cdots \ar[r] & \calA(p,\coprod_{j\in J_1}p) \ar[r] & \calA(p,\coprod_{j\in J_0}p)\ar[r,two heads] & M\ar[r] & 0
\end{diagram*}
making the above diagram commute, and by the fully faithfulness of \(\calA(p,-)\) we obtain an exact sequence
\[ \cdots \to \coprod_{j\in J_2} p \to \coprod_{j\in J_1}p\to \coprod_{j\in J_0}p. \]
Taking the cokernel of the right-most map, we have an exact sequence
\[ \cdots \to \coprod_{j\in J_1}p\to \coprod_{j\in J_0}p \sur x \to 0 \]
and now, since \(\calA(p,-)\) is exact since \(p\) is projective, the uniqueness of cokernels means \(\calA(p,x) \cong M\). We conclude that
\(\calA(p,-)\) is both fully faithful and essentially surjective, hence an equivalence.
\end{proof}

\begin{exercise}
	In the proof of Theorem \ref{thm:abelian-cat-with-compact-projective-generator}, we made two unjustified statements. They are the following:
	\begin{enumerate}[label=(\arabic*)]
	\item In the proof that \(\calA(p,-)\) is full, the constructed function \(f\) induces a morphism \(f_*\!:\calA(p,x)\to\calA(p,y)\), and this agrees with the original map \(\varphi\).
	\item In the proof that \(\calA(p,-)\) is essentially surjective, we use that any faithful functor \(F\!:\calA\to\calB\) between Abelian categories reflects exact sequences.
	\end{enumerate}
	Prove these statements.
\end{exercise}

\begin{remark}
	There is a version of Theorem \ref{thm:abelian-cat-with-compact-projective-generator} for stable \(\infty\)-categories which is originally due to Schwede \& Shipley in \cite{SCHWEDE2003103} in the setting of stable model categories,
	later generalized to the aforementioned setting by Lurie in \cite[Thm.\ 7.1.2.1]{lurie-ha}.
\end{remark}

\begin{remark}
	There is \emph{almost} a version of Theorem \ref{thm:abelian-cat-with-compact-projective-generator} for triangulated categories. Hoshino, Kato, \& Miyachi in \cite{hkm02} prove that a triangulated category \(\calT\) with a compact generator
	\(s\in\calT\) satisfying \(\forall i>0,\,\calT(s,s[i]) = 0\), admits a \emph{t-structure} whose heart is equivalent to \(\Mod_{\End(s)}\). In general, however, this kind of thing
	is typically the best you can do with just triangulated categories. We will explain all this terminology in future lectures.
\end{remark}


%!TEX root = ../lectures.tex

\section{Localizations of categories}\label{section:localization-of-categories}
Localization is the procedure by which one formally inverts morphisms in a category. It's a ``categorification'' of the analogous process for rings, and various deeply interesting
categories in homological algebra and homotopy theory are formed in this way. We mainly follow \cite{krause-homological-theory-of-representations} and \cite{kashiwara-schapira-book}.

\subsection{Weak \& strict localizations}
\begin{definition}
	Let \(\calC\) be a category, and let \(W\subseteq\Mor\calC\) be a set of morphisms in \(\calC\). A \emph{weak localization} of \(\calC\) with respect to \(W\)
	is a category \(\calC_W\) together with a functor \(Q\!:\calC\to\calC_W\) satisfying the following properties:
	\begin{enumerate}[label=(\arabic*)]
	\item For any \(f\in\calS\), the morphism \(Q(f)\) in \(\calC_W\) is invertible.
	\item For any category \(\calE\), composition with \(Q\) induces an equivalence
	\[ (\circ Q)\!:\Fun(\calC_W,\calE) \iso \Fun_W(\calC,\calE) \]
	where \(\Fun_W(\calC,\calE)\) denotes the full subcategory of \(\Fun(\calC,\calE)\) spanned by those functors sending morphisms in \(\calS\) to invertible morphisms in \(\calE\).
	\end{enumerate}
	We say that the pair \((\calC_W,Q)\) is a \emph{strict localization} if the equivalence in (2) is an isomorphism. A weak localization is unique up to canonical equivalence, while
	a strict localization is unique up to canonical isomorphism.
\end{definition}
\begin{remark}\label{remark:1-categorical-property-of-localization}
	The universal property of the localization means that given any functor \(F\!:\calC\to\calE\) sending morphisms in \(W\) to isomorphisms in \(\calE\), there is an essentially unique factorization
	\begin{diagram*}
		\calC\ar[d,"Q"']\ar[dr,"F",""{name=A,left}] & \\
		\calC_W\ar[r,dashed,"F'"'] & \calE\ar[from=l,to=A,symbol=\cong]
	\end{diagram*}
	through \(Q\!:\calC\to\calC_W\) up to natural isomorphism. In the strict case, the difference comes down to demanding that \(F'\) is genuinely unique and that the diagram commutes on the nose, i.e.\ that \(F'\circ Q = F\).
	Clearly, any strict localization is also a weak localization.
\end{remark}
\begin{lemma}\label{lemma:1-strict-localization-is-2-strict-localization}
	Let \(\calC\) be a category, and let \(W\) be some collection of morphisms in \(\calC\). Then a pair \((\calC_W,Q\!:\calC\to\calC_W)\) is a strict localization of \(\calC\) with respect to \(W\) if and only if
	it satisfies the following a priori weaker property described above: for all categories \(\calE\) with a functor \(F\!:\calC\to\calE\) inverting morphisms in \(W\), there is a unique functor \(F'\!:\calC_W\to\calE\)
	such that \(F = F'\circ Q\).
\end{lemma}
\begin{proof}
Condition (2) of being a strict localization, the isomorphism \((\circ Q)\!:\Fun(\calC_W,\calE)\iso\Fun_W(\calC,\calE)\), clearly implies the ``weaker'' condition, since it is just a specialization
to the objects of the aforementioned categories. For the converse, we are given that \((\circ Q)\) is bijective on objects, and need to show that it is fully faithful. For this, suppose we have
\(F,G\in\Fun_W(\calC,\calE)\) and a natural transformation \(\sigma\!:F\To G\). This corresponds uniquely to a functor \(\sigma'\!:\calC\to\Fun(\2,\calE)\) under the isomorphisms of categories
\[ \Fun(\2,\Fun(\calC,\calE)) \cong \Fun(\2\times\calC,\calE)\cong\Fun(\calC,\Fun(\2,\calE)). \]
Explicitly, \(\sigma'\) is given by the assignment \(\calC\ni x \mapsto (Fx\overset{\sigma_x}\to Gx) \) on objects, and on morphisms \(f\!:x\to y\) is sent to the square
\begin{diagram*}
	Fx\ar[r,"\sigma_x"]\ar[d,"F(f)"'] & Gx\ar[d,"G(f)"] \\
	Fy\ar[r,"\sigma_y"] & Gy
\end{diagram*}
and in particular, it is clear that for all \(f\in W\), the morphism \(\sigma'(f)\) in \(\Fun(\2,\calE)\) is invertible. Therefore, we obtain a unique functor \(\tau'\!:\calC_W\to\Fun(\2,\calE)\)
such that \(\sigma' = \tau'\circ Q\). Now \(\tau'\) corresponds uniquely to a natural transformation \(\tau\!:FQ\To GQ\) in \(\Fun(\calC_W,\calE)\) for which \(\tau = Q\sigma\).
\end{proof}
\begin{remark}
	The above argument actually has almost nothing to do with the specific situation at hand. Instead, it is really a statement about particular 2-categories.
\end{remark}
\begin{exercise}
	Let \(\calC\) be a category, and \(W\subseteq\Mor(\calC)\). Let \((\calC_W,Q)\) be a strict localization of \(\calC\) with respect to \(W\). Show that \(Q\) is bijective on objects. Hint: consider the category
	\(\calC'\) whose objects are those of \(\calC\), and where all objects are isomorphic by a unique isomorphism.
\end{exercise}
\begin{exercise}
	Prove the rest of Remark \ref{remark:1-categorical-property-of-localization}.
\end{exercise}
\begin{notation}
	The localization of a category \(\calC\) with respect to \(W\) is variously denoted \(\calC_W\), \(\calC[W^{-1}]\), and \(W^{-1}\calC\).
\end{notation}

Since strict localizations are automatically also weak localizations, it suffices to construct strict ones in all situations of interest in order to know that all weak localizations
also exist. It turns out that this is possible, through a construction originally due to Gabriel \& Zisman. The idea is that we want morphisms in \(\calC[W^{-1}]\) to represent
fractions
\[ f_1s_1^{-1}f_2s_2^{-1}\cdots f_ns_n^{-1}, \]
where the \(f_i\) are just ``arbitrary'' morphisms and \(s_i\in W\). If there was some commutation relationship, this could be collapsed down to just \(fs^{-1}\), which we'll look
at later, but in general there is no way around having arbitrarily long chains. This same issue is present when localizing noncommutative rings.
\begin{construction}\label{construction:gabriel-zisman-localization}
	Let \(\calC\) be a category, and let \(W\subseteq\Mor(\calC)\). We construct a category \(\calC[W^{-1}]\) and functor \(Q\!:\calC\to\calC[W^{-1}]\) as suggested in \cite{krause-homological-theory-of-representations}.
	Consider the quiver \(\sfC\) with vertices \(\Ob(\calC)\), and edges given by the disjoint union of \(\Mor(\calC)\) where \(W^- := \{ s^{-1}\!: y\to x \mid W\ni s\!:x\to y \}\) consists
	of edges labeled \(s^{-1}\) for all \(s\in W\), and which are reversed in direction compared to their corresponding element of \(W\). Let \(\calP\) be the path category of \(\sfC\), and
	denote the composition (i.e.\ concatenation) in \(\calP\) by \(\circ_\calP\). We let \(\calC[W^{-1}]\) be the quotient of \(\calP\) given by the equivalence relation on the morphisms generated by
	the following relations:
	\begin{enumerate}[label=(\arabic*)]
	\item \(g\circ_\calP f = g\circ f\) for all \(f\!:x\to y\), \(g\!:y\to z\) in \(\calC\).
	\item Let \(\id^\calP_x\) denote the identity of \(x\) in \(\calP\). Then we demand \(\id^\calP_x = \id_x\).
	\item For all \(W\ni s\!:x\to y\), we have \(s^{-1}\circ_\calP s = \id_x\), \(s\circ_\calP s^{-1} = \id_y\).
	\end{enumerate}
	The functor \(Q\!:\calC\to\calC[W^{-1}]\) is the identity on objects, and on morphisms is the composition
	\[ \Mor(\calC)\inj \Mor(\calC)\amalg W^- \to \Mor(\calP) \sur \Mor(\calC[W^{-1}]). \]
	Explicitly, this means the following: the objects of \(\calC[W^{-1}]\) are just the objects of \(\calC\), while a morphism from \(x\) to \(y\) is an equivalence class of a zigzagging sequence of morphisms, e.g.
	\[ x \longto \bullet \overset{\in W}\longot\bullet\longto\cdots\overset{\in W}\longot\bullet\longto y. \]
	Two such sequences are equivalent if they can be related
	\begin{enumerate}[label=(\arabic*')]
	\item by composing adjacent arrows going the same direction,
	\item by removing identities,
	\item by removing instances of \(\bullet \overset{s}\longto \bullet \overset{s}\longot \bullet \) and \(\bullet \overset{s}\longot \bullet \overset{s}\longto \bullet \) whenever \(s\in W\), or
	\item through the existence of a commutative diagram
	\begin{diagram*}[cramped, row sep=tiny]
		 & \bullet\ar[dd] & \bullet\ar[l,"\in W"']\ar[r]\ar[dd] & \cdots & \bullet\ar[l,"\in W"']\ar[dr]\ar[dd] & \\
		x \ar[ur]\ar[dr] & & & & & y. \\
		& \bullet & \bullet\ar[l,"\in W"']\ar[r] & \cdots & \bullet\ar[l,"\in W"']\ar[ur] &
	\end{diagram*}
	\end{enumerate}
	One regards the ``empty zigzag'' as the identity morphism. The functor \(Q\) just sends a morphism \(f\!:x\to y\) to the ``trivial zigzag'' \(f\!:x\to y\).
\end{construction}

Note that this category can fail to be locally small: one can imagine non-equivalent morphisms from \(x\) to \(y\) in \(\calC[W^{-1}]\) having
any object of \(\calC\) as an intermediate at some point, so that \(\calC[W^{-1}](x,y)\) may have as many morphisms as \(\calC\) does objects.

\begin{theorem}
	Let \(\calC\) be a category, and \(W\subseteq\Mor(\calC)\). Then the pair \((\calC[W^{-1}],Q)\) from Construction \ref{construction:gabriel-zisman-localization} is a strict localization
	of \(\calC\) with respect to \(W\).
\end{theorem}
\begin{proof}
Observe that for all \(s\in W\), the morphism \(Q(s)\) is invertible. Indeed, we have
\[ (\bullet\overset{s}\longto\bullet)\circ(\bullet\overset{s}\longot\bullet) = \bullet\overset{s}\longot\bullet\overset{s}\longto\bullet = \bullet\overset{\id}\longto\bullet. \]
with the other composition being the same. Let \(\calE\) be a category, and suppose we have a functor \(F\!:\calC\to\calE\) which inverts elements of \(W\). We need
to produce a unique functor \(F'\!:\calC[W^{-1}]\to\calE\) for which \(F = F'Q\). For each object \(x\in\Ob(\calC[W^{-1}]) = \Ob(\calC)\), we are thus forced
to set \(F'x := Fx\) since \(Q\) is the identity on objects. For morphisms, observe that every zigzag can be broken up as
\[ \bullet \longto \bullet \longot \bullet \cdots \bullet \longot \bullet \longto \bullet = (\bullet\to \bullet)\circ(\bullet\ot\bullet)\circ\cdots\circ(\bullet\ot\bullet)\circ(\bullet\to\bullet). \]
In particular, the value of \(F'\) at such a zigzag is determined by what it does to morphisms \(\bullet\to\bullet\) and \(\bullet\ot\bullet\). The former are exactly the morphisms
in the image of \(Q\), hence we have \(F'(\bullet\to\bullet) = F(\bullet\to\bullet)\). For a morphism \((s\!:x\to y)\in W\), as remarked, we have that
\[ Q(s)^{-1} = (y\overset{s}\longot x) \]
and every zigzag of the form \(\bullet\ot\bullet\) arises in this way. In particular, forcibly we must have \(F'(Q(s)^{-1}) = (F'Q(s))^{-1}\). This completely determines the functor \(F'\)
in a unique way; in particular, the value is given by
\[ F'(\bullet \overset{f_1}\longto \bullet \overset{s_1}\longot  \cdots  \overset{s_n}\longot \bullet \overset{f_{n+1}}\longto \bullet)
 = F(\bullet) \overset{Ff_1}\longto F(\bullet) \overset{F(s_1)^{-1}}\longto  \cdots  \overset{F(s_n)^{-1}}\longto F(\bullet) \overset{Ff_{n+1}}\longto F(\bullet). \]
By Lemma \ref{lemma:1-strict-localization-is-2-strict-localization}, we are done, recognizing that the above is a well-defined functor since the value is invariant
under the operations (1')--(3') described in Construction \ref{construction:gabriel-zisman-localization}.
\end{proof}

\subsection{Calculus of fractions}
The construction of an arbitrary localization \(\calC[W^{-1}]\) can be quite hard to control the properties of. As noted, even when \(\calC\) is locally small, the result may not
end up being so. It is therefore desirable to find conditions on \(W\) where the localization is more well-behaved. There are several ways of doing this; we
will consider the \emph{calculus of fractions} approach, similar to the work of Ore on the localization of noncommutative rings: one can get control over localizations by imposing
a ``weak'' commutation rule, collapsing an arbitrary zigzag
\[ \bullet \longto \bullet \longot  \cdots  \longot \bullet \longto \bullet \]
into one of the form
\[ \bullet \longto \bullet \longot \bullet\quad\text{or}\quad\bullet\longot\bullet\longto\bullet. \]
\begin{definition}
	Let \(\calC\) be a category, and let \(W\) be a collection of morphisms closed under composition and containing all identity morphisms. We say that \(\calC\) is a \emph{right multiplicative system} (or
	has a \emph{left calculus of fractions}) if it satisfies the following two conditions:
	\begin{enumerate}[label=(M\arabic*)]
	\item For any diagram of solid arrows
	\begin{diagram*}
		x\ar[r,"f"]\ar[d,"s\in W"'] & y \ar[d,dashed,"t\in W"] \\
		x' \ar[r,dashed,"g"] & y'
	\end{diagram*}
	with \(s\in W\), there are dashed arrows as displayed.
	\item For any solid diagram
	\begin{diagram*}
		x'\ar[r,"s"] & x\ar[r,shift left,"f"]\ar[r,shift right,"g"'] & y \ar[r,dashed,"t"] & y'
	\end{diagram*}
	where \(s\in W\), there is a dashed arrow \(t\in W\) as displayed.
	\end{enumerate}
	We say that \(W\) is a \emph{left multiplicative system} (or has a \emph{right calculus of fractions}) if \(W^\op\) is a right multiplicative system in \(\calC^\op\).
\end{definition}
\begin{remark}
	The terminology is very scattered, and most sources tend to pick arbitrarily what things they call \emph{left} and \emph{right.}
\end{remark}
\begin{remark}
	The way to interpret (M1) is that it allows us to commute a fraction \(fs^{-1}\) into \(t^{-1}g\).
\end{remark}

To any collection of arrows \(W\) in a category \(\calC\), we can associate some categories and functors.
\begin{definition}
	Let \(\calC\) be a category, and let \(W\) be a collection of morphisms in \(\calC\). We define the category
	\begin{align*}
		W_{/x} &:= \{ s\!:x'\to x\mid s\in W \},\\
		W_{/x}( (s\!:x'\to x), (t\!:x''\to x) ) &:= \left\{ f\in\calC(x',x'')\,\left\vert %
																		\begin{tikzcd}[column sep=small, ampersand replacement=\&]
																		x'\ar[rr,"f"] \& \& x''\\
																		 \& x\ar[from=ul,"s"']\ar[from=ur,"t"] \&
																		\end{tikzcd} \text{ commutes}\right. \right\},\\
		\pi_{/x}\!:W_{/x}&\to \calC,\quad ( s\!:x'\to x ) \mapsto x'.\\
	\end{align*}
	Note that this is a full subcategory of the slice \(\calC/x\).

	Dually, we define the full subcategory \(W^{x/}\) of \(x/\calC\) by
	\begin{align*}
		W^{x/} &:= \{ s\!:x\to x'\mid s\in W \},\\
		W^{x/}( (s\!:x\to x'), (t\!:x\to x'') ) &:= \left\{ f\in\calC(x',x'')\,\left\vert %
																		\begin{tikzcd}[column sep=small, ampersand replacement=\&]
																		 \& x\ar[dl,"s"']\ar[dr,"t"] \& \\
																		x'\ar[rr,"f"] \& \& x''
																		\end{tikzcd} \text{ commutes}\right. \right\},\\
		\pi^{x/}\!:W^{x/}&\to\calC,\quad ( s\!:x\to x' ) \mapsto x'.
	\end{align*}
\end{definition}

\begin{proposition}
	Let \(\calC\) be a category, let \(W\) be a collection of morphisms in \(\calC\), and let \(x\in \calC\).
	\begin{enumerate}[label=(\arabic*)]
	\item Suppose \(W\) is a right multiplicative system. Then \(W^{x/}\) is filtered.
	\item Suppose \(W\) is a left multiplicative system. Then \(W_{/x}\) is cofiltered.
	\end{enumerate}
\end{proposition}
\begin{proof}
Statement (2) is formally dual to (1), so we prove only (1). Since \(W\) contains the identities, \(W^{x/}\) is non-empty. To show that \(W^{x/}\) is filtered,
it suffices to prove the following:
\begin{enumerate}[label=(\alph*)]
\item Any two objects in \(W^{x/}\) map to a common object: let \(s\!:x\to x'\) and \(t\!:x\to x''\) be objects of \(W^{x/}\). By (M1), we obtain a diagram
\begin{diagram*}
	x\ar[r,"s"]\ar[d,"t"] & x'\ar[d,dashed,"t'"] \\
	x''\ar[r,dashed,"s'"] & x'''
\end{diagram*}
where \(t'\in W\), so \(t'\circ s \in W\) by closure under composition, hence we may consider it as an element of \(W^{x/}\). By commutativity of the diagram, we have morphisms \(s\to t'\circ s\) and \(t\to t'\circ s\).
\item Any two parallel arrows in \(W^{x/}\) are equalized by some morphism: If we have parallel arrows \(f,g\!:s\to t\), we apply (M2) to obtain
\begin{diagram*}[cramped,column sep=small]
	& x\ar[dl,"s"]\ar[dr,"t"] & & \\
	x'\ar[rr,shift left,"f"]\ar[rr,shift right,"g"'] & & x''\ar[r,dashed,"t'"] & x'''
\end{diagram*}
where now \(t'\circ t \in W\). We obtain a morphism \(t'\!: t\to t'\circ t\) which equalizes \(f,g\).
\end{enumerate}
We conclude that \(W^{x/}\) is filtered.
\end{proof}

This category being filtered means we can drastically simplify the morphisms in the localization not only in their form but also in terms of their equivalence relation.
To prove this, we will basically do another construction of the localization exploiting this property, and deduce by universal property that the aforementioned statement is true.
Preliminarily, we make this definition:
\begin{definition}
	Let \(\calC\) be a category, and \(W\) be a right multiplicative system. We set
	\[ \calC_W^r(x,y) := \injlim\calC(x,\pi^{y/}) = \injlim_{(y\to y')\in W^{y/}}\calC(x,y'). \]
	That is, the colimit of the (filtered) diagram
	\[ W^{y/}\xrightarrow{\pi^{y/}} \calC \xrightarrow{\calC(x,-)} \Set. \]
	Dually, if \(W\) is a left multiplicative system, we set
	\[ \calC_W^l(x,y) := \injlim\calC(\pi_{/x},y) = \injlim_{(x'\to x)\in W_{/x}}(x',y). \]
	That is, the colimit of the (filtered) diagram
	\[ (W_{/x})^\op \xrightarrow{\pi_{/x}^\op}\calC^\op\xrightarrow{\calC(-,y)}\Set \]
	where we note that this is filtered since \(W_{/x}\) is cofiltered.
\end{definition}
\begin{remark}
	Note that we have a canonical map
	\[ \calC(x,y) \to \calC_W^r(x,y) \]
	and
	\[ \calC(x,y) \to \calC_W^l(x,y) \]
	since \(\id_x\in W^{x/}\) and \(W_{/x}\).
\end{remark}
\begin{remark}
	An element of \(\calC_W^r(x,y)\) consists of a choice of a morphism \(s\!:y\to y'\) and a morphism \(f\!:x\to y'\), i.e.\ a zig-zag
	\[ x \to y' \ot y. \]
	Since \(W^{y/}\) is filtered, we have an explicit description of the equivalence relation. Two such zig-zags are equivalent if and only if we have a commutative diagram
	\begin{diagram*}
		 & y'\ar[d,dashed] & \\
		x\ar[ur,"f"]\ar[dr,"f'"'] & y''' & y\ar[ul,"s"']\ar[dl,"s'"]\ar[l,dashed,"\in W"'] \\
		 & y''\ar[u,dashed] &
	\end{diagram*}
\end{remark}

The idea is to show that this defines a category.

\begin{theorem}\label{thm:right-multiplicative-system-localization-category}
	Let \(\calC\) be a category, and \(W\) be a right multiplicative system. Then we have a category \(\calC_W^r\) whose objects are the same as \(\calC\), and Hom-sets are given by
	\(\calC_W^r(-,-)\). Furthermore, the functor \(Q\!:\calC\to\calC_W^r\) induced by the canonical maps \(\calC(-,-)\to\calC_W^r(-,-)\) exhibits \(\calC_W^r\) as a strict
	localization of \(\calC\) by \(W\).
\end{theorem}

To prove this, we need a lemma which is how we actually produce the composition law.
\begin{lemma}\label{lemma:right-multiplicative-system-morphism-bijection}
	Let \(\calC\) be a category, and let \(W\) be a right multiplicative system. Suppose we have a morphism \(s\!:x\to x'\) in \(W\). Then composition with \(s\) induces an
	isomorphism
	\[ s^*\!: \calC_W^r(x',y)\iso\calC_W^r(x,y). \]
\end{lemma}
\begin{proof}
The map sends \(x'\to y'\ot y\) to \(x \to x' \to y' \ot y\), and that this is well-defined (i.e.\ respects the equivalence relation) is clear. We need to check that we get a bijection.
\begin{enumerate}[label=(\alph*)]
\item \(s^*\) is surjective: given a morphism \(x \to y' \ot y\), we apply (M1) to get
\begin{diagram*}
	& y'' & & \\
	x'\ar[ur,dashed] & & y'\ar[ul,dashed] & \\
	& x\ar[ul,"s"']\ar[ur] & & y\ar[ul]
\end{diagram*}
and this provides the desired preimage.
\item \(s^*\) is injective: Suppose we have
\[ x' \overset{f}\to y'\overset{t}\ot y,\quad x' \overset{f'}\to y''\overset{t'}\ot y \]
and that these are sent to the same thing by \(s^*\). This means that we have a diagram
\begin{diagram*}[column sep=large]
	 & y'\ar[d,"q"] & \\
	x\ar[ur,"f\circ s"]\ar[dr,"f'\circ s"'] & z & y\ar[ul,"t"']\ar[dl,"t'"]\ar[l,"\in W"'] \\
	 & y''\ar[u,"q'"'] &
\end{diagram*}
and so, applying (M2) we get a morphism \(r\!:z\to z'\) in \(W\) such that
\begin{diagram*}[column sep=large]
	 & y'\ar[d,"r\circ q"] & \\
	x'\ar[ur,"f"]\ar[dr,"f'"'] & z' & y\ar[ul,"t"']\ar[dl,"t'"]\ar[l,"\in W"'] \\
	 & y''\ar[u,"r\circ q'"'] &
\end{diagram*}
commutes, so that the two morphisms we started with are the same.
\end{enumerate}
\end{proof}

\begin{proof}[Proof of Theorem \ref{thm:right-multiplicative-system-localization-category}]
We split this into two parts.
\begin{enumerate}[label=(\alph*)]
\item \(\calC_W^r\) is a category: the composition
\[ \calC_W^r(y,z) \times \calC_W^r(x,y) \to \calC_W^r(x,z) \]
is given by
\begin{align*}
	\calC_W^r(y,z) \times \injlim_{\mathclap{(y\to y')\in W^{y/}}}\calC(x,y') &\cong \injlim_{y\to y'}\left(\calC_W^r(y,z) \times \calC(x,y')\right) \\
	\intertext{since \(W^{y/}\) is filtered,}
	&\cong \injlim_{y\to y'}\left(\calC_W^r(y',z) \times \calC(x,y')\right) \\
	\intertext{by the lemma,}
	&\cong \injlim_{y\to y'}\injlim_{z\to z'}\left(\calC(y',z') \times \calC(x,y')\right) \\
	&\to \injlim_{y\to y'}\injlim_{z\to z'}\calC(x,z') \cong \calC_W^r(x,z).
\end{align*}
It is clear that this composition law has identity morphisms, given simply by
\[ x\overset\id\to x \overset\id\ot x. \]
One needs to check that it's associative, but for this, the proof of Lemma \ref{lemma:right-multiplicative-system-morphism-bijection}
and the above tell us that composing three morphisms can be written in terms of a diagram
\begin{diagram*}[cramped, sep=small]
	& & & v & & & \\
	& & u\ar[ur] & & u'\ar[ul] & & \\
	& y'\ar[ur] & & z'\ar[ul]\ar[ur] & & w'\ar[ul] & \\
	x\ar[ur] & & y\ar[ul]\ar[ur] & & z\ar[ul]\ar[ur] & & w\ar[ul]
\end{diagram*}
so we conclude that \(\calC_W^r\) is a category.
\item \(\calC_W^r\) is a strict localization of \(\calC\) by \(W\): consider the functor \(Q\!:\calC\to\calC_W^r\) given by \(x\mapsto x\) and \((x\overset{f}\to y)\mapsto (x\overset{f}\to y\overset\id\ot y)\).
One easily sees that if \(s\in W\), then \(Q(s)\) is invertible. Furthermore, if \(F\!:\calC\to\calD\) is a functor inverting morphisms in \(W\), then one can without much effort verify that the rule
\[ x\mapsto Fx,\quad (x\overset{f}\to y'\overset{s}\ot y)\mapsto F(s)^{-1}\circ F(f) \]
yields a unique well-defined functor \(F'\!:\calC_W^r\to\calD\) such that \(F'\circ Q = F\).
\end{enumerate}
This concludes the proof.
\end{proof}
\begin{exercise}
	Verify the details in part (b) of the proof of Theorem \ref{thm:right-multiplicative-system-localization-category}.
\end{exercise}
\begin{corollary}
	Let \(\calC\) be a category, and \(W\) be a right multiplicative system. Then there is a canonical isomorphism of categories \(\calC[W^{-1}]\cong\calC_W^r\). In particular,
	for all \(x,y\in\calC\), the induced map
	\[ \calC[W^{-1}](x,y) \to \calC_W^r(x,y) \]
	is bijective.% In particular, \(\calC[W^{-1}]\) is locally small. % This last statement isn't quite true; you need the assumption that W^{y/} is cofinally small.
\end{corollary}
\begin{corollary}
	Let \(\calC\) be a category, and let \(W\) be a collection of morphisms. Consider the functor \(Q\!:\calC\to\calC[W^{-1}]\).
	\begin{enumerate}[label=(\arabic*)]
	\item If \(W\) is a right multiplicative system, then \(Q\) preserves finite colimits.
	\item If \(W\) is a left multiplicative system, then \(Q\) preserves finite limits.
	\end{enumerate}
\end{corollary}
\begin{proof}
Statement (2) is formally dual to (1), and (1) is a consequence of filtered colimits commuting with finite limits in \(\Set\).
\end{proof}


\subsection{Interaction with adjoints}
\begin{construction}\label{construction:localization-of-adjoints}
	Let \(\calC\) and \(\calC'\) be categories, and consider collections of morphisms \(W\subseteq\Mor(\calC)\) and \(W'\subseteq\Mor(\calC')\). Consider an adjunction
	\begin{tikzcd}[cramped]
		\calC\ar[r,bend left,"L",""{name=A,below}] & \calC' \ar[l,bend left,"R",""{name=B,above}]\ar[from=A,to=B,symbol=\dashv]
	\end{tikzcd}
	and assume that \(L(W)\subseteq W'\) and \(R(W')\subseteq W\). By universal property, we obtain a pair of functors
	\begin{diagram*}[row sep=large]
		\calC\ar[r,bend left=20,"L",""{name=A,below}]\ar[d] & \calC'\ar[d] \ar[l,bend left=20,"R",""{name=B,above}]\ar[from=A,to=B,symbol=\dashv] \\
		\calC[W^{-1}]\ar[r,dashed,bend left=20,"L'"] & \calC'[W'^{-1}]. \ar[l,dashed,bend left=20,"R'"]
	\end{diagram*}
\end{construction}
\begin{proposition}\label{prop:localization-of-adjoints}
	Consider the situation in Construction \ref{construction:localization-of-adjoints}. Then \(L'\ladj R'\).
\end{proposition}
\begin{proof}
Let \(\eta\!:\1\To RL\) and \(\varepsilon\!:LR\To\1\) be the unit and counit, respectively. We obtain natural transformations
\[ \eta'\!:\1\To R'L',\quad \varepsilon\!:L'R'\To\1 \]
which are given on components simply by \(\eta\) and \(\varepsilon\), i.e.\ \(\varepsilon'_x = (x\overset{\eta_x}\to RLx = R'L'x)\). To see that this defines natural transformations,
note that since \(L\) and \(R\) send weak equivalences to weak equivalences, we have a commutative diagram
\begin{diagram*}
	x\ar[d]\ar[r,"\eta_x"] & RLx\ar[d] \\
	\bullet\ar[r,"\eta"] & RL\bullet \\
	\vdots\ar[u,"W\ni"]\ar[d] & \vdots\ar[u,"\in W"']\ar[d]\\
	y\ar[r,"\eta_y"] & RLy
\end{diagram*}
in \(\calC\). This shows that
\[ (x\overset{\eta_x}\to RLx \to RL\bullet \ot \cdots\to RLy) = (x\to\bullet\ot\cdots\to y\overset{\eta_y}\to RLy) \]
so that the induced square commutes in \(\calC[W^{-1}]\). The same reasoning applies to \(\varepsilon'\). That the unit and counit \(\eta'\) and \(\varepsilon'\)
satisfy the triange identities now follows trivially from the fact that the same holds of \(\eta\) and \(\varepsilon\).
\end{proof}

\subsection{Localizations of additive categories are additive}
\begin{lemma}\label{lemma:product-of-localizations}
	Let \(\calC\) and \(\calC\) be categories, along with collections of morphisms \(W\subseteq\Mod(\calC)\) and \(W'\subseteq\Mor(\calC')\). Then there is a canonical
	isomorphism of categories
	\[ (\calC\times\calC')[(W\times W')^{-1}] \cong \calC[W^{-1}]\times\calC'[W'^{-1}].\]
\end{lemma}
\begin{proof}
Observe that we may compute
\begin{align*}
	\Fun(\calC[W^{-1}]\times\calC'[W'^{-1}],\calE) &\cong \Fun(\calC[W^{-1}],\Fun(\calC'[W'^{-1}],\calE))\\
	&\cong \Fun_W(\calC,\Fun(\calC'[W'^{-1}],\calE))\\
	&\cong \Fun_W(\calC,\Fun_{W'}(\calC',\calE))\\
	&\cong \Fun_{W\times W'}(\calC\times\calC',\calE) \cong \Fun( (\calC\times\calC')[(W\times W')^{-1}],\calE )
\end{align*}
where we use the definition of a (strict) localization, and that \(\Fun_W(\calC,\Fun_{W'}(\calC',\calE)) \cong \Fun_{W\times W'}(\calC\times\calC',\calE)\) can be checked easily.
\end{proof}
\begin{corollary}\label{corollary:localization-admits-coproducts}
	Let \(I\) be a finite set, and let \(\calC\) be a category admitting \(I\)-indexed coproducts. Suppose that \(W\subseteq\Mor(\calC)\) is a collection of morphisms containing
	the coproducts \(\amalg_{i\in I}s_i\) of all \(I\)-indexed families of morphisms \(s_i\in W\). Then \(\calC[W^{-1}]\) admits \(I\)-indexed coproducts, and
	\[ \calC\to\calC[W^{-1}] \]
	preserves them.
\end{corollary}
\begin{proof}
The functor sending an \(I\)-indexed tuple \((x_i)_{i\in I}\in\prod_{i\in I}\calC\) to its coproduct \(\coprod_{i\in I}x_i\) is the left adjoint of the constant functor
\begin{diagram*}
	\calC\ar[from=r,bend right,shift right,"\coprod_{i\in I}-"',""{name=A,below}] & \mathmakebox[\widthof{\calC}][l]{\prod_{i\in I}\calC} \ar[from=l,"\Delta"',""{name=B,above}]\ar[from=A,to=B,symbol=\dashv]
\end{diagram*}
and so, applying Proposition \ref{construction:localization-of-adjoints} and Lemma \ref{lemma:product-of-localizations} we get
\begin{diagram*}[column sep=large]
	\calC\ar[from=r,bend right,shift right,"\coprod_{i\in I}-"',""{name=A,below}]\ar[d] & \mathmakebox[\widthof{\calC}][l]{\prod_{i\in I}\calC} \ar[from=l,"\Delta"',""{name=B,above}]\ar[from=A,to=B,symbol=\dashv]\ar[d] \\
	\mathmakebox[\widthof{\calC}][r]{\calC[W^{-1}]}\ar[from=r,bend right,shift right,dashed,""{name=C,below}] & \mathmakebox[\widthof{\calC}][l]{\prod_{i\in I}\calC[W^{-1}]} \ar[from=l,"\Delta"',""{name=D,above}]\ar[from=C,to=D,symbol=\dashv]
\end{diagram*}
which exhibits the existence of \(I\)-indexed coproducts in \(\calC[W^{-1}]\). It is clear from the commutativity of the diagram that \(\calC\to\calC[W^{-1}]\) preserves \(I\)-indexed coproducts.
\end{proof}
\begin{exercise}\label{exercise:localization-admits-products}
	Use the same kind of argument to show that if \(\calC\) admits finite products, then so does \(\calC[W^{-1}]\) and the localization functor preserves them.
\end{exercise}
\begin{remark}
	When the categories and collections of morphisms are more well-behaved, this can be extended to larger indexing sets \(I\). For example, when the pair \((\calC,W)\) comes from
	a model category, or more generally a (co)fibration category, arbitrary products of the localizations are the localizations of the products. See \cite[Thm.\ 7.1.1]{radulescubanu2009cofibrationshomotopytheory}.
\end{remark}
\begin{lemma}\label{lemma:induced-additive-structure}
	Let \(\calC\) be an additive category, let \(\calD\) be a category admitting finite products and coproducts, and suppose there is an essentially surjective
	functor \(Q\!:\calC\to\calD\) which preserves finite products and coproducts. Then \(\calD\) is additive, and \(Q\) is an additive functor.
\end{lemma}
\begin{proof}
We check the conditions in Remark \ref{remark:additive-internal-characterization}. Conditions (1) and (2) are already fulfilled.

First, note that finite products and coproducts agree in \(\calD\): indeed, for \(x,y\in\calD\), we find \(x_0,y_0\in\calC\) such that \(x\cong Q(x_0)\), \(y\cong Q(y_0)\). Then
\[ x\amalg y \cong Q(x_0)\amalg Q(y_0) \cong Q(x_0\amalg y_0) \cong Q(x_0\times y_0)\cong Q(x_0)\times Q(y_0) \cong x\times y \]
as desired. This shows that (3) is satisfied.

To prove that (4) holds, let \(x\in\calD\) and write \(x\cong Q(x_0)\). Since \(\calC\) is additive, we have a morphism \(a\!:x_0\to x_0\) such that
\[ x_0\xrightarrow{\Delta_{x_0}} x_0\oplus x_0 \xrightarrow{(a,\id_{x_0})} x_0\oplus x_0 \xrightarrow{\nabla_x} x_0 \]
is the zero map. Since \(Q\) preserves both products and coproducts, we see that
\[ Qx_0\xrightarrow{\Delta_{Qx_0}} Qx_0\oplus Qx_0 \xrightarrow{(Qa,\id_{Qx_0})} Qx_0\oplus Qx_0 \xrightarrow{\nabla_{Qx_0}} Qx_0 \]
is zero, so that \(x\cong Qx_0 \overset{Qa}\to Qx_0\cong x\) provides (4).
\end{proof}

\begin{theorem}\label{thm:localization-of-additive-is-additive}
	Let \(\calC\) be an additive category, and let \(W\subseteq\Mor(\calC)\) be a collection of morphisms containing the identities and closed under direct sums of maps. Then
	\(\calC[W^{-1}]\) is an additive category, and \(Q\!:\calC\to\calC[W^{-1}]\) an additive functor.
\end{theorem}
\begin{proof}
Note that \(Q\) is an essentially surjective (in fact, bijective-on-objects) functor, and that by Corollary \ref{corollary:localization-admits-coproducts} and Exercise \ref{exercise:localization-admits-products}
the localization admits finite products and finite coproducts which commute with \(Q\). Hence, by Lemma \ref{lemma:induced-additive-structure}, \(\calC[W^{-1}]\) is additive and \(Q\)
is an additive functor.
\end{proof}

\begin{example}
	Recall that we defined chain complexes in an Abelian category \(\calA\), and that given a chain complex \(x^\bullet\) one may form the cohomologies \(\HH^i(x^\bullet)\in\calA\).
	A \emph{morphism} \(f\!:x^\bullet\to y^\bullet\) of chain complexes consists of a system of morphisms \((f^i\!:x^i\to y^i)\) such that
	\begin{diagram*}[cramped]
		\cdots \ar[r] & x^{i} \ar[r]\ar[d,"f^i"] & x^{i+1} \ar[r]\ar[d,"f^{i+1}"] & \cdots \\
		\cdots \ar[r] & x^{i} \ar[r] & x^{i+1} \ar[r] & \cdots
	\end{diagram*}
	commutes. In particular, this defines a category \(\Ch(\calA)\), and one can check that it is Abelian. It is also not so hard to check that taking cohomology extends to a
	collection of functors \(\HH^i\!:\Ch(\calA)\to\calA\). One says that a morphism \(f\!:x^\bullet\to y^\bullet\) of chain complexes is a \emph{quasi-isomorphism} if \(\HH^i(f)\) is an isomorphism for all \(i\in\Z\).

	Let \(\Qis_\calA\subseteq\Mor(\Ch(\calA))\) denote the quasi-isomorphisms. The \emph{derived category} of \(\calA\) is the localization
	\[ \sfD(\calA) := \Ch(\calA)[\Qis_\calA^{-1}]. \]
	By the above theorem, this is an additive category. It is very important to note, however, that it is absolutely not an Abelian category; the structure it carries is instead
	that of a \emph{triangulated category.} Tautologically, the functors \(\HH^i\!:\Ch(\calA)\to\calA\) descend to functors \(\HH^i\!:\sfD(\calA)\to\calA\) by universal property.
\end{example}



%!TEX root = ../lectures.tex

\section{Homotopical algebra through deformations}\label{lecture:homotopical-algebra-through-deformations}
In Lecture \ref{section:localization-of-categories}, we covered the basic theory of how to localize categories at certain morphisms; in other words, we take a category \(\calC\) and
a collection of morphisms \(W\subseteq\Mor(\calC)\) and ask for a universal choice of category \(\calC[W^{-1}]\) wherein the morphisms in \(W\) become invertible. The most basic functoriality
properties of this construction demand that these morphisms are appropriately preserved: if we have pairs \((\calC,W)\) and \((\calC',W')\), and a functor \(F\!:\calC\to\calC'\) such that \(F(W)\subseteq W'\),
then we have an induced functor making the diagram
\begin{diagram*}
	\calC\ar[r,"F"]\ar[d,"Q"'] & \calC'\ar[d,"Q'"] \\
	\calC[W^{-1}]\ar[r,dashed] & \calC'[W'^{-1}]
\end{diagram*}
commute. This follows by the universal property, since if \(F(W)\subseteq W'\) then \(Q'F(W)\) consists of invertible morphisms.

In practice, it is very rare to have a functor that behaves like this. In most situations, the functor \(F\) will fail to preserve weak equivalences, and so there is no immediate way
to produce a functor between the localizations. Of course, one still wants to do this, for example to get functors between derived categories: if \(F\!:\calA\to\calA'\) is a functor
between Abelian categories, it induces a functor \(F\!:\Ch(\calA)\to\Ch(\calA')\), and one would like to obtain from this some kind of functor \(\sfD(\calA)\to\sfD(\calA')\). Instinctually,
one would look to the universal property for this kind of thing, but that doesn't work, so another approach is required. That is the topic of this lecture.

There are various approaches to derived functors, and they may be covered on many different levels depending on one's preferences. Here, we follow one which is particularly beautiful, and also
somewhat rare, provided in \cite{riehl-categorical-homotopy-theory}. It leads to a very clear presentation of the ideas behind the construction of derived functors,
removing many technical aspects, thought with the downside that it is not entirely practical.

\subsection{Homotopical structures \& derived functors}
\begin{definition}
	Let \(\calC\) be a category. A \emph{wide subcategory} of \(\calC\) is a subcategory containing all objects of \(\calC\).
\end{definition}
\begin{definition}
	A \emph{relative category} is a pair \((\calC,W)\) consisting of a category \(\calC\) and a wide subcategory \(W\subseteq\calC\).
	\begin{enumerate}[label=(\arabic*)]
		\item We say \((\calC,W)\) is a \emph{pseudo-homotopical category} if \(W\) contains all isomorphisms and satisfies the \(2\)-out-of-\(3\) property: for any two morphisms
		\(x\overset{f}\to y\overset{g}\to z\), if any two of \(f\), \(g\), or \(g\circ f\) are in \(W\) then so is the third.
		\item We say \((\calC,W)\) is a \emph{homotopical category} if \(W\) satisfies the \(2\)-out-of-\(6\) property: for any three composable morphisms
		\begin{diagram*}[cramped,row sep=small, column sep=large]
			  & y\ar[ddrr,"hg"] & & \\
			x\ar[ur,"f"]\ar[ddrr,"gf"']\ar[drrr,near start,"hgf"] & & & \\
			  & &   & w \\
			  & & z\ar[ur,"h"']\ar[from=uuul,crossing over,"g" near end] &
		\end{diagram*}
		if \(hg\in W\) and \(gf\in W\), all other arrows in the 3-simplex above are in \(W\).
	\end{enumerate}
	The morphisms in \(W\) are called \emph{weak equivalences.}
\end{definition}
\begin{remark}
	The notion of a relative category is almost exactly what was implicitly studied in Lecture \ref{section:localization-of-categories}. Indeed, the wide subcategory \(W\) can be specified by giving
	a collection of morphisms containing all identities and which is closed under composition.
\end{remark}
\begin{remark}
	Any homotopical category is a pseudo-homotopical category; the 2-out-of-3 property follows by replacing appropriate arrows by identities. Furthermore, if \((\calC,W)\) is a homotopical category, applying the 2-out-of-6 property yields
	that \(W\) contains all isomorphisms: if \(f\!:x\iso y\), then one can consider the 3-simplex formed by
	\[ x\overset{f}\to y\overset{f^{-1}}\to x\overset{f}\to y \]
	and note that the partial compositions are the identities, which are in \(W\).

	Any category \(\calC\) can be promoted to a homotopical category in a trivial way by letting \(W\) be the largest groupoid contained in \(\calC\), i.e.\ the subcategory given by all the isomorphisms.
\end{remark}
\begin{notation}
	Let \((\calC,W)\) be a relative category. The localization \(W^{-1}\calC\) is sometimes denoted by \(\ho(\calC,W)\) (or \(\ho(\calC)\) if \(W\) is left implicit), and is called the \emph{homotopy category} of \((\calC,W)\).

	We will typically leave the wide subcategory \(W\) implicit in the notation.
\end{notation}

A functor \(\calC\to\calD\) between (pseudo-)homotopical categories is called \emph{homotopical} if if sends weak equivalences to weak equivalences. Trivially, a homotopical
functor induces a functor \(\ho(\calC)\to\ho(\calD)\). When the functor fails to be homotopical, however, it is not clear what to do in
order to obtain a functor on the level of homotopy categories. Derived functors are an attempt to formalize the notion of a \emph{best approximation} of the functor on this level.


There are many technically different definitions of derived functors at our level of generality, and they are not necessarily equivalent. We take the conventions used \cite{riehl-categorical-homotopy-theory},
as they seem reasonable enough, and also correspond in a straightforward way to e.g.\ what is used in \cite{kashiwara-schapira-book}.
\begin{definition}
	Let \(F\!:\calC\to\calD\) be a functor between relative categories. A \emph{total left derived functor} \(\bfL F\) of \(F\) is a right Kan extension
	\begin{diagram*}
		\calC\ar[r,"F"]\ar[d,"\gamma"'] & \calD\ar[d,"\delta"] \\
		\ho(\calC)\ar[r,dashed,"\bfL F"] & \ho(\calD)\ar[from=l,to=u,Rightarrow,shorten=2mm]
	\end{diagram*}
	of the composition \(\calC\overset{F}\to\calD\to\ho(\calD)\) along the localization functor \(\calC\to\ho(\calC)\). We say that the total derived functor \(\bfL F\) is \emph{absolute}
	if the right Kan extension above is absolute.

	A \emph{left derived functor} of \(F\) is a pair \((\bbL F,\lambda)\) of a homotopical functor \(\bbL F\!:\calC\to\calD\) and a natural transformation \(\lambda\!:\bbL F\To F\)
	such that the induced functor \(\delta\bbL F\!:\ho(\calC)\to\ho(\calD)\) defines a total left derived functor
	\begin{diagram*}
		\calC\ar[r,"F"]\ar[d,"\gamma"'] & \calD\ar[d,"\delta"] \\
		\ho(\calC)\ar[r,dashed,"\delta\bbL F"'] & \ho(\calD)\ar[from=l,to=u,Rightarrow,"\delta\lambda"',shorten=2mm]
	\end{diagram*}
	of \(F\). We say \(\bbL F\) is an \emph{absolute} left derived functor if the above total left derived functor is absolute.
\end{definition}
\begin{exercise}
	Dualize the above notions to define \emph{(absolute) (total) right derived functors.}
\end{exercise}

As bare Kan extensions are not particularly well-behaved, nor are derived functors, and establishing their existence is generally very hard. Absolute derived functors exhibit far more desirable properties.
\begin{proposition}
	Consider an adjunction
	\begin{diagram*}[cramped]
		\calC\ar[r,bend left,"F",""{name=A,below}] & \calD \ar[l,bend left,"G",""{name=B,above}]\ar[from=A,to=B,symbol=\dashv]
	\end{diagram*}
	between relative categories, and assume that \(F\) admits an absolute total left derived functor and \(G\) admits an absolute total right derived functor. Then the adjunction
	descends to an adjunction
	\begin{diagram*}[cramped,column sep=large]
		\ho(\calC)\ar[r,bend left,"\bfL F",""{name=A,below}] & \ho(\calD). \ar[l,bend left,"\bfR G",""{name=B,above}]\ar[from=A,to=B,symbol=\dashv]
	\end{diagram*}
\end{proposition}
\begin{proof}
We have that the diagrams
\begin{center}
	\begin{tikzcd}
		\calC\ar[r,"F"]\ar[d,"\gamma"'] & \calD\ar[d,"\delta"] \\
		\ho(\calC)\ar[r,"\bfL F"] & \ho(\calD)\ar[from=l,to=u,Rightarrow,"\alpha",shorten=4mm]
	\end{tikzcd}\quad\quad
	\begin{tikzcd}
		\calC\ar[from=r,"G"']\ar[d,"\gamma"'] & \calD\ar[d,"\delta"] \\
		\ho(\calC)\ar[from=r,"\bfR G"'] & \ho(\calD)\ar[from=ul,Rightarrow,"\beta",shorten=4mm]
	\end{tikzcd}
\end{center}
provide a right and left absolute Kan extensions, respectively. Using the Kan extensions \((\bfL F,\alpha)\) and \((\bfR G,\beta)\) are absolute, we see that the pairs
\[ (\bfR G\bfL F, \bfR G\bfL F\gamma\overset{\bfR G\alpha}\To \bfR G\delta F),\quad (\bfL F\bfR G, \bfL\gamma G \overset{\bfL F\beta}\To \bfL F\bfR G \delta ) \]
are left (resp.\ right) Kan extension along \(\gamma\) (resp.\ \(\delta\)). Now, hitting \(\beta\) with \(F\) and \(\alpha\) with G, and combining this with the
adjunction \(F\ladj G\) (with unit/counit \(\eta\) and \(\varepsilon\)), we get composite natural transformations
\begin{center}
\begin{tikzcd}[column sep=small]
	\gamma\ar[dr,dashed,Rightarrow,"\exists!"'] \ar[r,Rightarrow,"\gamma\eta"] & \gamma GF \ar[r,Rightarrow,"\beta F"] & \bfR G \delta F \\
	& \bfR G\bfL F\gamma\ar[ur,Rightarrow,"\bfR G\alpha"'] &
\end{tikzcd}\quad\quad
\begin{tikzcd}[column sep=small]
	\bfL F\gamma G\ar[dr,Rightarrow,"\bfL F\beta"'] \ar[r,Rightarrow,"\alpha G"] & \delta FG \ar[r,Rightarrow,"\delta\varepsilon"] & \delta \\
	& \bfL F\bfR G\delta \ar[ur,Rightarrow,dashed,"\exists!"'] &
\end{tikzcd}
\end{center}
producing unique dashed natural transformations by universality. By the universal property of the localization, these correspond uniquely to natural transformations
\[ \eta'\!:\1\To \bfR G \circ \bfL F,\quad \varepsilon'\!:\bfL F\circ\bfR G\To\1. \]

We must show that the natural transformations \(\eta'\) and \(\varepsilon'\) satisfy the triangle identities. Observe that by universal property of the localization and of right Kan extensions,
the commutativity of the below left diagram
\begin{center}
	\begin{tikzcd}[column sep=small]
		& \bfL F \bfR G \bfL F\ar[dr,Rightarrow,"\varepsilon'\bfL F"] & \\
		\bfL F\ar[rr,equal]\ar[ur,Rightarrow,"\bfL F\eta'"] & & \bfL F
	\end{tikzcd}\(\quad \leftrightsquigarrow \quad\)
	\begin{tikzcd}[column sep=small]
		& \bfL F \bfR G \bfL F\gamma \ar[dr,Rightarrow,"\varepsilon'\bfL F \gamma "] & & \\
		\bfL F \gamma \ar[rr,equal]\ar[ur,Rightarrow,"\bfL F\eta'\gamma"] & & \bfL F\gamma\ar[r,Rightarrow,"\alpha"] & \delta F
	\end{tikzcd}
\end{center}
is equivalent to the two obvious compositions in the above right diagram being equal. Now, applying \(\bfL F\) on the left to the diagram defining \(\eta'\) and \(F\) on the right to the diagram defining \(\varepsilon'\),
as well as using the naturality of \(\varepsilon'\) and \(\alpha\), we see that the diagram
\begin{diagram*}[row sep=large]
	& \delta F\ar[r,Rightarrow,"\delta F\eta"] & \delta F G F \ar[dr,Rightarrow,"\delta\varepsilon F"] & \\
	\bfL F\gamma\ar[r,Rightarrow,"\bfL F\gamma\eta"]\ar[dr,Rightarrow,"\bfL F\eta'\gamma"']\ar[ur,Rightarrow,"\alpha"] & \bfL F\gamma GF\ar[r,Rightarrow,"\bfL F\beta F"]\ar[ur,Rightarrow,"\alpha G F"] & \bfL  F\bfR G\delta F\ar[r,Rightarrow,"\varepsilon'\delta F"] & \delta F \\
	& \bfL F\bfR G\bfL F\gamma\ar[r,Rightarrow,"\varepsilon'\bfL F\gamma"']\ar[ur,Rightarrow,"\bfL F\bfR G\alpha"'] & \bfL F \gamma\ar[ur,Rightarrow,"\alpha"'] &
\end{diagram*}
commutes. The identity then follows from the triangle identity \(\varepsilon F \circ F\eta = \id_F\). The other case is handled similarly.
\end{proof}
\begin{exercise*}
	In the diagram at the end of the above proof there are two squares. It is claimed that they commute by naturality. Check this.
\end{exercise*}

\subsection{Deformations}
While a typical functor \(\calC\to\calD\) between (pseudo-)homotopical categories fails to be homotopical, there are usually subcategories of \(\calC\) on which the functor does preserve weak equivalences.
Restricted to such subcategories, it is straightforward to compute the derived functor, as it is simply induced by universal property. However, the resulting functor of course need not be related
to the original, due to the restriction step.

In favourable situations, \emph{every} object in \(\calC\) can be replaced by one from a subcategory on which a functor acts homotopically. In even more favourable situations, this can be done functorially. This
is captured in the following definition.
\begin{definition}
	Let \((\calC,W)\) be a relative category,
	\begin{enumerate}[label=(\arabic*)]
	\item Consider two functors \(F,G\!:\calD\to\calC\). A natural transformation \(\eta\!:F\To G\) is a \emph{natural weak equivalence} if all components
	are weak equivalences, i.e.\ \(\forall x\in\calD,\, \eta_x\in W\).
	\item A \emph{left deformation} of \(\calC\) is a pair \((Q,q)\) consisting of a functor \(Q\!:\calC\to\calC\) and a natural weak equivalence \(q\!:Q\To\1\). Dually, a right deformation
	is a pair \((R,r)\) consisting of a functor \(R\!:\calC\to\calC\) and a natural weak equivalence \(r\!:\1\To Q\).
	\item Let \((Q,q)\) be a left deformation. A full subcategory \(\calC_Q\) of \(\calC\) which contains the image of \(Q\) is called a \emph{category of} \(Q\)\emph{-cofibrant objects.} This is promoted to a relative category
	by letting the weak equivalences be \(\calC_Q\cap W\).

	Dually, let \((R,r)\) be a right deformation. A full subcategory \(\calC_R\) of \(\calC\) which contains the image of \(R\) is called a \emph{category of} \(R\)\emph{-fibrant objects.} This is promoted to a relative
	category by letting the weak equivalences be \(\calC_R\cap W\).
	\item Consider another relative category \((\calC',W')\) and some functor \(F\!:\calC\to\calC'\). We say \(F\) is \emph{left deformable} if there is a left deformation \((Q,q)\) of \(\calC\)
	and a category of \(Q\)-cofibrant objects \(\calC_Q\) such that \(F|_{\calC_Q}\!:\calC_Q\to\calD\) is homotopical, where \(\calC_Q\) is endowed with weak equivalences \(\calC_Q\cap W\).
	\end{enumerate}
\end{definition}

\begin{proposition}\label{prop:deformation-is-homotopical-and-cofibrant-objects-homotopy-equivalence}
	Let \((\calC,W)\) be a pseudo-homotopical category, and let \((Q,q)\) be a left deformation. Then the following statements hold.
	\begin{enumerate}[label=(\arabic*)]
	\item The functor \(Q\) is homotopical.
	\item For any category of \(Q\)-cofibrant objects \(\calC_Q\), the induced functor
	\[ \ho(\calC_Q) \to \ho(\calC) \]
	is an equivalence.
	\end{enumerate}
\end{proposition}
\begin{proof}
(1) Let \(f\!:x\to y\) be a weak equivalence. Then we have a diagram
\begin{diagram*}
	x\ar[r,"q_x"]\ar[d,"f"'] & Qx\ar[d,"Qf"] \\
	y\ar[r,"q_y"] & Qy
\end{diagram*}
so by the 2-out-of-3 property, \(Qf\) is a weak equivalence. Hence \(Q\) is homotopical.

(2) By the definition of the induced relative category structure on \(\calC_Q\), \(\calC_Q\inj\calC\) is homotopical, giving the induced functor
\[ I\!:\ho(\calC_Q)\to\ho(\calC). \]
On the other hand, by (1), \(Q\) is homotopical and so induces a functor
\[ Q'\!:\ho(\calC)\to\ho(\calC_Q). \]
Now, observe that both the composites
\[ \calC_Q\inj\calC\overset{Q}\to\calC_Q,\quad \calC\overset{Q}\to\calC_Q\inj\calC \]
have natural weak equivalences to \(\1_{\calC_Q}\) and \(\1_\calC\), respectively, formed from the natural weak equivalence \(q\!:Q\To\1\). On the level of the homotopy categories,
these give rise to the desired natural isomorphisms
\[ Q'I\cong\1_{\ho(\calC_Q)},\quad IQ'\cong\1_{\ho(\calC)} \]
which yields the result.
\end{proof}

\subsection{Existence of absolute derived functors via deformations}
Functors that admit left or right deformations are nice enough that they induce not only a total derived functor, but an \emph{absolute} total derived functor; even more,
one gets a derived functor without even passing to the localization.
\begin{theorem}
	Let \(F\!:\calC\to\calD\) be a left deformable functor between pseduo-homotopical categories, with left deformation \((Q,q)\). Then \(F\) admits an absolute left derived functor
	given by
	\[ (\bbL F := FQ\!:\calC\to\calD, F q\!: FQ\To F). \]
\end{theorem}
\begin{proof}
Let \(\gamma\!:\calC\to\ho(\calC)\) and \(\delta\!:\calD\to\ho(\calD)\) denote the localization functors. There are three steps to this proof.
\begin{enumerate}[label=(\arabic*)]
\item To check that \(\bfL F\) exists and is absolute, it suffices to work with \(\bbL F\) (as \(\bbL F = FQ\) is homotopical since \(Q\) is a deformation of a pseudo-homotopical category),
functors \(H\!:\ho(\calD)\to\calE\), and to consider homotopical functors \(G\!:\calC\to\calE\) with natural transformations \(\sigma\!:G\To H\delta F\). This is by the universal property of localizations, and in
particular, the isomorphism
\[\gamma^*\!:\Fun(\ho(\calC),\calE)\cong\Fun_W(\calC,\calE),\]
where \(W\) are the weak equivalences on \(\calC\).
\item Factorizations exist. Let \(\sigma\!:G\To H\delta F\) be a natural transformation. Then, by naturality and \(G\) being homotopical,
\begin{center}
	\begin{tikzcd}
		GQ\ar[d,Rightarrow,"Gq"']\ar[r,Rightarrow,"\sigma Q"] & H\delta F Q\ar[d,Rightarrow,"H\delta F q"] \\
		G\ar[r,Rightarrow,"\sigma"] & H\delta F
	\end{tikzcd}\(\quad\implies\quad \sigma = (G\overset{(Gq)^{-1}}\To GQ\overset{\sigma Q}\To H\delta FQ\overset{H\delta Fq}\To H\delta F)\)
\end{center}
which gives the desired factorization.
\item Factorizations are unique: given another factorization as in the below left,
\begin{center}
	\begin{tikzcd}[cramped, column sep=small]
		G\ar[dr,Rightarrow,"\sigma'"']\ar[rr,Rightarrow,"\sigma"] & & H\delta F \\
		& H\delta F Q\ar[ur,Rightarrow,"H\delta F q"']&
	\end{tikzcd}\(\quad\overset{Q}\leadsto\quad\)
	\begin{tikzcd}[cramped, column sep=small]
		GQ\ar[dr,Rightarrow,"\sigma'Q"']\ar[rr,Rightarrow,"\sigma Q"] & & H\delta F Q \\
		& H\delta F Q^2\ar[ur,Rightarrow,"H\delta F qQ"']&
	\end{tikzcd}
\end{center}
we get the above right diagram by applying \(Q\) on the right. Since \(qQ\) lives in a category of \(Q\)-cofibrant objects, \(FqQ\) is a natural weak equivalence. Therefore, \(H\delta FqQ\) is a natural isomorphism, so that
\(\sigma'Q\) is uniquely determined. However, \(\sigma'\) is determined by \(\sigma'Q\), as demonstrated by the naturality square
\begin{diagram*}
	GQ\ar[d,Rightarrow,"Gq"']\ar[r,Rightarrow,"\sigma' Q"] & H\delta F Q^2\ar[d,Rightarrow,"H\delta F Q q"] \\
	G\ar[r,Rightarrow,"\sigma'"] & H\delta F Q
\end{diagram*}
from which we conclude that \((FQ, F q)\) defines an absolute left derived functor.
\end{enumerate}
This completes the proof. Note that the above specializes to show that \((FQ, Fq)\) is a left derived functor by setting \(H = \id_{\ho(\calD)}\).
\end{proof}
\begin{remark}
	In the above proof, the assumption that \(\calD\) is pseudo-homotopical can be weaked to it merely being a relative category. Indeed, the only aspect of the proof
	that relies on a pseudo-homotopical assumption is that the deformation functor \(Q\!:\calC\to\calD\) is homotopical, which demands the 2-out-of-3 property. Obviously,
	this does not involve \(\calD\).
\end{remark}

\subsection{Pseudofunctoriality}
We have seen that left deformable functors \(F\!:\calC\to\calD\) between pseudo-homotopical categories admit absolute total left derived functors \(\bfL F\). This suggests
that there should be some assignment \(\bfL\!:F\mapsto \bfL F\), which we may hope is functorial. However, this is not generally true: the composition of two absolute total left derived
functors need not be a total left derived functor. On the other hand, with some mild assumptions in place, one can arrange for a form of weak functoriality.

Recall that a (strict) 2-category is a category enriched in categories. The prototypical example of this is \(\underline{\Cat}\), the 2-category of categories (subject to some set-theoretical constraint).
As this case is fairly clear, and all other examples we work with here will essentially be derived from it, we will not expound upon the definition any further for the moment.
\begin{definition}
	We define a 2-category \(\underline{\cat{LDef}}\) as follows.
	\begin{itemize}[label=\(\star\)]
	\item The objects are tuples \((\calC,W,Q,q,\calC_Q)\) consisting of a pseudo-homotopical category \((\calC,W)\), a left deformation \((Q,q)\) for \(\calC\), and
	a choice of a distinguished category of \(Q\)-cofibrant objects \(\calC_Q\).
	\item A 1-morphism \((\calC,W,Q,q,\calC_Q)\to(\calC',W',Q',q',\calC'_{Q'})\) consists of a left deformable functor \(F\!:\calC\to\calC'\) for which a choice
	of left deformation is \(Q\), and such that \(F\) sends distinguished \(Q\)-cofibrant objects to distinguished \(Q\)-cofibrant objects, i.e.\ restricts to a functor \(\calC_Q\to\calC'_{Q'}\).
	\item A 2-morphism \(F\To F'\) is a natural transformation, with no added compatibility required.
	\end{itemize}
\end{definition}

\begin{theorem}\label{thm:deformable-functor-derived-functor-pseudofunctoriality}
	There is a pseudofunctor
	\[ \bfL\!: \underline{\cat{LDef}} \to \underline{\Cat} \]
	which
	\begin{enumerate}[label=(\arabic*)]
	\item sends a tuple \((\calC,W,Q,q,\calC_Q)\) to \(\ho(\calC)\),
	\item sends a left deformable functor \(F\!:(\calC,W,Q,q,\calC_Q)\to (\calC',W',Q',q',\calC'_{Q'})\) to its total left derived functor \(\bfL F\!:\ho(\calC)\to\ho(\calD)\), and
	\item sends a natural transformation \(F\To F'\) to the induced natural transformation \(\bfL F\To\bfL F'\).
	\end{enumerate}
\end{theorem}

For the purposes of explaining the above theorem, we include below the definition of a pseudofunctor between strict 2-categories. In the less strict setting of a bicategory,
the definition is similar but with small variations (since e.g.\ the associativity of the composition there is up to natural isomorphism).

\begin{definition}\label{definition:pseudofunctor}
	A pseudofunctor \(P\!:\underline{\calC}\to\underline{\calD}\) between 2-categories consists of the following data and conditions.
	\begin{enumerate}[label=(\arabic*)]
	\item A function \(P\!:\Ob\underline{\calC}\to\Ob\underline{\calD}\).
	\item For each pair of objects \((x,y)\) in \(\underline{\calC}\), a functor
	\[ P_{x,y}\!:\underline{\calC}(x,y) \to \underline{\calD}(Px,Py). \]
	\item For each triple of objects \((x,y,z)\) in \(\underline{\calC}\), two natural isomorphisms
	\begin{center}
		\begin{tikzcd}
			\underline{\calC}(y,z)\times\underline{\calC}(x,y)\ar[r,"\circ"]\ar[d,"P\times P"'] & \underline{\calC}(x,z)\ar[d,"P"] \\
			\underline{\calD}(Py,Pz)\times\underline{\calD}(Px,Py)\ar[r,"\circ"'] & \underline{\calD}(Px,Pz) \ar[from=l,to=u,Rightarrow,shorten=7mm]
		\end{tikzcd}\quad\quad
		\begin{tikzcd}
			\text{[0]}\ar[r,"\id_x"]\ar[dr,bend right,"\id_{Px}"',""{name=A,right}] & \underline{\calC}(x,x) \ar[d,"P"] \\
			& \underline{\calD}(Px,Px) \ar[from=A,to=u,Rightarrow,shorten=3mm]
		\end{tikzcd}
	\end{center}
	with components
	\[ Pg\circ Pf \overset{P_{g,f}}\Longrightarrow P(g\circ f),\quad \id_{Px}\overset{P_{\id_x}}\Longrightarrow P\id_x. \]
	\item For each 1-morphism \(f\!:x\to y\) in \(\underline{\calC}\), the diagrams of 2-morphisms
	\begin{center}
		\begin{tikzcd}
			\id_{Py}\circ Pf \ar[r,equal] \ar[d,Rightarrow,"P_{\id_y}Pf"'] & Pf \\
			P\id_y\circ Pf\ar[r,Rightarrow,"P_{\id_y,f}"] & P(\id_y\circ f)\ar[u,equal]
		\end{tikzcd}\quad\quad
		\begin{tikzcd}
			Pf\circ\id_{Px} \ar[r,equal] \ar[d,Rightarrow,"PfP_{\id_x}"'] & Pf \\
			Pf\circ P\id_x\ar[r,Rightarrow,"P_{f,\id_x}"] & P(f\circ \id_x)\ar[u,equal]
		\end{tikzcd}
	\end{center}
	commute.
	\item For each triple of 1-morphisms \(x\overset{f}\to y\overset{g}\to z\overset{h}\to w\) in \(\underline{\calC}\), the diagram of 2-morphisms
	\begin{diagram*}
		Ph\circ (Pg\circ Pf)\ar[d,equal] \ar[r,Rightarrow,"PhP_{g,f}"] & Ph\circ P(g\circ f)\ar[r,Rightarrow,"P_{h,g\circ f}"] & P(h\circ(g\circ f))\ar[d,equal] \\
		(Ph\circ Pg)\circ Pf\ar[r,Rightarrow,"P_{h,g}Pf"] & P(h\circ g)\circ Pf\ar[r,Rightarrow,"P_{h\circ g,f}"] & P((h\circ g)\circ f)
	\end{diagram*}
	commutes.
	\end{enumerate}
\end{definition}

\begin{proof}[Proof of Theorem \ref{thm:deformable-functor-derived-functor-pseudofunctoriality}]
The statement of the theorem provides us with (1) and (2) in Definition \ref{definition:pseudofunctor}. For simplicity, let us denote an object of \(\underline{\cat{LDef}}\) simply by \(\calC\), leaving
all the other data implicit. We have to give natural isomorphisms
\begin{center}
	\begin{tikzcd}
		\underline{\cat{LDef}}(\calC',\calC'')\times\underline{\cat{LDef}}(\calC,\calC')\ar[r,"\circ"]\ar[d,"\bfL\times\bfL"'] & \underline{\cat{LDef}}(\calC,\calC'')\ar[d,"\bfL"] \\
		\underline{\Cat}(\ho(\calC'),\ho(\calC''))\times\underline{\Cat}(\ho(\calC,\ho(\calC'))\ar[r,"\circ"'] & \underline{\calC'}(\ho(\calC),\ho(\calC'')) \ar[from=l,to=u,Rightarrow,shorten=7mm]
	\end{tikzcd}\quad\quad
	\begin{tikzcd}
		\text{[0]}\ar[r,"\id_x"]\ar[dr,bend right,"\id_{Px}"',""{name=A,right}] & \underline{\cat{LDef}}(\calC,\calC) \ar[d,"\bfL"] \\
		& \underline{\Cat}(\ho(\calC),\ho(\calC)) \ar[from=A,to=u,Rightarrow,shorten=3mm]
	\end{tikzcd}
\end{center}
and on components, these will be of the form
\[ \bfL G \circ \bfL F \To \bfL(G\circ F),\quad \1_{\ho(\calC)} \To \bfL\1_\calC \]
where \(F\!:\calC\to\calC'\) and \(G\!:\calC'\to\calC''\). Note that we have a natural transformation
\[ \bbL G \bbL F = GQ'FQ \overset{Gq'FQ}\Longrightarrow GFQ = \bbL (GF). \]
This is a natural weak equivalence since \(F\) maps \(\calC_Q\) to \(\calC'_{Q'}\), which \(G\) acts homotopically on; similarly, this is why \(GFQ = \bbL(GF)\). Secondly, note
that
\[ \bbL\1_{\calC} = Q \overset{q}\To \1_\calC \]
is a natural weak equivalence. These two natural weak equivalences descend to the components
\[ \mu_{G,F}\!:\bfL G \circ \bfL F \cong \bfL(G\circ F),\quad \alpha_\calC\!:\1_{\ho(\calC)} \cong \bfL\1_\calC \]
that we want. We leave it to the reader to check that these components form natural isomorphisms.

We now prove (4). For the left diagram, it suffices to check that \( q \bbL F = \1_\calC q FQ \), but this holds by definition. For the right diagram, we are comparing
\(\bbL Fq\) and \(Fq\1_\calC Q\). That is, the natural transformations \(FQq\) and \(FqQ\). By naturality, \(q\circ qQ = q\circ Qq\), so for all \(x\in\calC\), we have
\[ F(q)\circ F(q_{Qx}) = F(q)\circ F(Qq_x) \]
which, if \(x\in\calC_Q\), implies that \(F(q_{Qx}) = F(Qq_x)\). In \(\ho(\calC)\), \(q\) provides an isomorphism between any object and a \(Q\)-cofibrant one, so this extends to everything in \(\ho(\calC)\).

To prove (5), it suffices to show that
\[ Hq''GFQ \circ HQ'' Gq'FQ = HGq'FQ\circ Hq''GQ'FQ. \]
This follows by the naturality square
% \begin{diagram*}
% 	HQ''GQ'FQ \ar[r,Rightarrow,"HQ''Gq'FQ"]\ar[d,Rightarrow,"Hq''GQ'FQ"'] & HQ''GFQ \ar[d,Rightarrow,"Hq''GFQ"] \\
% 	HGQ'FQ\ar[r,Rightarrow,"HGq'FQ"'] & HGFQ
% \end{diagram*}
\begin{diagram*}
	Q''GQ' \ar[r,Rightarrow,"Q''Gq'"]\ar[d,Rightarrow,"q''GQ'"'] & Q''G \ar[d,Rightarrow,"q''G"] \\
	GQ'\ar[r,Rightarrow,"Gq'"'] & G
\end{diagram*}
which completes the proof.
\end{proof}


\subsection{Appendix: When functoriality fails}

\begin{remark}
	In the presence of a model structure, even without functorial (co)fibrant replacement, absolute derived functors can still be produced. However, it will usually fail to lift to a functor prior to localizing.
	See \cite{426439} for a brief explanation, or \cite{cisinski-book} for a more detailed one, of how one would construct the left derived functor in such a situation.
\end{remark}

There is another approach to producing absolute total derived functors, taken in \cite{kashiwara-schapira-book}, which does not require a functorial deformation.
Instead, they exploit the features of having a nicer collection of weak equivalences, namely a multiplicative system as defined in Lecture \ref{section:localization-of-categories}.

In Proposition \ref{prop:deformation-is-homotopical-and-cofibrant-objects-homotopy-equivalence}, we see that categories of \(Q\)-cofibrant objects \(\calC_Q\) induce equivalences
\(\ho(\calC_Q)\simeq\ho(\calC)\). Similarly, in the situation where you have a \emph{potentially non-functorial} left deformation (so, for every object \(x\in\calC\) an object \(Qx\in\calC_Q\) and a weak equivalence
\(Qx\to x\)), but the weak equivalences \(W\) form a right multiplicative system, one can still get such an equivalence (and use it to compute the total derived functor, and show it is absolute).

We take for granted the following lemma.

\begin{lemma}
	Let \(\varphi\!:J\to I\) be a functor, and assume that \(I\) is filtered, \(\varphi\) is fully faithful, and for any \(i\in I\) there is some \(j\in J\) with a morphism \(i\to\varphi(j)\). Then
	\(J\) is filtered, and \(\varphi\) is cofinal.
\end{lemma}
\begin{proof}
See \cite[Prop.\ 3.2.4]{kashiwara-schapira-book}.
\end{proof}

We consider the situation of a right multiplicative system (and constructing right derived functors), leaving the other case implicit by duality.

\begin{proposition}
	Let \((\calC,W)\) be a relative category such that \(W\) forms a right multiplicative system. Let \(\calC'\) be a full subcategory of \(\calC\), and set \(W' := \calC'\cap W\).
	\begin{enumerate}[label=(\arabic*)]
	\item Assume that for all \(f\!:x\to y\) in \(W\) with \(x\in\calC'\), there is some \(g\!:y\to z\) such that \(z\in\calC'\) and \(g\circ f \in W\). Then \(W'\) is a right multiplicative system,
	and the induced map \(\ho(\calC')\to\ho(\calC)\) is fully faithful.
	\item Assume that for all \(x\in\calC\) there is some \(x'\in\calC'\) and a weak equivalence \(x\to x'\). Then \(W'\) is a right multiplicative system, and \(\ho(\calC')\to\ho(\calC)\) is an equivalence.
	\end{enumerate}
\end{proposition}
\begin{proof}
(1) First, we show that \(W'\) is a right multiplicative system. For this, both (M1) and (M2) follow by applying them in \(\calC\) then using the given assumption to move them back to \(\calC'\).

Now, the inclusion \(\calC'\inj\calC\) is homotopical, hence induces a functor \(\ho(\calC')\to\ho(\calC)\). On the level of Hom-sets, this is given by the canonical map
\[ \injlim_{(y\to y')\in W'^{y/}}\calC(x,y') \to \injlim_{(y\to y')\in W^{y/}}\calC(x,y') \]
coming from the inclusion \(W'^{y/} \inj W^{y/}\). The inclusion is seen to be cofinal by applying the above lemma; indeed, given some \(y\to y'\) in \(W^{y/}\), we find some \(y'\to y''\) with \(y''\in\calC'\)
and for which the composition is in \(W\), i.e.\ in \(W'\). In particular, the canonical map above is an isomorphism.

(2) The criterion in (1) is clearly satisfied, so \(W'\) is a right multiplicative system and the functor \(\ho(\calC')\to\ho(\calC)\) is fully faithful. However, by assumption it is also essentially surjective,
hence an equivalence.
\end{proof}

We may think of the above proposition as saying that when we can produce non-functorial replacements in the context of a right multiplicative system, this can be promoted to a functorial
scheme on the level of the homotopy categories. Indeed, by taking a quasi-inverse of the inclusion \(\ho(\calC')\inj\ho(\calC)\), which is an equivalence, we get a functor
\[ Q\!:\ho(\calC)\to\ho(\calC') \]
and we may even compose this with the canonical localization functor to get a homotopical functor \(\calC\to\ho(\calC')\), which is very close to a deformation.

\begin{lemma}\label{lemma:kan-extension-criterion}
	Consider functors
	\[ \calC\overset{Q}\longto\calC'\overset{G}\longto\calA. \]
	Assume that for any \(x\in\calC'\), there is some \(y\in\calC\) with a morphism \(s\!:x\to Qy\) such that the following conditions are satisfied.
	\begin{enumerate}[label=(\alph*)]
	\item \(G(s)\) is an isomorphism.
	\item For any \(y'\in\calC\) and any morphism \(t\!:x\to Qy'\), there is some \(y''\in\calC\) and morphisms \(s'\!:y'\to y''\), \(t'\!:y\to y''\) such that \(G(s')\) is an isomorphism and the diagram
	\begin{diagram*}
		x\ar[r,"s"]\ar[d,"t"'] & Qy\ar[d,"Qt'"] \\
		Qy'\ar[r,"Qs'"] & Qy''
	\end{diagram*}
	commutes.
	\end{enumerate}
	Then \(G\) is an absolute right Kan extension of \(GQ\) along \(Q\).
\end{lemma}
\begin{proof}
Here, the proposed structural natural transformation exhibiting \(G\) as \(\Ran_Q(GQ)\) is just the identity. As a result, what we have to show is that
\[ Q_*\!:\Fun(\calC',\calA)(F,G) \to \Fun(\calC,\calA)(FQ,GQ) \]
is bijective.
\begin{enumerate}[label=(\arabic*)]
\item Injectivity: consider two natural transformations \(\theta,\theta'\!:F\To G\) for which \(\theta Q = \theta'Q\). For \(x\in\calC'\), pick some \(y\in\calC\) with a morphism \(s\!:x\to Qy\)
for which \(Gs\) is an isomorphism. We have a diagram
\begin{diagram*}[column sep=large]
	Fx\ar[d,"Fs"']\ar[r,shift left,"\theta_x"]\ar[r,shift right,"\theta'_x"'] & Gx\ar[d,"Gs"] \\
	FQy\ar[r,"\theta_{Qy} = \theta'_{Qy}"] & GQy
\end{diagram*}
which commutes, except for knowing that the two arrows at the top are identical. However, as \(Gs\) is invertible and the two arrows at the bottom agree, the arrows on top agree.

\item Surjectivity: suppose we are given a natural transformation \(\theta\!:FQ\To GQ\). For each \(x\in\calC'\), choose a morphism \(s_x\!:x\to Qy\) satisfying (a) and (b), and set
\[ \theta'_x := (Gs_x)^{-1}\circ\theta_y\circ Fs_x\!:Fx\to Gx. \]
We want to show this is independent of the choices made for \(s_x\), and assembles into a natural transformation.

Let \(f\!:x_1 \to x_2\) be a morphism in \(\calC'\), and consider any choices \(s_i\!:x_i\to Qy_i\) of morphisms satisfying (a) and (b); following the above, these yield morphisms
\(\theta'_{x_i}\). Applying (b) to \(s_1\) and \(s_2\circ f\), we find morphisms
\[ t_1\!:y_1 \to y_3,\quad t_2\!:y_2\to y_3 \]
for which \(Gt_2\) is an isomorphism, and \(Qt_1\circ s_1 = Qt_2\circ s_2\circ f\). One then sees that in the diagram
\begin{diagram*}
	Fx_1\ar[rrrrr,"\theta'_{x_1}"]\ar[dddd,"Ff"']\ar[dr,"Fs_1"] &       &       &       &       & Gx_1\ar[dddd,"Gf"]\ar[dl,"\sim"',"Gs_1"] \\
	     & FQy_1 \ar[rrr,"\theta_{y_1}"]\ar[dr,"FQt_1"] &       &       & GQy_1 \ar[dl,"GQt_1"] & \\
	     &       & FQy_3\ar[r,"\theta_{y_3}"] & GQy_3 &       & \\
	     & FQy_2 \ar[rrr,"\theta_{y_2}"]\ar[ur,"FQt_2"] &       &       & GQy_2 \ar[ul,"\sim"',"GQt_2"] & \\
	Fx_2\ar[rrrrr,"\theta'_{x_2}"]\ar[ur,"Fs_2"] &       &       &       &       & Gx_2\ar[ul,"\sim"',"Gs_2"]
\end{diagram*}
all the inner diagrams commute, hence the outer diagram commutes. Taking \(f=\id_x\), we see that any two choices of \(s_x\) yield the same \(\theta'_x\), and in general, the above
diagram shows we have a natural transformation.

We see that \(\theta'Q = \theta\) by the independence of choices. In particular, for any \(Qy\), we may choose the identity \(Qy \to Qy\) to see that \(\theta'_{Qy} = \theta_y\).
\end{enumerate}
We conclude that \(G\) is a right Kan extension of \(GQ\) along \(Q\). To see that it is absolute, let \(H\!:\calA\to\calA'\) be another functor. Then
\[ \calC\overset{Q}\longto\calC'\overset{HG}\longto\calA' \]
satisfies hypotheses (a) and (b), since \(H\) preserves isomorphisms. Therefore, \(HG\) is the right Kan extension of \(HGQ\) along \(Q\), and we are done.
\end{proof}

\begin{theorem}\label{thm:right-multiplicative-system-right-derived-functor-exists}
	Let \((\calC,W)\) be a relative category for which \(W\) forms a right multiplicative system. Consider a functor \(F\!:\calC\to\calD\), where \((\calD,E)\) is a relative category.
	Assume that there is a full subcategory \(\calC'\inj\calC\) satisfying the following conditions, where \(W' := \calC'\cap W\):
	\begin{enumerate}[label=(\alph*)]
	\item For any \(x\in\calC\), there is a weak equivalence \(x\to x'\) where \(x'\in\calC'\).
	\item For any \(s\in W'\), the morphism \(Fs\) is a weak equivalence in \(\calD\), i.e.\ \(Fs\in E\).
	\end{enumerate}
	Then \(F\) has an absolute total right derived functor \(\bfR F\) such that
	\[ (\calC'\inj \calC \to \ho(\calC) \overset{\bfR F}\to \ho(\calD)) \cong (\calC'\inj\calC\overset{F}\to\calD\to\ho(\calD)). \]
\end{theorem}
\begin{proof}
Let \(\iota\!:\calC'\inj\calC\) be the inclusion. By our assumptions, the functor \(\iota'\!:\ho(\calC')\to\ho(\calC)\) is an equivalence. Let \(Q\) be a quasi-inverse. Since \(F\) acts homotopically
on \(\calC'\), we have an induced functor \(F'\!:\ho(\calC')\to\ho(\calD)\). For any \(H\!:\ho(\calD)\to\calE\), we then have the diagram
\begin{diagram*}
	\calC'\ar[r,hook,"\iota"]\ar[d,"\gamma'"'] & \calC\ar[r,"F"]\ar[d,"\gamma"'] & \calD\ar[d,"\delta"] &  \\
	\ho(\calC')\ar[rr,bend right,"F'"']\ar[r,shift right,"\iota'"'] & \ho(\calC)\ar[l,shift right,"Q"']\ar[r,dashed] & \ho(\calD)\ar[r,"H"] & \calE.
\end{diagram*}
Now, for any \(G\!:\ho(\calC)\to\calE\) we have a chain of natural morphisms
\begin{align*}
	\Fun(\calC,\calE)(H\delta F,G\gamma) &\to \Fun(\calC',\calE)(H\delta F\iota,G\gamma\iota) \\
	&\cong \Fun(\calC',\calE)(F'\gamma',G\iota'\gamma') \\
	&\cong \Fun(\ho(\calC'),\calE)(HF',G\iota') \\
	&\cong \Fun(\ho(\calC),\calE)(HF'Q,G).
\end{align*}
It thus suffices to check that the first morphism is an isomorphism. This follows by applying Lemma \ref{lemma:kan-extension-criterion}; indeed, it clearly applies to
\[ \calC'\overset{\iota}\inj\calC\overset{\gamma}\to\ho(\calC) \]
by our assumptions and property (M1) of being right multiplicative, so that \(\gamma\) is an absolute right Kan extension of \(\gamma\iota\) along \(\iota\). By
absoluteness, \(HG\gamma\) is a right Kan extension of \(HG\gamma\iota\) along \(\iota\), which means precisely that the given morphism is bijective.

Setting \(H=\id_{\ho(\calD)}\), we see that \(F'Q = \bfR F\) is a total right derived functor, and the above computation shows it is also absolute.
\end{proof}

In this setting, we still get pseudofunctoriality. We sketch how this is done.
\begin{definition}
	Define the strict 2-category \(\underline{\cat{LMul}}\) by the following.
	\begin{itemize}[label=\(\star\)]
	\item On objects: tuples \((\calC,W,\calC')\) where \((\calC,W)\) is a relative category with \(W\) a left multiplicative system, and \(\calC'\) is a subcategory of \(\calC\)
	such that for all \(x\in\calC\) there is a weak equivalence \(x'\to x\) with \(x'\in\calC'\).
	\item On 1-morphisms: a morphism \((\calC,w,\calC')\to(\calD,E,\calD')\) is a functor \(F\!:\calC\to\calD\) such that \(F\calC' \subseteq \calD'\) and \(F|_{\calC'}\) is homotopical.
	\item On 2-morphisms: just take natural transformations.
	\end{itemize}
	Dually, define \(\underline{\cat{RMul}}\).
\end{definition}
\begin{theorem}
	We have pseudofunctors
	\[ \bfL\!:\underline{\cat{LMul}}\to\underline{\Cat},\quad \bfR\!:\underline{\cat{RMul}}\to\underline{\Cat} \]
	which
	\begin{enumerate}[label=(\arabic*)]
	\item send objects \((\calC,W,\calC')\) to \(\ho(\calC) = \calC[W^{-1}]\),
	\item send 1-morphisms \(F\) to \(\bfL F\) and \(\bfR F\), respectively, and
	\item send 2-morphisms \(F\To F'\) to the induced natural transformations \(\bfL F\To\bfL F'\) and \(\bfR F\To\bfR F'\), respectively.
	\end{enumerate}
\end{theorem}
\begin{proofsketch}
We examine the case of \(\underline{\cat{RMul}}\). There is lots of data to produce, and we only do one part, which is to give the natural isomorphism
\[ \bfR G\circ \bfR F \cong \bfR(G\circ F). \]
Let \(F\!:\calC\to\calD\) and \(G\!:\calD\to\calE\) be 1-morphisms in \(\underline{\cat{RMul}}\), and let \(\gamma\!:\calC\to\ho(\calC)\), \(\gamma'\!:\calD\to\ho(\calD)\) and \(\gamma''\!:\calE\to\ho(\calE)\)
be the localization functors. Then we have canonical natural transformations
\[ \gamma''\circ G\To \bfR G \circ \gamma',\quad \gamma'\circ F\To \bfR F\circ \gamma, \]
and thus may form the composite
\[ \gamma''\circ G\circ F \To \bfR G\circ\gamma'\circ F \To \bfR G\circ \bfR F\circ \gamma \]
which induces a canonical natural transformation
\[ \sigma\!: \bfR(G\circ F)\To\bfR G\circ\bfR F. \]
We must show this is a natural isomorphism, i.e.\ show that for all \(x\in\ho(\calC)\), \(\sigma_x\) is an isomorphism. Write \(x\cong \gamma(x_0)\), and choose a weak equivalence \(x_0\to x_0'\)
with \(x_0'\in\calC'\). The crux is that then \(\bfR Fx \cong Fx_0'\), and since \(F\calC'\subseteq\calD'\), we also have \(\bfR G(Fx_0') \cong GFx_0' \). Thus, \(\sigma_x\) factors
as a composition of isomorphisms, and as such is an isomorphism.
% consider the diagram
% \begin{diagram*}
% 	& \calC'\ar[dl,hook]\ar[rr,"F|_{\calC'}"]\ar[dd] & & \calD'\ar[dl,hook]\ar[rr,"G|_{\calD'}"]\ar[dd] & & \calE'\ar[dl,hook]\ar[dd] \\
% 	\calC\ar[dd] & & \calD\ar[from=ll,crossing over,"F" near end] & & \calE\ar[from=ll,crossing over,"G" near end] & \\
% 	& \ho(\calC')\ar[rr,dashed,"F'" near start]\ar[dl,shift right] & & \ho(\calD')\ar[rr,dashed,"G'" near start]\ar[dl,shift right,"\iota'"'] & & \ho(\calE')\ar[dl,shift right,"\iota''"'] \\
% 	\ho(\calC)\ar[rr,dashed,"\bfR F"]\ar[ur,shift right,"Q"'] & & \ho(\calD)\ar[from=uu,crossing over]\ar[rr,dashed,"\bfR G"]\ar[ur,shift right,"Q'"'] & & \ho(\calE)\ar[from=uu,crossing over]\ar[ur,shift right,"Q''"'] &
% \end{diagram*}
% where one notes that \(\bfR F = \iota'F'Q\) and \(\bfR G = \iota'' G' Q'\). By the assumptions on \(F\) and \(G\), we have an induced functor \(\ho(\calC')\to\ho(\calE')\) coming from the composition \(G|_{\calD'}\circ F|_{\calC'}\)
% and it is exactly the composition \(G'\circ F'\) (by uniqueness). Now,
% \[ \bfR G\circ \bfR F = \iota''G'Q' \circ \iota'F'Q \cong \iota''G'F'Q = \bfR(G\circ F). \]
% This gives us our desired natural transformation.
\end{proofsketch}


\subsection{Appendix: Kan extensions}
Kan extensions formalize the notion of a universal best approximation of an extension of a functor along another functor.
\begin{definition}
	Let \(F\!:\calC\to\calE\), \(K\!:\calC\to\calD\) be functors. A \emph{left Kan extension} of \(F\) along \(K\) is a functor is given by the data of a functor and natural transformation
	\begin{diagram*}[cramped]
		\calC\ar[rr,"F",""{below,name=A}]\ar[dr,"K"'] & & \calE \\
		& \calD\ar[ur,"\Lan_K{F}"'] & \ar[from=A,to=l,Rightarrow,"\eta",shorten=2mm]
	\end{diagram*}
	universal in the sense that any other functor with this data factors uniquely as below:
	\begin{center}
	\begin{tikzcd}
		\calC\ar[rr,"F",""{below,name=A}]\ar[dr,"K"'] & & \calE \\
		& \calD\ar[ur,"G"'] & \ar[from=A,to=l,Rightarrow,"\sigma",shorten=2mm]
	\end{tikzcd}\(\quad=\quad\)
	\begin{tikzcd}
		\calC\ar[rr,"F",""{below,name=A}]\ar[dr,"K"'] & & \calE \\
		& \calD\ar[ur,bend left,"\Lan_K{F}" description,""{right,name=B}]\ar[ur,shift right,bend right,"G"',""{right,name=C}] & \ar[from=A,to=l,Rightarrow,shift right,"\eta"',shorten=2mm]\ar[from=B,to=C,Rightarrow,"\exists!"']
	\end{tikzcd}
	\end{center}
	i.e.\ there is a unique natural transformation \(\sigma'\!:\Lan_K{F}\to G\) such that \(\sigma = \sigma'K\circ\eta=\sigma\).

	The dual notion of a \emph{right Kan extension} \((\Ran_K{F},\varepsilon)\) of \(F\) along \(K\) is given by reversing the directions of the natural transformations.
\end{definition}

Note that a functor which has a genuine extension ``on the nose'' may not have an actual Kan extension, as the aforementioned extension may not be appropriately universal.

\begin{exercise*}
	Let \(\calC\) be a category, and let \(x\!:[0]\to\calC\) be the functor picking out an object \(x\in\calC\); accordingly, let \(*\!:[0]\to\Set\) be the functor picking out a one-point set.
	Show that the left Kan extension \(\Lan_x{*}\) is given by \(\calC(x,-)\).
\end{exercise*}

Left (or right) Kan extensions on their own are not generally very well-behaved. On the other hand, there are slight variations on the notion which has more desirable properties.

\begin{definition}
	Consider functors \(F\!:\calC\to\calE\) and \(K\!:\calC\to\calD\). A functor \(G\!:\calE\to\calE'\) is said to \emph{preserve} the left Kan extension \((\Lan_K{F},\eta)\) of \(F\) if
	\((G\circ\Lan_K{F},G\eta)\) is a left Kan extension of \(G\circ F\) along \(K\),
	\begin{diagram*}[cramped]
		\calC\ar[rr,"F",""{below,name=A}]\ar[dr,"K"'] & & \calE\ar[r,"G"] & \calE'. \\
		& \calD\ar[ur,"\Lan_K{F}"']\ar[urr,shift right,bend right,"\Lan_K{GF}"'] & \ar[from=A,to=l,Rightarrow,"\eta"',shorten=2mm] &
	\end{diagram*}
	One says that the left Kan extension \((\Lan_K{F},\eta)\) is \emph{pointwise} if it is preserved by all corepresentable functors \(\calE(x,-)\), \(x\in\calE\), and \emph{absolute} if it preserved by all functors with domain \(\calE\).
\end{definition}

All concepts in 1-category theory can be expressed in terms of Kan extensions. Notably, (co)limits and adjoints can be formulated in this language.


%!TEX root = ../lectures.tex

\section{Triangulated categories}\label{lecture:triangulated-categories}
Homological algebra is essentially concerned with properties of the cohomology of chain complexes, and so is primarily interested in the category of chain complexes \(\Ch(\calA)\) in an Abelian
category \emph{considered up to quasi-isomorphism.} The way this is done, classically, is by passing from the category of chain complexes \(\Ch(\calA)\) to the \emph{derived category} \(\sfD(\calA)\),
which is obtained from \(\Ch(\calA)\) by formally inverting the quasi-isomorphisms (i.e.\ morphisms which induce an isomorphism on cohomology) through the machinery of Lecture \ref{section:localization-of-categories}.

In the process of forming the derived category \(\sfD(\calA)\), many properties of \(\calA\) (and \(\Ch(\calA)\)) are destroyed: whereas the latter two are Abelian categories, the former is not.
However, one nonetheless sees a shadow of its Abelian origins; for one, \(\sfD(\calA)\) is an additive category. Furthermore, it sees the remains of kernels and cokernels with nice properties, though
they are no longer canonical. In the 1960s, Grothendieck \& Verdier developed the formalism of \emph{triangulated categories} to deal with the problem of encapsulating the properties exhibited
by derived categories.

\subsection{Pre-triangulated categories \& triangulated categories}
The basic initial concept of triangulated categories is to mimick the property that Abelian categories admit \emph{exact sequences.} In the context of triangulated categories, the analogous
concept is a \emph{triangle.}
\begin{definition}
	Let \(\calT\) be an additive category equipped with an automorphism \(\Sigma\!:\calT\to\calT\). A \emph{triangle} for \(\Sigma\) is a sequence of maps
	\[ x \overset{f}\to y \overset{g}\to z \overset{h}\to \Sigma x \]
	in \(\calT\). One says that a triangle as above is a \emph{candidate triangle} if \(g\circ f = 0\) and \(h\circ g = 0\).

	A \emph{morphism of triangles} from \(x\to y \to z \to \Sigma x\) to \(x'\to y'\to z'\to \Sigma x'\) is a triple of maps \((a,b,c)\) in a commutative diagram
	\begin{diagram*}
		x\ar[r]\ar[d,"a"] & y\ar[r]\ar[d,"b"] & z\ar[r]\ar[d,"c"] & \Sigma x.\ar[d,"\Sigma a"] \\
		x'\ar[r] & y'\ar[r] & z'\ar[r] & \Sigma x'
	\end{diagram*}
\end{definition}

Recall that we briefly discussed \emph{Quillen exact categories} in Appendix \ref{appendix:quillen-exact-categories}. These were additive categories which admitted some kernels and cokernels,
in the form of certain distinguished exact sequences. Triangulated categories are very similar: one specifies a distinguished class of triangles to be considered ``exact.''

\begin{definition}
	A \emph{(Neeman) pre-triangulated category} is a triple \((\calT, \Sigma, \calE)\) of an additive category \(\calT\), an automorphism \(\Sigma\!:\calT\to\calT\), and a set \(\calE\) of triangles for \(\Sigma\), called the
	\emph{distinguished triangles} or \emph{exact triangles,} closed under isomorphisms of triangles. These are required to satisfy the following axioms:
	\begin{enumerate}[label=(TR\arabic*)]
	\item For any \(x\in\calT\), \(x\overset{\id}\to x \to 0 \to \Sigma x\) is a distinguished triangle, and for any morphism \(f\!:x\to y\) in \(\calT\) there is a distinguished triangle
	\[ x\overset{f}\to y \to z \to \Sigma x. \]
	\item The triangle
	\[ x\overset{u}\to y\overset{v}\to z\overset{w}\to \Sigma x \]
	is distinguished if any only if the triangle
	\[ y\overset{v}\to z\overset{w}\to \Sigma x\overset{-\Sigma u}\to \Sigma y \]
	is distinguished.
	\item For any commutative diagram
	\begin{diagram*}
		x\ar[r]\ar[d,"a"] & y\ar[r]\ar[d,"b"] & z\ar[r]\ar[d,dashed] & \Sigma x\ar[d,"\Sigma a"] \\
		x'\ar[r] & y'\ar[r] & z'\ar[r] & \Sigma x'
	\end{diagram*}
	of solid arrows where the rows are distinguished triangles, a dashed arrow exists making the diagram into a morphism of triangles. It need not be unique!
	\end{enumerate}
	We often abuse notation by saying that \(\calT\) is a pre-triangulated category, leaving the rest of the data implicit. We call the functor \(\Sigma\) the \emph{shift functor.}
\end{definition}
\begin{remark}
	One will often see distinguished triangles notated as
	\[ x\to y\to z \overset{+1}\to \]
	as shorthand. There is also other common notation used for the shift functor, such as \(T\) or \([1]\), the latter being written as e.g.\ \(x[1]\). We will likely switch between
	different notations depending on the context.
\end{remark}
\begin{terminology}
	Consider a pre-triangulated category \(\calT\) and a morphism \(f\!:x\to y\). An object \(z\) with a morphism \(g\!:y\to z\) as in (TR1), i.e.\ sitting in a distinguished triangle
	\[ x\overset{f}\to y \overset{g}\to z \to \Sigma x \]
	is called a \emph{cone} of \(f\). Dually, one calls \(f\) a \emph{cocone} of \(g\).
\end{terminology}
\begin{remark}
	One should think of a cone of \(f\) as roughly like a cokernel (which we will justify the dual of in Corollary \ref{corollary:weak-kernel-property-of-cocones}), except \emph{up to homotopy} in some sense. Notably, it is
	absolutely not unique! On the other hand, we will see that it certainly is unique up to isomorphism (see Proposition \ref{prop:triangulated-five-lemma}), just not a canonical one.

	One can constrast this situation with the one for exact categories, which are similar except that the cokernel actually is canonical.
\end{remark}
\begin{exercise}\label{exercise:partial-morphism-of-dts-fills}
	Let \(\calT\) be a pre-triangulated category. Show that an analogue of axiom (TR3) holds for any ``partial morphism of distinguished triangles'' missing only one arrow.
\end{exercise}

\begin{definition}
	Let \(\calT\) and \(\calT'\) be triangulated categories. A \emph{lax-triangulated functor} is a pair \((F,\sigma)\) of an additive functor \(F\!:\calT\to\calT'\) and a natural transformation \(\sigma\!:F\circ\Sigma \to \Sigma\circ F\)
	such that for all distinguished triangles
	\[ x \overset{f}\to y \overset{g}\to z \overset{h}\to \Sigma x\text{ d.t.\ in }\calT\quad\implies\quad Fx\overset{Ff}\longto Fy\overset{Fg}\longto Fz\overset{\sigma_{z}\circ Fh}\longto \Sigma Fx\text{ d.t.\ in }\calT' \]
	We say a lax-triangulated functor is \emph{triangulated} (or \emph{strict}) if \(\sigma\) is a natural isomorphism.
\end{definition}
\begin{remark}
	Triangulated functors are sometimes called \emph{exact} functors, in analogy with exact functors between Abelian categories. We will not use this terminology.
\end{remark}

\begin{lemma}
	Let \(\calT\) be a pre-triangulated category. Then any distinguished triangle is a candidate triangle.
\end{lemma}
\begin{proof}
Consider a distinguished triangle
\[ x\overset{f}\to y\overset{g}\to z\to \Sigma x. \]
By (TR2), it suffices to show that \(g\circ f = 0\). To do this, apply (TR1) and (TR3) to see that we have a morphism of triangles
\begin{diagram*}
	x\ar[r,equal]\ar[d,equal] & x\ar[d,"f"]\ar[r] & 0\ar[r]\ar[d] & \Sigma x\ar[d,equal] \\
	x\ar[r,"f"] & y\ar[r,"g"] & z\ar[r] & \Sigma x
\end{diagram*}
from which it follows that \(g\circ f = 0\).
\end{proof}

\begin{definition}
	A \emph{triangulated category} is a pre-triangulated category \(\calT\) in which the distinguished triangles satisfy the following additional axiom relating the cones
	of composable morphisms with those of their composition:
	\begin{itemize}[label=(TR4)]
	\item Let \(f\!:x\to y\) and \(g\!:y\to z\) be morphisms in \(\calT\), lying in a commutative diagram of solid arrows as below
	\begin{diagram*}
		x\ar[r,"f"']\ar[d,equal] & y\ar[d,"g"] \ar[r] & w\ar[r]\ar[d,dashed] & \Sigma x\ar[d,equal] \\
		x\ar[r,"g\circ f"]\ar[d,"f"'] & z\ar[r]\ar[d,equal] & w' \ar[r]\ar[d,dashed] & \Sigma x\ar[d,"\Sigma f"] \\
		y\ar[r,"g"]\ar[d] & z\ar[r]\ar[d] & w''\ar[r]\ar[d,equal] & \Sigma y\ar[d] \\
		w\ar[r,dashed] & w'\ar[r,dashed] & w''\ar[r,dashed] & \Sigma w
	\end{diagram*}
	in which the rows are distinguished triangles. Then there exist dashed arrows as indicated making the diagram commute and making the bottom row a distinguished triangle.
	\end{itemize}
\end{definition}
\begin{remark}
	One can think of (TR4) as analogous to the third isomorphism theorem. Heuristically, cones are like cokernels, so if we set \(w = y/x\), \(w'=z/x\), and \(w'' = z/y\),
	then (TR4) says that \((z/x)/(y/x) = z/y\).
\end{remark}

\begin{exercise}
	Let \(\calT\) be a (pre-)triangulated category. Show that \(\calT^\op\) can be endowed with the structure of a (pre-)triangulated category, where the shift is given by the inverse of the shift on \(\calT\).
\end{exercise}

\subsection{Cohomological functors \& the ``triangulated Yoneda lemma''}
One of the magical miracles that pre-triangulated categories allow us is a way to formulate what it means for a functor to be cohomological.

\begin{definition}
	Let \(\calT\) be a triangulated category, and \(\calA\) an Abelian category. Consider an additive functor \(H\!:\calT\to\calA\). We say \(H\) is \emph{cohomological}
	if for any distinguished triangle
	\[ x\to y\to z \to \Sigma x \]
	in \(\calT\), the sequence
	\[ Hx\to Hy\to Hz \]
	in \(\calA\) is exact.
\end{definition}
\begin{remark}
	The mantra is: cohomological functors take short exact sequences to long exact sequences. The above makes this precise; given a ``short homotopy exact sequence''
	\[ x\to y\to z \to \Sigma x, \]
	we extend this to a long sequence of morphisms
	\[ \cdots \to \Sigma^{-1}y \to \Sigma^{-1}z\to x\to y \to z\to \Sigma x \to \Sigma y \to \cdots \]
	and after applying \(H\), obtain a long exact sequence
	\[ \cdots \to H\Sigma^{-1}y \to H\Sigma^{-1}z\to Hx\to Hy \to Hz\to H\Sigma x \to H\Sigma y \to \cdots \]
	as expected.
\end{remark}
\begin{proposition}[``Triangulated Yoneda lemma'']\label{prop:pre-triangulated-representable-functors-are-cohomological}
	Let \(\calT\) be a pre-triangulated category, and let \(x\in\calT\). Then the functors
	\[ \calT(x,-)\!:\calT\to\Ab,\quad \calT(-,x)\!:\calT^\op\to\Ab \]
	are cohomological.
\end{proposition}
\begin{proof}
We prove the proposition for \(\calT(x,-)\); the other case is essentially dual. Fix a distinguished triangle
\[ x'\overset{f}\to y'\overset{g}\to z'\to \Sigma x' \]
and consider the sequence
\[ \calT(x,x')\overset{f_*}\longto\calT(x,y')\overset{g_*}\longto\calT(x,z'). \]
Since \(g\circ f = 0\), this is a complex, i.e.\ \(\img(f_*)\subseteq\ker(g_*)\). For the other inclusion, let \(v\!:x\to y'\) be such that \(g\circ v = 0\). We need to
find a map \(u\!:x\to x'\) such that \(u = f\circ v\). Applying Exercise \ref{exercise:partial-morphism-of-dts-fills}, we have a filling dashed arrow in the solid diagram
\begin{diagram*}
	x\ar[r,equal]\ar[d,dashed,"u"] & x\ar[d,"v"]\ar[r] & 0\ar[r]\ar[d] & \Sigma x\ar[d,dashed,"\Sigma u"] \\
	x'\ar[r,"f"] & y'\ar[r,"g"] & z'\ar[r] & \Sigma x'
\end{diagram*}
providing the desired map \(u\!:x\to x'\).
\end{proof}
\begin{corollary}\label{corollary:weak-kernel-property-of-cocones}
	Let \(\calT\) be a pre-triangulated category, and let \(x\overset{f}\to y\overset{g}\to z\to \Sigma x\) be a distinguished triangle.
	Then the morphism \(f\) satisfies the following ``weak'' universal property with respect to \(g\): for any morphism \(u_0\!:x_0\to y\) such that \(g\circ u = 0\),
	there exists a morphism \(u\!:x_0\to x\) making the diagram
	\begin{diagram*}
		 & x_0\ar[dl,dashed,"u"']\ar[d,"u_0"]\ar[dr,"0"] & & \\
		x\ar[r] & y\ar[r] & z\ar[r] & \Sigma x
	\end{diagram*}
	commute.
\end{corollary}
\begin{terminology}
	A morphism \(f\!:x\to y\) satisfying the property described above for a morphism \(g\!:y\to z\) is called a \emph{weak kernel} for \(g\). Dually, one defines the notion of a \emph{weak cokernel} for \(f\).
\end{terminology}
\begin{exercise}
	Write out a proof for the unproven part of Proposition \ref{prop:pre-triangulated-representable-functors-are-cohomological}, and deduce the dual of Corollary \ref{corollary:weak-kernel-property-of-cocones}.
	That is, show that cones are weak cokernels.
\end{exercise}

\begin{remark}
	The above implies that distinguished triangles form \emph{weak kernel/cokernel pairs,} analogous to the observation that the distinguished exact sequences in an exact category
	form distinguished kernel/cokernel pairs.
\end{remark}

\begin{proposition}[``Triangulated five lemma'']\label{prop:triangulated-five-lemma}
	Let \(\calT\) be a pre-triangulated category. Consider a morphism
	\begin{diagram*}
		x\ar[r]\ar[d,"a"] & y\ar[r]\ar[d,"b"] & z\ar[r]\ar[d,"c"] & \Sigma x\ar[d,"\Sigma a"] \\
		x'\ar[r] & y'\ar[r] & z'\ar[r] & \Sigma x'
	\end{diagram*}
	of distinguished triangles. If any two of \(a,b,c\) are isomorphisms, then so is the third.
\end{proposition}
\begin{proof}
By applying (TR2), it suffices to show that if \(a\) and \(b\) are isomorphisms, then so is \(c\). Let \(w\in\calT\). Applying Proposition \ref{prop:pre-triangulated-representable-functors-are-cohomological}, we have a
commutative diagram
\begin{diagram*}
	\calT(w,x)\ar[r]\ar[d,"a_*"] & \calT(w,y)\ar[d,"b_*"]\ar[r] & \calT(w,z)\ar[d,dashed,"c_*"]\ar[r] & \calT(w,\Sigma x)\ar[r]\ar[d,"(\Sigma a)_*"] & \calT(w,\Sigma y)\ar[d,"(\Sigma b)_*"] \\
	\calT(w,x')\ar[r] & \calT(w,y')\ar[r] & \calT(w,z')\ar[r] & \calT(w,\Sigma x')\ar[r] & \calT(w,\Sigma y')
\end{diagram*}
with exact rows. By assumption, all arrows are isomorphisms except the dashed one, so by the classical five lemma, the dashed morphism is also an isomorphism. Since this holds for all \(w\),
the Yoneda lemma tells us that \(c\) is an isomorphism.
\end{proof}

\begin{corollary}
	Let \(\calT\) be a pre-triangulated category, and let \(f\!:x\to y\) be a morphism in \(\calT\). Then the following are equivalent.
	\begin{enumerate}[label=(\arabic*)]
	\item \(f\) is an isomorphism.
	\item The triangle \(x\overset{f}\to y\to 0 \to \Sigma x\) is distinguished.
	\end{enumerate}
\end{corollary}
\begin{proof}
We have a morphism of triangles
\begin{diagram*}
	x\ar[r,equal]\ar[d,equal] & x\ar[r]\ar[d,"f"] & 0\ar[d,equal]\ar[r] & \Sigma x\ar[d,equal] \\
	x\ar[r,"f"] & y\ar[r] & 0 \ar[r] & \Sigma x
\end{diagram*}
so by the triangulated five lemma, Proposition \ref{prop:triangulated-five-lemma}, the morphism \(f\) is an isomorphism if and only if the lower triangle is distinguished.
\end{proof}

\subsection{Uniqueness issues}
In the axioms for a pre-triangulated category, there are many postulates of the existence of some object. However, as emphasized, there is no guarantee of uniqueness.
We explore here some situations where one can get a unique choice. These have significance because one can use these simple criteria in order to build functors
in favourable situations.
\begin{lemma}\label{lemma:simple-tr3-uniqueness}
	Let \(\calT\) be a pre-triangulated category, and suppose we have a partial morphism of distinguished triangles
	\begin{diagram*}
		x\ar[r]\ar[d] & y\ar[r,"g"]\ar[d] & z \ar[r,"h"]\ar[d,dashed] & \Sigma x\phantom. \ar[d] \\
		x' \ar[r] & y'\ar[r,"g'"] & z'\ar[r,"h'"] & \Sigma x'.
	\end{diagram*}
	If \(\calT(\Sigma x, z') = 0\) or \(\calT(z,y') = 0\), then there exists a unique dashed morphism making the diagram commute.
\end{lemma}
\begin{proof}
By (TR3), some morphism exists, and we must show it is unique. Let \(a,b\!:z\to z'\) be two such morphisms. Then the morphism \(a-b\) satisfies
\begin{diagram*}
	x\ar[r]\ar[d] & y\ar[r,"g"]\ar[d]\ar[dr,"0"] & z \ar[r,"h"]\ar[d,"a-b"]\ar[dr,"0"] & \Sigma x \ar[d] \\
	x' \ar[r] & y'\ar[r,"g'"] & z'\ar[r,"h'"] & \Sigma x'
\end{diagram*}
so that Corollary \ref{corollary:weak-kernel-property-of-cocones} (and its dual) yield morphisms \(\Sigma x\to z'\) and \(z\to y'\) factorizing \(a-b\). Thus, if
either of the assumptions of the lemma are true, \(a-b\) factors through the zero morphism, so \(a=b\).
\end{proof}
Here is a more complex condition, which we can use to produce a useless but neat condition for a cone to be unique.
\begin{proposition}\label{prop:complicated-tr3-uniqueness}
	Let \(\calT\) be a triangulated category, and suppose we have a partial morphism of distinguished triangles
	\begin{diagram*}
		x\ar[r,"f"]\ar[d,"a"] & y\ar[r,"g"]\ar[d,"b"] & z \ar[r,"h"]\ar[d,dashed] & \Sigma x \ar[d,"\Sigma a"] \\
		x' \ar[r,"f'"] & y'\ar[r,"g'"] & z'\ar[r,"h'"] & \Sigma x'
	\end{diagram*}
	and assume that \(\calT(y,x') = 0\) and \(\calT(\Sigma x, y') = 0\). Then there exists a unique dashed morphism making the diagram commute.
\end{proposition}
\begin{proof}
Again, by (TR3), a morphism exists. We may assume that \(a=0\) and \(b=0\) by considering the difference of induced two dashed morphisms. Thus, we must show that
given a candidate dashed morphism \(c\!:z\to z'\), we have \(c=0\). Expanding these assumptions and applying Corollary \ref{corollary:weak-kernel-property-of-cocones}, we have a (non-commuting) diagram
\begin{diagram*}
	x\ar[r,"f"]\ar[d,"0"] & y\ar[r,"g"]\ar[d,"0"] & z \ar[r,"h"]\ar[d,"c"]\ar[dl,"p"'] & \Sigma x\ar[dl,"q"'] \ar[d,"0"] \\
	x' \ar[r,"f'"] & y'\ar[r,"g'"] & z'\ar[r,"h'"] & \Sigma x'
\end{diagram*}
which provides for us, via an application of (TR3), a morphism of distinguished triangles
\begin{diagram*}
	y\ar[r,"g"]\ar[d,dashed,"r"] & z\ar[r,"h"]\ar[d,"p"] & \Sigma x \ar[r,"-\Sigma f"]\ar[d,"q"] & \Sigma y \ar[d,dashed,"\Sigma r"] \\
	x' \ar[r,"f'"] & y'\ar[r,"g'"] & z'\ar[r,"h'"] & \Sigma x'
\end{diagram*}
but since \(\calT(y,x') = 0\), we see that \(r=0\). Applying Corollary \ref{corollary:weak-kernel-property-of-cocones} again, we see that \(p\) factors through a
map \(\Sigma x\to y'\), but since \(\calT(\Sigma x,y')=0\), this means \(p=0\). Since \(c\) factors through \(p\), we have \(c=0\).
\end{proof}
\begin{corollary}
	Let \(f\!:x\to y\) be a morphism in a pre-triangulated category \(\calT\), and assume \(\calT(y,x) = 0\) and \(\calT(\Sigma x,y)=0\). Then \(f\) has a cone
	unique up to unique isomorphism of distinguished triangles.
\end{corollary}
\begin{proof}
Suppose we have two distinguished triangles,
\begin{diagram*}
	x\ar[r,"f"]\ar[d,equal] & y\ar[r]\ar[d,equal] & z\ar[r]\ar[d,dashed,"\sim" labl] & \Sigma x\ar[d,equal] \\
	x\ar[r,"f"] & y\ar[r] & z' \ar[r] & \Sigma x
\end{diagram*}
with the dashed morphism induced by (TR3). By Proposition \ref{prop:complicated-tr3-uniqueness}, the assumptions of this corollary imply that the morphism is unique,
and it is an isomorphism by the triangulated five lemma, Proposition \ref{prop:triangulated-five-lemma}.
\end{proof}

Given a sequence of maps
\[ x\to y \to z, \]
one may wonder how many ways this occurs as part of a distinguished triangle. In general, there could be many ways to extend it, but under the assumption of Lemma \ref{lemma:simple-tr3-uniqueness},
one can make sure it is unique.
\begin{lemma}\label{lemma:simple-uniqueness-of-cone-shift-map}
	Let \(\calT\) be a pre-triangulated category, and suppose we have a pair of maps \(x\overset{f}\to y\overset{g}\to z\) such that \(g\circ f = 0\). Suppose that
	\(\calT(\Sigma x,z) = 0\). Then there is at most one distinguished triangle
	\[ x\overset{f}\to y\overset{g}\to z \to \Sigma x. \]
\end{lemma}
\begin{proof}
Suppose that we have two; let \(h_i\!:z\to\Sigma x\) be the associated maps completing the sequence to an exact triangle. Then we have a partial morphism
\begin{diagram*}
	x\ar[d,equal]\ar[r,"f"] & y\ar[d,equal]\ar[r,"g"] & z\ar[d,dashed,"c"',"\sim" labl]\ar[r,"h_1"] & \Sigma x\ar[d,equal] \\
	x\ar[r,"f"] & y\ar[r,"g"] & z\ar[r,"h_2"] & \Sigma x
\end{diagram*}
which induces a unique dashed isomorphism by Lemma \ref{lemma:simple-tr3-uniqueness}. In particular, \(c\circ g = g\), hence
\((\id_y-c)\circ g = 0\), so Corollary \ref{corollary:weak-kernel-property-of-cocones} says that \(\id_y-c\) factors through a morphism \(\Sigma x\to z\).
By assumption, any such morphism is zero, so \(c = \id_y\) and \(h_1 = h_2\).
\end{proof}

Conditions of the form \(\calT(\Sigma x,z)=0\) turn out to be practical, as such \emph{orthogonality} properties show up in other situations.

\subsection{Interesting triangles}
\begin{proposition}
	Let \(\calT\) be a pre-triangulated category, and let \(I\) be some indexing set. Suppose that we have a family of distinguished triangles
	\[ \left(x_i\overset{f_i}\longto y_i\overset{g_i}\longto z_i\overset{h_i}\longto \Sigma x_i\right)_{i\in I} \]
	in \(\calT\) for which the coproducts
	\[ \coprod_{i\in I}x_i,\quad \coprod_{i\in I}y_i,\quad \coprod_{i\in I}z_i \]
	exist. Then the triangle
	\[ \coprod_{i\in I}x_i \xrightarrow{\coprod_i f_i} \coprod_{i\in I}y_i\xrightarrow{\coprod_i g_i}\coprod_{i\in I}z_i\xrightarrow{\coprod_ih_i} \coprod_{i\in I}\Sigma x_i \]
	is a distinguished triangle.
\end{proposition}
\begin{proof}
The strategy is to take a cone of the left morphism and show that one gets the correct output. We have a distinguished triangle
\[ \coprod_{i\in I}x_i \xrightarrow{\coprod_i f_i} \coprod_{i\in I}y_i\longto z' \longto \coprod_{i\in I}\Sigma x_i \]
by (TR1), and by (TR3) induced morphisms of distinguished triangles
\begin{diagram*}
	x_i \ar[r,"f_i"]\ar[d,hook] & y_i\ar[r,"g_i"]\ar[d,hook] & z_i\ar[d,dashed] \ar[r, "h_i"] & \Sigma x_i\ar[d,hook] \\
	\coprod_{i\in I}x_i \ar[r,"\coprod_i f_i"] & \coprod_{i\in I}y_i\ar[r] & z' \ar[r] & \coprod_{i\in I}\Sigma x_i 
\end{diagram*}
which combine into a morphism of triangles
\begin{diagram*}
	\coprod_{i\in I}x_i \ar[r,"\coprod_{i}f_i"]\ar[d,equal] & \coprod_{i\in I}y_i\ar[r,"\coprod_{i}g_i"]\ar[d,equal] & \coprod_{i\in I}z_i\ar[d] \ar[r, "\coprod_{i}h_i"] & \coprod_{i\in I}\Sigma x_i\ar[d,equal] \\
	\coprod_{i\in I}x_i \ar[r,"\coprod_i f_i"] & \coprod_{i\in I}y_i\ar[r] & z' \ar[r] & \coprod_{i\in I}\Sigma x_i 
\end{diagram*}
and after application of \(\calT(-,w)\) for some \(w\in\calT\) and unraveling one more step, we have a diagram
\begin{diagram*}
	\calT(\coprod_{i}\Sigma y_i,w) \ar[r]\ar[d,equal] & \calT(\coprod_{i}\Sigma x_i,w)\ar[d,equal]\ar[r] & \calT(\coprod_{i}z_i,w)\ar[d]\ar[r] & \calT(\coprod_{i}y_i,w)\ar[r]\ar[d,equal] & \calT(\coprod_{i\in I}x_i,w)\ar[d,equal] \\
	\calT(\coprod_{i}\Sigma y_i,w) \ar[r] & \calT(\coprod_{i}\Sigma x_i,w)\ar[r] & \calT(z',w)\ar[r] & \calT(\coprod_{i}y_i,w)\ar[r] & \calT(\coprod_{i\in I}x_i,w)
\end{diagram*}
where the bottom row is exact. Commuting out the coproducts on the top row and using that the original triangles are exact yields that the top row is exact, so the five lemma implies that the middle vertical
morphism is an isomorphism. By the Yoneda lemma, we deduce that the morphism
\[ \coprod_{i\in I}z_i \to z \]
is an isomorphism, and since the class of distinguished triangles is closed under isomorphism, we are done.
\end{proof}

While the above may seem somewhat innocuous, it has the following remarkable corollary which is actually tremendously useful.

\begin{corollary}\label{corollary:direct-sum-triangle}
	Let \(\calT\) be a triangulated category, and let \(x,y\in \calT\). Then the triangle
	\[ x\inj x\oplus y \sur y \overset{0}\to \Sigma x \]
	is distinguished. In particular, any distinguished triangle
	\[ x\to e\to y\overset{0}\to \Sigma x \]
	is isomorphic to the first distinguished triangle.
\end{corollary}
\begin{proof}
The first statement follows by taking the direct sum of the two distinguished triangles
\[ x\overset{\id}\to x\to 0 \to \Sigma x,\quad 0\to y\overset{\id}\to y \to \Sigma 0, \]
where we note that \(\Sigma 0 = 0\). For the second, construct the obvious partial morphism of triangles and apply (TR3) together with Proposition \ref{prop:triangulated-five-lemma}.
\end{proof}

The immense value of this corollary is that it allows us to identify an element as a direct summand of another by exhibiting the existence of a suitable distinguished triangle.
This turns out to be surprisingly doable in many interesting situations.

\subsection{Appendix: The five lemma}
One of the most frequently used results in homological algebra is the \emph{five lemma.} We dedicate this appendix to proving it.
The five lemma is primarily useful to us because it allows us to prove Proposition \ref{prop:triangulated-five-lemma}, but it also has relevance when understanding
extensions in Abelian categories. We will aim to prove a generalization of the five lemma found in \cite{kashiwara-schapira-book}. We begin with a prerequisite lemma.

\begin{lemma}\label{lemma:five-lemma-technical-necessary-lemma}\textup{\cite[Lemma 8.3.12]{kashiwara-schapira-book}}
	Let \(\calA\) be an Abelian category, and suppose we have morphisms \(x' \overset{f}\to x \to \overset{g}\to x'' \) in \(\calA\) such that \(g\circ f = 0\).
	Then the following are equivalent.
	\begin{enumerate}[label=(\arabic*)]
		\item The pair of morphisms form an exact sequence, i.e.\ the canonical map \(\img{f} \to \ker{f}\) is an isomorphism.
		\item For any \(h\!:z\to x\) such that \(g\circ h = 0\), there is an epimorphism \(f'\!:z'\to z\) and a commutative diagram
		\begin{diagram*}
			z'\ar[r,two heads,"f'"]\ar[d] & z\ar[dr,"0"]\ar[d,"h"] & \\
			x'\ar[r,"f"] & x\ar[r,"g"] & x''.
		\end{diagram*}
	\end{enumerate}
\end{lemma}
\begin{proof}
(1) \(\Rightarrow\) (2). The maps \(f\!:x'\to x\) and \(h\!:z\to x\) both factor through \(\ker{g}\inj x\). Thus, we set \(z' := x'\times_{\ker{g}}z\).
By exactness, \(x'\to\ker{g}\) is an epimorphism, and pullbacks of epimorphisms in Abelian categories are epimorphisms, so the map \(z'\to z\) is an epimorphism.

(2) \(\Rightarrow\) (1). Take \(z = \ker{g}\) and let \(h\!:\ker{g}\inj x\) be the canonical inclusion. We find an epimorphism \(z'\sur\ker{g}\),
but then the composition \(z'\to x'\to\ker{g}\) is also necessarily epic, which implies that \(x'\to\ker{g}\) is epic, so \(\img{f}\cong\ker{g}\).
\end{proof}

\begin{lemma}\label{lemma:generalized-five-lemma}\textup{\cite[Lemma 8.3.13]{kashiwara-schapira-book}}
	Let \(\calA\) be an Abelian category, and consider a commutative diagram
	\begin{diagram*}
		x^0 \ar[r]\ar[d,"f^0"] & x^1 \ar[r]\ar[d,"f^1"] & x^2\ar[r]\ar[d,"f^2"] & x^3 \ar[d,"f^3"] \\
		y^0 \ar[r] & y^1 \ar[r] & y^2\ar[r] & y^3 
	\end{diagram*}
	where
	\begin{enumerate}[label=(\roman*)]
		\item each row is a complex (i.e.\ adjacent morphisms compose to zero), and
		\item \(x^1 \to x^2 \to x^3\) and \(y^0 \to y^1 \to y^2\) are exact.
	\end{enumerate}
	Then the following statements hold.
	\begin{enumerate}[label=(\arabic*)]
		\item If \(f^0\) is an epimorphism and \(f^1\), \(f^3\) are monomorphisms, then \(f^2\) is a monomorphism.
		\item If \(f^3\) is a monomorphism and \(f^0\), \(f^2\) are epimorphisms, then \(f^1\) is an epimorphism.
	\end{enumerate}
\end{lemma}
\begin{proof}
The statements of (1) and (2) are dual, so it suffices to prove (1). As we endeavour to prove \(f^2\) is a monomorphism, suppose we have a morphism
\(a\!:z\to x^2\) such that \(f^2\circ a = 0\). We must prove that \(a=0\). By the commutativity of the diagram, we have
\[ (z \overset{a}\to x^2 \to x^3 \overset{f^3}\inj y^3) = (z \overset{a}\to x^2 \overset{f^2} \to y^2 \to y^3) = 0 \implies (z \overset{a}\to x^2 \to x^3) = 0. \]
Using Lemma \ref{lemma:five-lemma-technical-necessary-lemma}, we find some epimorphism \(z^1 \sur z\) sitting in the solid diagram
\begin{diagram*}
	w\ar[rr,two heads,dashed]\ar[ddr,dashed] & & z^1\ar[d,"a^1"']\ar[r,two heads] & z\ar[d,"a"]\ar[dr,"0"] & \\
	& x^0 \ar[r]\ar[d,two heads,"f^0"] & x^1 \ar[r]\ar[d,hook,"f^1"] & x^2\ar[r]\ar[d,"f^2"] & x^3 \ar[d,hook,"f^3"] \\
	& y^0 \ar[r] & y^1 \ar[r] & y^2\ar[r] & y^3 
\end{diagram*}
where we observe that
\[ (z^1 \overset{a^1}\to x^1 \overset{f^1}\inj y^1 \to y^2) = (z^1\sur z \overset{a}\to x^2 \overset{f^2}\to y^2) = 0. \]
Applying Lemma \ref{lemma:five-lemma-technical-necessary-lemma} again, we find an epimorphism \(w\sur s^1\) as indicated by the dashed morphisms.

Now, here is a slightly tricky bit: let \(z^0 := w\times_{y^0}x^0\) be the pullback of \(f^0\) along \(w\to y^0\). Since pullbacks of epimorphims are epic (since \(\calA\) is Abelian), we have an epimorphism
\(z^0 \sur w\) such that \((z^0\sur w \to y^0) = z^0 \to x^0 \overset{f^0}\sur y^0\). As a result, we have a commutative diagram
\begin{diagram*}
	z^0\ar[d,"a^0"']\ar[r,two heads] & z^1\ar[d,"a^1"']\ar[r,two heads] & z\ar[d,"a"]\ar[dr,"0"] & \\
	x^0 \ar[r]\ar[d,two heads,"f^0"] & x^1 \ar[r]\ar[d,hook,"f^1"] & x^2\ar[r]\ar[d,"f^2"] & x^3 \ar[d,hook,"f^3"] \\
	y^0 \ar[r] & y^1 \ar[r] & y^2\ar[r] & y^3 
\end{diagram*}
where we note that the top left square in the diagram commutes since it commutes after composition with the monomorphism \(f^1\). Finally, we now see that
\[ (z^0 \sur z^1 \sur z \overset{a}\to x^2) = (z^0\overset{a^0}\to x^0 \to x^1 \to x^2) = 0 \]
and therefore \(a = 0\).
\end{proof}
\begin{remark}
	The above proof is perhaps illuminated by the philosophy of generalized elements. What we start off with is an element \(a\in x^2\) such that
	\(f^2(a) = 0\). We then lift this to an element \(a^1 \in x^1\) which is sent to \(a\), and \(a^1\) is in turn lifted to an element \(a^0\in x^0\)
	which is sent to \(a^1\), meaning that \(a\) is in the image of the composite of two adjacent morphisms, so \(a=0\). The repeatedly used
	Lemma \ref{lemma:five-lemma-technical-necessary-lemma} is just a formalization of these lifting steps (or most of them).
\end{remark}
\begin{corollary}[Five lemma]
	Let \(\calA\) be an Abelian category, and consider a commutative diagram with exact rows
	\begin{diagram*}
		x^0\ar[r] \ar[d,"\sim" labl] & x^1\ar[r] \ar[d,"\sim" labl] & x^2\ar[r]\ar[d] & x^3\ar[r] \ar[d,"\sim" labl] & x^4 \ar[d,"\sim" labl] \\
		y^0\ar[r] & y^1\ar[r] & y^2\ar[r] & y^3\ar[r] & y^4 
	\end{diagram*}
	in \(\calA\), with isomorphisms as indicated. Then the middle vertical morphism is an isomorphism.
\end{corollary}
\begin{proof}
Apply Lemma \ref{lemma:generalized-five-lemma} to the left part of the diagram and then to the right to see that the middle morphism is both monic and epic,
thus an isomorphism.
\end{proof}

\subsection{Appendix: Adjoints of triangulated functors are triangulated}

There is a somewhat tricky business in triangulated categories coming from the shift functor, namely that one should really demand compatibility with it at
all times, including e.g.\ with natural transformations. Consider the following definition, which we hope is obviously a natural one.
\begin{definition}
	Let \((F,\sigma),(F',\sigma')\!:\calT\to\calT'\) be lax-triangulated functors. A natural transformation \(\alpha\!:F\To F'\) is \emph{triangulated} if
	\begin{diagram*}
		F\Sigma\ar[r,Rightarrow,"\sigma"]\ar[d,Rightarrow,"\alpha\Sigma"'] & \Sigma F \ar[d,Rightarrow,"\Sigma\alpha"] \\
		F'\Sigma\ar[r,Rightarrow,"\sigma'"]  & \Sigma F'
	\end{diagram*}
	commutes.
\end{definition}
\begin{remark}
	These are sometimes called \emph{trinatural transformations,} but we choose to avoid this terminology due to its proximity to the completely unrelated notion
	of a \emph{dinatural} transformation.
\end{remark}

This definition gives rise to a 2-category \(\underline{\cat{TCat}}\), the objects of which are triangulated categories, 1-morphisms are triangulated
functors, and 2-morphisms are triangulated natural transformations. Now, any 2-category automatically induces a notion of adjunction between 1-morphisms. In the case
of the 2-category \(\underline{\cat{TCat}}\), it yields the following:
\begin{definition}
	Let \((F,\sigma)\!:\calT\to\calT'\) and \((G,\sigma')\!:\calT'\to\calT\) be triangulated functors between triangulated categories. We say that they form
	a \emph{triangulated adjunction} \((F,\sigma)\ladj(G,\sigma')\) if \(F\ladj G\) and the unit and counit are triangulated natural transformations.
\end{definition}

One would like for ordinary adjunctions between functors which happen to have a triangulated structure to yield a triangulated adjunction, but this is a non-trivial
statement. As such, we prove the following result.

\begin{theorem}\label{thm:adjoints-of-triangulated-functors-are-triangulated}
	Let \((G,\sigma)\!:\calT'\to\calT\) be a triangulated functor between triangulated categories. If \(G\) has a left adjoint \(F\), then there is a canonical triangulated structure on \(F\)
	for which the unit and counit of the adjunction are triangulated.
\end{theorem}
\begin{proof}
We have a natural isomorphism \(\sigma\!:G\circ\Sigma\To\Sigma\circ G\). This yields a natural isomorphism
\[ \Sigma^{-1}\sigma \Sigma^{-1}\!: \Sigma^{-1}\circ G \To G\circ\Sigma^{-1}. \]
Let \(\sigma'\!:F\circ \Sigma\To\Sigma\circ F\) be the composition
\begin{diagram*}
	F\Sigma \ar[r,"F\Sigma\eta"] & F\Sigma GF \ar[r," F\sigma^{-1}F "] & FG\Sigma F \ar[r,"\varepsilon\Sigma F"] & \Sigma F.
\end{diagram*}
One can check that this natural transformation is also implemented using the Yoneda lemma and the chain of natural isomorphisms
\begin{align*}
	\calT'(F\Sigma-,-) &\cong \calT(\Sigma-,G-) \\
	&\cong \calT(-,\Sigma^{-1}G-) \\
	&\cong \calT(-,G\Sigma^{-1}-) \\
	&\cong \calT'(F-,\Sigma^{-1}-) \cong \calT'(\Sigma F-,-).
\end{align*}
In particular, we conclude that \(\sigma'\) is a natural isomorphism. As a trivial remark, note that \(F\) is a left adjoint and hence commutes with colimits, and so is
automatically additive.

Before we check that \((F,\sigma')\) is triangulated, let us assume it is the case, and check that the unit \(\eta\!:\1\to GF\) and counit \(\varepsilon\!:FG\To\1\) are then triangulated.
First, note that the compositions \(GF\) and \(FG\) are triangulated with shift compatibilities
\[ \sigma F \circ G\sigma' \!: GF\Sigma\To \Sigma GF,\quad  \sigma' G \circ F\sigma \!: FG\Sigma \To \Sigma FG. \]
To see that \(\varepsilon\Sigma = \Sigma\varepsilon \circ (\sigma' G \circ F\sigma) \), we have the commutative diagram
\begin{diagram*}
	FG\Sigma\ar[r,Rightarrow,"F\sigma"] & F\Sigma G\ar[r,Rightarrow,"F\Sigma\eta G"]\ar[dr,equal] & F\Sigma GFG\ar[r,Rightarrow,"F\sigma^{-1}FG"]\ar[d,Rightarrow,"F\Sigma G\varepsilon"] & FG\Sigma FG \ar[r,Rightarrow,"\varepsilon\Sigma FG"]\ar[d,Rightarrow,"FG\Sigma\varepsilon"] & \Sigma FG\ar[d,Rightarrow,"\Sigma\varepsilon"] \\
	 & & F\Sigma G\ar[r,Rightarrow,"F\sigma^{-1}"'] & FG\Sigma\ar[r,Rightarrow,"\varepsilon\Sigma"'] & \Sigma
\end{diagram*}
by naturality and one of the triangle identities. Checking that \( (\sigma'F\circ G\sigma) \circ \eta\Sigma = \Sigma\eta \) is similar, and uses
the other triangle identity.

Now we show that \((F,\sigma')\) is triangulated. Consider a distinguished triangle
\[ x \overset{f}\to y \overset{g}\to z \overset{h}\to \Sigma x \]
in \(\calT\). We must show that the induced triangle
\[ Fx \overset{Ff}\longto Fy \overset{Fg}\longto Fz \overset{\sigma'_{x}\circ Fh}\longto \Sigma Fx \]
is distinguished. We will show that it is isomorphic to a distinguished triangle. For this, first take the cone of \(Ff\) to get a distinguished triangle
\[ Fx \overset{Ff}\to Fy \overset{g'}\to z' \overset{h'}\to \Sigma Fx. \]
Now, applying \(G\), we have a distinguished triangle
\[ GFx \overset{GFf}\longto GFy \overset{Gg'}\longto Gz' \overset{\sigma_{Fx}\circ Gh'}\longto \Sigma GFx. \]
Let \(\eta\!:\1\To GF\) be the unit of the adjunction \(F\ladj G\). By (TR3), we obtain a morphism of distinguished triangles
\begin{diagram*}
	x \ar[r,"f"]\ar[d,"\eta_x"] & y \ar[r,"g"]\ar[d,"\eta_y"] & z \ar[r,"h"]\ar[d,dashed,"\theta"] & \Sigma x \ar[d,"\Sigma\eta_x"] \\
	GFx \ar[r,"GFf"] & GFy \ar[r,"Gg'"] & Gz' \ar[r,"\sigma_{Fx}\circ Gh'"] & \Sigma GFx
\end{diagram*}
which now allows us to define, for any \(w\in\calT'\), the map
\[ \calT'(z',w) \to \calT'(z,Gw),\quad q \mapsto G(q)\circ\theta. \]
In particular, by applying \(\calT'(-,w)\) to the triangle defining \(z'\) and \(\calT(-,Gw)\) to the top row above, we get a commutative diagram
\begin{diagram*}
	\calT(x,Gw) \ar[d,"\sim" labl] & \calT(y, Gw) \ar[l]\ar[d,"\sim" labl] & \calT(z, Gw) \ar[l]\ar[d] & \calT(\Sigma x, Gw) \ar[l]\ar[d,"\sim" labl] & \calT(\Sigma y, Gw) \ar[d,"\sim" labl] \ar[l] \\
	\calT'(Fx, w)  & \calT'(Fy, w) \ar[l] & \calT'(z',w) \ar[l] & \calT'(w,\Sigma Gx) \ar[l] & \calT'(w,\Sigma Gy) \ar[l]
\end{diagram*}
with exact rows. It follows by the five lemma that \(\calT(z,Gw) \iso \calT'(z,w) \), which implies that for any morphism \(q\!:z\to Gw\), there is a unique morphism \(q'\!:z'\to w\) such that \(q = Gq'\circ\theta\).
This is the universal property of the unit, and we deduce that the morphism \(Fz\to z'\) corresponding to \(\theta\) is an isomorphism. That is, we have an isomorphism
of triangles
\begin{diagram*}
	Fx \ar[r,"Ff"]\ar[d,equal] & Fy \ar[r,"Fg"]\ar[d,equal] & Fz \ar[r,"\sigma'_{x}\circ Fh"]\ar[d,"\sim" labl] & \Sigma Fx \ar[d,equal] \\
	Fx \ar[r,"Ff"] & Fy \ar[r,"g'"] & z' \ar[r,"h'"] & \Sigma Fx 
\end{diagram*}
as desired.
\end{proof}
\begin{remark}
	Note that by dualizing Theorem \ref{thm:adjoints-of-triangulated-functors-are-triangulated}, we get that right adjoints of triangulated functors are canonically triangulated.
	Observe that the proof relies on the functor in question being triangulated and not just lax-triangulated.
\end{remark}
\begin{remark}
	The above proof is adapted from \cite[Thm.\ 47]{murfet-triangulated-categories} (which is based on \cite[Lemma 5.3.6]{neeman-triangulated-categories}) and \cite{4984372}.
\end{remark}

\subsection{Appendix: The most elementary \texorpdfstring{\(K\)}{K}-theory}
Let \(X\) be some kind of geometric object. One of the fundamental tools available in all ``geometric'' areas (say, differential or algebraic) is to study the
\emph{vector bundles} on \(X\), and one of the standard ways to do this is to consider invariants like Euler characteristics. In his work on algebraic geometry,
Grothendieck introduced \(K\)-theory, which one may see as the \emph{universal} receptacle of Euler characteristics.

While \(K\)-theory was initially defined in a concrete way, it was realized that many aspects of \(K\)-theory were visible more clearly when one realized it as
a purely categorical kind of invariant. In this appendix, we will describe one way to define the easiest \(K\)-group, namely \(K_0\), in the context of a triangulated
category. Much the same strategy we use here can be used to define \(K_0\) of a Quillen exact category as well. The idea is that the Euler characteristic splits short exact sequences,
sending the middle term to the sum of the adjacent terms.

\begin{definition}
	Let \(\calT\) be a triangulated category. The \emph{zeroth} \(K\)-group of \(\calT\), denoted \(K_0(\calT)\), is the quotient of the free Abelian group
	generated by isomorphism classes of objects in \(\calT\), by the relation
	\[ [y] = [x] + [z] \quad\text{if}\quad \exists\text{d.t. } x\to y\to z \to \Sigma x.  \]
	As implied above, we write \([x]\) for the image of \(x\in\calT\) in \(K_0(\calT)\).
\end{definition}
\begin{remark}
	For any \(x,y\in\calT\) we have the distinguished triangle
	\[ x \to x\oplus y \to y \overset0\to \Sigma x \]
	which implies that
	\[ \forall x,y\in\calT,\quad [x\oplus y] = [x] + [y] \in K_0(\calT). \]
	Furthermore, by the distinguished triangle
	\[ x \to 0 \to \Sigma x \to 0 \]
	given by shifting the cone of the identity, we see that
	\[ \forall x\in\calT,\quad [\Sigma x] = -[x]. \]
	As a result, we could've instead defined \(K_0(\calT)\) as the commutative monoid whose elements are isomorphism classes of objects in \(\calT\) and addition
	is given by taking direct sums, modulo the distinguished triangle relations. The above calculation would then show that it is automatically a group.
\end{remark}

While this definition seems perfectly good, it has a ``flaw'' of sorts which means one has to be careful when applying it. The below is an example of the
famous ``Eilenberg swindle''.

\begin{proposition}
	Let \(\calT\) be a triangulated category admitting infinite coproducts. Then \(K_0(\calT) \cong 0\).
\end{proposition}
\begin{proof}
Let \(x\in\calT\). Simply observe that
\[ [x^{\oplus\N}] = [x\oplus x^{\oplus\N}] = [x] + [x^{\oplus\N}] \]
and therefore \([x] = 0\).
\end{proof}

\begin{remark}
	Because of this result, one often has to jump through some hoops, placing finiteness conditions on one's objects, in order to get a non-trivial \(K_0\)-group. For the sake
	of this discussion, let us assume that we have a sensible understanding of the derived category \(\sfD(\calA)\) of an Abelian category \(\calA\).
	A common fix is to consider the full subcategory of compact objects.

	Rather remarkably, the vast majority of \(K_0\)-groups of interest can be formed using the above machinery by choosing an appropriate triangulated category. Some
	exceptions where it is tricker are in the world of operator algebras.
\end{remark}

To get a small taste for how \(K\)-theory works, let's prove an easy result.
\begin{proposition}
	Let \(F\!:\calT\to\calT'\) be a triangulated functor of triangulated categories. Then \(F\) induces a group homomorphism
	\[ K_0(F)\!:K_0(\calT)\to K_0(\calT'). \]
\end{proposition}
\begin{proof}
The recipe for the morphims is simple:
\[ K_0(F)\!: [x]\mapsto [Fx]. \]
We must show that this is well-defined and a group homomorphism. However, this is a trivial matter since \(F\) preserves distinguished triangles:
\[ x \to y \to z \to \Sigma x \quad \leadsto \quad Fx \to Fy \to Fz \to \Sigma Fx \]
means that
\[ [y] = [x] + [z] \quad \leadsto \quad [Fy] = [Fx] + [Fz]. \]
\end{proof}

There is an independent way one can define the \(K_0\)-group of an Abelian category. We include it here so we can prove a similar result as above, except about cohomological functors.
\begin{definition}
	Let \(\calA\) be an Abelian category. Then \(K_0(\calA)\) is the quotient of the free Abelian group on the isomorphism classes of \(\calA\) by the relation
	\[ [y] = [x] + [z] \quad\text{if}\quad \exists\text{s.e.s } 0 \to x \to y \to z \to 0. \]
\end{definition}
\begin{remark}
	The Eilenberg swindle works just as well for Abelian categories as it does for triangulated ones. As a result, one sees that whenever \(\calA\) admits
	infinite (co)products, one will have \(K_0(\calA)=0\). For example, \(K_0(\Mod_{\Z}) = 0\). On the other hand, as mentioned earlier, these kinds of ``issues''
	can be fixed by considering compact objects. In the case of this example, the compact objects are exactly the finitely generated Abelian groups.
\end{remark}

In a sense, the below will illustrate how \(K_0(\calT)\) acts as a receptacle for Euler characteristics, supposing the reader is familiar with the latter (and setting aside the fact
that it is, in a sense, by definition).

\begin{lemma}
	Consider a long exact sequence of the form
	\[ 0 \to w^0 \overset{d^0}\to w^1 \overset{d^1}\to \cdots \overset{d^{n-1}}\to w^n \overset{d^n}\to 0 \]
	in an Abelian category \(\calA\). Then
	\[ [w^0] = \sum_{i=1}^n(-1)^i[w^i] \in K_0(\calA). \]
\end{lemma}
\begin{proof}
At any stage of the exact sequence, we can extract an exact sequence
\[ 0 \to \img{d^{i-1}} \to w^i \to \img{d^i} \to 0 \]
and thus have
\[ [w^i] = [\img{d^{i-1}}] + [\img{d^i}] \implies [\img{d^{i-1}}] = [w^i] - [\img{d^i}]. \]
At \(i=1\), this says \([w^0] = [w^1] - [\img{d^1}]\). Continuing inductively, one gets the result.
\end{proof}

\begin{proposition}
	Let \(H\!:\calT\to\calA\) be a cohomological functor, and assume that for all \(x\in\calT\), at most a finite number of the objects \(H\Sigma^ix\), \(i\in\Z\), are non-zero.
	Then \(H\) induces a group homomorphism
	\[ K_0(\calT) \to K_0(\calA),\quad [x]\mapsto \sum_{i\in\Z} (-1)^i[H\Sigma^ix]. \]
\end{proposition}
\begin{proof}
To see this, just note that if we have a distinguished triangle
\[ x \to y \to z \to \Sigma x \]
then we have a long exact sequence
\[ \cdots \to H\Sigma^{i-1}z \to H\Sigma^ix\to H\Sigma^iy\to H\Sigma^iz \to H\Sigma^{i+1} x \to \cdots \]
which by our assumption has only finitely many non-zero terms. For the sake of simplicity, we assume everything lies in the range \([0,n]\), i.e.
\[ 0 \to Hx\to Hy\to Hz \to H\Sigma x \to \cdots \to H\Sigma^ny \to H\Sigma^n z \to 0. \]
Applying the lemma, we see that
\[ [Hx] = [Hy] - [Hz] + [H\Sigma x] - [H\Sigma y] + [H\Sigma z] - [H\Sigma^2x] + \cdots \]
and upon rearranging terms, one has
\[ \sum_{i=0}^n (-1)^i[H\Sigma^iy] = \sum_{i=0}^n (-1)^i[H\Sigma^ix] + \sum_{i=0}^n (-1)^i[H\Sigma^iz] \]
as desired.
\end{proof}

\begin{remark}
	Just to give some clarity: at the start of the appendix, we mentioned that what we have done above can be easily modified to give a definition of \(K_0\) for Quillen exact categories.
	The way to do this is to take the free Abelian group on the isomorphism classes, as always, and then quotient that by the obvious relations imposed by the admissible
	exact sequences.
\end{remark}


%!TEX root = ../lectures.tex

\section{Localizations of triangulated categories}\label{lecture:localizations-of-triangulated-categories}
We have discussed the topic of localizing a category at some collection of arrows. In the Abelian/stable case, one can usually pass from working with an ordinary category to working with
a triangulated category, and in many cases, working with the localization of this other category is more convenient. Thus, we are compelled to explain what localization looks like in the
context of triangulated categories.

While much of the theory is the same, the additive nature of triangulated categories allows one to encode the weak equivalences in a system of objects one wishes to equate to zero,
making the theory of localizations of triangulated categories very similar to the theory of quotients in ordinary algebra. There are more general settings in which one may localize
triangulated categories in a nice way, but we will not discuss them.

\subsection{Triangulated subcategories \& null systems}
\begin{definition}
	Let \(\calT\) be a pre-triangulated category with shift denoted \(\Sigma\). A \emph{pre-triangulated subcategory} of \(\calT\) consists of an additive subcategory \(\calT'\subseteq\calT\)
	such that \(\Sigma\calT' = \calT'\), along with a pre-triangulated structure on \(\calT\) with shift given by the restriction of \(\Sigma\) and for which the inclusion
	\[ \calT'\inj\calT \]
	is a triangulated functor.
\end{definition}
\begin{definition}
	Let \(\calT\) be a pre-triangulated category. A replete full subcategory \(\calN\) of \(\calT\) is a \emph{null system} if:
	\begin{enumerate}[label=(N\arabic*)]
	\item \(0\in \calN\).
	\item \(\calN\) is closed under shifts: \(\Sigma\calN = \calN\).
	\item \(\calN\) is closed under extensions: for any d.t.\ \(x\to y\to z \to \Sigma x\) in \(\calT\), if \(x,z\in\calN\) then \(y\in\calN\).
	\end{enumerate}
\end{definition}
\begin{exercise}\label{exercise:null-system-N3}
	Let \(\calT\) be a pre-triangulated category, and let \(\calN\) be a null system in \(\calT\). Show that \(\calN\) satisfies the following stronger version of (N3): for any
	distinguished triangle
	\[ x\to y \to z\to \Sigma x \]
	in \(\calT\), if any two of \(x,y,z\) are in \(\calN\) then so is the third.

	In fact, prove that these are both equivalent to the following a priori different statement:
	\begin{itemize}[label=(N3')]
	\item \(\calN\) is closed under cones: for any d.t.\ \(x\to y\to z\to \Sigma x\) in \(\calT\), if \(x,y\in\calN\) then \(z\in\calN\).
	\end{itemize}
\end{exercise}
\begin{proposition}\label{prop:null-system-subcategory-characterization}
	Let \(\calT\) be a pre-triangulated category, and let \(\calN\) be a replete full subcategory. Then the following statements are equivalent.
	\begin{enumerate}
	\item \(\calN\) is a null system.
	\item \(\calN\) is non-empty and can be endowed with the structure of a pre-triangulated category making it into a pre-triangulated full subcategory of \(\calT\).
	\end{enumerate}
	Furthermore, the same statements hold with ``pre-triangulated'' replaced by ``triangulated''.
\end{proposition}
\begin{proof}
(1) implies (2). First, note that \(\calN\) is an additive subcategory of \(\calT\). Indeed, it is clearly pre-additive, and furthermore, contains \(0\) by (N1), and is closed
under finite direct sums by (N3) together with Corollary \ref{corollary:direct-sum-triangle}. In particular, the inclusion \(\calN\inj\calT\) is an additive functor. Since \(\Sigma\calN = \calN\),
the shift on \(\calT\) restricts to a functor \(\Sigma|_{\calN}\!:\calN\to\calN\). Now, let a triangle in \(\calN\) be distinguished if and only if it is distinguished in \(\calT\). Then \(\calN\)
is pre-triangulated: indeed, (TR1) and (TR2) are clear, and (TR3) follows from \(\calN\) being a full subcategory. When \(\calT\) is in addition triangulated, (TR4) follows for \(\calN\) similarly.

It is clear that \(\calN\inj\calT\) is a triangulated functor, so we are done.

(2) implies (1). Since \(\calN\inj\calT\) is triangulated, it is additive, hence \(0\in\calN\) agrees with \(0\in\calT\). That is, (N1) is satisfied. Since \(\calN\) is a pre-triangulated subcategory,
\(\Sigma\calN = \calN\) is satisfied by definition, so (N2) holds. To see that (N3) holds, we use that it is equivalent to (N3') by Exercise \ref{exercise:null-system-N3}, and note that if we have a diagram of solid arrows
\begin{diagram*}
	x\ar[r]\ar[d,equal] & y\ar[r]\ar[d,equal] & z\ar[r]\ar[d,dashed,"\sim" labl] & \Sigma x\ar[d,equal] \\
	x\ar[r] & y\ar[r] & z'\ar[r] & \Sigma x
\end{diagram*}
where the top row is distinguished in \(\calN\) and the bottom row is distinguished in \(\calT\), then by Proposition \ref{prop:triangulated-five-lemma} we have an induced isomorphism \(z\cong z'\).
Therefore, \(z'\in\calN\) since \(\calN\) is replete.
\end{proof}
\begin{remark}
	As a slogan, we can say that null systems are exactly replete full triangulated subcategories.
\end{remark}

\subsection{The Verdier quotient}
Let \(\calN\) be a null system in a pre-triangulated category \(\calT\). We want to make sense of what it would mean to take the quotient \(\calT/\calN\). The idea we want to utilize now is based
on the following basic fact about pre-triangulated categories: a morphism \(f\!:x\to y\) is an isomorphism if and only if
\[ x\overset{f}\to y\to 0 \to \Sigma x \]
is a distinguished triangle. This gives us a way to conceptualize which morphisms should be sent to isomorphism in \(\calT/\calN\). It should be exactly those \(f\!:x\to y\) for which there
is a distinguished triangle
\[ x\overset{f}\to y\to z \to \Sigma x \]
where \(z\in\calN\).
\begin{notation}
	Let \(\calT\) be a pre-triangulated category, and let \(\calN\) be a null system. We form the collection of morphism
	\[ \calS(\calN) := \{ f\!:x\to y\mid \exists\text{d.t.\ } x\overset{f}\to y\to z \to \Sigma x\text{ such that }z\in\calN \}. \]
\end{notation}
\begin{definition}
	Let \(\calT\) be a pre-triangulated category, and let \(\calN\) be a null system. Then the \emph{Verdier quotient} of \(\calT\) by \(\calN\) is the localization
	\[ \calT/\calN := \calT[\calS(\calN)^{-1}]. \]
\end{definition}

The rest of this subsection is dedicated to giving a well-chosen triangulated structure on \(\calT/\calN\) whenever \(\calT\) is triangulated, and showing it is well-behaved.
\begin{proposition}
	Let \(\calT\) be a triangulated category, and let \(\calN\) be a null system in \(\calT\). Then \(\calS(\calN)\) is a multiplicative system.
\end{proposition}
\begin{proof}
Note that \(\calN\) is a null system in \(\calT\) if and only if \(\calN^\op\) is a null system in \(\calT^\op\), so it suffices to check that \(\calN\) is
a right multiplicative system.
\begin{enumerate}[label=(\arabic*)]
\item \(\calS(\calN)\) contains the identities: since \(0\in\calN\), in fact every isomorphism is contained in \(\calS(\calN)\).
\item \(\calS(\calN)\) is closed under composition: given morphisms \(x\overset{f}\to y\overset{g}\to z\), let \(c_f\) (resp.\ \(c_g\), \(c_{g\circ f}\)) be a cone of \(f\) (resp.\ \(g\), \(g\circ f\)).
Applying (TR4), we have a distinguished triangle
\[ c_f \to c_{g\circ f}\to c_g\to \Sigma c_f. \]
Since \(c_f\) and \(c_g\) are in \(\calN\) and \(\calN\) is closed under extension, it follows that \(c_{g\circ f}\) is in \(\calN\), so \(g\circ f\in\calS(\calN)\).
\item \(\calS(\calN)\) satisfies (M1): consider \(x'\overset{s}\ot x\overset{f}\to y\), where \(s\in\calS(\calN)\). Using (N2) and (TR2), we deduce the existence of a \(z\in\calN\) and a distinguished triangle
\[ z\overset{h}\to x\overset{s}\to x' \to \Sigma z. \]
Taking the cone of \(f\circ h\) and applying (TR3), we get
\begin{diagram*}
	z\ar[d,equal]\ar[r,"h"] & x\ar[d,"f"]\ar[r,"s"] & x'\ar[d,dashed,"g"]\ar[r] & \Sigma z \ar[d,equal] \\
	z\ar[r,"f\circ h"] & y\ar[r,dashed,"t"] & y'\ar[r] & \Sigma z
\end{diagram*}
and one notes that \(t\in\calS(\calN)\) since \(z\in\calN\).
\item \(\calS(\calN)\) satisfies (M2): it suffices to show that given a solid diagram
\begin{diagram*}
	z\ar[r,"s"] & x\ar[r,"f"] & y\ar[r,dashed,"t"] & z'
\end{diagram*}
where \(f\circ s = 0\), a dashed arrow \(t\) exists such that \(t\circ f = 0\). Suppose we are given the solid diagram. Taking a cone \(g\!:y\to c_f\) of \(f\), weak cokernel property of
Corollary \ref{corollary:weak-kernel-property-of-cocones} yields a map \(h\!:c_f\to y\) such that \(h\circ g = f\), and we may take a cone \(t\!:y\to z'\) of \(h\). All in all, we have
\begin{diagram*}
	z\ar[r,"s"] & x\ar[r,"g"]\ar[dr,"f"'] & c_f\ar[r]\ar[d,"h"] & \Sigma z \\
	& & y\ar[d,"t"] & \\
	& & z' &
\end{diagram*}
such that \(t\circ f = t\circ h \circ g = 0\).
\end{enumerate}
This completes the proof.
\end{proof}

The above lets us apply the non-functorial derived functor machinery later, which is beneficial in many cases, as functoriality even on the level of homotopy categories can
be slightly tricky to arrange.

Before we can place a triangulated structure on \(\calT/\calN\), we have to find a shift functor on it. For this, consider the localization functor \(\gamma\!:\calT\to\calT/\calN\) and note that
the composition \(\gamma\circ\Sigma\) is homotopical: we have that \(\Sigma\calS(\calN) = \calS(\calN)\), so for all \(f\in\calS(\calN)\) the morphism \(\gamma\Sigma f\) is an isomorphism.
By universal property of the localization, this means we have an induced functor
\begin{diagram*}
	\calT\ar[r,"\Sigma"]\ar[d,"\gamma"'] & \calT\ar[d,"\gamma"] \\
	\calT/\calN\ar[r,dashed,"\Sigma"] & \calT/\calN
\end{diagram*}
which is our prospective shift. Note that \(\gamma\Sigma = \Sigma\gamma\).

\begin{exercise}
	Show that the above functor \(\Sigma\!:\calT/\calN\to\calT/\calN\) is an automorphism, with inverse induced from the inverse of \(\Sigma\!:\calT\to\calT\). Hint: use the uniquenss in the universal property.
\end{exercise}

\begin{theorem}\label{thm:verdier-quotient-is-triangulated}
	Let \(\calT\) be a triangulated category, and let \(\calN\) be a null system. Then the following statements hold,
	\begin{enumerate}[label=(\arabic*)]
	\item \(\calT/\calN\) is an additive category, and the localization functor \(\gamma\!:\calT\to\calT/\calN\) is additive.
	\item Let a triangle \(x\to y\to z\to \Sigma x\) in \(\calT/\calN\) be distinguished if it is isomorphic to the image under \(\gamma\) of a distinguished triangle in \(\calT\). Then
	this endows \(\calT\) with the structure of a triangulated category.
	\item With the triangulated structure from (2), the functor \(\gamma\!:\calT\to\calT/\calN\) is triangulated.
	\end{enumerate}
\end{theorem}
\begin{proof}
(1) This is an immediate corollary of Theorem \ref{thm:localization-of-additive-is-additive}, given that \(\calS(\calN)\) is a multiplicative system.

(2) We sketch how to show that (TR1)--(TR4) hold. The only non-trivial one is (TR1). Consider a morphism \(fs^{-}\!:\gamma(x)\to\gamma(y)\), i.e.\ maps \(x\overset{s}\ot x'\overset{f}\to y\) in \(\calT/\calN\), which we note is the same
as \(\gamma(f)\circ\gamma(s)^{-1}\). Take a cone \(c_f\) of \(f\) in \(\calT\); we then have an isomorphism of triangles
\begin{diagram*}
	\gamma(x')\ar[d,"\gamma(s)"', "\sim" labl]\ar[r,"\gamma(f)"] & \gamma(y)\ar[d,equal]\ar[r] & \gamma(c_f)\ar[d,equal]\ar[r] & \Sigma\gamma(x')\ar[d,"\sim" labl] \\
	\gamma(x)\ar[r,"fs^{-1}"] & \gamma(y) \ar[r] & \gamma(c_f)\ar[r] & \Sigma\gamma(x')
\end{diagram*}
where the upper triangle is distinguished, hence so is the lower one. For (TR2), just lift the required distinguished triangles to ones in \(\calT\) and apply (TR2) there.
The strategies for (TR3) and (TR4) are very similar.

(3) We have that \(\gamma\Sigma = \Sigma\gamma\), and by definition, \(\gamma\) sends distinguished triangles to distinguished triangles.
\end{proof}
\begin{exercise}
	Complete the proof of (2) in Theorem \ref{thm:verdier-quotient-is-triangulated}.
\end{exercise}

From now on, we tacitly endow \(\calT/\calN\) with the triangulated structure provided above. In this context, one can restate the universal property of the localization
in terms of triangulated categories.

\begin{theorem}\label{thm:Verdier-quotient-universal-property}
	Let \(\calT\) be a triangulated category, let \(\calN\) be a null system, and let \(\gamma\!:\calT\to\calT/\calN\) be the localization functor. Then the following statements hold.
	\begin{enumerate}[label=(\arabic*)]
	\item For all \(x\in\calN\), we have \(\gamma(x)\cong 0\).
	\item \(\calT/\calN\) is the universal pre-triangulated category satisfying (1): for any triangulated functor \(F\!:\calT\to\calD\) such that \(Fx\cong 0\) for all \(x\in\calN\),
	there is a unique triangulated functor \(F'\!:\calT/\calN\to\calE\) such that \(F = \gamma\circ F\).
	\item \(\calT/\calN\) is the universal source of cohomological functors out of \(\calT\) sending \(\calN\) to zero: for any cohomological functor \(H\!:\calT\to\calA\) such that
	\(Hx\cong 0\) for all \(x\in\calN\), there is a unique cohomological functor \(H'\!:\calT/\calN\to\calA\) such that \(H = \gamma\circ H'\).
	\end{enumerate}
\end{theorem}
\begin{proofsketch}
(1) For any \(x\in\calN\), the distinguished triangle \(0\to x \to x\to \Sigma 0\) shows that \(0\to x\) is in \(\calS(\calN)\). In particular, \(0\cong\gamma(0)\cong\gamma(x)\).

(2) If \(F\) sends everything in \(\calN\) to zero, then for any \(s\!:x\to y\) in \(\calS(\calN)\), taking a cone and applying \(F\) yields
\[ Fx\overset{Fs}\longto Fy\longto 0 \longto \Sigma Fx. \]
Therefore, \(Fs\) is an isomorphism. In particular, the universal property of localizations then guarantees a unique functor \(F'\!:\calT/\calN\to\calD\). To see that it is triangulated,
one needs two things: that it commutes with the shift (up to a specified natural isomorphism), and that it sends distinguished triangles to distinguished triangles. For the former, use
the natural commutation isomorphism for \(F\) along with the universal property of localizations. For the latter, use the defining properties of \(F'\) and distinguished triangles in \(\calT/\calN\).

(3) The details of this are similar to (2).
\end{proofsketch}

\subsection{Thick subcategories}
\begin{definition}
	Let \(F\!:\calC\to\calD\) be a functor between categories with zero objects. We define the \emph{kernel} of \(F\) to be the full category of \(\calC\) spanned by objects sent
	to zero by \(F\), i.e.
	\[ \ker{F} := \{ x\in\calC\mid Fx\cong 0 \}. \]
\end{definition}

The kernel of a functor is by definition a replete full subcategory. Consider the localization functor \(\gamma\!:\calT\to\calT/\calN\) of a Verdier quotient. As explained,
we have \(\calN\subseteq\ker{\gamma}\). Is this an equality? In general, the answer is no, and for a very simple reason: the kernel \(\ker\gamma\) is closed under taking direct summands.
Indeed, consider the general situation of the definition, and assume that \(F\) is an additive functor between additive categories. If we have \(F(x\oplus y) \cong 0\), then
we have \(Fx\oplus Fy \cong 0\), and this can only happen if \(Fx\cong Fy\cong 0\).
\begin{definition}
	Let \(\calT\) be a pre-triangulated category. A \emph{thick} (pre-)triangulated subcategory \(\calT'\) of \(\calT\) is a full (pre-)triangulated subcategory which is closed under summands.
	That is, if \(x\oplus y\in\calT'\), then \(x,y\in\calT'\).
\end{definition}

What we have observed is that \(\ker\gamma\) is a thick triangulated subcategory of \(\calT\). This is not necessarily true of \(\calN\), so clearly it is not necessary that it is equal to \(\ker\gamma\).
\begin{notation}
	Let \(\calT\) be a triangulated category, and let \(\calC\) be a subcategory. We denote by \(\thick(\calC)\) the smallest thick triangulated subcategory of \(\calT\) containing \(\calC\).
\end{notation}

Our goal for this subsection is to prove that \(\ker\gamma = \thick(\calN)\). As a result, we will also be able to deduce that there is a canonical isomorphism of triangulated categories
\(\calT/\calN \cong \calT/\ker\gamma\). Fix a triangulated category \(\calT\) and a null system \(\calN\), and denote the localization functor by \(\gamma\!:\calT\to\calT/\calN\).

\begin{exercise}\label{exercise:S(N)-2-out-of-3-property}
	Show that \((\calT,\calS(\calN))\) is a pseudo-homotopical category, i.e.\ show that \(\calS(\calN)\) satisfies the 2-out-of-3 property.
\end{exercise}

\begin{exercise}
	Let \(F\!:\calT\to\calT'\) be a triangulated functor between triangulated categories. Check that \(\ker{F}\) inherits a triangulated structure from \(\calT\).
\end{exercise}

\begin{lemma}\label{lemma:equivalence-class-of-identity-in-Verdier-quotient}
	If a morphism
	\[ x\overset{s}\ot x'\overset{f}\to x \]
	in \(\calT/\calN\) is in the equivalence class of the identity, then \(f\in\calS(\calN)\).
\end{lemma}
\begin{proof}
By assumption, we have a diagram
\begin{diagram*}
	& x'\ar[dl,"s"']\ar[dr,"f"] & \\
	x & x''\ar[l,"t"']\ar[d,"\psi"]\ar[u,"\phi"]\ar[r,"h"] & x \\
	& x\ar[ul,"\id_x"]\ar[ur,"\id_x"] &
\end{diagram*}
where \(t\in\calS(\calN)\). By Exercise \ref{exercise:S(N)-2-out-of-3-property}, \(\phi,\psi\in\calS(\calN)\). But now \(\psi = f\circ\phi\), so the same property yields \(f\in\calS(\calN)\).
\end{proof}
\begin{lemma}\label{lemma:equivalence-invertible-map-in-Verdier-quotient}
	Consider a morphism
	\[ x\overset{s}\ot x'\overset{g}\to y \]
	in \(\calT/\calN\). Then the following statements are equivalent.
	\begin{enumerate}[label=(\arabic*)]
	\item The morphism \(gs^{-1}\) is invertible.
	\item There are morphisms \(f\!:y'\to x'\), \(h\!:y\to z'\) in \(\calT\) such that \(g\circ f,h\circ g\in\calS(\calN)\).
	\end{enumerate}
\end{lemma}
\begin{proof}
Assume (2) holds. Then \(\gamma(g\circ f)\) and \(\gamma(h\circ g)\) are invertible, and so see that
\[ \gamma(g)\circ\gamma(f)\circ\gamma(g\circ f)^{-1} = \id,\quad \gamma(h\circ g)^{-1}\circ\gamma(h)\circ\gamma(g) = \id \]
so that \(\gamma(g)\) is invertible. Now, \(\gamma(s)\) is invertiblle, and \(gs^{-1} = \gamma(g)\circ\gamma(s)^{-1}\), so \(gs^{-1}\) is invertible.

Assume now that (1) holds. We must find suitable \(f\) and \(h\). To this end, note that we have an inverse \(\gamma(g)^{-1} = ft^{-1}\!:\gamma(y)\to\gamma(x')\) displayed by the zigzag
\[ y\overset{t}\ot y'\overset{f}\to x' \]
and composing this with \(\gamma(g)\), we see that
\[ y\overset{t}\ot y'\overset{g\circ f}\to y \]
is in the equivalence class of \(\id_y\). By Lemma \ref{lemma:equivalence-class-of-identity-in-Verdier-quotient}, \(g\circ f\in\calS(\calN)\). Performing a dual computation produces the morphism \(h\).
\end{proof}

\begin{proposition}
	Consider an object \(x\in\calT\). Then the following are equivalent.
	\begin{enumerate}[label=(\arabic*)]
	\item The unique morphism \(x\to 0\) in \(\calT\) maps to an isomorphism in \(\calT/\calN\).
	\item \(x\in\ker\gamma\).
	\item There exists some \(y\in\calT\) such that \(x\oplus y\in\calN\).
	\end{enumerate}
	In other words, \(\ker\gamma = \thick(\calN)\).
\end{proposition}
\begin{proof}
The equivalence between (1) and (2) is clear. We show that (1) and (3) are equivalent. Suppose that \(\gamma(x)\to0\) is an isomorphism. By Lemma \ref{lemma:equivalence-invertible-map-in-Verdier-quotient},
we can find some \(y\in\calT\) and some morphism \(0\to \Sigma y\) such that the zero map \(x\to0\to\Sigma y\) is in \(\calS(\calN)\). On the other hand, we have a distinguished triangle
\[ y \to x\oplus y\to x \overset{0}\to \Sigma y \]
so that \(\Sigma(x\oplus y)\) is a cone of \(x\overset{0}\to \Sigma y\), hence is in \(\calN\). We conclude that \(x\oplus y\in\calN\).

Conversely, assume (3). Given \(x\oplus y\in\calN\), the same distinguished triangle as above shows that \(x\overset{0}\to\Sigma y\) is in \(\calS(\calN)\). We finally observe that the maps
\[ 0 \to x\to 0,\quad x\to 0\to\Sigma y \]
are thus both in \(\calS(\calN)\), so \(x\to 0\) is sent to an isomorphism in \(\calT/\calN\) by Lemma \ref{lemma:equivalence-invertible-map-in-Verdier-quotient}.
\end{proof}

\begin{corollary}
	Let \(f\!:x\to y\) be a morphism in \(\calT\). Then \(f\) is sent to an isomorphism in \(\calT/\calN\) if and only if any cone of \(f\) is a direct summand of an object in \(\calN\).
\end{corollary}
\begin{corollary}
	Let \(\calT\) be a triangulated category, and let \(\calN\) be a null system in \(\calT\). Then there is a canonical isomorphism of categories
	\[ \calT/\calN \cong \calT/\thick(\calN). \]
\end{corollary}
\begin{proof}
One easily checks that \(\calT/\thick(\calN)\) satisfies the same universal property as \(\calT/\calN\).
\end{proof}

\subsection{Derived functors for triangulated categories}
We have provided (at least) two bits of machinery for constructing derived functors in the context of ordinary categories. However, in the context of triangulated categories,
one naturally wants a slightly different context. Ordinarily, one works in the 2-category \(\underline\Cat\), but here, we really want to work in another 2-category consisting
just of the triangulated categories.
\begin{definition}
	We define a strict 2-category \(\underline{\cat{TCat}}\) as follows.
	\begin{itemize}[label=\(\star\)]
	\item For objects, we have triangulated categories.
	\item For 1-morphisms, we have triangulated functors \((F,\sigma)\) where \(F\!:\calT\to\calT'\) and \(\sigma\!:F\circ\Sigma\To\Sigma'\circ F\) is a specified natural isomorphism.
	\item For 2-morphisms \((F,\sigma)\To (F',\sigma')\) we have natural transformations \(\eta\!:F\To F'\) compatible with the specified commutation natural isomorphisms, i.e.
	\begin{diagram*}
		F\circ\Sigma\ar[r,Rightarrow,"\sigma"]\ar[d,Rightarrow,"\eta\Sigma"'] & \Sigma'\circ F\ar[d,Rightarrow,"\Sigma'\eta"] \\
		F'\circ\Sigma\ar[r,Rightarrow,"\sigma'"] & \Sigma'\circ F'
	\end{diagram*}
	commutes.
	\end{itemize}
\end{definition}

One then mimicks the definition of an (absolute) Kan extension in \(\underline\Cat\) to get a definition of (absolute) (total) derived functors in the context of triangulated categories.
This is straightforward to do, so we do not spell it out. Furthermore, as the proofs involving deformations are in essense purely formal, they naturally extend very easily to the above context,
though we replace the notion of a relative category \((\calC,W)\) with that of a ``Verdier pair'' \((\calT,\calN)\) of a triangulated category \(\calT\) and a null system \(\calN\) in \(\calT\).

\begin{definition}
	Let \(F\!:\calT\to\calT'\) be a triangulated functor, and let \(\calN\) (resp.\ \(\calN'\)) be a null system in \(\calT\) (resp.\ \(\calT'\)). Consider a full triangulated subcategory \(\calC\)
	of \(\calT\). We say that \(\calC\) is \(F\)\emph{-injective} (with respect to \(\calN\) and \(\calN'\)) if the following conditions are satisfied:
	\begin{enumerate}[label=(\arabic*)]
	\item \(F(\calN\cap\calC)\subseteq\calN'\).
	\item For all \(x\in\calT\), there is an object \(x'\in\calC\) and a morphism \(x\to x'\) in \(\calS(\calN)\).
	\end{enumerate}
	Dually, we say \(\calC\) is \(F\)\emph{-projective} if it satisfies (1) above, and the following condition:
	\begin{itemize}[label=(2')]
	\item For all \(x\in\calT\), there is an object \(x'\in\calC\) and a morphism \(x'\to x\) in \(\calS(\calN)\).
	\end{itemize}
	We will say \(\calT\) \emph{has enough} \(F\)\emph{-injectives} (resp.\ \(F\)\emph{-projectives}) if there exists a full triangulated \(F\)-injective (resp.\ \(F\)-projective) category \(\calC\).
\end{definition}
\begin{theorem}
	Let \(F\!:\calT\to\calT'\) be a triangulated functor between triangulated functors equipped with null systems \(\calN\) and \(\calN'\). Then the following statements hold.
	\begin{enumerate}[label=(\arabic*)]
	\item Suppose that there exists enough \(F\)-injectives. Then \(F\) has an absolute total right derived functor, and it is a triangulated functor.
	\item Suppose that there exists enough \(F\)-projectives. Then \(F\) has an absolute total left derived functor, and it is a triangulated functor.
	\end{enumerate}
	Furthermore, in both of the above situations, the absolute total right/left derived functors are also absolute left/right Kan extensions in \(\underline{\cat{TCat}}\).
\end{theorem}
\begin{proof}
(1) and (2) are dual, so it suffices to prove only one of them. We prove (1). To this end, we wish to apply Theorem \ref{thm:right-multiplicative-system-right-derived-functor-exists}. Let
\(\calC\) be our promised full triangulated subcategory of \(F\)-injectives with respect to \(\calN\) and \(\calN'\). We see by assumption that (a) in the theorem is certainly satisfied.
To see that (b) is satisfied, note that the weak equivalences in \(\calC\) are exactly the morphisms which have a cone in \(\calN\cap\calC\), which is sent to \(\calN'\) under \(F\).
Therefore, we may apply Theorem \ref{thm:right-multiplicative-system-right-derived-functor-exists} to see that an absolute total right derived functor \(\bfR F\!:\calT/\calN\to\calT'/\calN\)
exists, and furthermore, that it is given by a composition
\[ \calT/\calN \overset{Q}\to \calC/(\calN\cap\calC)\overset{F'}\to\calT'/\calN'. \]
To see that \(\bfR F\) can be made into a triangulated functor, it suffices to see that the above functors \(Q\) and \(F'\) are triangulated. However, \(F'\) is induced by
the universal property Theorem \ref{thm:Verdier-quotient-universal-property}, and \(Q\) is a quasi-inverse of the triangulated equivalence \(\calC/(\calN\cap\calC)\iso\calT/\calN\)
also induced by this universal property from the inclusion \(\calC\inj\calT\). Therefore, both of them are automatically triangulated functors.

That the resulting functor is also a total left Kan extension in \(\underline{\cat{TCat}}\), it suffices to note that the proof of Theorem \ref{thm:right-multiplicative-system-right-derived-functor-exists}
(and in partcular Lemma \ref{lemma:kan-extension-criterion}) can be carried out entirely internally to this 2-category.
\end{proof}

\begin{remark}
	Of course, an obvious version of pseudofunctoriality also holds in this setting. In particular, given triangulated functors \(\calT\overset{F}\to\calT'\overset{G}\to\calT''\)
	between triangulated categories with null systems \(\calN\), \(\calN'\), and \(\calN''\), along with distinguished \(F\)-injectives \(\calC\subseteq\calT\) and \(G\)-injectives
	\(\calC'\subseteq\calT'\), there is a canonical natural transformation of triangulated functors
	\[ \bfR(G\circ F) \To \bfR G\circ\bfR F \]
	and this is a natural isomorphism if \(F\calC\subseteq\calC'\).
\end{remark}


%!TEX root = ../lectures.tex

\section{Localization sequences \& recollements of triangulated categories}\label{lecture:localization-sequences-of-triangulated-categories}
In the theory of localizations of categories, there is something called \emph{Bousfield localization.} Essentially, it concerns a way in which one can recognize and recover a localization from
certain other data. In the ``stable'' setting of triangulated categories, it takes on a particularly simple form, which is the topic of this lecture. It's a useful thing to learn about
for several reasons:
\begin{enumerate}[label=(\arabic*)]
\item Localization sequences, or rather recollements, appear often in nature, and give a nice formalism for explaining ``gluing'' phenomena in many situations, such as in algebraic geometry.
Often, in topics where one is trying to do geometry abstractly, one will seek recollements as a form of aid, indicating e.g.\ what should be considered open and closed.
\item Understanding the theory in the special case of triangulated categories is useful in motivating the general theory of Bousfield localizations.
\item The proofs involved are tremendously helpful in demonstrating how one handles the machinery of a tringulated category, and give useful intuition for topics like t-structures.
\end{enumerate}

\subsection{Orthogonal complements, \& adjoints arising from them}
\begin{definition}
	Let \(\calT\) be a triangulated category, and let \(\calC\subseteq\calT\) be a full subcategory. Define the full subcategories
	\begin{align*}
		\calC^\perp := \{ y\in\calT\mid \forall x\in\calC,\, \calT(x,y) \cong 0 \}, \\
		\bperp\calC := \{ y\in\calT\mid \forall x\in\calC,\, \calT(y,x) \cong 0 \}.
	\end{align*}
\end{definition}
\begin{proposition}\label{prop:basic-triangulated-category-orthogonal-properties}
	Let \(\calT\) be a triangulated category, and let \(\calT'\) be a triangulated subcategory. Then the following statements hold.
	\begin{enumerate}[label=(\arabic*)]
	\item \(\calT'^\perp\) and \(\bperp{\calT'}\) are thick replete full triangulated subcategories of \(\calT\).
	\item \(\calT'\cap\calT'^\perp \simeq \{0\}\simeq \calT'\cap\bperp{\calT'}\).
	\item \(\calT' \subseteq \bperp{(\calT'^\perp)}\) and \(\calT'\subseteq (\bperp{\calT'})^\perp\).
	\end{enumerate}
\end{proposition}
\begin{proof}
(1) By definition, \(\calT'^\perp\) is full; that it is furthermore replete is trivial. To see that it is closed under direct summands, observe that if \(x,y\in\calT'^\perp\), then for all \(z\in\calT'\)
\[ 0 \cong \calT(z,x\oplus y) \cong \calT(z,x)\oplus\calT(z,y)\quad\implies\quad \calT(z,x)\cong\calT(z,y)\cong 0.  \]
It remains to see that \(\calT'^\perp\) is a triangulated subcategory. By Proposition \ref{prop:null-system-subcategory-characterization}, it suffices to check that \(\calT'^\perp\)
is a null system. Clearly, \(0\in\calT'^\perp\). Next, if \(x\in\calT'^\perp\), then for all \(z\in\calT'\) we have
\[ \calT(z,\Sigma x) \cong \calT(\Sigma^{-1}z,x) \cong 0 \]
since \(\Sigma^{-1}z\in\calT'\), on account of \(\calT'\) being a triangulated subcategory of \(\calT\). Finally, we need to check that \(\calT'^\perp\) is closed under extensions. For this,
consider a distinguished triangle
\[ x \to x' \to x'' \to \Sigma x \]
where \(x,x''\in\calT'^\perp\). Applying \(\calT(z,-)\), we have the exact sequence
\[ 0 = \calT(z,x) \to \calT(z,x')\to\calT(z,x'') = 0 \]
so that \(\calT(z,x') \cong 0\) as desired.

(2) If \(x\in\calT'\cap\calT'^\perp\), then for all \(z\in\calT'\) we have
\[ \calT'(z,x) \cong 0 \cong \calT'(z,0) \]
so by the Yoneda lemma, \(x\cong 0\).

(3) Let \(x\in\calT'\). If \(z\in\calT'^\perp\), then clearly \(\calT(x,z) = 0\), so \(x\in\bperp{(\calT'^\perp)}\).

The remaining statements are dual.
\end{proof}
\begin{proposition}\label{prop:adjoint-from-orthogonal-triangle}
	Let \(\calT\) be a triangulated category, and suppose we have a replete full subcategory \(\calT'\) of \(\calT\) satisfying the following properties.
	\begin{enumerate}[label=(\alph*)]
	\item For all \(x\in\calT\), there is a distinguished triangle
	\[ x'\to x\to x''\to \Sigma x' \]
	where \(x'\in\calT'\) and \(x''\in(\calT')^\perp\).
	\item \(\Sigma\calT' \subseteq \calT'\).
	\end{enumerate}
	Then the inclusion \(\iota'\!:\calT'\inj\calT\) admits a right adjoint \(\tau'\!:\calT\to\calT'\) given by \(x\mapsto x'\), and for all \(x\in\calT\) the distinguished triangle in (a) has the form
	\[ \tau'(x) \to x \to x'' \to \Sigma\tau'(x) \]
	where \(\tau'(x)\to x\) is the counit of the adjunction.
\end{proposition}
\begin{proof}
The idea is that the distinguished triangle in (a) gives us everything.

We produce a right adjoint to \(\iota'\). For \(x\in\calT\), pick a distinguished triangle as in (b) and let \(\tau'(x) := x'\). We show that this object represents the functor \(\calT(\iota'(-),x)\!:\calT'\to\Ab\). For this, just
apply \(\calT(y',-)\) with \(y'\in\calT'\) to the induced distinguished triangle
\[ \Sigma^{-1}x''\to \tau'(x) \overset{\varepsilon_x}\to x\to x'' \]
to get, using Proposition \ref{prop:pre-triangulated-representable-functors-are-cohomological}, the exact sequence
\[ 0 \overset{(b)}= \calT(y',\Sigma^{-1}x'')\to\calT(y',\tau'(x))\overset{\varepsilon_{x,*}}\to\calT(y',x)\to\calT(y',x'') = 0 \]
so that we have a natural isomorphism \(\calT(y',\tau'(x))\iso\calT(y',x)\), i.e.\ the natural isomorphism
\[\varepsilon_{x,*}\!:\calT'(-,\tau'(x))\To\calT(\iota'(-),x).\]
The rest now follows by abstract nonsense.
\end{proof}

\subsection{Short exact sequences of triangulated categories}
We introduce the below terminology, which is not entirely standard.
\begin{definition}
	A \emph{(short) exact sequence of triangulated categories} is a pair of functors
	\[ \calT' \overset{i}\longto \calT \overset{p}\longto \calT'' \]
	between triangulated categories such that
	\begin{enumerate}[label=(\arabic*)]
	\item the functor \(i\) is fully faithful, and has essential image a (definitionally replete) thick triangulated subcategory of \(\calT\),
	\item the functor \(p\) is essentially surjective, and
	\item \(p\) exhibits \(\calT''\) as the Verdier quotient \(\calT/i(\calT')\) in the strict sense, i.e.\ \(p\) induces an isomorphism of categories \(\calT/i(\calT')\cong\calT''\).
	\end{enumerate}
	We say it is a \emph{weak} exact sequence if the condition in (3) only holds up to equivalence.
\end{definition}
\begin{remark}
	We see that, by definition, any null system (i.e.\ replete full triangulated subcategory) \(\calN\) in \(\calT\) induces a canonical exact sequence
	\[ \calN\inj\calT\sur\calT/\calN. \]
	Furthermore, basically also by definition, all exact sequences can be written in this form up to equivalence.
\end{remark}
\begin{exercise}
	Define a suitable notion of morphism of exact sequences of triangulated categories. Show that any exact sequence of triangulated functors can
	be written in the form indicated in the above remark.
\end{exercise}

In traditional settings, one has the following intuition: given a short exact sequence of Abelian groups
\[ 0 \to A \to B \to C \to 0, \]
we may think of \(B\) as a ``sum'' of \(C\) and \(A\). Of course, it need not be an actual direct sum: indeed, such a sequence exhibits \(B\) as \(A\oplus C\) if and only if it splits.
On the other hand, it is still useful to think of extensions as describing some potentially non-trivial way to combine \(C\) with \(A\) to get \(B\).

In the non-Abelian setting, such as an exact sequence of \emph{groups}
\[ 1 \to H \to G \to K \to 1 \]
one still has this intuition, but one \emph{loses} the ability to detect that \(G = H\times K\) using splitting. More precisely, if it splits on the left, all is as usual; if it splits on the \emph{right,} however,
one cannot know that \(G\cong H\times K\), as there are examples where this fails. These examples are provided by \emph{semidirect} products.

For triangulated categories, the situation most resembles that of the Abelian case. This is no surprise, as triangulated categories model ``homotopy Abelian'' or ``stable'' phenomena, where we have some form of commutativity.
For example, we will see that splitting on the left is the same as splitting on the right. On the other hand, the setting of triangulated categories is also more subtle, and in this way has some of the flavour of the non-Abelian case:
there are several ways in which one side can split (given by having a left or right adjoint), and even in the presence of a splitting, one cannot reconstruct \(\calT\) from \(\calT'\) and \(\calT''\). In a sense,
this last subtlety is a deficiency of triangulated categories specifically, as in enhanced contexts (such as dg-categories, or stable \(\infty\)-categories) it disappears.

We will be interested in exact sequences with nice properties.
\begin{definition}
	An exact sequence
	\[ \calT' \overset{i}\longto \calT \overset{p}\longto \calT'' \]
	is called a
	\begin{enumerate}[label=(\arabic*)]
	\item \emph{localization sequence} if \(i\) and \(p\) admit right adjoints;
	\item \emph{colocalization sequence} if \(i\) and \(p\) admit left adjoints.
	\item \emph{recollement} if it is both a localization sequence and a colocalization sequence.
	\end{enumerate}
\end{definition}

\subsection{A recognition theorem for localizations}
\begin{theorem}\label{thm:fully-faithful-adjoint-gives-localization}
	Suppose we have an adjunction
	\begin{tikzcd}[cramped]
		\calD\ar[from=r,bend right,"L"',""{name=A,below}] & \calC, \ar[from=l,bend right,"R"',""{name=B,above}]\ar[from=A,to=B,symbol=\dashv]
	\end{tikzcd}
	between categories. Let \(W_L\) (resp.\ \(W_R\)) be the wide subcategory of \(\calC\) (resp.\ \(\calD\)) spanned by those morphisms sent to isomorphisms by \(L\) (resp.\ by \(R\)).
	Then the following statements hold.
	\begin{enumerate}[label=(\arabic*)]
	\item \(R\) is fully faithful if and only if the induced functor \(L'\!:\calC[W_L^{-1}] \to \calD\) is an equivalence.
	\item \(L\) is fully faithful if and only if the induced functor \(R'\!:\calD[W_R^{-1}] \to \calC\) is an equivalence.
	\end{enumerate}
\end{theorem}
\begin{proof}
The statements (1) and (2) are dual, so it suffices to prove (1). By Lemma \ref{lemma:fully-faithful-adjoint}, we know that \(R\) is fully faithful if and only if the counit
\(\varepsilon\!:LR\To\1_\calD\) is a natural isomorphism, so we must show that this occurs if and only if \(L'\) is an equivalence. Let \(\eta\!:\1_\calC\To RL\) be the unit,
and let \(\gamma_L\!:\calC\to\calC[W_L^{-1}]\) be the localization functor.

Assume that \(\varepsilon\) is a natural isomorphism. We show that \(\gamma_L R\) is an inverse to \(L'\). One composition is easy:
\[ L'\gamma_L R \cong LR \cong \1. \]
Here, we used the definition of \(L'\), and the counit \(\varepsilon\!:LR\cong\1\). For the other direction, we use that, by Lemma \ref{lemma:fully-faithful-adjoint}, the
natural transformation \(L\eta\) is a natural isomorphism; in particular, every component of \(\eta\) is thus contained in \(W_L\), so \(\eta\) is a weak natural equivalence.
One deduces that \(\gamma_L\eta\!:\gamma_L\To\gamma_L RL\) is a natural isomorphism, so
\[ \gamma_L RL'\gamma_L \cong \gamma_L RL \cong \gamma_L \]
which by universal property implies that \(\gamma_L R L'\cong\1\).

Conversely, assume that \(L'\!:\calC[W_L^{-1}]\simeq\calD\). We show that \(\1_\calD\) is left adjoint to \(LR\) with counit \(\varepsilon\); in that case, the adjunction map \(\calD(x,y)\cong\calD(x,LRy)\)
tells us that the counit components \(\varepsilon_y\!: LRy\to y\) are isomorphisms. So, to this end, note that for any category \(\calD'\), composition with \(L\)
gives a fully faithful functor \(L^*\!:\Fun(\calD,\calD')\to\Fun(\calC,\calD')\), as it is naturally isomorphic to the composite
\begin{diagram*}
	\Fun(\calD,\calD') \ar[r,"\sim","(\circ L')"'] & \Fun(\calC[W_L^{-1}],\calD') \ar[r,"\sim","(\circ\gamma_L)"'] & \Fun_{W_L}(\calC,\calD') \ar[r,hook] & \Fun(\calC,\calD').
\end{diagram*}
Choosing \(\calD' = \calD\), composition with \(L\) gives a fully faithful functor \(\Fun(\calD,\calD)\to\Fun(\calC,\calD)\), and in particular, we may find a natural
transformation \(\eta'\!:\1_\calD\To LR\) in the preimage of \(L\eta\!:L\To LRL\), so that \(L\eta = \eta'L\). This \(\eta'\) will be our unit. It remains
to check the triangle identities; from the ones for the adjunction \(L \ladj R\), we have
\[ \id_L = \varepsilon L\circ L\eta = \varepsilon L \circ \eta'L = (\varepsilon\circ\eta')L \]
which provides one of the triangle identities. For the other, note that we have
\[ \id_R = R\varepsilon\circ\eta R \implies \id_{LR} = LR\varepsilon\circ L\eta R = LR\varepsilon\circ\eta'LR. \]
This completes the proof.
\end{proof}
\begin{remark}
	Note that the above theorem necessarily only holds up to equivalence, i.e.\ in the ``weaker'' sense of localization. Inspecting the proof shows that this is essentially
	unavoidable.
\end{remark}
\begin{corollary}\label{corollary:triangulated-fully-faithful-adjoint-gives-localization}
	Suppose we have an adjunction
	\begin{tikzcd}[cramped]
		\calT'\ar[from=r,bend right,"L"',""{name=A,below}] & \calT, \ar[from=l,bend right,"R"',""{name=B,above}]\ar[from=A,to=B,symbol=\dashv]
	\end{tikzcd}
	between triangulated categories. Let \(\calN_L = \ker{L}\) and \(\calN_R = \ker{R}\). Then the following statements hold.
	\begin{enumerate}[label=(\arabic*)]
	\item \(R\) is fully faithful if and only if \(L\) induces an equivalence of triangulated categories \(\calT/\calN_L \simeq \calT'\).
	\item \(L\) is fully faithful if and only if \(R\) induces an equivalence of triangulated categories \(\calT'/\calN_R \simeq \calT\).
	\end{enumerate}
\end{corollary}
\begin{proof}
(2) is dual to (1), so we prove (1). Let \(W_L\) be the morphisms in \(\calT\) sent to isomorphisms by \(L\). By Theorem \ref{thm:fully-faithful-adjoint-gives-localization}, \(R\)
is fully faithful if and only if \(L\) induces an equivalence \(\calT[W_L^{-1}] \simeq \calT'\). Now, simply note that \(W_L = \calS(\calN_L)\). Indeed, \(Lf\) is an isomorphism
if and only if its cone is zero, if and only if the cone of \(f\) is in \(\ker{L} = \calN_L\). Furthermore, note that the induced equivalence is triangulated by universal property.
\end{proof}

\subsection{Properties of the adjoints in localization sequences}
\begin{lemma}
	Let \(F\!:\calC\inj\calD\) be a fully faithful functor. If \(F\) has a right adjoint \(F\ladj R\), then \(R\) is essentially surjective.
\end{lemma}
\begin{proof}
By Lemma \ref{lemma:fully-faithful-adjoint}, since \(F\) is fully faithful, we know that the unit \(\eta\!:\1\To RF\) is a natural isomorphism. In particular, for any \(x\in\calC\), we have \(x \cong RFx\).
\end{proof}
\begin{theorem}\label{thm:localization-sequence-adjoints-properties}
	Suppose we have an exact sequence of solid arrows
	\begin{diagram*}
		\calT'\ar[r,"i",""{above,name=A}] & \calT \ar[r,"p",""{above,name=C}]\ar[l,bend left,shift left,dashed,"i_R",""{below,name=B}] & \mathmakebox[\widthof{\calT'}][l]{\calT''.} \ar[l,bend left,shift left,dashed,"p_R",""{below,name=D}] \ar[from=B,to=A,symbol=\vdash]\ar[from=D,to=C,symbol=\vdash]
	\end{diagram*}
	Then the following statements hold.
	\begin{enumerate}[label=(\arabic*)]
	\item If \(i\) has a right adjoint \(i_R\), then \(i_R\) is essentially surjective.
	\item If \(p\) has a right adjoint \(p_R\), then \(p_R\) is fully faithful.
	\item In the situation of (2), \(p_R\) induces an equivalence \(\calT''\simeq i(\calT')^\perp\) with quasi-inverse given by the composite
	\[ i(\calT')^\perp \inj \calT \overset{p}\to \calT''.  \]
	\end{enumerate}
\end{theorem}
\begin{proof}
(1) Since \(i\) is fully faithful, we are done by the above lemma.

(2) The idea is to explot Theorem \ref{thm:fully-faithful-adjoint-gives-localization}, or rather, Corollary \ref{corollary:triangulated-fully-faithful-adjoint-gives-localization}. We note
that \(\ker{p} = i(\calT')\), so the requirement that \(p\) induces an equivalence \(\calT/\ker{p}\simeq\calT''\) is satisfied, hence \(p_R\) is fully faithful.

(3) It suffices to check that the essential image of \(p_R\) is \(i(\calT')^\perp\). Let \(z\cong i(z_0)\in i(\calT')\). For any \(y\cong p_R(y_0)\in p_R(\calT'')\), we then have
\[ \calT(x,y) \cong \calT(i(z_0),p_R(y_0)) \cong \calT''(pi(z_0),y_0) \cong 0 \]
since \(i(\calT') = \ker{p}\). We deduce that \(p_R(\calT'')\subseteq i(\calT')^\perp\). Conversely, let \(x\cong i(x_0) \in i(\calT')^\perp\), and consider the unit \(\eta\!:\1\To p_Rp\). Taking the cone of \(\eta_x\)
\[ x \overset{\eta_x}\to (p_Rp)x \to x' \to \Sigma (p_Rp)x, \]
we note that by (2), \(p_R\) is fully faithful, so \(p\eta\!: p \To pp_Rp\) is a natural isomorphism, and hence \(p(x'') \cong 0\). Therefore, \(x'\in i(\calT')\). Now, both \(x\) and \(p_Rpx\) are in \(i(\calT')^\perp\),
which is a full triangulated subcategory of \(\calT\) and is thus closed under cones; in particular, \(x' \in i(\calT')\cap i(\calT')^\perp\simeq \{0\}\), so \(x' \cong 0\). We conclude that \(\eta_x\!:x\cong p_Rpx\).
\end{proof}

\subsection{Splittings of exact sequences}
\begin{lemma}\label{lemma:orthogonal-complement-is-kernel-of-adjoint}
	Consider a fully faithful triangulated functor \( i\!:\calT' \inj \calT \) and assume that \(i\) has a right adjoint \(i_R\). Then
	\[ i(\calT')^\perp = \{ x\in\calT\mid i_R(x) \cong 0 \} = \ker{i_R}. \]
\end{lemma}
\begin{proof}
If \(x\in \calT\), then for all \(z\cong i(z_0)\in i(\calT')\), we have
\[ \calT(z,x) \cong \calT(i(z_0),x) \cong \calT'(z_0,i_R(x)). \]
Thus, if \(x\in i(\calT')^\perp\), we see that \(\calT'(-, i_R(x)) \cong 0\), so \(x\in\ker{i_R}\). Conversely, if \(x\in\ker{i_R}\), then the above right Hom-set is zero, hence the above left is too.
\end{proof}

\begin{lemma}\label{lemma:perp-of-perp-in-presence-of-orthogonal-triangle}
	Consider a fully faithful triangulated functor \( i\!:\calT' \to \calT \) and assume that for all \(x\in\calT\), there is a distinguished triangle
	\[ x' \to x \to x'' \to \Sigma x' \]
	where \(x'\in i(\calT')\) and \(x''\in i(\calT')^\perp\). Then \(\bperp{(i(\calT')^\perp)} = \thick(i(\calT'))\).
\end{lemma}
\begin{proof}
Let \(x\in \bperp{(i(\calT')^\perp)}\). We have a distinguished triangle
\[ x' \to x \to x'' \to \Sigma x' \]
by assumption, and by definition of \(x\), we have \(\calT(x,x'') \cong 0\). Shifting around, we thus have
\[ \Sigma^{-1}x'' \to x' \to x \overset{0}\to x'' \]
which by Corollary \ref{corollary:direct-sum-triangle} implies that \(x' \cong \Sigma^{-1}x'' \oplus x\). In particular, \(x\) is a direct summand of \(x'\in i(\calT')\),
so \(x\in\thick(i(\calT'))\). This shows \(\bperp{(i(\calT')^\perp)} \subseteq\thick(i(\calT'))\).

We already know that \(i(\calT')\subseteq\bperp{(i(\calT')^\perp)}\), and since \(\bperp{(i(\calT')^\perp)}\) is thick, this means that \(\thick(i(\calT'))\subseteq\bperp{(i(\calT')^\perp)}\).
\end{proof}

\begin{theorem}
	Consider an exact sequence
	\[ \calT' \overset{i}\longto \calT \overset{p}\longto \calT''. \]
	Then the following statements are equivalent.
	\begin{enumerate}[label=(\arabic*)]
	\item The sequence is a localization sequence.
	\item The functor \(i\) has a right adjoint \(i_R\).
	\item The functor \(p\) has a right adjoint \(p_R\).
	\item For all \(x\in\calT\), there are \(x'\in i(\calT')\), \(x''\in i(\calT')^\perp\) sitting in a distinguished triangle
	\[ x' \to x \to x'' \to \Sigma x'. \]
	%\item The composite \(i(\calT')^\perp \inj \calT \overset{p}\to \calT''\) is an equivalence of triangulated categories.
	\end{enumerate}
	Furthermore, in the above situation, we have an equivalence of triangulated categories and equality
	\[ \calT/(i(\calT')^\perp) \iso \calT',\quad \bperp{(i(\calT')^\perp)} = i(\calT'). \]
\end{theorem}
\begin{proof}
Clearly, if (2) through (4) are equivalent, then (1) is equivalent to them all. We have already seen in Proposition \ref{prop:adjoint-from-orthogonal-triangle} that (4) implies (2). We will show that
\((2)\Rightarrow(4)\Rightarrow(3)\Rightarrow(2)\).% and \((3) \Leftrightarrow (5)\).

\((2)\Rightarrow(4)\). Let \(\varepsilon\!:ii_R\To\1\) be the counit. By taking the cone, we then have a distinguished triangle
\[ i(i_Rx) \to x \to x'' \to \Sigma i(i_Rx) \]
for all \(x\in\calT\). Since \(i\) is fully faithful, \(i_R\varepsilon\!: i_Rii_R\To i_R\) is a natural isomorphism, so applying \(i_R\) to the above (and noting that adjoints of triangulated
functors are triangulated), we have the distinguished triangle
\[ (i_Rii_R)x \iso i_Rx \to i_Rx'' \to \Sigma (i_Rii_R)x \]
so that \(i_Rx'' \cong 0\). By Lemma \ref{lemma:orthogonal-complement-is-kernel-of-adjoint}, we see that \(x''\in i(\calT')^\perp\).

\((4)\Rightarrow(3)\). Applying the dual of Proposition \ref{prop:adjoint-from-orthogonal-triangle}, we see that the inclusion \(i'\!:i(\calT')^\perp \inj \calT\) admits a \emph{left adjoint} \(L\).
Since \(L\) has fully faithful right adjoint \(i'\), it induces an equivalence \(F\!:\calT/\ker{L}\simeq i(\calT')^\perp\) by Corollary \ref{corollary:triangulated-fully-faithful-adjoint-gives-localization}.
Now, by the dual of Lemma \ref{lemma:orthogonal-complement-is-kernel-of-adjoint}, we have \(\ker{L} = \bperp{(i(\calT')^\perp)}\), and since \(i(\calT')\) is thick, Lemma \ref{lemma:perp-of-perp-in-presence-of-orthogonal-triangle}
tells us that \(\ker{L} = i(\calT')\). Therefore, we get an equivalence \(\calT'' \simeq i(\calT')^\perp\) which we abusively also call \(F\).
The following computation shows that \(i'F\) is right adjoint to \(p\):
\[ \calT''(p-,-) \cong i(\calT)^\perp(Fp-,F-) = i(\calT)^\perp(L-,F-) \cong \calT(-,i'F-). \]

\((3)\Rightarrow(2)\). Let \(p\) have a right adjoint \(p_{R}\). By taking the cocone of the components of the unit \(\eta\!:\1\To p_Rp\), we have distinguished triangles
\[ x' \to x \to p_Rp(x) \to \Sigma x' \]
By Theorem \ref{thm:localization-sequence-adjoints-properties}, \(p_R\) is fully faithful, hence \(p\eta\!:p \To pp_Rp\) is a natural isomorphism. Therefore, applying \(p\) above yields
that \(x'\in\ker{p} = i(\calT')\). Furthermore, one trivially sees that \(p_R(\calT'') \subseteq i(\calT')^\perp\), so we may apply Proposition \ref{prop:adjoint-from-orthogonal-triangle}.

% \((3)\Rightarrow(5)\). This is Theorem \ref{thm:localization-sequence-adjoints-properties}. We spell out the missing step: since \(p_R\) is fully faithful, \(pp_R \cong \1\), but on the other hand,
% since we know what the essential image is, \(p_R\) factors as \(\calT'' \simeq i(\calT')^\perp \inj \calT\), so the proposed composite is automatically a left quasi-inverse, from which it follows formally
% that it is also a right quasi-inverse.

% \((5)\Rightarrow(3)\). Let \(q\!:\calT''\to i(\calT')^\perp\) be a quasi-inverse of the equivalence \(i(\calT')^\perp\iso\calT''\). Then the composition
% \[ \calT'' \overset{q}\to i(\calT')^\perp \inj \calT  \]
% is a right adjoint of \(p\). Indeed, we have
% \[ \calT''(p(x),y) \cong \calT(q(p(x)),q(y))  \]

For the final statements, note that we already showed that (4) implies the equality
\[ i(\calT') = \bperp{(i(\calT')^\perp)}\]
in Lemma \ref{lemma:perp-of-perp-in-presence-of-orthogonal-triangle}; for the equivalence, since \(i\) is fully faithful, by Corollary \ref{corollary:triangulated-fully-faithful-adjoint-gives-localization} the right
adjoint \(i_R\) induces an equivalence \(\calT/\ker{i_R} \simeq \calT'\). Since \(\ker{i_R} = i(\calT')^\perp\), we are done.
\end{proof}
\begin{remark}
	Observe that, in fact, the implications \((2)\Leftrightarrow(4)\Rightarrow(3)\) hold without assuming that the essential image of \(\calT'\) in \(\calT\) is thick. Indeed,
	the only step where thickness appears there is in \((4)\Rightarrow(3)\), showing that \(\ker{L} = i(\calT')\); without thickness, however, one has \(\ker{L} = \thick(i(\calT'))\). However,
	we know that \(\calT/\calN \simeq \calT/\thick(\calN)\), so this actually does not matter.

	The other implications, however, strictly make use of thickness in an essential way, and so may not be naturally generalized.
\end{remark}

\subsection{Recollements}\label{subsection:recollements}
Recall that an exact sequence of triangulated categories is a \emph{recollement} if it both a localization sequence and a colocalization sequence, i.e.\ if all adjoints exist.
In other words, a recollement looks like this:
\begin{diagram*}[column sep=large]
	\calT'\ar[r,"i" description,""{below,name=A},""{above,name=AA}] &
		\calT \ar[r,"p" description,""{below,name=C},""{above,name=CC}]\ar[l,bend left,shift left,"i_R",""{above,name=B}]\ar[l,bend right,shift right,"i_L"',""{below,name=BB}] &
		\mathmakebox[\widthof{\calT'}][l]{\calT''.} \ar[l,bend left,shift left,"p_R",""{above,name=D}]\ar[l,bend right,shift right,"p_L"',""{below,name=DD}]
		\ar[from=B,to=A,symbol=\vdash]\ar[from=D,to=C,symbol=\vdash]
		\ar[from=AA,to=BB,symbol=\vdash]\ar[from=CC,to=DD,symbol=\vdash]
\end{diagram*}

The results we have produced (and their duals) tell us that having a recollement is the same as having distinguished triangles
\begin{align*}
	p_Lp(x) \to x \to ii_L(x) \to \Sigma p_Lp(x), \\
	ii_R(x) \to x \to p_Rp(x) \to \Sigma ii_R(x),
\end{align*}
telling us about the adjoints. Of course, the above make use of the adjoints, so really we should say
\begin{align*}
	x' \to x \to x'' \to \Sigma x',\quad &x'\in\bperp{(i(\calT'))},\quad x''\in i(\calT'), \\
	x' \to x \to x'' \to \Sigma x',\quad &x'\in i(\calT'),\quad x''\in i(\calT')^\perp.
\end{align*}
Perhaps surprisingly, these things do appear in nature, and one may see them as providing a way to distinguish between \emph{open} and \emph{closed} sets.
In particular, for any topological space \(X\), let \(\sfD(X)\) denote the derived category of sheaves of Abelian groups on \(X\), and consider a closed
set \(Z\subseteq X\). Write \(U := X\backslash Z\), and let \(i\!:Z\inj X\), respectively \(j\!:U\inj X\), be the inclusions. Then there is a recollement of the form
\begin{diagram*}[column sep=large]
	\sfD(Z)\ar[r,"i_*" description,""{below,name=A},""{above,name=AA}] &
		\sfD(X) \ar[r,"j^*" description,""{below,name=C},""{above,name=CC}]\ar[l,bend left,shift left,"i^!",""{above,name=B}]\ar[l,bend right,shift right,"i^*"',""{below,name=BB}] &
		\sfD(U). \ar[l,bend left,shift left,"j_*",""{above,name=D}]\ar[l,bend right,shift right,"j_!"',""{below,name=DD}]
		\ar[from=B,to=A,symbol=\vdash]\ar[from=D,to=C,symbol=\vdash]
		\ar[from=AA,to=BB,symbol=\vdash]\ar[from=CC,to=DD,symbol=\vdash]
\end{diagram*}
In the above, we have suppressed explicitly writing how the functors are derived, as it is not important in the context of merely presenting this as an example.
The important point is that this kind of situation arises whenever one considers the complement of a closed subspace, even in geometric situations such as for schemes. More
generally, one can go backwards and use this to ``identify'' which kinds of objects should be considered open versus closed, by placing them within an appropriate
recollement.

In a recollement
\begin{diagram*}[column sep=large]
	\calT'\ar[r,"i" description,""{below,name=A},""{above,name=AA}] &
		\calT \ar[r,"p" description,""{below,name=C},""{above,name=CC}]\ar[l,bend left,shift left,"i_R",""{above,name=B}]\ar[l,bend right,shift right,"i_L"',""{below,name=BB}] &
		\mathmakebox[\widthof{\calT'}][l]{\calT''} \ar[l,bend left,shift left,"p_R",""{above,name=D}]\ar[l,bend right,shift right,"p_L"',""{below,name=DD}]
		\ar[from=B,to=A,symbol=\vdash]\ar[from=D,to=C,symbol=\vdash]
		\ar[from=AA,to=BB,symbol=\vdash]\ar[from=CC,to=DD,symbol=\vdash]
\end{diagram*}
the functor \(i_R\circ p_L\) is sometimes referred to as the \emph{gluing functor.} This can be made sense of using the above example of sheaves: the functor \(j_!\) is given
by ``extending by zero'' while \(i^*!\) considers those sections with support in \(Z\).

\subsection{Appendix: Verdier quotients of stable \(\infty\)-categories as cofibers}
The appropriate \(\infty\)-categorical version of triangulated categories is given by stable \(\infty\)-categories. A natural question is thus to ask in
what form the Verdier quotient construction appears in the theory of stable \(\infty\)-categories. There is one evident approach: do the same thing as we did
for triangulated categories. That is, given a stable \(\infty\)-category \(\calD\), we consider a stable subcategory \(\calC\inj\calD\), and form
a class of morphsisms \(\calS(\calC) := \{ (f\!:x\to y)\in\calD\mid \cofib{f}\in\calC \}\). Then the Verdier quotient should be the localization \(\calD[\calS(\calC)^{-1}]\).

On the other hand, there is a different approach, and one which is a sense more elegant. The theory of colimits of stable \(\infty\)-categories is very well-behaved,
particularly if the \(\infty\)-categories in question are presentable. This leads to the
\begin{definition}
	Let \(F\!:\calC\to\calD\) be a fully faithful exact functor of presentable stable \(\infty\)-categories. Then the Verdier quotient \(\calD/\calC\)
	is the cofiber of \(F\). That is, it is the pushout
	\begin{diagram*}
		\calC\ar[r,"F"]\ar[d] & \calD\ar[d] \\
		*\ar[r] & \calD/\calC\ar[ul,pushout]
	\end{diagram*}
	in the \(\infty\)-category of presentable stable \(\infty\)-categories.
\end{definition}
This is a rather beautiful and simple definition, capturing precisely our intuition for what exact sequences are. One can show that, in various settings,
this agrees with the previously suggested definition as a localization; this is done in \cite{Blumberg_Gepner_Tabuada_2013} and \cite{drew2015verdierquotientsstablequasicategories}.
Similarly, they show that the concept agrees with that on triangulated categories: it is proven that \(\ho(\calD/\calC)\simeq\ho(\calD)/\ho(\calC)\).


%!TEX root = ../lectures.tex

\section{Abstract cohomology through t-structures}\label{lecture:abstract-cohomology-through-t-structures}
An important topic of homological algebra is \emph{(co)homology.} Indeed, the entire subject is named after it. Typically, cohomology is handled (one way or another) through
derived categories, by using that the cohomology functors on chain complexes descend to cohomology functors on the derived category. One can then study various properties in a nice and systematic way.
However, there is also an abstract formulation of cohomology on the level of triangulated categories, given by the theory of \emph{t-structures.} These were initially
invented in \cite{faisceaux-pervers}, where they acted as a way to find Abelian categories living inside triangulated categories (in particular, Abelian categories of so-called \emph{perverse sheaves}).

While the formalism of t-structures should really be motivated by seeing the concrete example of derived categories, we choose to reverse this
and instead view t-structures as a formal setting in which cohomology arises, later showing that this may be employed in the standard cases.
As a result, some of the axioms may at first seem a bit unmotivated, but the hope is that they are still convincing enough to be considered reasonable. We essentially follow
\cite{kashiwara-schapira-book-2}.

\subsection{Truncation functors \& t-structures}
In essence, the cohomology functors on chain complexes come about due to the natural \emph{grading} that chain complexes have: indeed, given a chain complex \(x^\bullet\in\sfC(\calA)\), we can consider
the piece \(x^i\) living in degree \(i\). Cohomology comes out of the relation between this piece and its neighbouring pieces. Accordingly, t-structures start with the perspective that this
should be doable in more generality by \emph{specifying} this grading as data.

\begin{definition}
	Let \(\calT\) be a triangulated category. A \emph{t-structure} on \(\calT\) is a pair \((\calT^{\leq 0}, \calT^{\geq 0})\) of replete full subcategories of \(\calT\) (the \emph{aisle} and \emph{co-aisle,} respectively;
	alternatively, the \emph{coconnective piece} and the \emph{connective piece}) satisfying the below conditions. For any \(n\in\Z\), we set \(\calT^{\leq n} := \Sigma^{-n}\calT^{\leq 0}\) and \(\calT^{\geq n} := \Sigma^{-n}\calT^{\geq 0}\).
	\begin{enumerate}[label=(T\arabic*)]
		\item \((\calT^{\leq 0})^\perp \supseteq \calT^{\geq 1}\). That is, if \(x\in\calT^{\leq 0}\) and \(y\in\calT^{\geq 1}\), then \(\calT(x,y) = 0\).
		\item \(\calT^{\leq -1}\subseteq\calT^{\leq 0}\) and \(\calT^{\geq 1}\subseteq\calT^{\geq 0}\).
		\item For all \(x\in\calT\), there is a distinguished triangle
		\[ x' \to x \to x'' \to \Sigma x' \]
		such that \(x'\in\calT^{\leq 0}\) and \(x''\in\calT^{\geq 1}\).
	\end{enumerate}
\end{definition}
\begin{remark}
	Requiring the subcategories in a t-structure to be replete is not \emph{strictly} necessary for all purposes, but it is a nice simplifying assumption, and furthermore, it is harmless to reduce to this case
	since one can always just enlarge one's subcategories to be replete.
\end{remark}

\begin{example}
	Let \(\calT\) be a triangulated category. There are two trivial t-structures one may put on \(\calT\), namely \((\calT,0)\) and \((0,\calT)\).
\end{example}
\begin{example}
	When we get around to the derived category in later lectures, we will see that for an Abelian category \(\calA\), the pair
	\[ (\sfD^{\leq 0}(\calA),\sfD^{\geq 0}(\calA)) \]
	given by the subcategories spanned by those complexes concentrated in non-positive or non-negative cohomological degrees determines a t-structure. This is called the \emph{standard t-structure}
	on \(\sfD(\calA)\).
\end{example}
\begin{exercise}
	Let \((\calT^{\leq0},\calT^{\geq 0})\) be a t-structure. Show that, for all \(n\in\Z\), the pair \((\calT^{\leq n},\calT^{\geq n})\) is a t-structure.
\end{exercise}

We may employ the results developed in Lecture \ref{lecture:localization-sequences-of-triangulated-categories} immediately. In particular, we can make use of
Proposition \ref{prop:adjoint-from-orthogonal-triangle} (and its dual); inspection of the assumptions reveal that one demands almost exactly half the conditions of a t-structure.
We deduce that the inclusions
\[ \iota^{\leq n}\!:\calT^{\leq n}\inj\calT,\quad \iota^{\geq n}\!:\calT^{\geq n}\inj\calT \]
admit a right (resp.\ left) adjoint.
\begin{definition}
	Let \((\calT^{\leq 0},\calT^{\geq 0})\) be a t-structure on \(\calT\). The right adjoint of \(\iota^{\leq n}\), respectively the left adjoint of \(\iota^{\geq n}\),
	are denoted
	\[ \tau^{\leq n}\!:\calT\to\calT^{\leq n},\quad \text{resp.\ } \tau^{\geq n}\!:\calT \to \calT^{\geq n} \]
	and are called the \emph{truncation functors.}
\end{definition}
By construction, the counit \(\varepsilon\) of the adjunction \(\iota^{\leq n}\ladj\tau^{\leq n}\) and the unit \(\eta\) of the adjunction \(\tau^{\geq n+1}\ladj\iota^{\geq n+1}\)
lie in distinguished triangles
\[ \tau^{\leq n}x \overset{\varepsilon_x}\longto x \overset{\eta_x}\longto \tau^{\geq n+1}x \longto \Sigma \tau^{\leq n}x. \]
Furthermore, any other distinguished triangle
\[ x' \to x \to x'' \to \Sigma x', \]
such that \(x'\in\calT^{\leq n}\) and \(x''\in\calT^{\geq n+1}\), is canonically isomorphic to the first one. The details of this are in Proposition \ref{prop:adjoint-from-orthogonal-triangle}.

\begin{proposition}
	For all \(n\in\Z\) and \(x\in\calT\), there is a \emph{unique} morphism \(d^n_x\!:\tau^{\geq n+1}x\to \Sigma\tau^{\leq n}x\) such that
	\[ \tau^{\leq n}x \overset{\varepsilon_x}\longto x \overset{\eta_x}\longto \tau^{\geq n+1}x \overset{d^n_x}\longto \Sigma \tau^{\leq n}x \]
	is a distinguished triangle. Moreover, the morphisms \(d^n_x\) assemble into a natural transformation \(d^n\!:\tau^{\geq n+1}\To \Sigma\tau^{\leq n}\).
\end{proposition}
\begin{proof}
We use Lemma \ref{lemma:simple-uniqueness-of-cone-shift-map}. All we have to do is check that
\[ \calT(\Sigma \tau^{\leq n}x, \tau^{\geq n+1}x) = 0 \]
but \(\Sigma \tau^{\leq n}x\in \Sigma \calT^{\leq n} = \calT^{\leq n-1} \subseteq \calT^{\leq n}\), and \((\calT^{\leq n})^\perp\supseteq\calT^{\geq n+1}\), so this is clear.

To see that the (now unique) morphisms \(d^n_x\) assemble into a natural transformation \(d^n\!:\tau^{\geq n+1}\To \Sigma\circ \tau^{\leq n}\), suppose we have a
map \(f\!:x\to y\). Applying \(\tau^{\leq n}\) to \(f\) and using adjointness, we get a morphism of distinguished triangles
\begin{diagram*}
	\tau^{\leq n}x\ar[d,"\tau^{\leq n}f"']\ar[r] & x\ar[r]\ar[d,"f"] & \tau^{\geq n+1}x\ar[d,dashed]\ar[r,"d^n_x"] & \Sigma\tau^{\leq n}x\ar[d,"\Sigma\tau^{\leq n}f"] \\
	\tau^{\leq n}y\ar[r] & y\ar[r] & \tau^{\geq n+1}y\ar[r,"d^n_y"] & \Sigma\tau^{\leq n}y
\end{diagram*}
which by uniqueness (see Lemma \ref{lemma:simple-tr3-uniqueness}) implies that the dashed arrow is \(\tau^{\geq n+1}f\).
\end{proof}

\begin{exercise}
	Prove that for all \(n\in\Z\),
	\[ \tau^{\leq n} = \Sigma^{-n}\circ\tau^{\leq 0}\circ\Sigma^{n},\quad \tau^{\geq n} = \Sigma^{-n}\circ\tau^{\leq 0}\circ\Sigma^{n}. \]
\end{exercise}

\subsection{Properties of the aisle and co-aisle}
The truncation functors can be used to detect if something is in the aisle or the co-aisle.
\begin{proposition}
	Let \((\calT^{\leq0},\calT^{\geq0})\) be a t-structure. Then \(x\in\calD^{\leq n}\) if and only if \(\tau^{\geq n+1}x \cong 0\).
\end{proposition}
\begin{proof}
Since \(\tau^{\leq n}\) is right adjoint to a fully faithful functor, we have that \(x\in\calD^{\leq n}\) if and only if then the counit component \(\tau^{\leq n}x\to x\) is an isomorphism.
This, in turn, is equivalent to the cone (which is given by \(\tau^{\geq n+1}x\)) being zero.
\end{proof}

\begin{exercise}\label{exercise:t-structure-zero-is-in-aisle-and-coaisle}
	Let \((\calT^{\leq 0}, \calT^{\geq 0})\) be a t-structure. Show that \(0\in\calT^{\leq 0}\cap\calT^{\geq 0}\).
\end{exercise}

\begin{corollary}\label{corollary:t-structure-orthogonality}
	Let \((\calT^{\leq 0},\calT^{\geq 0})\) be a t-structure. Then the following statements hold.
	\begin{enumerate}[label=(\arabic*)]
		\item \(y\in\calT^{\geq 1}\) if and only if \(\calT(x,y) = 0\) for all \(x\in\calT^{\leq 0}\).
		\item \(x\in\calT^{\leq 0}\) if and only if \(\calT(x,y) = 0\) for all \(y\in\calT^{\geq 1}\).
	\end{enumerate}
	In particular, \(\calT^{\geq 1} = (\calT^{\geq 0})^\perp\) and \(\calT^{\leq 0} = \bperp{(\calT^{\geq 1})}\). Furthermore, \(\calT^{\leq 0}\) and \(\calT^{\geq 0}\) are
	closed under direct summands.
\end{corollary}
\begin{proof}
We prove (1), since (2) follows dually. Since \(\calT^{\geq 1}\subseteq (\calT^{\geq0})^\perp\), one direction is clear. For the other, let \(y\in\calT\) and
assume that \(\calT(x,y) = 0\) for all \(x\in \calT^{\geq 0}\). We now note that
\[ 0 \cong \calT(x,0) \cong \calT(x,y) \cong \calT(x,\tau^{\leq 0}y) \]
so that \(\tau^{\leq 0}y \cong 0\), hence \(y\in\calT^{\geq 1}\).

To see that \(\calT^{\leq 0}\) is closed under direct summands, let \(x \cong x'\oplus x'' \in \calT^{\leq 0}\). Then for all \(y\in\calT^{\geq 1}\) we have
\[ 0\cong \calT(x,y) \cong \calT(x',y)\oplus \calT(x'',y) \implies \calT(x',y) \cong \calT(x'',y)\cong 0 \]
so that \(x',x''\in\calT^{\leq 0}\) by (2). The other case is dual.
\end{proof}

We thus see that the aisle and co-aisle in a t-structure are almost thick subcategories, missing only the property of being closed under a particular shift.
We can say more: these subcategories are also closed under extensions. To prove this, we need a lemma.
\begin{lemma}\label{lemma:t-structure-aisle-from-trivial-mapping-to-truncation}
	Let \((\calT^{\leq 0},\calT^{\geq0})\) be a t-structure, and let \(x\in\calT\). If \(\calT(x,\tau^{\geq 1}x)=0\), then \(x\in\calT^{\leq 0}\).
\end{lemma}
\begin{proof}
By assumption, we have a distinguished triangle
\[ \tau^{\leq 0}x\to x \overset{0}\to \tau^{\geq 1}x \to \Sigma\tau^{\leq 0}x. \]
We recognize that any distinguished triangle involving a zero map induces a direct sum decomposition by Corollary \ref{corollary:direct-sum-triangle}, and in particular, we see that
\[ \tau^{\leq 0}x\cong \Sigma^{-1}\tau^{\geq 1}x \oplus x. \]
Since \(\calT^{\leq 0}\) is closed under direct summands, we see that \(x\in\calT^{\leq 0}\).
\end{proof}
\begin{proposition}\label{prop:t-structure-closure-under-extension}
	Let \((\calT^{\leq 0},\calT^{\geq0})\) be a t-structure. Then \(\calT^{\leq 0}\) and \(\calT^{\geq 0}\) are closed under extensions.
\end{proposition}
\begin{proof}
We prove the first statement, as the other follows by duality. Consider a distinguished triangle
\[ x' \to x \to x'' \to \Sigma x' \]
where \(x',x''\in\calT^{\leq 0}\). We want to show that \(x\in\calT^{\leq 0}\). We apply the cohomological functor \(\calT(-,\tau^{\geq 1}x)\) to see that
\[ 0 = \calT(x'',\tau^{\geq 1}x) \to \calT(x,\tau^{\geq 1}x) \to \calT(x',\tau^{\geq 1}x) = 0 \]
is exact, so \(\calT(x,\tau^{\geq 1}x) = 0\). By the preceding lemma, we conclude that \(x\in\calT^{\leq 0}\).
\end{proof}
\begin{exercise}\label{exercise:t-structures-cones-of-morphisms-in-aisle}
	Let \((\calT^{\leq 0},\calT^{\geq0})\) be a t-structure, and let \(f\!:x\to y\) be a morphism in \(\calT\). Fix some \(n\in\Z\). Prove the following statements.
	\begin{enumerate}[label=(\arabic*)]
		\item If \(x,y\in\calT^{\leq n}\), then any cone of \(f\) is in \(\calT^{\leq n}\) and any cocone of \(f\) is in \(\calT^{\leq n+1}\).
		\item If \(x,y\in\calT^{\geq n}\), then any cone of \(f\) is in \(\calT^{\geq n-1}\) and any cocone of \(f\) is in \(\calT^{\geq n}\).
	\end{enumerate}
\end{exercise}

\subsection{The Abelian heart of a t-structure}
One of the original motivations for t-structures is the following construction.

\begin{definition}
	Let \((\calT^{\leq 0},\calT^{\geq 0})\) be a t-structure. The \emph{heart} of \(\calT\) with respect to \((\calT^{\leq 0},\calT^{\geq 0})\) is
	\[ \calT^\heartsuit := \calT^{\leq 0}\cap\calT^{\geq 0}. \]
\end{definition}

The heart consists of objects ``concentrated in degree zero'' and so, from the example of the derived category, we expect this to be an Abelian category. Proving that is our next aim.
An easy step towards this is showing that \(\calT^\heartsuit\) is additive.

\begin{proposition}\label{prop:t-structure-heart-is-additive}
	Let \((\calT^{\leq0},\calT^{\geq0})\) be a t-structure. Then \(\calT^\heartsuit\) is additive.
\end{proposition}
\begin{proof}
By Exercise \ref{exercise:t-structure-zero-is-in-aisle-and-coaisle}, \(0\in\calT^\heartsuit\). Furthermore, it is clearly pre-additive, since it is a full subcategory of a pre-additive category.
Therefore, we need only show it admits finite direct sums. However, if \(x,x'\in\calT^{\leq0}\) then for all \(y\in\calT^{\geq 1}\), we have
\[ 0\cong\calT(x\oplus x',y) \cong \calT(x,y)\oplus\calT(x',y) \]
so \(x\oplus x'\in\calT^{\leq 0}\) by Corollary \ref{corollary:t-structure-orthogonality}. The same statement holds for \(\calT^{\geq 0}\) by a similar argument. Therefore, \(\calT^\heartsuit\) is
also closed under finite direct sums.
\end{proof}

To prove that \(\calT^\heartsuit\) is Abelian, we will make use of the uniqueness properties coming from the orthogonality assumptions in a t-structure, along
with the intuition that cones in triangulated categories are like ``homotopy cokernels''. Actually, the cone is a more subtle construction: the cocone, which should be a ``homotopy kernel'',
is just a shift of the cone. In other words, the cone contains both information about the kernel and cokernel of a morphism. One way in which this appears in other results
is that it suffices to know a morphism has trivial cone to know that it is an isomorphism, in contrast to an Abelian category where one needs to check both the kernel and cokernel.

\begin{theorem}\label{thm:t-structure-heart-is-abelian}
	Let \((\calT^{\leq 0},\calT^{\geq0})\) be a t-structure. Then \(\calT^\heartsuit\) is an Abelian category.
\end{theorem}
\begin{proof}
Since we already know \(\calT^\heartsuit\) is additive, what remains is to check that it admits kernels and cokernels, and that the image coincides with the coimage.
Let \(f\!:x\to y\) be a morphism in \(\calT^\heartsuit\), and take the cone to get a distinguished triangle
\[ x\overset{f}\to y \to z \to \Sigma x. \]
Inspecting the triangle and using closure under extension, one sees that \(z\in\calT^{\leq 0}\cap\calT^{\geq -1}\).
\begin{enumerate}[label=(\arabic*)]
	\item The kernel of \(f\) is given by \(\tau^{\leq0}\Sigma^{-1} z \to x\). To see this, let \(w\in\calT^\heartsuit\). Applying \(\calT(w,-)\), we see that we have an exact sequence
	\[  0 = \calT(w,\Sigma^{-1}y) \to \calT(w,\Sigma^{-1}z) \to \calT(w,x)\overset{f_*}\to\calT(w,y). \]
	Therefore, we see that \(\calT(w,\Sigma^{-1}z)\) is the kernel of \(f_*\!:\calT(w,x)\to\calT(w,y)\). Using the canonical isomorphism \(\calT(w,\Sigma^{-1}z) \cong \calT(w,\tau^{\leq 0}\Sigma^{-1}z)\),
	we are done.
	\item The cokernel of \(f\) is given by \(y\to \tau^{\geq 0}z\). This is identical to (1).
	\item The image and coimage agree. By (1) and (2), we obtain the image of \(f\) by truncating the cocone of the composition \(y\to z \to \tau^{\geq 0}z\). Let us consider the distinguished triangle
	\[ e \to y \to \tau^{\geq 0}z \to \Sigma e, \]
	so that \(\img{f} = \tau^{\geq 0}e\). We observe that \(e\in\calT^{\leq1}\cap\calT^{\geq0}\). Applying (TR4) to the composition \(y \to z \to \tau^{\geq 0}z\), we obtain a distinguished triangle
	\[ \Sigma x \to \Sigma e \to \Sigma\tau^{\leq-1}z \to \Sigma^2 x \quad\leadsto\quad x \to e \to \tau^{\leq-1}z \to \Sigma x.\]
	By closure under extension, we see that \(e\in\calT^{\leq 0}\), so in fact \(e\in\calT^\heartsuit\). Shifting to the left, we have the distinguished triangle
	\[ \Sigma^{-1}\tau^{\leq-1}z \to x \to e \to \tau^{\leq-1}z \]
	where we note that \(\Sigma^{-1}\tau^{\leq-1}z = \tau^{\leq 0}\Sigma^{-1} z = \ker{f}\), so really we have
	\[ \ker{f} \to x \to e \to \Sigma(\ker{f}). \]
	Since \(e \cong \tau^{\geq 0}e\), this exhibits \(e\) also as the cokernel of the kernel map, i.e.\ \(e \cong \coimg{f} \). On the other hand,
	\(e\cong\img{f}\), so we are done.
\end{enumerate}
We conclude that \(\calT^\heartsuit\) is Abelian.
\end{proof}

Remarkably, this means that whenever we have a t-structure on \(\calT\), we may find an Abelian subcategory where the kernel and cokernel are, in a sense, given by the
triangulated structure on \(\calT\).

\subsection{Cohomology functors}
By truncating to the left and right at position \(n\), we are left with something in \(\calT^{\leq n}\cap\calT^{\geq n} \simeq \Sigma^{-n}\calT^\heartsuit\). This
indudces the all-important \emph{cohomology functors} intrinsic to any t-structure.
\begin{definition}
	Let \((\calT^{\leq0},\calT^{\geq0})\) be a t-structure. The \emph{zeroth cohomology functor} with respect to the t-structure is
	\[ \HH^0\!:\calT\to\calT^\heartsuit,\quad \HH^0 := \tau^{\geq 0}\circ\tau^{\leq 0}. \]
	More generally, we define the \(n\)th cohomology functor, for any \(n\in\Z\), by
	\[ \HH^n\!:\calT\to\calT^\heartsuit,\quad \HH^n := \HH^0\circ\Sigma^n. \]
\end{definition}
\begin{exercise}
	Show that \(\HH^n = \Sigma^n\circ \tau^{\geq n}\circ\tau^{\leq n}\).
\end{exercise}

Now, we made a choice in the above definition of what order to do the truncations. However, clearly there should be no difference between truncating on the left then the right, or vice versa.
We verify this now, in (3) below.
\begin{proposition}\label{prop:t-structure-truncation-relations}
	Let \((\calT^{\leq 0},\calT^{\geq 0})\) be a t-structure. Then the following statements hold.
	\begin{enumerate}[label=(\arabic*)]
		\item If \(m\leq n\), then
		\[ \tau^{\leq n}\circ\tau^{\leq m} \cong \tau^{\leq m}\circ\tau^{\leq n} \cong \tau^{\leq m},\quad \text{and}\quad \tau^{\geq n}\circ\tau^{\geq m} \cong \tau^{\geq m}\circ\tau^{\geq n} \cong \tau^{\geq n}. \]
		\item If \(m > n\), then
		\[ \tau^{\geq m}\circ\tau^{\leq n} \cong 0 \cong \tau^{\leq n}\circ\tau^{\geq m}. \]
		\item For all \(m,n\in\Z\), there is a unique natural isomorphism \(\beta\!:\tau^{\geq m}\circ\tau^{\leq n}\cong\tau^{\leq n}\circ\tau^{\geq m}\) such that the diagram
		of natural transformations
		\begin{diagram*}
			\tau^{\leq n}\ar[r,Rightarrow,"\varepsilon"]\ar[d,Rightarrow,"\eta\tau^{\leq n}"'] & \1 \ar[r,Rightarrow,"\eta"] & \tau^{\geq m} \\
			\tau^{\geq m}\tau^{\leq n} \ar[rr,Rightarrow,"\sim","\beta"'] & & \tau^{\leq n}\tau^{\geq m}\ar[u,Rightarrow,"\varepsilon\tau^{\geq m}"']
		\end{diagram*}
		commutes, where \(\varepsilon\) is the counit of \(\iota^{\leq n}\ladj\tau^{\leq n}\) and \(\eta\) is the unit of \(\tau^{\geq m}\ladj\iota^{\geq m}\).
	\end{enumerate}
\end{proposition}
\begin{proof}
(1) We prove the leftmost statement, as the other one is dual. Observe that we have a natural transformation \(\varepsilon\tau^{\leq m}\!: \tau^{\leq n}\circ\tau^{\leq m} \To \tau^{\leq m}\).
Since \(\calT^{\geq n}\) is a reflective subcategory and \(\tau^{\leq m}x\in\calT^{\leq m}\subseteq\calT^{\leq n}\) for all \(x\in\calT\), we see that this natural transformation
must be a natural isomorphism by Lemma \ref{lemma:reflective-subcategory-essential-image}, showing that \(\tau^{\leq n}\circ\tau^{\leq m} \cong \tau^{\leq m}\). For the
other isomorphism, note that we have natural isomorphisms
\[ \calT^{\leq m}(-, (\tau^{\leq m}\circ\tau^{\leq n})(-)) \cong \calT^{\leq n}(\iota^{\leq m}(-),\tau^{\leq n}(-)) \cong \calT(\iota^{\leq m}(-), -) \]
so by uniqueness of adjoints, we get a natural isomorphism \( \tau^{\leq m}\circ\tau^{\leq n}\cong\tau^{\leq m} \).

(2) The natural transformation \( \tau^{\leq m-1}\tau^{\leq n} \To \tau^{\leq n} \) is a natural isomorphism since \(n < m\), by (1). In particular, the cone of each component is zero.
However, the cone here is canonically given by \(\tau^{\geq m}\circ \tau^{\leq n}\), hence we get one of the claimed isomorphisms. The other is dual.

(3) By (2), the claim already holds when \(m > n\). Hence, we may assume that \(m \leq n\). We have the distinguished triangles
\begin{align*}
	\tau^{\leq m-1}\tau^{\leq n}x \longto \tau^{\leq n}x \overset{(\eta\tau^{\leq n})_x}\longto \tau^{\geq m}\tau^{\leq n}x \longto \Sigma\tau^{\leq m-1}\tau^{\leq n}x, \\
	\tau^{\leq n}\tau^{\geq m}x \overset{(\varepsilon\tau^{\geq m})_x}\longto \tau^{\geq m}x \longto \tau^{\geq n+1}\tau^{\geq m}x \longto \Sigma\tau^{\leq n}\tau^{\geq m}x.
\end{align*}
Applying Exercise \ref{exercise:t-structures-cones-of-morphisms-in-aisle}, we see that \(\tau^{\leq n}\tau^{\geq m}x, \tau^{\geq m }\tau^{\leq n}x \in \calT^{\leq n}\cap\calT^{\geq m}\).
Now we use the universal properties of the truncations (being adjoints):
\begin{center}
	\begin{tikzcd}[row sep=small]
		\tau^{\leq n}x \ar[r,"(\eta\tau^{\leq n})_x"]\ar[d,"\varepsilon_x"'] & \tau^{\geq m}\tau^{\leq n}x\ar[ddl,dashed,"\exists!"]  \\
		x\ar[d,"\eta_x"'] & \\
		\tau^{\geq m}x & \tau^{\leq n}\tau^{\geq m}x \ar[l, "(\varepsilon\tau^{\geq m})_x"]
	\end{tikzcd}
	\(\quad\leadsto\quad\)
	\begin{tikzcd}[row sep=small]
		\tau^{\leq n}x \ar[r,"(\eta\tau^{\leq n})_x"]\ar[d,"\varepsilon_x"'] & \tau^{\geq m}\tau^{\leq n}x\ar[dd,dashed,"\exists!"]  \\
		x\ar[d,"\eta_x"'] & \\
		\tau^{\geq m}x & \tau^{\leq n}\tau^{\geq m}x \ar[l, "(\varepsilon\tau^{\geq m})_x"]
	\end{tikzcd}
\end{center}
where we use that \(\tau^{\geq m}\tau^{\leq n}x\in\calT^{\leq n}\). We see that there is a unique natural transformation \(\beta\!:\tau^{\geq m}\circ\tau^{\leq n}\To\tau^{\leq n}\tau^{\geq m}\)
making the provided diagram of natural transformations commute. We must now show it is an isomorphism.

By part (1), we may rewrite the triangles as
\begin{align*}
	\tau^{\leq m-1}x \longto \tau^{\leq n}x \overset{(\eta\tau^{\leq n})_x}\longto \tau^{\geq m}\tau^{\leq n}x \longto \Sigma\tau^{\leq m-1}x, \\
	\tau^{\leq n}\tau^{\geq m}x \overset{(\varepsilon\tau^{\geq m})_x}\longto \tau^{\geq m}x \longto \tau^{\geq n+1}x \longto \Sigma\tau^{\leq n}\tau^{\geq m}x.
\end{align*}
We apply (TR4) to the composition \(\tau^{\leq m-1}x \to \tau^{\leq n}x \to x\). This yields a distinguished triangle
\[ \tau^{\geq m}\tau^{\leq n}x \to \tau^{\geq m}x \to \tau^{\geq n+1}x \to \Sigma\tau^{\geq m}\tau^{\leq n}x. \]
However, we now have a distinguished triangle around \(\tau^{\geq m}x\) whose left term is in \(\calT^{\leq n}\) and right term is in \(\calT^{\geq n+1}\), which implies (by construction
of the truncation adjunction) that the canonical map \(\beta_x\!:\tau^{\geq m}\tau^{\leq n}x \to \tau^{\leq n}\tau^{\geq m}x \) is an isomorphism.
\end{proof}

\begin{corollary}\label{corollary:t-structure-kernel-and-cokernel-in-heart-cohomology}
	Let \((\calT^{\leq 0},\calT^{\geq 0})\) be a t-structure, and consider a morphism \(f\!:x\to y\) in \(\calT^\heartsuit\). Given a distinguished triangle
	\[ x \to y \to z \to \Sigma x, \]
	we have canonical isomorphisms
	\[ \ker{f} \cong \HH^{-1}(z),\quad \text{and}\quad \coker{f} \cong \HH^0(z).  \]
\end{corollary}
\begin{proof}
Recall from the proof of Theorem \ref{thm:t-structure-heart-is-abelian} that we have canonical isomorphisms
\[ \ker{f} \cong \tau^{\leq -1}z,\quad \text{and}\quad \coker{f} \cong \tau^{\geq 0}z  \]
and that \(z\in\calT^{\geq -1}\cap\calT^{\leq 0}\). In particular, we have canonical isomorphisms
\[ \tau^{\geq -1}z \cong z \cong \tau^{\leq 0}z. \]
Thus,
\[ \ker{f} \cong \tau^{\leq -1}\tau^{\geq -1}z \cong \HH^{-1}(z),\quad \text{and}\quad \coker{f} \cong \tau^{\geq 0}\tau^{\leq 0}z \cong \HH^0(z) \]
where, for the left computation, we use Proposition \ref{prop:t-structure-truncation-relations}.
\end{proof}

\begin{corollary}\label{corollary:t-structure-exact-sequence-in-heart-induces-unique-distinguished-triangle}
	Let \((\calT^{\leq 0},\calT^{\geq 0})\) be a t-structure, and consider a short exact sequence
	\[ 0 \to x \inj y \sur z \to 0 \]
	in \(\calT^\heartsuit\). Then there is a unique morphism \(z \to \Sigma x\) for which
	\[ x \to y \to z \to \Sigma x \]
	is a distinguished triangle in \(\calT\).
\end{corollary}
\begin{proof}
If such a morphism exists, it is unique by Lemma \ref{lemma:simple-uniqueness-of-cone-shift-map}, since there are no non-zero maps from \(\calT^{\leq -1}\) to \(\calT^{\geq 0}\). To see that one exists,
pick a distinguished triangle
\[ x \to y \to z' \to \Sigma x. \]
We aim to show that \(z' \cong z\). We have that \(z'\in\calT^{\leq 0}\cap\calT^{\geq -1}\), and therefore the distinguished triangle
\[ \tau^{\leq -1}z' \to z' \to \tau^{\geq 0}z' \to \Sigma\tau^{\leq -1}z' \]
turns into
\[ \HH^{-1}(z') \to z' \to \HH^0(z') \to \Sigma \HH^{-1}(z'). \]
However, by Corollary \ref{corollary:t-structure-kernel-and-cokernel-in-heart-cohomology}, we have
\[ \HH^{-1}(z') \cong \ker(x\inj y) \cong 0,\quad \HH^0(z')\cong \img(y\sur z) \cong z. \]
Thus, \(z' \cong \HH^0(z') \cong z\).
\end{proof}

\subsection{Appendix: Cohomology is cohomological}
Of central importance about the cohomology functors induced by a t-structure is the fact that they are actually genuine cohomological functors, so
that a t-structure gives rise to some new \emph{non-trivial} and \emph{interesting} constructions. We now prove this.
\begin{exercise}\label{exercise:abelian-category-kernel-invariant-under-composition-with-monic}
	Let \(\calA\) be an Abelian category, and let \(f\!:x\to y\) be a morphism in \(\calA\). Suppose that we have a factorization
	\[ f = (x \overset{g}\to z \overset{h}\inj y). \]
	Show that there is a canonical isomorphism \(\ker{f} \cong \ker{g}\).
\end{exercise}
\begin{exercise}\label{exercise:t-structure-cohomology-of-object-in-aisle}
	Let \((\calT^{\leq 0},\calT^{\geq 0})\) be a t-structure, and let \(x\in\calT^{\leq n}\). Show that \(\HH^i(x) \cong 0\) for all \(i > n\). Prove the dual statement also.
\end{exercise}
\begin{theorem}\label{thm:t-structure-cohomology-is-cohomological}
	Let \((\calT^{\leq 0},\calT^{\geq 0})\) be a t-structure. Then the cohomology functors \(\HH^n\!:\calT\to\calT^{\heartsuit}\) are cohomological.
\end{theorem}
\begin{proof}
Note that \(\calT^\heartsuit\) is Abelian by Theorem \ref{thm:t-structure-heart-is-abelian}, so the proposition is a valid one to consider. The functors are clearly additive, so we must check that
they send distinguished triangles to exact sequences. Consider a distinguished triangle
\[ x \to y \to z \to \Sigma x \]
in \(\calT\). It suffices to show that \(\HH^0\) sends this to an exact sequence
\[ \HH^0(x) \to \HH^0(y) \to \HH^0(z). \]
The proof proceeds in three steps, increasing in generality.
\begin{enumerate}[label=(\arabic*)]
	\item The case when \(x,y,z\in\calT^{\leq 0}\). We show that, in fact,
	\[ \HH^0(x) \to \HH^0(y) \to \HH^0(z) \to 0 \]
	is exact. To do this, note that for all \(w\in\calT^\heartsuit\) and \(u\in\calT^{\leq 0}\), we have natural isomorphisms
	\[ \calT^\heartsuit(\HH^0(u),w) = \calT(\tau^{\geq 0}\tau^{\leq 0}u, w) \cong \calT(\tau^{\geq 0}u, w) \cong \calT(u,w) \]
	since \(\tau^{\leq 0}u \iso u\). Furthermore, \(\calT(\Sigma u,w) \cong 0\) since \(\Sigma u \in \calT^{\leq -1}\). Therefore, applying \(\calT(-,w)\) to our original distinguished triangle yields
	exact sequences
	\begin{diagram*}[cramped]
		\calT(\Sigma x,w) \ar[r]\ar[d,"\sim" labl] & \calT(z,w)\ar[r]\ar[d,"\sim" labl] & \calT(y,w)\ar[d,"\sim" labl]\ar[r] & \calT(x,w)\ar[d,"\sim" labl] \\
		0 \ar[r] & \calT(\HH^0(z),w)\ar[r] & \calT(\HH^0(y),w)\ar[r] & \calT(\HH^0(x),w)
	\end{diagram*}
	for all \(w\in\calT^\heartsuit\), from which we deduce that the claimed sequence is exact.
	\item The case when \(x\in\calT^{\leq 0}\). We again show that
	\[ \HH^0(x) \to \HH^0(y) \to \HH^0(z) \to 0 \]
	is exact, by reduction to (1). Let \(w\in\calT^{\geq 1}\). We see that \(\calT(x,w) \cong \calT(\Sigma x,w) \cong 0\). In particular, applying \(\calT(-,w)\) to our distinguished triangle yields an
	exact sequence
	\[ 0 \to \calT(z,w) \to \calT(y,w) \to 0 \]
	so that we have a natural isomorphism \(\calT(z,w)\iso\calT(y,w)\) when \(w\in\calT^{\geq 1}\). Therefore, we have natural isomorphisms
	\[ \calT(\tau^{\geq 1}z,w) \cong \calT(z,w) \iso \calT(y,w) \cong \calT(\tau^{\geq 1}y,w) \]
	from which we conclude that \(\tau^{\geq 1}y \iso \tau^{\geq 1}z\). To finally reduce to (1), we apply (TR4) to the composition \(y \to z \to \tau^{\geq 1}z\), yielding a distinguished triangle
	\[ \Sigma x \to \Sigma\tau^{\leq 0}y \to \Sigma\tau^{\leq 0}z \to \Sigma^2x \quad \leadsto \quad x \to \tau^{\leq 0}y \to \tau^{\leq 0}z \to \Sigma x \]
	where we are now in the situation of (1). We have a natural isomorphism \(\HH^0\cong\HH^0\circ\tau^{\leq 0}\), so in the end, applying \(\HH^0\) to the above yields the desired exact sequence.
	\item The case when \(x,y,z\) are all arbitrary. Consider the composition \(\tau^{\leq 0}x \to x \to y\), and take the cone so we have a distinguished triangle
	\[ \tau^{\leq 0}x \to y \to e \to \Sigma\tau^{\leq 0}x. \]
	Applying (TR4) to the composition, we have the commutative diagram
	\begin{diagram*}[cramped]
		\tau^{\leq 0}x\ar[r]\ar[d,equal] & x\ar[d]\ar[r] & \tau^{\geq 1}x \ar[r]\ar[d,dashed] & \Sigma \tau^{\leq 0}x \ar[d,equal] \\
		\tau^{\leq 0}x\ar[r]\ar[d] & y\ar[r]\ar[d,equal] & e\ar[r]\ar[d,dashed] & \Sigma\tau^{\leq 0}x \ar[d] \\
		x\ar[r]\ar[d] & y\ar[d]\ar[r] & z \ar[r]\ar[d,equal] & \Sigma x\ar[d] \\
		\tau^{\geq 1}x \ar[r,dashed] & e\ar[r,dashed] & z\ar[r,dashed] & \Sigma \tau^{\geq 1}x
	\end{diagram*}
	with rows distinguished triangles. Applying \(\HH^0\) to the second row, by (2), gives the exact sequence
	\[ \HH^0(x) \to \HH^0(y) \to \HH^0(e). \]
	By the dual of (2), shifting the bottom row and applying \(\HH^0\) yields the exact sequence
	\[ 0 \to \HH^0(e) \to \HH^0(z) \to \HH^1(x) \]
	and in particular, \(\HH^0(e) \to \HH^0(z)\) is a monomorphism. Noting that \(\HH^0(y)\to\HH^0(z)\) factors as \(\HH^0(y)\to\HH^0(e)\inj\HH^0(z)\) by the commutativity
	of the big diagram, we deduce by Exercise \ref{exercise:abelian-category-kernel-invariant-under-composition-with-monic} that
	\[ \img(\HH^0(x)\to\HH^0(y)) \cong \ker(\HH^0(y)\to\HH^0(e)) \cong \ker(\HH^0(y)\to\HH^0(z)) \]
	showing that
	\[ \HH^0(x) \to \HH^0(y) \to \HH^0(z) \]
	is exact.
\end{enumerate}
This completes the proof.
\end{proof}

\begin{corollary}
	Let \((\calT^{\leq 0},\calT^{\geq 0})\) be a t-structure, and suppose we have a distinguished triangle
	\[ x \to y \to z \to \Sigma x,\quad x,y,z\in\calT^\heartsuit. \]
	Then the sequence
	\[ 0 \to x \to y \to z \to 0 \]
	is exact. In particular, there is a bijection between exact sequences in \(\calT^\heartsuit\) and distinguished triangles whose terms lie in \(\calT^\heartsuit\).
\end{corollary}
\begin{proof}
We have a natural isomorphism \(\HH^0|_{\calT^\heartsuit} \cong \1_{\calT^\heartsuit}\), since on objects of the heart, the components of the (co)unit for the truncation at zero are isomorphisms.
In particular, Theorem \ref{thm:t-structure-cohomology-is-cohomological} specializes to say that given a distinguished triangle
\[ x \to y \to z \to \Sigma x,\quad x,y,z\in\calT^\heartsuit, \]
applying \(\HH^0\) yields exact sequences
\begin{diagram*}[cramped]
	0 \ar[r]\ar[d,equal] & \HH^0(x) \ar[r]\ar[d,"\sim" labl] & \HH^0(y) \ar[d,"\sim" labl]\ar[r] & \HH^0(y)\ar[d,"\sim" labl]\ar[r] & 0\ar[d,equal] \\
	0 \ar[r] & x \ar[r] & y \ar[r] & z \ar[r] & 0
\end{diagram*}
as desired. The final assertion follows by combining the above with Corollary \ref{corollary:t-structure-exact-sequence-in-heart-induces-unique-distinguished-triangle}.
\end{proof}

\subsection{Appendix: Bounded and non-degenerate t-structures}
Consider a complex \(x^\bullet\) in some Abelian category category \(\calA\):
\[ x^\bullet\!:\quad \cdots \to x^{i-1} \to x^i \to x^{i+1} \to \cdots. \]
The t-structure on \(\sfD(\calA)\) is such that \(x^\bullet\in\sfD(\calA)^{\leq 0}\) if and only if \(x^\bullet \in \sfD^{\leq 0}(\calA)\), i.e.\ if \(\HH^i(x^\bullet) \cong 0\) for all \(i > 0\).
In other words, one can use the cohomology functors to completely characterize the aisle (and co-aisle) of the t-structure.

Now consider some arbitrary triangulated category \(\calT\) with a t-structure \((\calT^{\leq 0},\calT^{\geq 0})\). The above statement no longer holds, as, in fact, there may be non-zero \(x\in\calT\)
for which \(\HH^i(x)\cong 0\) \emph{for all} \(i\in\Z\). Indeed, consider the trivial t-structure \((\calT,0)\). The left truncation functor does nothing, while the right truncation functor
is the functor sending everything to zero. In this t-structure, all cohomology functors are just the zero functor \(\calT \to 0\), and so in fact every object \(x\in\calT\) forms a counter-example.
We would like to find critera which prevent this kind of pathology.

\begin{proposition}\label{prop:t-structure-aisle-from-cohomology}
	Let \((\calT^{\leq 0},\calT^{\geq 0})\) be a t-structure, and let \(x\in\calT\). Assume one of the following conditions hold.
	\begin{enumerate}[label=(\roman*)]
		\item There is some \(n\in\Z\) such that \(x\in\calT^{\leq n}\).
		\item If \(\tau^{\geq 1}x \in \calT^{\geq n} \) for all \(n\geq 1\), then \(\tau^{\geq 1}x\cong 0\).
	\end{enumerate}
	Then \(x\in\calT^{\leq 0}\) if and only if \(\HH^i(x)\cong 0\) for all \(i > 0\).
\end{proposition}
\begin{proof}
If \(x\in\calT^{\leq 0}\), then \(\HH^i(x)\cong 0\) for all \(i>0\); this is Exercise \ref{exercise:t-structure-cohomology-of-object-in-aisle}. For the converse, we split into the cases presented.
\begin{enumerate}[label=(\roman*)]
	\item We proceed by induction. If \(n\leq 0\), we are done. Thus, suppose \(n > 0\), and that we know the result for all \(0 \leq m < n\). By assumption, \(\HH^n(x) \cong 0\), so the distinguished triangle
	\[ \tau^{\leq n-1}\tau^{\leq n}x \to \tau^{\leq n}x \to \tau^{\geq n}\tau^{\leq n}x \to \Sigma \tau^{\leq n-1}\tau^{\leq n}x \]
	becomes
	\[ \tau^{\leq n-1}x \to x \to 0 \to \Sigma\tau^{\leq n-1}x \]
	showing that \(\tau^{\leq -1}x\iso x\). In particular, \(x\in\calT^{\leq n-1}\), so by the induction hypothesis \(x\in\calT^{\leq 0}\).
	\item We wish to show that \(\tau^{\leq 0}x \to x\) is an isomorphism, to which it suffices to prove \(\tau^{\geq 1}x\cong 0\). To this end, let us show that for all \(n \geq 1\),
	the morphism \(\tau^{\geq n}x \to \tau^{\geq n+1}x\) is an isomorphism. Consider the distinguished triangle
	\[ \tau^{\leq n}\tau^{\geq n}x \to \tau^{\geq n}x \to \tau^{\geq n+1}\tau^{\geq n}x \to \Sigma\tau^{\leq n}\tau^{\geq n}x \]
	which we compute as
	\[ \Sigma^{-n}\HH^n(x) \to \tau^{\geq n}x \to \tau^{\geq n+1}x \to \Sigma^{-n+1}\HH^n(x). \]
	Since \(n>0\), we see that \(\HH^n(x)\cong 0\) and therefore \(\tau^{\geq n}x\iso\tau^{\geq n+1}x\). We see that \(\tau^{\geq 1}x\cong \tau^{\geq n}x\) for all \(n\geq 1\),
	so that \(\tau^{\geq 1}x \in \bigcap_{n\geq 1}\calT^{\geq n}\). By assumption, \(\tau^{\geq 1}x\) is such that if it satisfies this, it is zero.
\end{enumerate}
This completes the proof.
\end{proof}

This proposition hopefully motivates the following definitions being interesting.

\begin{definition}
	Let \((\calT^{\leq 0},\calT^{\geq 0})\) be a t-structure. We say it is \emph{left bounded,} resp.\ right bounded, if
	\[ \calT = \bigcup_{n\in\Z}\calT^{\geq n},\quad \text{resp.\ } \calT = \bigcup_{n\in\Z}\calT^{\leq n}. \]
	If the t-structure is both left and right bounded, we say is is bounded.

	We say the t-structure is \emph{left non-degenerate,} resp.\ \emph{right non-degenerate} if
	\[ \{0\} \simeq \bigcap_{n\in\Z}\calT^{\leq n},\quad \text{resp.\ } \{0\} \simeq \bigcap_{n\in\Z}\calT^{\geq n}. \]
	If the t-structure is both left and right non-degenerate, we say it is non-degenerate.
\end{definition}

\begin{proposition}
	Let \((\calT^{\leq 0},\calT^{\geq 0})\) be a t-structure. If it is right bounded, then it is right non-degenerate.
\end{proposition}
\begin{proof}
Suppose that \(x\in\bigcap_{k\in\Z}\calT^{\geq k}\). Since the t-structure is right bounded, we also have that \(x\in\calT^{\leq n}\) for some \(n\in\Z\).
In particular, \(x\in\calT^{\leq n}\cap\calT^{\geq n+1}\), so
\[ x \cong \tau^{\geq n+1}\tau^{\leq n}x \cong 0 \]
by Proposition \ref{prop:t-structure-truncation-relations}. Therefore, the t-structure is non-degenerate.
\end{proof}

\begin{corollary}\label{corollary:t-structure-right-non-degenerate-aisle-from-cohomology}
	Let \((\calT^{\leq 0},\calT^{\geq 0})\) be a right non-degenerate t-structure. Then
	\[ \calT^{\leq 0} = \{ x\in\calT \mid \forall i>0,\, \HH^i(x) \cong 0 \}. \]
\end{corollary}
\begin{proof}
Clearly, we have
\[ \calT^{\leq 0} \subseteq \{ x\in\calT \mid \forall i>0,\, \HH^i(x) \cong 0 \}. \]
For the converse, we apply Proposition \ref{prop:t-structure-aisle-from-cohomology}. In applying (ii), note that since \(\calT^{\geq m}\subseteq\calT^{\geq n}\) for all \(m\geq n\), we have
\[ \bigcap_{n\in\Z}\calT^{\geq n} = \bigcap_{n\geq 1}\calT^{\geq n}. \]
By assumption, the right term is thus zero, which means (ii) in the proposition is satisfied for all \(x\in\calT\).
\end{proof}

By appropriately dualizing what we have proven, we see that in a non-degenerate t-structure (for example, one which is bounded) the aisle and co-aisle are completely
determined by the cohomology functors \(\HH^n\!:\calT\to\calT^\heartsuit\). In this case, the cohomology functors act nicely as a family.
\begin{proposition}
	Let \((\calT^{\leq 0},\calT^{\geq 0})\) be a non-degenerate t-structure. Then \(\{\HH^n\}_{n\in\Z}\) is a conservative family of functors. That is, a morphsism \(f\!:x\to y\)
	is an isomorphism if and only if the morphisms \(\HH^n(f)\!:\HH^n(x)\to\HH^n(y)\) are isomorphisms for all \(n\in\Z\).
\end{proposition}
\begin{proof}
One direction is easy: if \(f\) is already an isomorphism, the latter statement is clear. Conversely, assume \(\HH^n(f)\) is an isomorphism for all \(n\in\Z\). Taking the cone of \(f\),
we have a distinguished triangle
\[ x \overset{f}\to y \to z \to \Sigma x. \]
From Theorem \ref{thm:t-structure-cohomology-is-cohomological}, the functors \(\HH^n\) are cohomological, so we have the exact sequence
\begin{diagram*}
	\HH^n(x) \ar[r,"\sim","\HH^n(f)"'] & \HH^n(y) \ar[r] & \HH^n(z) \ar[r] & \HH^{n+1}(x) \ar[r,"\sim","\HH^{n+1}(f)"'] & \HH^{n+1}(y)
\end{diagram*}
which implies that \(\HH^n(z)\cong 0\) for all \(n\in\Z\). In particular, since the t-structure is right non-degenerate, Corollary \ref{corollary:t-structure-right-non-degenerate-aisle-from-cohomology} tells us that
\[ z \in \bigcap_{n\in\Z}\calT^{\leq n}. \]
However, since the t-structure is also left non-degenerate, this means that \(z\cong 0\), so \(f\) is an isomorphism.
\end{proof}

In fact, this property completely characterizes non-degenerate t-structures.
\begin{theorem}
	Let \((\calT^{\leq 0},\calT^{\geq 0})\) be a t-structure. Then the following are equivalent.
	\begin{enumerate}[label=(\arabic*)]
		\item The t-structure is non-degenerate.
		\item The family of cohomological functors \(\{\HH^n\!:\calT\to\calT^{\heartsuit}\}_{n\in\Z}\) is conservative.
	\end{enumerate}
\end{theorem}
\begin{proof}
The implication \((1)\Rightarrow(2)\) has already been proven. For the converse, let us first assume that
\[ x \in\bigcap_{n\in\Z}\calT^{\leq n}. \]
Then \(\HH^n(x)\cong 0\) for all \(n\in\Z\) by Exercise \ref{exercise:t-structure-cohomology-of-object-in-aisle}. In particular, \(x \to 0\) is an isomorphism, by conservativity. This shows
that the t-structure is left non-degenerate. Showing that it is right non-degenerate is dual.
\end{proof}

\begin{remark}
	The above characterization of non-degenerate t-structures suggests that perhaps one ought to be able to go the other way: starting from a cohomological functor \(H\!:\calT\to\calA\)
	such that the family \(\{H\circ\Sigma^n\}_{n\in\Z}\) is conservative, construct a t-structure on \(\calT\) such that \(\calT^\heartsuit\simeq\calA\) and the cohomology functors
	are given by \(H\circ\Sigma^n\).

	In general, one cannot hope for this to be possible, but for particular kinds of functors it actually is. A prominent example is the theorem
	of Hoshino--Kato--Miyachi \cite{hkm02}, previously discussed briefly in Appendix \ref{appendix:abelian-categories-with-a-compact-projective-generator}, which proves that
	this works for the functor \(\calT(s,-)\!:\calT\to\Mod_{\End(s)}\) as long as \(s\in\calT\) is a compact generator and satisfies
	\[ \forall i > 0,\quad \calT(s,s[i]) \cong 0. \]
	Proving the theorem takes some work, but involves many interesting ingredients which we will cover later.
\end{remark}

\subsection{Appendix: Stable t-structures}
Let \(\calT\) be a triangulated category. The trivial t-structure \((\calT,0)\) has a special property.
\begin{definition}
	A t-structure \((\calT^{\leq 0},\calT^{\geq 0})\) is called \emph{stable} if \(\Sigma\calT^{\leq 0} = \calT^{\leq 0}\).
\end{definition}
We dedicate this short appendix to a simple characterization of such t-structures, which also demonstrates that they are somewhat pathological. Everything is taken
from \cite{chen2022extensionststructures}, where a little more can be found too.

\begin{proposition}
	Let \((\calT^{\leq 0},\calT^{\geq 0})\) be a t-structure. The following statements are equivalent.
	\begin{enumerate}[label=(\arabic*)]
		\item The t-structure \((\calT^{\leq 0},\calT^{\geq 0})\) is stable.
		\item Both \(\calT^{\leq0}\) and \(\calT^{\geq0}\) are triangulated subcategories of \(\calT\).
		\item The heart of the t-structure \((\calT^{\leq 0},\calT^{\geq 0})\) satisfies \(\calT^\heartsuit\simeq\{0\}\).
	\end{enumerate}
\end{proposition}
\begin{proof}
That (1) and (2) are equivalent is pretty easy. In particular, (2) trivially implies (1), while to see that (1) implies (2), use that by definition of being
stable, \(\calT^{\leq 0} = \calT^{\leq -1}\) is a triangulated subcategory and that \(\calT^{\geq 0} = (\calT^{\leq -1})^\perp\).

To see that (1) \(\Rightarrow\) (3), we compute
\[ \calT^\heartsuit = \calT^{\leq 0}\cap\calT^{\geq0} = \calT^{\leq -1}\calT^{\geq 0} = \calT^{\leq -1} \cap (\calT^{\leq -1})^\perp \simeq \{0\}. \]
For the converse implication (3) \(\Rightarrow\) (1), it suffices to prove that for all \(x\in\calT^{\leq 0}\), we have \(\tau^{\leq -1}x \iso x\).
To this end, consider the distinguished triangle
\[ \tau^{\leq -1}x \to x \to \tau^{\geq 0}x \to \Sigma\tau^{\leq -1}x. \]
Since \(x\in\calT^{\leq 0}\), we have \(\tau^{\geq 0}x \cong \HH^0(x)\cong 0\) since \(\calT^\heartsuit\simeq\{0\}\), and therefore
the map on the left is an isomorphism.
\end{proof}

\subsection{Appendix: t-structures on stable ∞-categories (TBD)}



%!TEX root = ../lectures.tex

\section{Gluing t-structures}
In Lecture \ref{lecture:localization-sequences-of-triangulated-categories}, we framed \emph{recollements} of triangulated categories as exhibiting how a triangulated category
is glued together from two smaller pieces. The sense in which this could be considered true was only somewhat hinted at, as in reality it is largely speaking a heuristic,
but in this lecture we will see a concrete way this rears its head in practice.

The theory of t-structures (covered in Lecture \ref{lecture:abstract-cohomology-through-t-structures}) was introduced originally in \cite{faisceaux-pervers}, which in fact also
introduced the theory of recollements. The concrete application they had in mind was the construction of a particular Abelian category of \emph{perverse sheaves,} obtained
by \emph{gluing} t-structures in a recollement. The theory required to do this is what we cover in this lecture.

\subsection{t-Exactness}
A triangulated functor between triangulated categories is sometimes referred to as an \emph{exact} functor. We've avoided the usage of this term, as it may present confusion,
but the intuition comes from the fact that it preserves ``homotopy exact sequences'' (i.e.\ distinguished triangles). However, there are other exactness phenomena that arise
naturally, such as some functors preserving complexes living in particular degrees. The theory of t-structures provides a convenient setting in which to phrase these
properties more generally.

\begin{definition}
	Let \(F\!:\calT_1 \to \calT_2\) be a triangulated functor of triangulated categories with t-structures \((\calT^{\leq 0}_i,\calT^{\geq 0})_i\), \(i=1,2\).
	\begin{enumerate}[label=(\arabic*)]
		\item We say \(F\) is \emph{left t-exact} if \(F(\calT^{\geq 0}_1) \subseteq \calT^{\geq 0}_2\).
		\item We say \(F\) is \emph{right t-exact} if \(F(\calT^{\leq 0}_1) \subseteq \calT^{\leq 0}_2\).
		\item We say \(F\) is \emph{t-exact} if it is both left t-exact and right t-exact.
	\end{enumerate}
\end{definition}

\begin{lemma}
	Let \(F\!:\calT_1 \to \calT_2\) be a triangulated functor of triangulated categories with t-structures \((\calT^{\leq 0}_i,\calT^{\geq 0}_i)\), \(i=1,2\).
	\begin{enumerate}[label=(\arabic*)]
		\item If \(F\) is left t-exact, then for all \(n\in\Z\) we have \(F(\calT^{\geq n}_1) \subseteq \calT^{\geq n}_2\).
		\item If \(F\) is right t-exact, then for all \(n\in\Z\) we have \(F(\calT^{\leq n}_1) \subseteq \calT^{\leq n}_2\).
	\end{enumerate}
\end{lemma}
\begin{proof}
Statements (1) and (2) are dual, so it suffices to prove (1). We have \(\calT^{\geq n}_i = \Sigma^{-n}\calT^{\geq 0}_i\), and since \(F\) is triangulated, we thus have
\[ F(\calT^{\geq n}_1) = F(\Sigma^{-n}\calT^{\geq 0}_1) = \Sigma^{-n}F(\calT^{\geq 0}_1) \subseteq \Sigma^{-n}\calT^{\geq 0}_2 = \calT^{\geq n}_2 \]
as desired.
\end{proof}

\begin{remark}
	The property of being, say, right t-exact can be pictured (and thus related intuitively to ordinary exactness of functors) in terms of the following:
	\[ F(\cdots \to \bullet \to \bullet \to 0\to 0 \to\cdots) = \cdots \to \bullet \to \bullet \to 0\to 0 \to\cdots. \]
	In contexts where a t-structure is present, it is often of central interest, and whether a functor is t-exact or not can have significant impact. For example,
	you may have an equivalence \(\calT \simeq \calT'\) of triangulated categories, but if both have ``natural'' t-structures on them and the equivalence does not preserve them, then
	that could present difficulties if one is relying on e.g.\ ``boundedness'' assumptions in one's work.
\end{remark}

\subsection{t-Compatible exact sequences}
The notion of a functor being t-exact describes a way in which it is compatible with some specified t-structures. Naturally, in the presence of an exact sequence, this leads us to the following definition.
\begin{definition}
	We say an exact sequence
	\[ \calT_1 \overset{i}\to \calT \overset{p}\to \calT_2 \]
	of triangulated categories with t-structures is \emph{t-compatible} if \(i\) and \(p\) are t-exact.
\end{definition}
\begin{proposition}
	Consider a t-compatible exact sequence
	\[ \calT_1 \overset{i}\to \calT \overset{p}\to \calT_2 \]
	of triangulated categories with t-structures. Then
	\begin{align*}
		i(\calT_1^{\leq 0}) &= \calT^{\leq 0}\cap i(\calT_1), & \calT_2^{\leq 0} &= p(\calT^{\leq 0}), \\
		i(\calT_1^{\geq 0}) &= \calT^{\geq 0}\cap i(\calT_1), & \calT_2^{\geq 0} &= p(\calT^{\geq 0}).
	\end{align*}
\end{proposition}
\begin{proof}
It suffices to show the equalities on one row, as the other is dual. We begin with the equality on the left. It is clear that \(i(\calT_1^{\leq 0}) \subseteq \calT^{\leq 0}\cap i(\calT_1)\)
since \(i\) is t-exact. Conversely, let \(x\in\calT^{\leq 0}\cap i(\calT_1)\), and write \(x \cong i(x_1)\). We show that \(\tau^{\geq 0}x_1 \cong 0\). To this end, consider the distinguished triangle
\[ \tau^{\leq 0}x_1 \to x_1 \to \tau^{\geq 1}x_1 \to \Sigma \tau^{\leq 0}x_1. \]
By t-exactness, we have \(i(\tau^{\leq 0}x_1) \in \calT^{\leq 0}\) and \(i(\tau^{\geq 1}x_1) \in \calT^{\geq 1}\). Since \(i\) is a triangulated functor, we thus see that the induced triangle
\[ i(\tau^{\leq 0}x_1) \to x \to i(\tau^{\geq 1}x_1) \to \Sigma i(\tau^{\leq 0}x_1) \]
has left term in \(\calT^{\leq 0}\) and right term in \(\calT^{\geq 1}\). By uniqueness, we see that \(i(\tau^{\geq 1}x_1) \cong \tau^{\geq 1}x\), but since \(x\in\calT^{\leq 0}\), we thus
have \(i(\tau^{\geq 1}x_1)\cong 0\). Therefore, \(\tau^{\geq 1}x_1 \cong 0\) on account of \(i\) being fully faithful.

We must now show that \(p(\calT^{\leq 0}) = \calT_2^{\leq 0}\). By t-exactness, it is clear that \(p(\calT^{\leq 0}) \subseteq \calT_2^{\leq 0}\), so let us prove the converse inclusion.
Pick \(x_2 \in \calT_2^{\leq 0}\). Since \(p\) is essentially surjective, write \(x_2 \cong p(x)\) for some \(x\in\calT\). Consider the distinguished triangle
\[ \tau^{\leq 0}x \to x \to \tau^{\geq 1}x \to \Sigma\tau^{\leq 0}x. \]
Applying \(p\), we get
\[ p(\tau^{\leq 0}x) \to x_2 \to p(\tau^{\geq 1}x) \to \Sigma p(\tau^{\leq 0}x). \]
where the left term is in \(\calT_2^{\leq 0}\) and the right term is in \(\calT_2^{\geq 1}\), hence, similarly as before, \(p(\tau^{\geq 1}x) \cong \tau^{\geq 1}x_2 \cong 0\).
Finally, we just note that this implies \(p(\tau^{\leq 0}x) \cong x_2\), so we are done.
\end{proof}
\begin{remark}
	Oberve that we also prove that \(i\circ\tau^{\leq 0} \cong \tau^{\leq 0}\circ i\) and \(p\circ \tau^{\leq 0} \circ \tau^{\leq 0}\circ p\), along with the dual versions.
\end{remark}

The proposition says that in a t-compatible exact sequence, the t-structures on the left and right terms are completely determined by the functors \(i\) and \(p\), along
with the t-structure on the middle term. The converse is also true.
\begin{proposition}\label{prop:t-compatible-exact-sequence-middle-formula-from-sides}
	Consider a t-compatible exact sequence
	\[ \calT_1 \overset{i}\to \calT \overset{p}\to \calT_2 \]
	of triangulated categories with t-structures. Then
	\begin{align*}
		\calT^{\leq 0} &= \{ x\in\bperp{i(\calT_1^{\geq 1})}\phantom{.}\hspace{0.8mm} \mid p(x)\in\calT^{\leq 0} \},\\
		\calT^{\geq 0} &= \{ x\in i(\calT_1^{\leq -1})^\perp \mid p(x)\in\calT^{\geq 0} \}.
	\end{align*}
\end{proposition}

To prove this, we need a preparatory lemma.
\begin{lemma}\label{lemma:t-compatible-exact-sequence-right-aisle-criterion}
	Consider a t-compatible exact sequence
	\[ \calT_1 \overset{i}\to \calT \overset{p}\to \calT_2 \]
	of triangulated categories with t-structures, and let \(y\in\calT\). Then \(p(y)\in\calT_2^{\leq 0}\) if and only if \(\tau^{\geq 1}y \in i(\calT_1^{\geq 1})\).
\end{lemma}
\begin{proof}
We consider the distinguished triangle
\[ \tau^{\leq 0}y \to y \to \tau^{\geq 1}y \to \Sigma\tau^{\leq 0}y \]
which we note, by the same arguments as before, is sent by \(p\) to
\[ \tau^{\leq 0}p(y) \to p(y) \to \tau^{\geq 1}p(y) \to \Sigma\tau^{\leq 0}p(y). \]
Now we see that
\[ p(y)\in\calT_2^{\leq 0} \iff \tau^{\geq 1}p(y) \cong 0 \iff \tau^{\geq 1}y \in \ker{p} = i(\calT_1). \]
Since \(\tau^{\geq 1}y \in \calT^{\geq 1}\) and \(\calT^{\geq 1}\cap i(\calT_1) = i(\calT_1^{\geq 1})\), we are done.
\end{proof}

\begin{proof}[Proof of Proposition \ref{prop:t-compatible-exact-sequence-middle-formula-from-sides}]
We prove the first equality, since the second is dual.

Let \(x\in\calT^{\leq 0}\). Since \(i(\calT_1^{\geq 1})\subseteq\calT^{\geq 0}\), we see that
\[ \forall y_1\in\calT_1^{\geq 1},\quad \calT(x,i(y_1)) \cong 0 \]
so that \(x\in\bperp{i(\calT_1^{\geq 1})}\). That \(p(x)\in\calT_2^{\leq 0}\) follows by t-exactness. This shows the \(\subseteq\) inclusion.

Conversely, suppose that \(x\in\bperp{i(\calT_1^{\geq 1})}\) and \(p(x)\in\calT_2^{\leq 0}\). Applying Lemma \ref{lemma:t-compatible-exact-sequence-right-aisle-criterion}, we see that
\(\tau^{\geq 1}x \in i(\calT_1^{\geq 1})\), and therefore
\[ \calT(x,\tau^{\geq 1}x) \cong 0 \]
by the orthogonality assumption. Therefore, Lemma \ref{lemma:t-structure-aisle-from-trivial-mapping-to-truncation} says that \(x\in\calT^{\leq 0}\) as desired.
\end{proof}
\begin{corollary}
	Consider an exact sequence
	\[ \calT_1 \to \calT \to \calT_2 \]
	where \(\calT_1\) and \(\calT_2\) have t-structures on them. Then there is at most one t-structure on \(\calT\) for which the sequence is t-compatible.
\end{corollary}

\subsection{Gluing in a t-compatible recollement}
To collect what we have proven: in a t-compatible exact sequence
\[ \calT_1 \to \calT \to \calT_2, \]
the t-structures on the left/right terms uniquely determine the one in the middle, and vice versa. The question we now ask is: what if we drop the t-structure on \(\calT\)?

More precisely, the problem is as follows. Suppose we have t-structures on the \(\calT_i\), \(i=1,2\). Is it possible to find a t-structure on \(\calT\) such that the exact sequence
becomes t-compatible? A priori, there is no good answer to this, other than ``in general, no.'' Of course, we know that if one exists, it is unique and given by an explicit formula by
Proposition \ref{prop:t-compatible-exact-sequence-middle-formula-from-sides}. The trouble is in showing that this formula actually produces a t-structure.
The point of this final part of the lecture is that the situation is considerably different when the exact sequence is a recollement.
Throughout, let us fix a recollement
\begin{diagram*}[column sep=large]
	\calT_1\ar[r,"i" description,""{below,name=A},""{above,name=AA}] &
		\calT \ar[r,"p" description,""{below,name=C},""{above,name=CC}]\ar[l,bend left,shift left,"i_R",""{above,name=B}]\ar[l,bend right,shift right,"i_L"',""{below,name=BB}] &
		\calT_2 \ar[l,bend left,shift left,"p_R",""{above,name=D}]\ar[l,bend right,shift right,"p_L"',""{below,name=DD}]
		\ar[from=B,to=A,symbol=\vdash]\ar[from=D,to=C,symbol=\vdash]
		\ar[from=AA,to=BB,symbol=\vdash]\ar[from=CC,to=DD,symbol=\vdash]
\end{diagram*}
where \(\calT_1\) and \(\calT_2\) have t-structures on them. Our goal is the following.
\begin{theorem}\label{thm:t-structure-gluing-in-recollement}
	Define the full subcategories
	\begin{align*}
		\calT^{\leq 0} &:= \{ x \in \calT \mid i_L(x) \in \calT_1^{\leq 0},\, p(x)\in\calT_2^{\leq 0} \}, \\
		\calT^{\geq 0} &:= \{ x \in \calT \mid i_R(x) \in \calT_1^{\geq 0},\, p(x)\in\calT_2^{\geq 0} \}.
	\end{align*}
	Then \((\calT^{\leq 0},\calT^{\geq 0})\) is a t-structure on \(\calT\), and is the unique t-structure making the above sequence t-compatible.
\end{theorem}


\begin{lemma}\label{lemma:t-structure-gluing-recollement-adjoints-preserve}
	With the notation from Theorem \ref{thm:t-structure-gluing-in-recollement}, we have
	\[ p_L(\calT_2^{\leq 0}) \subseteq \calT^{\leq 0},\quad p_R(\calT_2^{\geq 0}) \subseteq \calT^{\geq 0} \]
	and
	\[ i(\calT_1^{\leq 0})\subseteq \calT^{\leq 0},\quad i(\calT_1^{\geq 0})\subseteq\calT^{\geq 0}. \]
\end{lemma}
\begin{proof}
We focus on the incluions on the left, as the others are dual; we prove the bottom one and leave the top as an exercise. If \(x_1\in\calT_1^{\leq 0}\),
then \(i(x_1)\) satisfies
\[ i_Li(x_1) \cong x_1 \in \calT_1^{\leq 0},\quad \text{and} \quad p(x_1) \cong 0 \in\calT_2^{\leq 0}. \]
Therefore, \(i(x_1)\in\calT^{\leq 0}\), so \(i(\calT_1^{\leq 0})\subseteq\calT^{\leq 0}\).
\end{proof}
\begin{exercise}
	Prove the rest of Lemma \ref{lemma:t-structure-gluing-recollement-adjoints-preserve}.
\end{exercise}

With the lemma in place, we can prove the gluing theorem.

\begin{proof}[Proof of Theorem \ref{thm:t-structure-gluing-in-recollement}]
Let us begin by proving (T2). Let \(x\in\calT^{\leq 0}\). We must show that \(\Sigma x \in\calT^{\leq 0}\). However,
\[ i_L(\Sigma x) \cong \Sigma i_L(x) \in \calT_1^{\leq -1} \subseteq \calT_1^{\leq 0} \]
and
\[ p(\Sigma x) \cong \Sigma p(x) \in \calT_2^{\leq -1} \subseteq \calT_2^{\leq 0}. \]
Therefore, \(\Sigma\calT^{\leq 0}\subseteq\calT^{\leq 0}\). Showing that \(\Sigma^{-1}\calT^{\geq 0}\subseteq\calT^{\geq 0}\) is similar.

We now prove (T1). Let \(x\in\calT^{\leq 0}\) and \(y\in\calT^{\geq 0}\). By the remarks of Section \ref{subsection:recollements}, we have a distinguished triangle
\[ p_Lp(x) \to x \to ii_L \to \Sigma p_Lp(x). \]
Applying \(\calT(-,\Sigma^{-1}y)\) and using adjointness, we get the exact sequences
\begin{diagram*}[cramped]
	\calT(ii_L(x),\Sigma^{-1}y) \ar[d,"\sim" labl] \ar[r] & \calT(x,\Sigma^{-1}y) \ar[d,equal] \ar[r] & \calT(p_Lp(x),\Sigma^{-1}y) \ar[d,"\sim" labl] \\
	\calT(i_L(x),\Sigma^{-1}i_R(y)) \ar[r] & \calT(x,\Sigma^{-1}y) \ar[r] & \calT(p(x),\Sigma^{-1}p(y))
\end{diagram*}
at which point we observe that the left and right terms of the bottom row are zero since \(i_R(y)\in\calT_1^{\geq 0}\) and \(p(y)\in\calT_2^{\geq 0}\). Therefore,
\(\calT(x,\Sigma^{-1}y)\cong 0\) as desired.

What remains now is (T3), which is the trickiest and most sophisticated part. Let \(x\in\calT\). We must find a distinguished triangle sandwiching \(x\)
between an object of \(\calT^{\leq 0}\) and an object of \(\Sigma^{-1}\calT^{\geq 0}\). Let us attempt an approximation first: project down to \(\calT_2\) using \(p\),
extract a distinguished triangle there, then apply \(p_R\) to get a distinguished triangle
\[ p_R\tau^{\leq 0}p(x) \to p_Rp(x) \to p_R\tau^{\geq 1}p(x) \to \Sigma p_R\tau^{\leq 0}p(x). \]
We now take the cocone of the composition \(x \to p_Rp(x) t\o p_R\tau^{\geq 1}p(x)\) to get a distinguished triangle
\[ x' \to x \to p_R\tau^{\geq 1}p(x) \to \Sigma x'. \]
This is almost what we want, as Lemma \ref{lemma:t-structure-gluing-recollement-adjoints-preserve} suggests, but we don't really have control over the left term and can't say much about it. So, let's do basically the same trick
to get a map \(x' \to i\tau^{\geq 1}i_L(x')\), and take the cocone to get a distinguished triangle
\[ a \to x' \to i\tau^{\geq 1}i_L(x') \to \Sigma a. \]
Finally, lets take the cone of the composition \(a \to x' \to x\) to get a distinguished triangle
\[ a \to x \to b \to \Sigma a. \]
With these steps done, we may now present the crux of the argument.

Applying (TR4) to the composition \(a \to x' \to x\) yields a distinguished triangle
\[ i\tau^{\geq 1}i_L(x') \to b \to p_R\tau^{\geq 1}p(x) \to \Sigma i\tau^{\geq 1}i_L(x'). \]
Hitting this with \(p\) yields
\[ 0 \to p(b) \iso \tau^{\geq 1}p(x) \to 0. \]
Hitting it with \(i_R\), on the other hand, yields
\[ \tau^{\geq 1}i_L(x') \iso i_R(b) \to 0 \to \Sigma \tau^{\geq 1}i_L(x'). \]
We conclude that \(b\in\calT^{\geq 1}\). It remains to see that \(a\in\calT^{\leq 0}\). To show this, apply \(i_L\) to the triangle defining \(a\), to get
\[ i_L(a) \to i_L(x') \to \tau^{\geq 1}i_L(x') \to \Sigma i_L(a). \]
By uniqueness of cocones (up to non-canonical isomorphism), this implies that \(i_L(a)\cong\tau^{\leq 0}i_L(x')\in\calT_1^{\leq 0}\). Finally,
applying \(p\) to the triangle defining \(b\), and using that \(p(b)\cong \tau^{\geq 1}p(x)\), yields
\[ p(a) \to p(x) \to \tau^{\geq 1}p(x) \to \Sigma p(a) \]
which implies that \(p(a) \cong \tau^{\leq 0}p(x) \in\calT_2^{\leq 0}\). Therefore, \(a\in\calT^{\leq 0}\) and we are done.
\end{proof}

\begin{example}
	Here is a simple application of Theorem \ref{thm:t-structure-gluing-in-recollement}. Consider any old recollement
	\begin{diagram*}[column sep=large]
		\calT'\ar[r,"i" description,""{below,name=A},""{above,name=AA}] &
			\calT \ar[r,"p" description,""{below,name=C},""{above,name=CC}]\ar[l,bend left,shift left,"i_R",""{above,name=B}]\ar[l,bend right,shift right,"i_L"',""{below,name=BB}] &
			\calT'' \ar[l,bend left,shift left,"p_R",""{above,name=D}]\ar[l,bend right,shift right,"p_L"',""{below,name=DD}]
			\ar[from=B,to=A,symbol=\vdash]\ar[from=D,to=C,symbol=\vdash]
			\ar[from=AA,to=BB,symbol=\vdash]\ar[from=CC,to=DD,symbol=\vdash]
	\end{diagram*}
	Recall that we may put certain \emph{trivial} t-structures on \(\calT'\) and \(\calT''\). For any triangulated category \(\calD\), let us write
	\[ t_1(\calD) := (\calD,0),\quad t_2(\calD) := (0,\calD). \]
	Gluing \(t_1(\calT')\) and \(t_1(\calT'')\), or \(t_2(\calT')\) and \(t_2(\calT'')\), does nothing interesting: we just get \(t_1(\calT)\) or \(t_2(\calT)\).
	On the other hand, gluing \(t_1(\calT')\) with \(t_2(\calT'')\), we get the t-structure
	\begin{align*}
		\calT^{\leq 0}_{1,2} &= \{ x\in\calT \mid i_L(x)\in\calT',\, p(x) = 0 \} \\
		&= \ker{p} = i(\calT'),\\
		\calT^{\geq 0}_{1,2} &= \{ x\in\calT \mid i_R(x) = 0,\, p(x)\in\calT'' \} \\
		&= \ker{i_R} = i(\calT')^\perp.
	\end{align*}
	In other words, we get the t-structure \((i(\calT'),i(\calT')^\perp)\) on \(\calT\). In light of our results on t-structures, in particular Corollary \ref{corollary:t-structure-orthogonality},
	this yields another proof that \(\bperp{(i(\calT')^\perp)} = i(\calT')\) when one is in a recollement situation. Explicitly, one applies the corollary and uses the fact
	that \(\Sigma^{-1}i(\calT')^\perp = i(\calT')^\perp\).
\end{example}
\begin{example}
	If we have a recollement
	\begin{diagram*}[column sep=large]
		\calT_1\ar[r,"i" description,""{below,name=A},""{above,name=AA}] &
			\calT \ar[r,"p" description,""{below,name=C},""{above,name=CC}]\ar[l,bend left,shift left,"i_R",""{above,name=B}]\ar[l,bend right,shift right,"i_L"',""{below,name=BB}] &
			\calT_2 \ar[l,bend left,shift left,"p_R",""{above,name=D}]\ar[l,bend right,shift right,"p_L"',""{below,name=DD}]
			\ar[from=B,to=A,symbol=\vdash]\ar[from=D,to=C,symbol=\vdash]
			\ar[from=AA,to=BB,symbol=\vdash]\ar[from=CC,to=DD,symbol=\vdash]
	\end{diagram*}
	with t-structures on \(\calT_1\) and \(\calT_2\), then we can observe that for all \(n\in\Z\), \((\calT_i^{\leq n},\calT_i^{\geq n})\) also defines
	a t-structure on \(\calT_i\). As a result, for any pair of integers \((n_1,n_2)\in\Z\times\Z\), we can produce a t-structure
	\begin{align*}
		\calT^{\leq 0}_{n_1,n_2} &= \{ x\in\calT \mid i_L(x)\in\calT_1^{\leq n_1},\, p(x)\in\calT_2^{\leq n_2} \}, \\
		\calT^{\geq 0}_{n_1,n_2} &= \{ x\in\calT \mid i_R(x)\in\calT_1^{\geq n_1},\, p(x)\in\calT_2^{\geq n_2} \}.
	\end{align*}
\end{example}



%!TEX root = ../lectures.tex

\section{Ind-categories \& compactness}
Recall that, in topology, there is the notion of a \emph{compact topological space.} This is, by definition, a topological space \(X\) such that any open cover has a finite subcover. In other words,
it is a space whose topology is in some sense ``finitary,'' and the space itself is not too big. More precisely, we can always understand some potentially infinite collection of data on \(X\)
in terms of a finite subset.

Inspired by the topological notion of compactness, we may produce a \emph{categorical} notion of compact objects which is similar in terms of intuition.

\subsection{Compact objects}
\begin{construction}
	Let \(\calC\) be a locally small category, let \(\kappa\) be an infinite regular cardinal, let \(I\) be a \(\kappa\)-filtered category, and suppose \(\calC\) admits \(I\)-shaped colimits. For a diagram \(D\!:I\to\calC\),
	we consider the colimit cone
	\[ D\To \projlim{D} \]
	and note, for \(x\in\calC\), that applying \(\calC(x,-)\) yields a cone
	\[ \calC(x,D)\To \calC(x,\projlim{D}) \]
	which induces a canonical morphism
	\[ \projlim\calC(x,D)\to\calC(x,\projlim{D}) \]
	which, for clarity, we also write as
	\[ \projlim_{i\in I}\calC(x,D(i))\to\calC(x,\projlim_{i\in I}{D(i)}). \]
\end{construction}
\begin{definition}
	Let \(\calC\) be a locally small category, let \(\kappa\) be an infinite regular cardinal, and let \(x\in\calC\). We say \(x\) is \(\kappa\)\emph{-compact} if, for any diagram \(D\!:I\to\calC\) where \(I\) is \(\kappa\)-filtered,
	the canonical map
	\[ \projlim_{i\in I}\calC(x,D(i))\to\calC(x,\projlim_{i\in I}{D(i)}) \]
	is an isomorphism. We say \(x\) is \emph{compact} if it is \(\aleph_0\)-compact (i.e.\ \(\omega\)-compact).

	We denote by \(\calC^\kappa\) the full subcategory of \(\calC\) spanned by the \(\kappa\)-compact objects.
\end{definition}
\begin{remark}
	One can rephrase \(\kappa\)-compactness as follows: \(x\in\calC\) is \(\kappa\)-compact if for any \(\kappa\)-filtered diagram \(D\!: I\to\calC\) and morphism
	\[ x \to \injlim_{i\in I}D(i) \]
	there is some \(i'\in I\) such that the above morphism factors uniquely as
	\begin{diagram*}[row sep=small]
		x \ar[rr]\ar[dr,dashed] & & \injlim_{i\in I}D(i)\\
		 & D(i')\ar[ur] &
	\end{diagram*}
	where the morphism \(D(i')\to\injlim_{i\in I}D(i)\) is the canonical morphism induced by universal property. Note the similarity to the concept of compactness in \emph{additive categories}
	explored in Appendix \ref{appendix:abelian-categories-with-a-compact-projective-generator}.
\end{remark}
\begin{example}\label{example:compact-objects-in-Set}
	The \(\kappa\)-compact objects in \(\Set\) are precisely the \(\kappa\)-small sets. Indeed, let \(A\) be some set, and assume it is \(\kappa\)-compact. Let \(\calS_A\) be the poset of
	\(\kappa\)-small subsets of \(A\). Note that
	\[ \injlim_{A'\in\calS_A}{A'} = \bigcup_{A'\in\calS_A} A' = A. \]
	This is just the statement that every set is the union of its \(\kappa\)-small subsets. Now, since \(A\) is \(\kappa\)-compact, we get a factorization through some \(\kappa\)-small set \(A'\)
	\begin{diagram*}
		A \ar[rr,equal]\ar[dr,dashed,"p"'] & & A\\
		& A'\ar[ur,hook,"i"'] &
	\end{diagram*}
	and in particular, \(i\circ p = \id_A\). The map \(i\), as indicated, is injective, but because it factors the identity, it must also be surjective, hence a bijection.
	That is, we get \(i\!:A'\iso A\), so \(A\) is \(\kappa\)-small. For the converse, by Proposition \ref{prop:compact-objects-closed-under-sufficiently-small-colimits-and-retracts}\ref{prop:compact-objects-closure-props:item:colimits}
	below the \(\kappa\)-compact objects are closed under \(\kappa\)-small colimits, e.g.\ disjoint unions of \(<\kappa\) points.
\end{example}
\begin{example}
	The compact objects in \(\Top\) are the finite discrete spaces. This is a bit harder to show, but we sketch the argument. First of all, if \(X\in\Top\) is compact,
	then \(X\) is finite: if \(X'\) is the topological space with underlying set \(X\) equipped with the indiscrete topology, then one may check that \(X'\) is the colimit
	of its finite subsets (with the indiscrete topology). Thus, the same trick as in \(\Set\) shows that \(X\) is finite. 

	To see that \(X\) is discrete, one constructs a sequence of spaces \(Y_n\) with underlying set \(\Z_{\geq n}\times\{0,1\}\) and topology given by the sets \(U_{n,m} = \Z_{\geq m}\times\{0\}\cup\Z_{\geq n}\times\{1\}\),
	for \(m\geq n\). These have canonical maps \(Y_{n}\to Y_{n+1}\), and one checks that \(Y_{\infty} = \injlim_nY_n\) is given by the set \(\{0,1\}\) with the indiscrete topology, the canonical map \(Y_n\to Y_{\infty}\) being
	the projection onto the second component. For any (!) subset \(U\subseteq X\), one may then consider the trivially continuous map \(1_U\!:X\to Y_{\infty}\) and factor it through some \(Y_n\) by compactness;
	projecting onto the first component, one can take an appropriate inverse image to see that \(U\) is open.
\end{example}
\begin{exercise}
	Consider an adjunction
	\begin{tikzcd}[cramped]
		\calD\ar[from=r,bend right,"L"',""{name=A,below}] & \calC. \ar[from=l,bend right,"R"',""{name=B,above}]\ar[from=A,to=B,symbol=\dashv]
	\end{tikzcd}
	\begin{enumerate}[label=(\arabic*)]
	\item Assume that \(\calC\), \(\calD\) admit \(\kappa\)-filtered colimits, and \(R\) commutes with \(\kappa\)-filtered colimits. Show that \(L\) preserves \(\kappa\)-compact objects, i.e.\ that it
	restricts to a functor \(\calC^\kappa\to\calD^\kappa\).
	\item Let \(\Bbbk\) be a field. Show that the forgetful functor \(U\!:\Vect_\Bbbk\to\Set\) preserves filtered colimits. Deduce using (1) that finite-dimensional
	vector spaces over \(\Bbbk\) are compact.
	\item Show that, in fact, the compact objects in \(\Vect_\Bbbk\) are exactly the finite-dimensional spaces.
	\end{enumerate}
\end{exercise}

We now turn to trying to characterize the compact objects in certain nice classes of categories. To do this, we will need to investigate a closure property that the category \(\calC^\kappa\)
of \(\kappa\)-compact objects in a category \(\calC\) satisfies.
\begin{definition}
	Let \(\calC\) be a category. An object \(x\in\calC\) is a \emph{retract} of \(y\in\calC\) if there is a diagram
	\begin{diagram*}
		x\ar[r]\ar[rr,bend left,"\id_x"] & y \ar[r] & x.
	\end{diagram*}
	We say that a full subcategory \(\calC'\subseteq\calC\) is \emph{closed under retracts} if, whenever \(y\in\calC'\) and \(x\) is a retract of \(y\), we have \(x\in\calC'\).
\end{definition}
\begin{lemma}\label{lemma:isomorphisms-closed-under-retracts}
	Let \(\calC\) be a category, and consider the category \(\Mor(\calC)\) of morphisms in \(\calC\), i.e.\ the functor category \(\Fun([1], \calC)\). Then the full subcategory
	of \(\Mor(\calC)\) spanned by the isomorphisms is closed under retracts.
\end{lemma}
\begin{proof}
The claim is that in a diagram
\begin{diagram*}
	x\ar[r,"a"]\ar[d,"f"']\ar[rr,bend left,"\id_x"] & x'\ar[d,"g"] \ar[r,"b"] & x\ar[d,"f"] \\
	y\ar[r,"a'"]\ar[rr,bend right,"\id_y"'] & y' \ar[r,"b'"] & y
\end{diagram*}
wherein the middle vertical arrow \(g\) is an isomorphism, all vertical arrows are isomorphisms.
We claim that the inverse of \(f\) is given by \(b\circ g^{-1}\circ a'\). To see this, we compute
\[ f\circ b \circ g^{-1} \circ a' = b' \circ a' = \id_y \]
and
\[ b\circ g^{-1} \circ a' \circ f = b \circ a = \id_x \]
as desired.
\end{proof}
\begin{proposition}\label{prop:compact-objects-closed-under-sufficiently-small-colimits-and-retracts}
	Let \(\calC\) be a locally small category. Recall that 
	Now, let \(\kappa\) be an infinite regular cardinal, and suppose that \(\calC\) admits \(\kappa\)-filtered colimits. Then
	\begin{enumerate}[label=(\arabic*)]
	\item\label{prop:compact-objects-closure-props:item:colimits} \(\calC^\kappa\) is closed under \(\kappa\)-small colimits in \(\calC\).
	\item\label{prop:compact-objects-closure-props:item:retracts} \(\calC^\kappa\) is closed under retracts, i.e.\ if \(c'\in\calC\) is \(\kappa\)-compact and \(c\in\calC\) is a retract of \(c'\), then \(c\) is \(\kappa\)-compact.
	\end{enumerate}
\end{proposition}
\begin{proof}
\ref{prop:compact-objects-closure-props:item:colimits} Suppose we have a \(\kappa\)-filtered diagram \(C\!:I\to\calC\) and a \(\kappa\)-small diagram \(J\to\calC^\kappa\). Then, by the definition
of \(\kappa\)-compactness and Theorem \ref{thm:filtered-colimits-commute-with-finite-limits-in-set}, we have
\begin{align*}
	\injlim_{i\in I}\calC(\injlim_{j\in J}C(j),D(i)) &\cong \injlim_{i\in I}\projlim_{j\in J}\calC(C(j),D(i))\\
	&\cong \projlim_{j\in J}\injlim_{i\in I}\calC(C(j),D(i)) \\
	&\cong \projlim_{j\in J}\calC(C(j),\injlim_{i\in I}D(i)) \cong \calC(\injlim_{j\in J}C(j),\injlim_{i\in I}D(i))
\end{align*}
as desired.

\ref{prop:compact-objects-closure-props:item:retracts} Let \(D\!: I\to\calC\) be a \(\kappa\)-filtered diagram, let \(y\) be \(\kappa\)-compact, and let \(x\) be a retract of \(y\). This data then induces a retraction
\begin{diagram*}
	\injlim_{i\in I}\calC(x,D(i)) \ar[r]\ar[d] & \injlim_{i\in I}\calC(y,D(i))\ar[d,"\sim" labl] \ar[r] & \injlim_{i\in I}\calC(x,D(i))\ar[d] \\
	\calC(x,\injlim_{i\in I}D(i))\ar[r] & \calC(y,\injlim_{i\in I}D(i)) \ar[r] & \calC(x,\injlim_{i\in I}D(i))
\end{diagram*}
and thus, by Lemma \ref{lemma:isomorphisms-closed-under-retracts}, we are done.
\end{proof}

\begin{remark}
	Though the notion of compactness in this lecture may seem irreconcilably different from the one in Appendix \ref{appendix:abelian-categories-with-a-compact-projective-generator}, the one here
	actually implies the one there. The reason for this is the following trick: one can write small coproducts as filtered colimits of finite coproducts. In particular, let \(I\) be some set,
	and consider a collection of objects \(\{y_i\}_{i\in I}\) in an additive category \(\calC\) admitting filtered colimits indexed by \(I\). Then one can consider the (obviously filtered) poset \(P\) given by all finite subsets of \(I\),
	and the induced diagram \(P\to\calC\) given by
	\[ (J\subseteq I) \mapsto \bigoplus_{j\in J} y_j. \]
	Taking the colimit of this diagram, one can fairly easily check that
	\[ \coprod_{i\in I} y_i = \injlim_{\substack{J\subseteq I \\ J\text{ finite}}}\left(\bigoplus_{j\in J}y_j\right). \]
	In particular, if \(x\in\calC\) is compact in the sense described in this lecture, then we get
	\begin{diagram*}
		x\ar[rr]\ar[dr,dashed] & & \coprod_{i\in I}y_i\ar[d,equal] \\
		& \bigoplus_{j\in J}y_j\ar[r] & \injlim_{J\subseteq I}\left(\bigoplus_{j\in J}y_j\right)
	\end{diagram*}
	so \(x\) is also compact in the sense given in Appendix \ref{appendix:abelian-categories-with-a-compact-projective-generator}.
\end{remark}

\subsection{Ind-categories}
To give examples of \(\kappa\)-compact objects, we will provide a construction which associates to any category \(\calC\) a category \(\Ind_\kappa(\calC)\) which contains \(\calC\) as a full subcategory (up to equivalence)
and in which every object of \(\calC\) becomes \(\kappa\)-compact. It is based on taking \(\calC\) and formally cocompleting it with respect to \(\kappa\)-filtered colimits. The construction
is also of independent interest, as it plays a role in numerous techniques of category theory. For example, the Freyd--Mitchell embedding theorem (that any Abelian category may be embedded into
a module category) makes use of the embedding \(\calC\inj\Ind(\calC)\) (or, in actuality, a small adjustment of it), at which point one applies the result in Appendix \ref{appendix:abelian-categories-with-a-compact-projective-generator}.

\begin{definition}
	Let \(\calC\) be a locally small category, and let \(\kappa\) be an infinite regular cardinal. We define the category
	\[ \Ind_\kappa(\calC) \subseteq \Fun(\calC^\op,\Set) \]
	as the full subcategory of \(\Fun(\calC^\op,\Set)\) spanned by functors isomorphic to a \(\kappa\)-filtered small colimit of representable functors.
\end{definition}
\begin{remark}
	Observe that, since the point-category \([0]\) is \(\kappa\)-filtered for all \(\kappa\), every representable functor is an object of \(\Ind_\kappa(\calC)\). In particular, the Yoneda embedding
	factors as
	\[ \calC \inj \Ind_\kappa(\calC) \subseteq \Fun(\calC^\op,\Set). \]
	In this way, we may identify \(\calC\) with a full subcategory of \(\Ind_\kappa(\calC)\).
\end{remark}
\begin{lemma}\label{lemma:ind-category-admits-filtered-colimits}
	Let \(\calC\) be a locally small category, and let \(\kappa\) be an infinite regular cardinal. Then \(\Ind_\kappa(\calC)\) admits small \(\kappa\)-filtered colimits, and the canonical inclusion
	\[ \Ind_\kappa(\calC) \inj \Fun(\calC^\op,\Set) \]
	preserves with them.
\end{lemma}

We defer a proof of this lemma for a later lecture, as it combines a few foundational categorical ingredients that we have not yet introduced, and which are not the focus of our attention at the moment.
\begin{proposition}\label{prop:ind-compact-objects-are-retracts-of-representables}
	Let \(\calC\) be a locally small category, and let \(\kappa\) be an infinite regular cardinal. Then an object \(A\in\Ind_\kappa(\calC)\) is \(\kappa\)-compact if
	and only if it is a retract of a representable functor. That is, \(\Ind_\kappa(\calC)^\kappa\) is the smallest full subcategory of \(\Ind_\kappa(\calC)\) which
	contains all representable functors and which is closed under retracts.
\end{proposition}
\begin{proof}
We begin by showing that for \(x\in\calC\), the functor \(h_x\in\Ind_\kappa(\calC)\) is \(\kappa\)-compact. This follows simply by the fact that colimits in functor categories are
computed point-wise. First, note that the inclusion \(\Ind_\kappa(\calC)\subseteq\Fun(\calC^\op,\Set)\) preserves \(\kappa\)-filtered colimits. Thus, we may compute
\begin{align*}
	\PSh(\calC)(h_x, \injlim_i D(i)) \cong (\injlim_i D(i))(x) = \injlim_i D(i)(x) \cong \injlim_i\PSh(\calC)(h_x,D(i))
\end{align*}
as desired. It now follows from Proposition \ref{prop:compact-objects-closed-under-sufficiently-small-colimits-and-retracts} that the retract of any representable functor is \(\kappa\)-compact.

Conversely, we must show that any \(\kappa\)-compact \(A\in\Ind_\kappa(\calC)\) is the retract of a representable functor. By definition,
\[ A \cong \injlim_{i\in I} h_{x_i} \]
but, on the other hand, the compactness of \(A\) thus means that we may find some \(j\in I\) and a factorization
\begin{diagram*}
	A\ar[rr,equal]\ar[dr] & & A \\
	& h_{x_j}\ar[ur] & 
\end{diagram*}
so that \(A\) is a retract of \(h_{x_j}\).
\end{proof}


\subsection{Idempotent completeness}
The above result establishes that the embedding \(\calC\inj\Ind_\kappa(\calC)\) factors through the \(\kappa\)-compact objects
\[ \calC\inj\Ind_\kappa(\calC)^\kappa\subseteq\Ind_\kappa(\calC). \]
An interesting question to ask is under what circumstances this induces an equivalence between \(\calC\) and \(\Ind_\kappa(\calC)^\kappa\). Since we know that the latter
is the ``closure'' of \(\calC\) with respect to retracts in \(\Ind_\kappa(\calC)\), in order for this equivalence to hold, we must know that the retract of a representable is
a representable. In order to find conditions enabling this, we will pass through a slightly different concept which is closely related to retracts: idempotents.
\begin{definition}
	Let \(\calC\) be a category. An endomorphism \(e\!:x\to x\) in \(\calC\) is an \emph{idempotent} if
	\[ e^2 = e. \]
\end{definition}
\begin{example}
	Let \(\calC\) be a category, and consider a retraction
	\[ y \overset{a}\to x\overset{b}\to y. \]
	Then the morphism \(e := a\circ b\!: c\to c\) is an idempotent. In particular,
	\[ e^2 = e\circ e = (a \circ b) \circ (a\circ b) = a\circ \id_{y} \circ b = e. \]
\end{example}
\begin{definition}
	We say that an idempotent \(e\!:x\to x\) in a category \(\calC\) \emph{splits into a retraction} if there is a retraction
	\[ y \overset{a}\to x\overset{b}\to y \]
	such that \(e = a\circ b\).
\end{definition}

\begin{lemma}\label{lemma:idempotent-splitting-uniqueness}
	Let \(\calC\) be a category, and let \(e\!:x\to x\) be an idempotent in \(\calC\). Suppose that \(e\) splits into a retraction in two ways:
	\begin{align*}
	y &\overset{a}\to x\overset{b}\to y, \\
	y' &\overset{a'}\to x\overset{b'}\to y'.
	\end{align*}
	Then the morphisms \(b'\circ a\!:y\to y'\) and \(b\circ a'\!:y'\to y\) are inverse to each other, and we have an isomorphism of retractions
	\begin{diagram*}
		y \ar[r,"a"]\ar[d,"\sim" labl, "b'\circ a"'] & x \ar[r,"b"]\ar[d,equal] & y\ar[d,"\sim" labl,"b'\circ a"'] \\
		y' \ar[r,"a'"] & x \ar[r,"b'"] & y'
	\end{diagram*}
	In particular, \(y\cong y'\). In other words, splittings of idempotents into retractions are unique.
\end{lemma}
\begin{proof}
The first assertion is just a computation: since, by assumption, we have \(e = a\circ b = a'\circ b'\), we may write
\[ b'\circ a \circ b \circ a' = b'\circ e\circ a' = b'\circ a'\circ b'\circ a' = \id_{y'}. \]
The other composition is similar (indeed, just swap the roles of the morphisms). That the diagram commutes is verified in an essentially identical manner.
\end{proof}

What we are looking for is a condition which ensures that the retract of a representable functor is representable. As it turns out, the above uniqueness in splitting an idempotent into
a retraction means that the splitting condition itself can be exploited for precisely this purpose:
\begin{lemma}\label{lemma:retract-of-representable-is-representable-if-idempotent-splits}
	Let \(\calC\) be a locally small category, and let \(F\!:\calC^\op\to\Set\) be a functor. Suppose that \(F\) is the retract of a representable functor
	\[ F \overset{a}\to h_x \overset{b}\to F. \]
	If the idempotent \(e\!:x\to x\) given by applying the Yoneda lemma to
	\[ a\circ b\!:h_x\to h_x \]
	splits into a retraction, then \(F\) is representable.
\end{lemma}
\begin{proof}
If the idempotent \(e\!:x\to x\) splits into a retraction
\[ y \overset{i}\to x \overset{p}\to y, \]
then we get a second splitting of \(a\circ b\) into a retraction
\[ h_{y} \to h_x \to h_{y} \]
which by the uniqueness in Lemma \ref{lemma:idempotent-splitting-uniqueness} means that \(A\cong h_{y}\).
\end{proof}

Clearly, in some generic category, not all idempotents split into retractions. However, there are conditions which ensure that a given idempotent does (in fact, which completely characterize this). We will need to
know one of them in order to justify a definition.
\begin{proposition}\label{prop:idempotents-split-iff-equalizer-exists}
	Let \(\calC\) be a category, and let \(e\!:x\to x\) be an idempotent in \(\calC\). Then the following are equivalent:
	\begin{enumerate}[label=(\roman*)]
	\item The idempotent \(e\) splits into a retraction.
	\item The equalizer
	\begin{diagram*}
		\eq(e,\id_x) \ar[r,hook] & x\ar[r,shift left,"e"]\ar[r,shift right,"\id_c"'] & x
	\end{diagram*}
	exists.
	\item The coequalizer
	\begin{diagram*}
		x\ar[r,shift left,"e"]\ar[r,shift right,"\id_x"'] & x \ar[r, two heads] & \coeq(e,\id_x)
	\end{diagram*}
	exists.
	\end{enumerate}
\end{proposition}
\begin{proof}
The only case we will care about is (ii) \(\Rightarrow\) (i).

Suppose we are given the existence of the equalizer. Then, since \(e^2 = e = e\circ\id\), we get an induced morphism
\begin{diagram*}
	& x\ar[d,"e"]\ar[dl,dashed,"p"'] & \\
	\eq(e,\id_x) \ar[r,hook,"i"] & x\ar[r,shift left,"e"]\ar[r,shift right,"\id_x"'] & x.
\end{diagram*}
The desired retraction is now
\begin{diagram*}
	\eq(e,\id_x) \ar[r,hook,"i"] & x \ar[r,"p"] & \eq(e,\id_x).
\end{diagram*}
Indeed, by definition, we have \(p\circ i = e\). Furthermore, by definition of the equalizer and since \(i\) is always a monomorphism, we have
\[ i\circ (p \circ i) = e\circ i = i \implies p\circ i = \id \]
as desired.
\end{proof}
\begin{exercise}
	Prove the rest of Proposition \ref{prop:idempotents-split-iff-equalizer-exists}.
\end{exercise}

\begin{proposition}\label{prop:idempotent-completion}
	Let \(\calC\) be a locally small category, and let \(\overline{\calC}\) denote the smallest full subcategory of \(\Fun(\calC^\op,\Set)\) which is closed under retracts and contains the representable functors.
	Then every idempotent in \(\overline{\calC}\) splits into a retraction, and the Yoneda embedding
	\[ \calC\inj\overline{\calC} \]
	is an equivalence if and only if every idempotent in \(\calC\) splits into a retraction.
\end{proposition}
\begin{proof}
That every idempotent in \(\overline{\calC}\) splits into a retraction is clear since they split in \(\Fun(\calC^\op,\Set)\) by Proposition \ref{prop:idempotents-split-iff-equalizer-exists} and since
the retract of a retract of a representable functor is a retract of a representable functor. For the second statement: clearly, if the functor is an equivalence, every idempotent in \(\calC\) splits;
conversely, if every idempotent in \(\calC\) splits, then given some retraction
\[ A \to h_x \to A \]
we may apply Lemma \ref{lemma:retract-of-representable-is-representable-if-idempotent-splits} to see that \(A\) is representable.
\end{proof}

\begin{definition}
	Let \(\calC\) be a category. We say \(\calC\) is \emph{idempotent complete} if every idempotent in \(\calC\) splits into a retraction. If \(\calC\) is locally small, we call
	\(\overline{\calC}\) the \emph{idempotent completion} (alternatively, the \emph{Cauchy} completion) of \(\calC\), so that \(\calC\) is idempotent complete if and only if it is equivalent to its
	idempotent completion.
\end{definition}

Together with the intuition from Lemma \ref{lemma:retract-of-representable-is-representable-if-idempotent-splits}, Proposition \ref{prop:idempotent-completion} justifies the above definition as reasonable.
Furthermore, they make it clear that this is exactly the condition we were looking for as a nice way of formulating when retracts of representables are representable.

\begin{proposition}
	Let \(\calC\) be a locally small category, and let \(\kappa\) be an infinite regular cardinal. If \(\calC\) is idempotent complete, then the embedding
	\[ \calC\inj\Ind_\kappa(\calC) \]
	induces an equivalence \(\calC\simeq\Ind_\kappa(\calC)^\kappa\) between \(\calC\) and the \(\kappa\)-compact objects of \(\Ind_\kappa(\calC)\). In particular, one obtains an equivalence
	\[ \Ind_\kappa(\calC) \simeq \Ind_\kappa(\Ind_\kappa(\calC)^\kappa). \]
\end{proposition}
\begin{proof}
By Proposition \ref{prop:ind-compact-objects-are-retracts-of-representables}, an object \(A\) of \(\Ind_\kappa(\calC)\) is \(\kappa\)-compact if and only if it is the retract of a representable functor.
Applying Lemma \ref{lemma:retract-of-representable-is-representable-if-idempotent-splits}, we see that \(A\) is representable if and only if \(A\) is \(\kappa\)-compact.
\end{proof}


\subsection{Appendix: Filtered colimits}
Here we give a definition which, in a sense, captures the notion of taking ``unions'' in a category.
\begin{definition}
	Let \(\kappa\) be a small infinite regular cardinal. We say a small category \( I \) is \(\kappa\)\emph{-filtered} if any diagram \( J\to I\) for which \(|{\Mor( J)}| < \kappa\) has a cocone, i.e.\ admits
	an extension
	\begin{diagram*}
		 J\ar[r]\ar[d] &  I \\
		 J^\triangleright \ar[ur]
	\end{diagram*}
	We say \( I\) is \emph{filtered} if it is \(\aleph_0\)-filtered.
	We say that a category \(\calC\) \emph{admits} \(\kappa\)\emph{-filtered colimits} if every diagram \( I\to\calC\) where \( I\) is \(\kappa\)-filtered has a colimit.
\end{definition}
\begin{remark}
	When \(\kappa=\aleph_0\), \(\kappa\)-filteredness can be phrased as follows:
	\begin{enumerate}[label=(\alph*)]
	\item \( I\) is non-empty.
	\item For all pairs of objects \(i,j\in I\), there is an object \(k\in I\) and morphisms \(i\to k\), \(j\to k\), as in the diagram
	\begin{diagram*}
		& k & \\
		i\ar[ur,dashed] & & j \ar[ul,dashed]
	\end{diagram*}
	\item For any two parallel morphisms \(f,g\!:i\to j\), there is an object \(k\in I\) and a morphism \(h\!:j\to k\) such that \(h\circ f = h\circ g\), as in the diagram
	\begin{diagram*}
		& k & \\
		i\ar[ur,dashed]\ar[rr,shift left,"f"]\ar[rr,shift right,"g"'] & & j \ar[ul,dashed,"h"']
	\end{diagram*}
	\end{enumerate}
	To prove this, one must simply check that these conditions are enough to build the desired extensions of any given finite diagram in \( I\), and this is not so hard.
\end{remark}
\begin{example}
	Any category which has a final object is \(\kappa\)-filtered for all \(\kappa\).
\end{example}
\begin{example}
	Any category \( I\) which admits \(\kappa\)-small colimits, i.e.\ colimits indexed by categories \( J\) with \(|{\Ob( J)}| < \kappa\), is \(\kappa\)-filtered.
\end{example}
\begin{example}\label{example:poset-filtered}
	A partially ordered set \((P,\leq)\), when regarded as a category, is \(\kappa\)-filtered if and only if \(P\) is \(\kappa\)-directed. A partially ordered set is \(\kappa\)-directed
	if any collection \(\{ x_{s} \}_{s\in S}\) indexed by some set \(S\) with \(|S| < \kappa\) has an upper bound, i.e.\ there exists an element \(x_\infty\in P\) such that \(x_s \leq x_\infty\) for all \(s\in S\).
\end{example}
\begin{example}
	Let \(\kappa\) be an infinite regular cardinal. As special cases of Example \ref{example:poset-filtered}, we have:
	\begin{enumerate}[label=(\alph*)]
	\item The partially ordered set \(\{\beta \mid \beta < \kappa\}\) is \(\kappa\)-filtered as a category.
	\item The partially ordered set \(\{ S\in 2^\kappa \mid |S| < \kappa \}\) is \(\kappa\)-filtered as a category. In particular, note that for any collection of \(\kappa\)-small
	subsets of \(\kappa\) indexed by a \(\kappa\)-small set, taking the union yields yet another \(\kappa\)-small subset by the regularity of \(\kappa\).
	\item Let \(X\) be a topological space. Then the partially ordered set of open subsets of \(X\) is \(\kappa\)-filtered.
	\end{enumerate}
\end{example}
\begin{example}
	Let \(\kappa\) be an infinite regular cardinal, let \(S\) be a set (as large as we want), and let \(\{  I_s \}_{s\in\calS}\) be an \(S\)-indexed collection
	of \(\kappa\)-filtered categories. Then the product
	\[ \prod_{s\in S} I_s \]
	is \(\kappa\)-filtered.
\end{example}
\begin{remark}\label{remark:filteredness-is-ordered}
	Let \(\kappa' < \kappa\) be two infinite regular cardinals. If \( I\) is \(\kappa\)-filtered, then it is also \(\kappa'\)-filtered.
\end{remark}

\begin{proposition}
	Consider a \(\kappa\)-filtered category \( I\) and a diagram \(D\!: I\to\Set\) in the category of small sets. We may then compute the colimit as
	\[ \injlim_{i\in I}D(i)\cong\left(\coprod_{i\in I}D(i)\right)/\sim \]
	where \(\sim\) is the equivalence relation given by \( (i, d) \sim (i',d') \) if and only if there is some \(k\in I\) and \(f\!:i\to k\), \(g\!:i'\to k\) such that \(D(f)(d) = D(g)(d')\).
\end{proposition}
\begin{proof}
First, a point of emphasis: it is non-trivial that the described relation is an equivalence relation, and this is really the non-trivial part of the proof. Supposing it is, that this
is the colimit follows quite easily. Thus, we show that \(\sim\) is an equivalence relation. Furthermore, by Remark \ref{remark:filteredness-is-ordered}, it suffices to show this for \(\kappa = \aleph_0\).
\begin{itemize}[label=\(\star\)]
\item (Reflexivity) For all \((i,d)\), we have \((i,d)\sim(i,d)\). This follows by noting that \(\id_i\!:i\to i\) does the job.
\item (Symmetry) Suppose that \((i,d)\sim (i',d')\). Clearly \((i',d')\sim (i,d)\) by swapping the roles of the data showing the former.
\item (Transitivity) Suppose that \((i,d)\sim (i',d')\) is exhibited by \(f\!:i\to k\), \(g\!:i'\to k\), and \((i',d') \sim (i'', d'')\) is exhibited by \(f'\!:i'\to k'\), \(g'\!:i''\to k'\).
Extend the data using that \( I\) is filtered:
\begin{diagram*}
	 & & k'' & &\\
	 & k\ar[ur,dashed,"h"] & & k'\ar[ul,dashed,"h'"'] & \\
	i\ar[ur,"f"] & & i'\ar[ul,"g"']\ar[ur,"f'"] & & i''\ar[ul,"g'"']
\end{diagram*}
and note that
\[ D(h\circ f)(d) = D(h\circ g)(d') = D(h'\circ f')(d') = D(h'\circ g')(d'') \]
so that \((i,d)\sim (i'',d'')\) as desired.
\end{itemize}
This completes the proof.
\end{proof}
\begin{remark}
	For an arbitrary colimit, the above argument fails, since the proof of transitivity crucially relies on the diagram being filtered. However, one can still compute the colimit by taking the equivalence relation
	\emph{generated} by the given relation.
\end{remark}

\begin{construction}
	Let \(\calC\), \(I\) and \(J\) be categories, and consider a diagram \(D:I\times J \to \calC\). From this, we can extract two other diagrams. In particular, there are isomorphisms
	\[ \Fun(J,\Fun(I,\calC)) \cong \Fun(I\times J,\calC)\cong \Fun(I,\Fun(J,\calC)) \]
	from which we obtain diagrams
	\[ D'\!:J \to \Fun(I,\calC),\quad D''\!:I\to\Fun(J,\calC) \]
	corresponding to \(D\). Explicitly, \(D'\) is given on objects by \(D'(j) = D(-,j)\), and \(D''\) is given by \(D''(i) = D(i,-)\). Suppose \(\calC\) admits \(I\)-shaped limits and \(J\)-shaped colimits.
	We may then have functors
	\[ \projlim\!:\Fun(I,\calC)\to\calC,\quad\injlim\!:\Fun(J,\calC)\to\calC. \]
	Consider the colimit of \(D'\). We have
	\begin{center}\(\forall (j\to j')\in J,\quad\)
	\begin{tikzcd}
		D'(j)\ar[r]\ar[d] & \injlim{D'} \\
		D'(j')\ar[ur]
	\end{tikzcd}\(\quad\leadsto\quad\)
	\begin{tikzcd}
		\projlim{D'(j)}\ar[r]\ar[d] & \projlim(\injlim{D'}) \\
		\projlim{D'(j')}\ar[ur]
	\end{tikzcd}
	\end{center}
	but also
	\begin{diagram*}
		\projlim{D(-,j)}\ar[r,equal]\ar[d] & \projlim{D'(j)}\ar[r] & \projlim(\injlim{D'}) \\
		\injlim_J\projlim_I{D(-,j)}\ar[r,equal] & \injlim(\projlim{D''})\ar[ur,dashed]
	\end{diagram*}
	so we get a canonical map
	\[ \injlim(\projlim{D''})\to\projlim(\injlim{D'}). \]
\end{construction}

\begin{definition}
	We say that \(I\)-shaped limits and \(J\)-shaped colimits \emph{commute} in \(\calC\) if for all \(D\!:I\times J\to\calC\), the canonical morphism
	\[ \injlim_{j\in J}\projlim_{i\in I}{D(i,j)}\to\projlim_{i\in I}\injlim_{j\in J}{D(i,j)} \]
	is an isomorphism.
\end{definition}

The following is a fundamental result about \(\kappa\)-filtered colimits in the category of sets, and it is one of the major reasons that \(\kappa\)-filtered colimits
are useful.
\begin{theorem}\label{thm:filtered-colimits-commute-with-finite-limits-in-set}
	Let \(\kappa\) be an infinite regular cardinal. Then \(\kappa\)-filtered colimits commute with \(\kappa\)-small limits in \(\Set\). More precisely, let \(I\) be a \(\kappa\)-small category, let \( J\) be a \(\kappa\)-filtered category, and
	consider a diagram \(D\!: I\times J\to\Set\). Then the canonical map
	\[ \injlim_{j\in J}\projlim_{i\in I} D(i,j) \to \projlim_{i\in I}\injlim_{j\in J} D(i,j) \]
	is an isomorphism (i.e.\ a bijection).
\end{theorem}
\begin{proof}
Fundamentally, this is just a computation. One computes the left and right sets, and deduce from their explicit forms that they are isomorphic. First of all, for a fixed \(j\in J\),
we can compute the limit as
\[ \projlim_{i\in I}D(i,j) \cong \left\{(x_i)_{i\in I}\in\prod_{i\in I}D(i,j) \bigmid \forall(\varphi\!:i\to i')\in I,\, D(\varphi,j)(x_i) = x_{i'} \right\}. \]
Now, as \(J\) is filtered, we can therefore compute the colimit as
\[ \injlim_{j\in J}\projlim_{i\in I}D(i,j) \cong \left(\coprod_{j\in J}\left\{(x_i)_{i\in I}\in\prod_{i\in I}D(i,j) \bigmid \forall(\varphi\!:i\to i')\in I,\, D(\varphi,j)(x_i) = x_{i'} \right\}\right)/\sim \]
where \(\sim\) identifies \((x_{i})_{i\in I}\in\projlim_{i\in I}D(i,j)\) with \((y_i)_{i\in I}\in\projlim_{i\in I}D(i,j')\) if and only if there are some
\[ \psi\!:j\to j'',\quad\psi'\!:j'\to j''\]
such that
\[ \forall i\in I,\quad D(i,\psi)(x_i) = D(i,\psi')(y_i). \]
This computes one side. For the other side, again since \(J\) is filtered, for a fixed \(i\in I\) we may compute the colimit as
\[ \injlim_{j\in J}D(i,j) \cong \left(\coprod_{j\in J}D(i,j)\right)/\simeq_i \]
where the equivalence relation is defined by \(D(i,j)\ni x \simeq_i y\in D(i,j')\) if and only if there is some \(\phi\!:j\to j''\) and \(\phi'\!:j'\to j''\) such that \( D(i,\phi)(x) = D(i,\phi')(y). \)
Now we can take the limit to get
\[ \projlim_{i\in I}\injlim_{j\in J}D(i,j) \cong \left\{ ([j_i,x_i])_{i\in I} \in\prod_{i\in I} \left(\coprod_{j\in J}D(i,j)\right)/\simeq_i \bigmid \forall(\varphi\!:i\to i'),\, [j_i,D(\varphi,j_i)(x_i)] = [j_{i'},x_{i'}] \right\} \]
where the canonical map can now be given by
\[ \alpha\!:\injlim_{j\in J}\projlim_{i\in I}D(i,j) \to \projlim_{i\in I}\injlim_{j\in J}D(i,j),\quad [j,(x_i)_{i\in I}] \mapsto ([j,x_i])_{i\in I}. \]
One may wish to sanity-check that this is well-defined (and this is not hard). Now it is just a matter of checking that \(\alpha\) is injective and surjective, which
is where one critically uses the assumption that \(I\) is \(\kappa\)-small and \(J\) is \(\kappa\)-filtered (though we have already partly used the latter in order to compute the colimit).

We construct an inverse \(\beta\) to \(\alpha\). Hence, consider an element \(([j_i,x_i^0])_{i\in I}\in\projlim_{i\in I}\injlim_{j\in J}D(i,j)\). By the assumption that \(I\) is \(\kappa\)-small
and \(J\) is \(\kappa\)-filtered, the elements \(j_i\in J\), \(i\in I\) have a common join \(j\in J\) with morphisms \(\phi_i\!:j_i\to j\). We set \(x_i = D(i,\phi_i)(x_i^0)\), so we have
\[ \forall i\in I,\quad [j_i,x_i^0] = [j,x_i],\quad \text{i.e.\ } x_i^0 \simeq_i x_i. \]
It follows that \(([j_i,x_i^0])_{i\in I} = ([j,x_i])_{i\in I}\). We define \(\beta\) by
\[ \beta\!: ([j_i,x_i^0])_{i\in I} = ([j,x_i])_{i\in I} \mapsto [j,(x_i)_{i\in I}]. \]
We need to show that this is well-defined. If we have \(([j,x_i])_{i\in I} = ([j',y_i])_{i\in I}\), then we find \(\psi_i\!:j\to j''_0\), \(\psi_i'\!:j'\to j''_0\) such that
\(D(i,\psi_i)(x_i) = D(i,\psi_i')(y_i)\). However, since \(J\) is \(\kappa\)-filtered, we may find a cone for the diagram given by the \(\psi_i\), \(\psi_i'\), which will provide a morphism
\(\phi\!:j''_0 \to j'' \) which equalizes all the \(\psi_i\) and all the \(\psi_i'\). We set \(\psi = \phi\circ\psi_i\), \(\psi' = \phi\circ\psi_i'\). Then \(D(i,\psi)(x_i) = D(i,\psi')(y_i)\)
so that \([j,(x_i)_{i\in I}] = [j',(y_i)_{i\in I}]\).

Now we need to ensure that the \((x_i)_{i\in I}\) coming from \(\beta\) can be picked to actually form an element of \(\projlim_{i\in I}D(i,j)\). By assumption, we have that
\begin{align*}
	\forall (\varphi\!:i\to i'),\hspace{3cm} [j, D(\varphi,j)(x_i)] &= [j,x_{i'}] \\
	\leadsto\quad \forall (\varphi\!:i\to i'),\, \exists (\phi^{\varphi}_1,\phi^{\varphi}_2\!: j\to j'),\quad (D(i,\phi^\varphi_1)\circ D(\varphi,j))(x_i) &= D(i,\phi^\varphi_2)(x_{i'}).
\end{align*}
However, again, since \(I\) is \(\kappa\)-small (hence has fewer than \(\kappa\) arrows) and \(J\) is \(\kappa\)-filtered, we can take a cone for the diagram given by the \(\phi^\varphi_k\)
to get a morphism \(\psi\!:j'\to j''\) for which \(\psi\circ\phi^\varphi_1 = \psi\circ\phi^\varphi_2\) for all \(\varphi\). Let \(\phi = \psi\circ\phi_1^\varphi\). We see that
\[ ([j,x_i])_{i\in I} = ([j'',D(i,\phi)(x_i)])_{i\in I} \]
and
\[ \forall (\varphi\!:i\to i'),\quad (D(\varphi,j'')\circ D(i,\phi))(x_i) = (D(i,\phi)\circ D(\varphi,j''))(x_i) = D(i,\phi)(x_{i'}) \]
so that \((D(i,\phi)(x_i))_{i\in I}\in\projlim_{i\in I}D(i,j)\). We conclude that \(\beta\) can be made into a well-defined function, and it is quite clear that it
is inverse to \(\alpha\).
\end{proof}

We can extend this slightly, using the results we've proven about reflective subcategories.
\begin{definition}
	A \emph{Grothendieck topos} is a Giraud subcategory of a presheaf category (with values in \(\Set\)).
\end{definition}
\begin{corollary}
	Let \(\calC\) be a Grothendieck topos. Then \(\calC\) admits all small limits and colimits, and filtered colimits commute with finite limits in \(\calC\).
\end{corollary}
\begin{proof}
Let \(\iota\!:\calC\inj\PSh(\calC_0)\) be the inclusion, with reflector \(\pi\!:\PSh(\calC_0)\to\calC\). From Theorem \ref{thm:filtered-colimits-commute-with-finite-limits-in-set}, we know that filtered colimits
commute with finite limits in \(\Set\), and hence this is also true in \(\PSh(\calC_0)\) since (co)limits are computed pointwise. By Theorem \ref{thm:limits-and-colimits-in-reflective-subcategories}, all
limits and colimits that \(\PSh(\calC_0)\) admits, i.e.\ all small limits and colimits, are also admitted by \(\calC\), and the diagram
\begin{diagram*}
	\Fun(J,\calC)\ar[r,"\injlim"]\ar[d,"\iota\circ"] & \calC \\
	\Fun(J,\PSh(\calC_0))\ar[r,"\injlim"] & \PSh(\calC_0)\ar[u,"\pi"]
\end{diagram*}
commutes (potentially up to natural isomorphism). By assumption, when \(J\) is filtered, all the functors preserve finite limits, so we are done.
\end{proof}
\begin{remark}
	Of course, if one can arrange for the reflector \(\pi\) to commute with larger limits, then one can deduce that more filtered colimits commute with certain limits.
\end{remark}
\begin{remark}
	The definition of a Grothendieck topos given above is only one of many possible definitions. We have chosen the one that fits us best in this situation, but in general
	one can characterize them
	\begin{enumerate}[label=(\arabic*)]
	\item in another \emph{concrete way,} as sheaves on sites, or
	\item \emph{abstractly,} as categories satisfying the \emph{Giraud axioms.}
	\end{enumerate}
	We may cover this more thoroughly in another lecture.
\end{remark}


%!TEX root = ../lectures.tex

\section{More about Ind-categories \& compact generation}
This lecture has a few goals. Previously, we gave a sort of ``extrinsic'' definition of Ind-categories, and relied upon Lemma \ref{lemma:ind-category-admits-filtered-colimits} in our dealings with it without
justification. We want to justfy this lemma, and also provide a more ``intrinsic'' perspective on Ind-categories. The first thing we want is a more canonical way to detect if a presheaf \(A\) lives in an Ind-category.
The definition we gave required that
\[ A \cong \injlim{D} \]
where \(D\) is some small (\(\kappa\)-)filtered diagram of representable functors. It turns out that there is always a canonical way to represent
a presheaf as a colimit, and that this representation can be used to characterize when \(A\in\Ind_\kappa(\calC)\).

Another major goal is to explain a phenomenon whereby a category may be \emph{generated} under filtered colimits by compact objects. Consider the following motivating sketches.
\begin{example}
	Trivially, \(\Set \cong \PSh(*)\), expressing the fact that the category of (small) sets is freely generated under (small) colimits by a point. Indeed, one only needs small coproducts.
	On the other hand, one also has \(\Set \cong \Ind(\cat{Fin})\), expressing that the category of sets is freely generated by finite sets under \emph{filtered colimits.} This was implicitly
	used in the proof computing the compact objects of \(\Set\); there, we noted that any set is the filtered colimit of its poset of finite subsets.
\end{example}
\begin{example}
	Let \(\Bbbk\) be a field. Then
	\[ \Vect_\Bbbk \cong \Ind(\Vect_\Bbbk^\fd). \]
	This follows by noting that any \(\Bbbk\)-vector space can be written as the union of its finite-dimensional subspaces.
\end{example}

In order to justify the above examples, our goal is to prove that any category admitting some small set of compact objects which generate the category under filtered colimits (in
the intuitive sense) can be written as the Ind-category of those compact objects.

\subsection{Category of elements}
\begin{definition}
	Let \(\calC\) be a locally small category, and let \(A\in\PSh(\calC)\). The \emph{category of elements} \(\calC/A\) of \(A\) is given by:
	\begin{itemize}[label=\(\star\)]
	\item An object of \(\calC/A\) is a tuple \((x,a)\) where \(x\in\calC\) and \(a\in Ax\).
	\item A morphism \((x,a)\to(x',a')\) is a morphism \(f\!:x\to x'\) such that \((Af)(a') = a\), i.e.\ \(f^*a' = a\).
	\end{itemize}
\end{definition}
\begin{remark}
	One easily sees that the category of elements \(\calC/A\) can also be described as follows: an object is a pair \((x,a)\) where \(x\in\calC\) and
	\(a\!:h_x\to A\); a morphism \((x,a)\to(x',a')\) is a morphism \(f\!:x\to x'\) such that the diagram
	\begin{diagram*}[cramped,column sep=small]
		h_x \ar[rr,"f_*"]\ar[dr,"a"'] & & h_{x'}\ar[dl,"a'"] \\
		& A & 
	\end{diagram*}
	commutes. This follows by the Yoneda lemma.
\end{remark}
\begin{remark}
	There is a ``projection'' functor, which we consider canonical, of type
	\[ \pi_A\!:\calC/A \to \calC, \]
	given by sending \((x,a)\) to \(x\). Observe that for each \(x\in\calC\), the ``fiber of \(\pi_A\) over \(x\)'' is exactly \(Ax\). More precisely, consider the subcategory
	of \(\calC/A\) spanned by those objects of the form \((x,a)\), and those arrows which map to \(\id_x\). This is a discrete category, whose underlying set of objects is just \(Ax\).
\end{remark}
\begin{notation}
	Let \(\calC\) be a category. We denote by
	\[ \yo = \yo_{\calC}\!:\calC\inj\PSh(\calC),\quad x\mapsto \yo(x) := h_x \]
	the Yoneda embedding.
\end{notation}
\begin{lemma}\label{lemma:presheaves-are-colimits-of-representable-functors}
	Let \(\calC\) be a category, and let \(A\) be a presheaf on \(\calC\). Then
	\[ A = \injlim(\yo_{\calC}\circ\pi_A). \]
	Colloquially,
	\[ A = \injlim_{h_x\to A}h_x. \]
\end{lemma}
\begin{proof}
Observe that we trivially have a collection of morphisms \(\{a\!:h_x\to A\}_{(x,a)\in\calC/A}\) compatible with the morphisms in \(\calC/A\). In other words,
we have a cone under \(\yo_{\calC}\circ\pi_A\) with tip \(A\). Suppose we have another such cone \(\{b_{x,a}\!:h_x\to B\}_{(x,a)\in\calC/A}\). By the Yoneda lemma,
this uniquely induces a morphism \(\alpha\!:A\to B\) for which the diagram
\begin{diagram*}[cramped, column sep=small]
	& h_x\ar[dl,"a"']\ar[dr,"b_{x,a}"] & \\
	A\ar[rr,"\alpha"] & & B
\end{diagram*}
commutes. In particular, the natural transformation \(\alpha\) is given by \(\alpha_x(a) := b_{x,a}\), where we identify the morphisms with their
corresponding elements under the Yoneda lemma. It follows that \(A\) forms a colimit of \(\yo_{\calC}\circ\pi_A\), as desired.
\end{proof}
\begin{corollary}
	Let \(\calC\) be a category, and let \(x\in\calC\). Then, for all \(z\in\calC\), we have a natural isomorphism
	\[ \calC(z,x) \cong \injlim_{(y\to x)\in\calC/x}\calC(z,y). \]
\end{corollary}
\begin{exercise}\label{exercise:presheaves-on-slice-is-slice-of-presheaves}
	Let \(\calC\) be a locally small category, and let \(A\in\PSh(\calC)\). Show that there is an equivalence \(\PSh(\calC)/A \simeq \PSh(\calC/A)\) such that the diagram
	\begin{diagram*}
		\calC/A\ar[r]\ar[dr,hook,"\yo"'] & \PSh(\calC)/A\ar[d,"\sim" labl] \\
		& \PSh(\calC/A)
	\end{diagram*}
	commutes, where the functor \(\calC/A\to\PSh(\calC)/A\) is the canonical inclusion induced by the Yoneda embedding \(\yo_{\calC}\!:\calC\inj\PSh(\calC)\).
\end{exercise}

\begin{lemma}\label{lemma:category-of-elements-induced-colimits}
	Let \(\calC\) be a category, and assume that \(\calC\) admits \(I\)-shaped colimits.
	Consider a presheaf \(A\in\PSh(\calC)\). If \(A\) commutes with \(I\)-shaped limits, so
	that \(A(\injlim{D}) \cong \projlim(A\circ D^\op)\) for all diagrams \(D\!:I\to\calC\), then \(\calC/A\) admits \(I\)-shaped colimits and \(\pi_A\!:\calC/A\to\calC\) preserves these colimits.
\end{lemma}
\begin{proof}
Consider a diagram \(D\!:I\to\calC/A\), \(i\mapsto (D(i), u_i\in A(D(i)))\). Since \(\calC\) admits \(I\)-shaped colimits, the colimit of \(\pi_A\circ D\) exists. Furthermore,
since \(A\) commutes with \(I\)-shaped limits, we have
\[ A(\injlim_{i\in I}D(i)) \cong \projlim_{i\in I}{A(D(i))}. \]
The elements \(u_i\) then provide an element of the latter, hence we get an element \(u\in A(\injlim(\pi_A\circ D))\). Thus, we can lift the colimit in \(\calC\) to an element
\[ (\injlim(\pi_A\circ D), u)\in\calC/A. \]
One easily sees that this is a colimit of \(D\).
\end{proof}

\begin{proposition}\label{prop:filtered-category-of-elements-iff-preserves-small-limits}
	Let \(\calC\) be a category admitting \(\kappa\)-small colimits, and let \(A\in\PSh(\calC)\). Then the following are equivalent:
	\begin{enumerate}
	\item \(\calC/A\) is \(\kappa\)-filtered.
	\item \(A\) preserves \(\kappa\)-small limits.
	\end{enumerate}
	In particular, \(A\) is left exact if and only if \(\calC/A\) is filtered.
\end{proposition}
\begin{proof}
Assume (1), and let \(D\!:I\to\calC\) be a \(\kappa\)-small diagram. Then
\begin{align*}
	A(\injlim{D}) &= \injlim(\yo_{\calC}\circ\pi_A)(\injlim{D})\\
	&= \injlim_{h_x\to A}\calC(\injlim_{i\in I}{D(i)},x)\\
	&= \injlim_{h_x\to A}\projlim_{i\in I}\calC(D(i),x)\\
	&\cong \projlim_{i\in I}\injlim_{h_x\to A}\calC(D(i),x) = \projlim_{i\in I}A(D(i))
\end{align*}
so that (2) holds. Here, we used that \(\kappa\)-filtered colimits commute with \(\kappa\)-small limits in \(\Set\).

Conversely, assume (2). Then \(\calC/A\) admits \(\kappa\)-small colimits by Lemma \ref{lemma:category-of-elements-induced-colimits}. But then \(\calC/A\) is \(\kappa\)-filtered: indeed, given a \(\kappa\)-small diagram \(I\to \calC/A\),
the colimit exists and hence gives rise to an extension \(I^\triangleright\to\calC/A\).
\end{proof}

\subsection{Cofinality \& an intrinsic characterization of Ind-categories}
\begin{definition}
	Let \(\varphi\!:I\to J\) be a functor. We say \(\varphi\) is \emph{cofinal} (or \emph{final}) if for any diagram \(D\!:J\to\calC\) in a category \(\calC\), the canonical comparison map
	\[ \injlim(D\circ\varphi)\to\injlim{D} \]
	is an isomorphism. We say \(J\) is cofinally small if there is a cofinal functor \(I\to J\) where \(I\) is small.
\end{definition}

While cofinality is an important concept, and widely used, we do not want to spend too much time on it here, as it can be rather technical.

\begin{proposition}\label{prop:cofinal-equivalent-conditions}
	Let \(\varphi\!:I\to J\) be a functor. The following are equivalent:
	\begin{enumerate}[label=(\arabic*)]
	\item \(\varphi\) is final.
	\item For all diagrams \(D\!:J\to\Set\), the canonical comparison map
	\[ \injlim(D\circ\varphi)\to\injlim{D} \]
	is an isomorphism.
	\item For all \(j\in J\), the comma category \(j/\varphi\) is connected, i.e.\ it is non-empty, and every two objects are connected by a zigzag of morphisms.
	\item For all \(j\in J\), we have \(\injlim_{i\in I}J(j,\varphi(i)) \cong *\).
	\end{enumerate}
\end{proposition}
\begin{proof}
See \cite[Prop.\ 2.5.2]{kashiwara-schapira-book}.
\end{proof}
\begin{remark}
	The comma category \(j/\varphi\) has as objects pairs \((i,t)\) where \(i\in I\) and \(t\!:j\to\varphi(i)\), and a morphism \((i,t)\to(i',t')\) is a morphism \(s\!:i\to i'\) such that \(t' = \varphi(s)\circ t\).
\end{remark}

\begin{construction}
	Consider a locally small category \(\calC\) and a small diagram \(D\!:I\to\calC\). We can form the formal colimit of \(D\) by taking the colimit
	\[ A := \injlim(\yo_{\calC}\circ D) \]
	in the category of presheaves. Let \(h_i\!:h_{D(i)}\to A\) be the canonical morphism. This induces a new diagram
	\[ \tilde{D}\!:I\to\calC/A \]
	given by \(i\mapsto (D(i), h_i)\) on objects, and \((t\!:i\to i')\mapsto Dt\) on morphisms.
\end{construction}

\begin{lemma}\label{lemma:induced-diagram-cofinal}
	Let \(\calC\) be a locally small category, and let \(D\!:I\to\calC\) be a small diagram. Then the induced diagram \(\tilde{D}\!:I\to\calC/A\) is a cofinal
	functor.
\end{lemma}
\begin{proof}
By (the same argument as) Lemma \ref{lemma:category-of-elements-induced-colimits}, the functor \(\yo/A\!:\PSh(\calC)/A\to\PSh(\calC)\) commutes with small colimits. Observe that we have
a commutative diagram
\begin{diagram*}
	I\ar[r,"\tilde{D}"]\ar[dr,"D"'] & \calC/A\ar[d,"\pi_A"]\ar[r,"\yo/A"] & \PSh(\calC)/A\ar[d] \\
	& \calC\ar[r,"\yo"] & \PSh(\calC)
\end{diagram*}
In particular, the underlying presheaf of \(\injlim(\yo/A\circ\tilde{D})\) is \(A\). One checks, using the equivalence \(\PSh(\calC/A)\simeq\PSh(\calC)/A\) of
Exercise \ref{exercise:presheaves-on-slice-is-slice-of-presheaves}, that the structure map is just the identity \(\id_A\!:A\to A\), so that \(\injlim(\yo/A\circ\tilde{D})\)
is the terminal object in \(\PSh(\calC)/A\). Using the characterization (4) in Proposition \ref{prop:cofinal-equivalent-conditions}, one sees that \(\tilde{D}\) is cofinal.
\end{proof}

% \begin{lemma}\label{lemma:cofinal-filtered-implies-filtered}
% 	Let \(\varphi\!:I\to J\) be a cofinal functor. If \(I\) is \(\kappa\)-filtered, then \(J\) is \(\kappa\)-filtered.
% \end{lemma}

\begin{theorem}\label{thm:ind-equivalent-characterizations}
	Let \(\calC\) be a locally small category, let \(\kappa\) be an infinite regular cardinal, and let \(A\in\PSh(\calC)\). Then the following are equivalent:
	\begin{enumerate}[label=(\arabic*)]
	\item \(A\in\Ind_\kappa(\calC)\).
	\item \(\calC/A\) is \(\kappa\)-filtered and cofinally small.
	\item \(A\) commutes with \(\kappa\)-small limits and \(\calC/A\) is cofinally small.
	\end{enumerate}
\end{theorem}
\begin{proof}
(3) implies (2) by Proposition \ref{prop:filtered-category-of-elements-iff-preserves-small-limits}. That (2) implies (1) is clear, by Lemma \ref{lemma:presheaves-are-colimits-of-representable-functors}.
Finally, (1) implies (3) by combining Lemma \ref{lemma:induced-diagram-cofinal} with the same argument as in Proposition \ref{prop:filtered-category-of-elements-iff-preserves-small-limits}.
\end{proof}

\begin{remark}
	If we drop the requirement that an Ind-object should be represented by a small diagram, then we can drop the cofinal smallness condition. Furthermore, if \(\calC\) is
	actually a \emph{small} category, rather than just locally small, then \(\calC/A\) too is small. Indeed, it is clearly locally small, and the objects can be realized
	as a disjoint union of small sets indexed by a small set. Thus, when \(\calC\) is small, the technicalities presented by cofinal smallness disappear.
\end{remark}

\begin{proof}[Proof of Lemma \ref{lemma:ind-category-admits-filtered-colimits}]
Let \(D\!:I\to\Ind_\kappa(\calC)\) be a small \(\kappa\)-filtered diagram. Let \(A\) be the colimit of \(D\) in \(\PSh(\calC)\).
We will use condition (2) in Theorem \ref{thm:ind-equivalent-characterizations} to show that \(A\in\Ind_\kappa(\calC)\); in doing this, we will neglect showing that \(\calC/A\) is
cofinally small, as it requires a very technical argument about the interplay between cofinality and \(\kappa\)-filteredness. A proof can be found in \cite[Thm.\ 6.1.8]{kashiwara-schapira-book}.

To see that \(\calC/A\) is \(\kappa\)-filtered, consider a diagram \(K\!:J\to\calC/A\) where \(J\) is \(\kappa\)-small. We need to find a cone under \(K\).
Let \(u_i\!:h_{D(i)}\to A\) be the canonical maps, and write \(\widehat{\calC} := \PSh(\calC)\). For any \((x,a)\in\calC/A\), we have
\begin{align*}
	(\widehat{\calC}/A)(h_x\to A,A\overset{\id}\to A)) &\cong \injlim_{i\in I}(\widehat{\calC}/A)(h_x\to A,h_{D(i)}\overset{u_i}\to A))\\
	&\cong \injlim_{i\in I}\injlim_{(y\to D(i))\in\calC/D(i)}\calC(x,y).
\end{align*}
Now, \(I\) is \(\kappa\)-filtered by assumption, and \(\calC/D(i)\) has a terminal object and in hence \(\kappa\)-filtered, and therefore
\begin{align*}
	\{*\} &\cong \projlim_{j\in J}(\widehat{\calC}/A)(K(j),A\overset{\id}\to A) \\
	&\cong \projlim_{j\in J}\injlim_{i\in I}\injlim_{(y\to D(i))\in\calC/D(i)}\calC(K(j),y) \\
	&\cong \injlim_{i\in I}\injlim_{(y\to D(i))\in\calC/D(i)}\projlim_{j\in J}\calC(K(j),y) \\
\end{align*}
so we find some \(i\in I\) and \(y_0\to D(i)\), which define an element \(y = (y_0, h_{y_0}\to h_{D(i)}\overset{u_i}\to A)\), for which
\[ \projlim_{j\in J}(\calC/A)(K(j),y) \not= \varnothing. \]
In other words, we can find a collection of morphisms forming a cone \(K\To y\) as desired.
\end{proof}

\subsection{Functoriality \& universal property of Ind}
Here, we should really be spelling out 2-categorical data. We will avoid doing this, however, and pretend that everything holds strictly for simplicity, as
otherwise the technical details riks obscuring the fundamental ideas, which are very simple.
\begin{proposition}
	Let \(\calC\) be a small category, \(\calD\) a locally small category, and let \(\kappa\) be an infinite regular cardinal. Consider a functor \(F\!:\calC\to\calD\). Then
	there is a unique functor \(\Ind_\kappa(F)\!:\Ind_\kappa(\calC)\to\Ind_\kappa(\calD)\) commuting with small \(\kappa\)-filtered colimits and for which the diagram
	\begin{diagram*}
		\calC\ar[r,"F"]\ar[d,hook,"\yo_{\calC}"'] & \calD \ar[d,hook,"\yo_{\calD}"] \\
		\Ind_\kappa(\calC)\ar[r,"\Ind_\kappa(F)"] & \Ind_\kappa(\calD)
	\end{diagram*}
	commutes.
\end{proposition}
\begin{proof}
Let \(A\in\Ind_\kappa(\calC)\). Then we have
\[  A = \injlim(\calC/A\overset{\pi_A}\longto\calC\overset{\yo_{\calC}}\longto\Ind_\kappa(\calC)). \]
Since \(\calC\) is small, \(\calC/A\) is also small. In particular, for \(\Ind_\kappa(F)\) to commute with small \(\kappa\)-filtered colimits and for the diagram to commute, we must have
\[ \Ind_\kappa(F)(A) = \injlim(\yo_{\calD}\circ F\circ\pi_A) \]
and for a morphism \(A\to B\) in \(\Ind_\kappa(\calC)\), the morphism \(\Ind_\kappa(F)(A)\to\Ind_\kappa(F)(B)\) must be the canonical morphism
\[ \injlim(\yo_{\calD}\circ F\circ\pi_A) \to \injlim(\yo_{\calD}\circ F\circ\pi_B). \]
So we define a unique functor by these requirements.
\end{proof}

The above functoriality statement makes it fairly easy to provide a universal property for \(\Ind_\kappa(\calC)\). It expresses that this category is the
formal cocompletion of \(\calC\) with respect to \(\kappa\)-filtered colimits.

\begin{lemma}
	Let \(\calC\) be a locally small category admitting small \(\kappa\)-filtered colimits for some infinite regular cardinal \(\kappa\).
	\begin{enumerate}[label=(\arabic*)]
	\item The functor \(\yo_{\calC}\!:\calC\to\Ind_\kappa(\calC)\) admits a left adjoint \(\sigma_{\calC}\!:\Ind_\kappa(\calC)\to\calC\), taking a ``formal'' \(\kappa\)-filtered
	colimit to its actual colimit object in \(\calC\).
	\item The functors compose to give \(\sigma_{\calC}\circ\yo_{\calC} \cong \1_{\calC}\).
	\end{enumerate}
\end{lemma}
\begin{proof}
Given the construction below (2) is clear, so we do not prove it explicitly. For (1), let \(A\in\Ind_\kappa(\calC)\), \(x\in\calC\). Then we have natural isomorphisms
\[ \Ind_\kappa(\calC)(A,\yo_{\calC}(x)) \cong \projlim\Ind_\kappa(\calC)(\yo_{\calC}\circ\pi_A,\yo_{\calC}(x)) \cong \projlim\calC(\pi_A,x) \cong \calC(\injlim\pi_A,x)  \]
so we are done, since the colimit \(\injlim\pi_A\) exists on account of \(\calC/A\) being cofinally small and \(\kappa\)-filtered.
\end{proof}
\begin{theorem}
	Let \(\calC\) be a small category, and let \(\kappa\) be an infinite regular cardinal. Then \(\Ind_\kappa(\calC)\) satisfies the following universal property:
	for any locally small category \(\calD\) admitting \(\kappa\)-filtered colimits, and any functor \(F\!:\calC\to\calD\), there is a functor \(F'\!:\Ind_\kappa(\calC)\to\calD\) unique
	up to unique isomorphism such that the diagram
	\begin{diagram*}
		\calC\ar[d,hook]\ar[r,"F"] & \calD \\
		\Ind_\kappa(\calC)\ar[ur,dashed,"F'"']
	\end{diagram*}
	commutes up to natural isomorphism and \(F'\) commutes with \(\kappa\)-filtered colimits.
\end{theorem}
\begin{proof}
The functor \(F'\) is given by the composition
\[ \Ind_\kappa(\calC)\overset{\Ind_\kappa(F)}\longto\Ind_\kappa(\calD)\overset{\sigma_{\calD}}\longto\calD. \]
One easily sees that this choice is unique from the uniqueness of \(\Ind_\kappa(F)\).
\end{proof}

\subsection{Accessibility}
\begin{lemma}\label{lemma:induced-ind-functor-fully-faithful}
	Let \(F\!:\calC\to\calD\) be a functor, where \(\calC\) is small and \(\calD\) is locally small. Assume further that the following conditions are satisfied:
	\begin{enumerate}[label=(\arabic*)]
	\item \(\calD\) admits small \(\kappa\)-filtered colimits.
	\item \(F\) is fully faithful.
	\item For all \(x\in\calC\), the object \(Fx\in\calD\) is \(\kappa\)-compact.
	\end{enumerate}
	Then the induced functor \(F'\!:\Ind_\kappa(\calC)\to\calD\) is fully faithful.
\end{lemma}
\begin{proof}
This is a computation. Let \(A,B\in\Ind_\kappa(\calC)\). Then
\begin{align*}
	\Ind_\kappa(\calC)(A,B) &\cong \projlim\injlim\Ind_\kappa(\calC)(\yo_{\calC}\circ\pi_A,\yo_{\calC}\circ\pi_B) \\
	&\cong \projlim\injlim\calC(\pi_A,\pi_B) \\
	&\cong \projlim\injlim\calD(F\circ\pi_A,F\circ\pi_B) \\
	&\cong \projlim\calD(F\circ\pi_A,\injlim(F\circ\pi_B)) \\
	&\cong \calD(\projlim(F\circ\pi_A),\injlim(F\circ\pi_B)) \cong \calD(F'(A),F'(B))
\end{align*}
as desired.
\end{proof}
\begin{theorem}
	Let \(\calC\) be a category, and let \(\kappa\) be an infinite regular cardinal. Then the following are equivalent:
	\begin{enumerate}[label=(\arabic*)]
	\item There is a small category \(\calC_0\) such that \(\calC\simeq\Ind_\kappa(\calC_0)\).
	\item The category \(\calC\) satisfies the following conditions:
		\begin{enumerate}[label=(\roman*)]
		\item \(\calC\) is locally small.
		\item \(\calC\) admits \(\kappa\)-filtered colimits.
		\item There is a small full subcategory \(\calC_0\) of \(\calC\) consisting of \(\kappa\)-compact objects for which every object of \(\calC\) can be written as the colimit of a \(\kappa\)-filtered diagram
		in \(\calC_0\).
		\end{enumerate}
	\end{enumerate}
\end{theorem}
\begin{proof}
It is clear that (1) implies (2), effectively by definition. To show that (2) implies (1), the idea is to show that \(\calC\simeq\Ind_\kappa(\calC_0)\) for the chosen full subcategory \(\calC_0\) of \(\kappa\)-compact
objects. To do this, consider the canonical functor \(\Ind_\kappa(\calC_0)\to\calC\) induced by universal property from the inclusion \(\calC_0\inj\calC\). By assumption, all the
conditions of Lemma \ref{lemma:induced-ind-functor-fully-faithful} are satisfied, so this functor is fully faithful. However, it is also essentially surjective by assumption, and therefore
an equivalence.
\end{proof}
\begin{exercise}
	Implicitly, in many places we have used that \(\Ind_\kappa(\calC)\) is locally small.
	\begin{enumerate}[label=(\arabic*)]
	\item Compute the Hom-sets in \(\Ind_\kappa(\calC)\) in terms of those in \(\calC\), using that representables are compact.
	\item Deduce that \(\Ind_\kappa(\calC)\) is locally small.
	\end{enumerate}
\end{exercise}
\begin{definition}
	Consider a regular infinite cardinal \(\kappa\). We say a category \(\calC\) is \(\kappa\)\emph{-accessible} if there is a small category \(\calC_0\) such that \(\calC\simeq\Ind_\kappa(\calC_0)\).
	We say that \(\calC\) is \emph{accessible} if it is \(\kappa\)-accessible for some \(\kappa\). A functor \(F\!:\calC\to\calD\) between accessible categories is called accessible
	if both \(\calC\) and \(\calD\) are \(\kappa\)-accesible for some common \(\kappa\), and \(F\) preserves \(\kappa\)-filtered colimits.
\end{definition}

Accessibility is \emph{very nearly} a purely set-theoretic condition.
\begin{proposition}
	Let \(\calC\) be a small category. Then \(\calC\) is accessible if and only if it is idempotent complete.
\end{proposition}
\begin{proof}
See \cite[Prop.\ 2.2.1 \& Thm.\ 2.2.2]{makkai-pare-accessible-categories}.
\end{proof}

\begin{definition}
	An accessible category is \emph{presentable} if it is cocomplete.
\end{definition}

Presentable categories are very useful in the context of adjoint functor theorems, because the automatically satisfy all the conditions for them to hold. In \(\infty\)-category theory,
this is often exploited by making use of the \(\infty\)-category \(\cat{Pr}^L\) of presentable \(\infty\)-categories and functors preserving small colimits. This \(\infty\)-category
is particularly nice, because it also has the structure of a symmetric monoidal \(\infty\)-category under the \emph{Lurie tensor product.}

Being presentable and \(\kappa\)-accessible (that is, \(\kappa\)-accessible and cocomplete) is sometimes refered to as being \(\kappa\)\emph{-compactly generated,} or \(\kappa\)-presentable.


%!TEX root = ../lectures.tex

% copied from tex.stackexchange @ https://tex.stackexchange.com/questions/615542/boxtimes-without-a-diagonal
% don't need it, actually, but I'm keeping it around just in case
% \DeclareFontFamily{U}{mathb}{\hyphenchar\font45}
% \DeclareFontShape{U}{mathb}{m}{n}{
%       <5> <6> <7> <8> <9> <10> gen * mathb
%       <10.95> mathb10 <12> <14.4> <17.28> <20.74> <24.88> mathb12
%       }{}
% \DeclareSymbolFont{mathb}{U}{mathb}{m}{n}
% \DeclareMathSymbol{\boxslash}{2}{mathb}{"6D}%%%%%% imported only two symbol of mathabx; here there are the slot of this package: https://mirror.las.iastate.edu/tex-archive/fonts/mathabx/texinputs/mathabx.dcl
% \DeclareMathSymbol{\boxbackslash}{2}{mathb}{"6E}

\section{Lifting problems \& Quillen's small object argument}\label{lecture:lifting-problems}

In essence, homotopy theory is the study of localizations of categories. For example, classically one was interested in the localization of some nice category of topological spaces
at the (weak) homotopy equivalences. Studying these kinds of structures takes many forms, and we have seen a few already in Lectures
\ref{section:localization-of-categories} \& \ref{lecture:homotopical-algebra-through-deformations}. However, modern homotopy theory is, largely speaking, fundamentally phrased in terms of
so-called \emph{lifting problems.} The crux is that much homotopy theory can be done in the setting of a \emph{model category,} which makes use of factorizations of morphisms into pairs
(a trivial fibration and a cofibration, or a fibration and a trivial cofibration) having some lifting property with respect to each other. Pairs of classes of maps where one can make such
factorizations are called \emph{factorization systems.}

The other topic of this lecture, the small object argument (due to Quillen), is a method for producing factorization systems out of some specified class of maps whose domains
are sufficiently compact objects. This is the origin of the name, as compact objects have historically also been referred to as ``small'' objects.

\subsection{Lifting problems}

This is our basic notion for study.
\begin{definition}
	Let \(\calC\) be a category. We say a morphism \(f\!:x\to y\) has the \emph{left lifting property} with respect to a morphism \(g\!:x'\to y'\),
	or that \(g\) has the \emph{right lifting property} with respect to \(f\), if for any solid diagram
	\begin{diagram*}
		x\ar[d,"f"']\ar[r] & x'\ar[d,"g"] \\
		y\ar[r]\ar[ur,dashed] & y'
	\end{diagram*}
	a dashed arrow exists.

	Consider some set of maps \(S\) in \(\calC\). We say \(f\) has the left lifting property with respect to \(S\) if it has the left lifting property with respect to all elements of \(S\).
	The case for the right lifting property is similar.
\end{definition}
\begin{remark}
	A diagram of the form
	\begin{diagram*}
		x\ar[d,"f"']\ar[r] & x'\ar[d,"g"] \\
		y\ar[r]\ar[ur,dashed] & y'
	\end{diagram*}
	describes a \emph{lifting problem.} A dashed arrow, if one exists, is called a \emph{solution} to the lifting problem. In this way, \(f\) has the left lifting property with
	respect to \(g\) if we can solve any lifting problem with \(f\) on the left and \(g\) on the right.
\end{remark}
\begin{example}
	Let \(\calA\) be an Abelian category. Recall that an object \(p\in\calA\) is \emph{projective} if \(\calA(p,-)\) is exact. This essentially reduces to asking that
	for any epimorphism \(x \sur y\) and morphism \(p\to y\), there is a lift factoring this through a morphism \(p\to x\). Observe that we can phrase this as a lifting problem:
	\begin{diagram*}
		0\ar[d]\ar[r] & x\ar[d,two heads] \\
		p\ar[r]\ar[ur,dashed] & y
	\end{diagram*}
	so that \(p\) is projective if and only if the unique map \(0 \to p\) has the left lifting property with respect to the set of epimorphisms in \(\calA\). Dually, an object \(q\in\calA\)
	is \emph{injective} if and only if the unique map \(0\to q\) has the right lifting property with respect to the set of monomorphisms.
\end{example}
\begin{remark}
	In the situation above, where one of the morphisms is trivial, it is common to omit it entirely. For example, one might write
	\begin{diagram*}
		 & x\ar[d,two heads] \\
		p\ar[r]\ar[ur,dashed] & y
	\end{diagram*}
	for the lifting problem associated to a projective.
\end{remark}
\begin{example}\label{example:surjections-in-terms-of-rlp}
	A map of sets \(f\!:A\to B\) is surjective if and only if it has the right lifting property with respect to \(\varnothing \inj *\). This is easy to see; indeed, the diagram is
	\begin{diagram*}
		\varnothing\ar[d]\ar[r] & A\ar[d,"f"] \\
		*\ar[r]\ar[ur,dashed] & B
	\end{diagram*}
	which tells us that for any element \(b\in B\), there is some element \(a\in A\) such that \(f(a)=b\), i.e.\ \(f\) is surjective.
\end{example}

So, many interesting maps and objects arise in terms of lifting problems admitting a solution. Let us fix some notation for these things, taken from \cite{riehl-categorical-homotopy-theory}.
\begin{notation}
	Let \(\calC\) be a category, and let \(f,g\) be two morphisms in \(\calC\). We write \(f\boxslash g\) to say that \(f\) has the left lifting property with respect to \(g\) (or, equivalently, that \(g\) has the
	right lifting property with respect to \(f\)).

	If \(S\) is a set of morphisms in \(\calC\), we write \(S^\boxslash\) for those morphisms which have the right lifting property with respect to \(S\), and dually, \(\prescript{\boxslash}{}{S}\)
	for those morphisms having the left lifting property with respect to \(S\). If \(T\) is another set of morphisms, we write \(S\boxslash T\) to say that \(S\subseteq \prescript{\boxslash}{}{T}\),
	or equivalently that \(T\subseteq S^\boxslash\).
\end{notation}
\begin{remark}
	To spell it out, writing \(S\boxslash T\) means that we can solve any lifting problem with an \(S\) on the left and \(T\) on the right.
\end{remark}
\begin{example}
	We see from Example \ref{example:surjections-in-terms-of-rlp} that
	\[ \{ \text{surjections in }\Set \} = \{\varnothing\inj *\}^\boxslash. \]
\end{example}

\begin{example}
	Any isomorphism has the left and right lifting property with respect to all maps. Indeed, you just compose with the inverse of the isomorphism to produce the desired lift.
\end{example}

\begin{proposition}
	Let \(\calC\) be a category, and let \(S\) and \(T\) be sets of morphisms in \(\calC\). Then the following statements hold.
	\begin{enumerate}[label=(\arabic*)]
		\item If \(S\subseteq T\), then \(\prescript{\boxslash}{}{T}\subseteq\prescript{\boxslash}{}{S}\).
		\item If \(S\subseteq T\), then \(T^\boxslash\subseteq S^\boxslash\).
		\item \(S \subseteq \prescript{\boxslash}{}{(S^\boxslash)}\).
		\item \(S^\boxslash = (\prescript{\boxslash}{}{(S^\boxslash)})^\boxslash\).
		\item \(\prescript{\boxslash}{}{S} = \prescript{\boxslash}{}{((\prescript{\boxslash}{}{S})^\boxslash)}\).
	\end{enumerate}
\end{proposition}
\begin{proof}
Statements (1), (2), and (3) are trivial. Statements (4) and (5) are dual, so we prove (3). First, note that one inclusion follows by (2) and (3), i.e.
\[ S \subseteq \prescript{\boxslash}{}{(S^\boxslash)} \implies (\prescript{\boxslash}{}{(S^\boxslash)})^\boxslash \subseteq S^\boxslash. \]
On the other hand, the other inclusion is also trivial: let \(g\in S^\boxslash\); what we have to check is that \(\prescript{\boxslash}{}{(S^\boxslash)} \boxslash g\),
but to test this, we use a lifting problem against a morphism \(f\) with the left lifting property with respect to \(S^\boxslash\).
\end{proof}

\subsection{Saturated classes}

Let \(S\) be a class of morphisms in some category \(\calC\). The set of morphisms \(\prescript{\boxslash}{}{S}\) has some special properties. We will now explain them.
One has already been mentioned: \(\prescript{\boxslash}{}{S}\) contains all isomorphisms, which is easy to see.

Most of the properties that \(\prescript{\boxslash}{}{S}\) will satisfy are fairly self-explanatory. One, however, is a bit trickier to make sense of. It is not too hard to see
that it is closed under composition. However, something much stronger is true: it is closed under \emph{transfinite composition.} For the purposes of explaining this,
we will need to make use of ordinals. Rest assured, we do not need to know much about them.

Recall that a totally ordered set is a \emph{well order} if, in addition, every non-empty subset has a least element. These can be regarded as categories in an obvious way, and an
ordinal is merely an isomorphism class of well ordered sets. In particular, we can view any ordinal \(\alpha\) as a category. Now, the class of ordinals happens to be well-ordered (intuitively,
by inclusion), meaning that for any two ordinals \(\alpha\) and \(\beta\) we can also talk about whether \(\beta < \alpha\). In general, one can identify an ordinal \(\alpha\) with
the set \(\{ \beta \mid \beta < \alpha \}\).

Taking an ordinal \(\alpha\) and freely adjoining a terminal element, one obtains the \emph{succesor} \(\alpha+1\). An ordinal \(\lambda\) is a \emph{limit} ordinal if it is non-zero and
not the successor of any other ordinal. Observe that this happens if and only if \(\lambda\) is the \emph{colimit} of all ordinals strictly lesser than \(\lambda\). Indeed, if \(\lambda = \beta+1\), then
the colimit would instead be given by \(\beta\).

\begin{definition}
	Let \(\calC\) be a category, let \(S\) be a set of morphisms in \(\calC\), and let \(\alpha\) be some ordinal. An \(\alpha\)-composable sequence of morphisms in \(S\) is a
	diagram \(x_\bullet\!:\alpha \to \calC\) with the following properties.
	\begin{enumerate}[label=(\arabic*)]
		\item For any ordinal \(\beta < \alpha\) with a succesor \(\beta + 1 < \alpha\), the corresponding morphism \(x_{\beta} \to x_{\beta+1}\) is in \(S\).
		\item \(x_\bullet\) is ``continuous''. That is, for any limit ordinal \(\lambda < \alpha\), the canonical morphism
		\[ x_\lambda \to \injlim_{\beta < \lambda}x_\beta \]
		is an isomorphism.
	\end{enumerate}
	The \emph{composition} of an \(\alpha\)-composable sequence of morphisms \(x_\bullet\) is the canonical map
	\[ x_0 \to x_\alpha := \injlim x_\bullet. \]
	Let \(\kappa\) be a cardinal. We say that \(S\) is closed under \(\kappa\)-transfinite compositions if the composition of any \(\alpha\)-composable sequence of morphisms
	in \(S\) is in \(S\) for all ordinals \(\alpha\) with cardinality strictly less than \(\kappa\).

	We say that \(S\) is \emph{closed under transfinite compositions} if it is closed under \(\kappa\)-transfinite compositions for all \(\kappa\).
\end{definition}

\begin{remark}
	The easiest non-trivial picture here is \(\alpha = \omega\). Then we just have a sequence
	\[ x_0 \to x_1 \to \cdots \to x_i \to \cdots \]
	and we require that each \(x_{i} \to x_{i+1}\) is in \(S\). If \(S\) is now closed under transfinite composition, it means that the map
	\[ x_0 \to x_\omega := \injlim_i x_i \]
	is in \(S\) too.
\end{remark}

\begin{proposition}\label{prop:llp-closure-properties}
	The set \(\prescript{\boxslash}{}{S}\) is closed under the following constructs.
	\begin{enumerate}[label=(\arabic*)]
		\item Retracts.
		\item Small coproducts, meaning that for any small indexing set \(I\) and collection of morphisms \(\{f_i\!:x_i\to y_i\}_{i\in I}\subseteq S\), the morphism
		\[ \coprod_{i\in I}f_i \!: \coprod_{i\in I}x_i \to \coprod_{i\in I}y_i \]
		is in \(S\) whenever it exists.
		\item Pushouts, meaning:
		\[
		\begin{tikzcd}
			x \ar[d,"S\ni"']\ar[r] & z \\
			y
		\end{tikzcd}\quad\leadsto\quad
		\begin{tikzcd}
			x\ar[d,"S\ni"'] \ar[r] & z \ar[d,dashed,"\in S"] \\
			y\ar[r,dashed] & y\amalg_xz\ar[ul,pushout]
		\end{tikzcd}.
		\]
		\item Transfinite compositions.
	\end{enumerate}
\end{proposition}
\begin{proof}
The proofs of all of these facts are very similar. We prove (4) by transfinite induction. Let \(\alpha\) be some ordinal, and consider an \(\alpha\)-composable sequence of morphisms
\(x_\bullet\!:\alpha\to\calC\) in \(\prescript{\boxslash}{}{S}\). The base case when \(\alpha=1\) is trivial, so assume we know the result for all \(\beta < \alpha\). If \(\alpha = \alpha'+1\) is a successor ordinal, then
if \(f\in S\), we are in the following situation
\[
	\begin{tikzcd}
		x_0\ar[d,"\prescript{\boxslash}{}{S}\ni"']\ar[r] & x\ar[dd,"f"] \\
		x_{\alpha'}\ar[d,"\prescript{\boxslash}{}{S}\ni"'] & \\
		x_{\alpha'+1}\ar[r]\ar[uur,dashed] & y
	\end{tikzcd}\quad = \quad
	\begin{tikzcd}
		x_0 \ar[d,"\prescript{\boxslash}{}{S}\ni"']\ar[r] & x\ar[d,"f"] \\
		x_{\alpha'}\ar[r]\ar[ur,dashed,"\exists"] & y
	\end{tikzcd}\text{ and }
	\begin{tikzcd}
		x_{\alpha'} \ar[d,"\prescript{\boxslash}{}{S}\ni"']\ar[r] & x\ar[d,"f"] \\
		x_{\alpha'+1}\ar[r]\ar[ur,dashed,"\exists"] & y
	\end{tikzcd}
\]
giving the desired solution. If \(\alpha\) is a limit ordinal, note that a map \(x_\alpha \to x\) is given by a compatible collection of maps \(x_\beta\to x\) for all \(\beta < \alpha\). In
particular, by the induction hypothesis, we then have
\[
	\forall \beta < \alpha,\quad
	\begin{tikzcd}
		x_0\ar[d,"\prescript{\boxslash}{}{S}\ni"']\ar[r] & x\ar[dd,"f"] \\
		x_\beta\ar[d]\ar[ur,dashed,"\exists"] & \\
		x_\alpha\ar[r] & y
	\end{tikzcd}\quad\leadsto\quad
	\begin{tikzcd}
		x_0\ar[d]\ar[r] & x\ar[d,"f"] \\
		x_\alpha\ar[r]\ar[ur,dashed,"\exists"] & y
	\end{tikzcd}
\]
so we have our lift here too. This completes the argument.
\end{proof}
\begin{exercise}
	Complete the above proof.
\end{exercise}

\begin{definition}
	Let \(\calC\) be a category and let \(\kappa\) be a cardinal. A set of morphisms \(T\) in \(\calC\) is said to be \emph{weakly} \(\kappa\)\emph{-saturated} if it satisfies the following
	properties.
	\begin{enumerate}[label=(\arabic*)]
		\item \(T\) is closed under retracts.
		\item \(T\) is closed under small coproducts.
		\item \(T\) is closed under pushouts.
		\item \(T\) is closed under \(\kappa\)-transfinite compositions.
		\item \(T\) contains all isomorphisms.
	\end{enumerate}
	We say \(T\) is \emph{weakly saturated} if it is weakly \(\kappa\)-saturated for all \(\kappa\). Dually, we say \(T\) is \emph{weakly} (\(\kappa\)-)\emph{cosaturated} if \(T^\op\)
	is weakly (\(\kappa\)-)saturated in \(\calC^\op\).
\end{definition}
\begin{corollary}\label{corollary:llp-weakly-saturated}
	Let \(S\) be a set of morphisms in \(\calC\). Then \(\prescript{\boxslash}{}{S}\) is weakly saturated and \(S^\boxslash\) is weakly cosaturated.
\end{corollary}
\begin{remark}
	In \cite[Remark 2.1.3]{cisinski-book}, it is pointed out that the above list has a redundancy. In fact, being closed under transfinite composition and pushouts implies being
	closed under small coproducts.
\end{remark}

\begin{notation}
	Let \(S\) be some collection of morphisms in \(\calC\). We write \(\overline{S}^{(\kappa)}\) for the \(\kappa\)\emph{-saturation} of \(S\), i.e.\ the smallest
	weakly \(\kappa\)-saturated collection of morphisms containing \(S\). Similarly, we write \(\overline{S}\) for the smallest weakly saturated collection of morphisms containing \(S\).
\end{notation}
\begin{remark}\label{remark:saturation-inclusions}
	Observe that since \(\prescript{\boxslash}{}{(S^\boxslash)}\) is saturated and contains \(S\), we have \(\overline{S} \subseteq \prescript{\boxslash}{}{(S^\boxslash)}\). We will
	see that in favourable situations, this inclusion is an equality. This is one of the consequences of the small object argument.
	
	Additionally, we have that for all cardinals \(\kappa < \kappa'\), \(\overline{S}^{\kappa} \subseteq \overline{S}^{\kappa'} \subseteq \overline{S}\).
\end{remark}

\subsection{Factorization systems}
We begin with the following motivating example.
\begin{example}
	We saw that the set of surjections in \(\Set\) were characterized by having the right lifting property with respect to \(\varnothing\inj*\). One may ask: what are
	\[  \{\text{surjections}\}^\boxslash \quad\text{and}\quad\prescript{\boxslash}{}{\{\text{surjections}\}}? \]
	Curiously, the answer is the same in both cases, namely you recover the collection of injective maps. Proving this makes use of the axiom of choice, and we will not discuss it.
	Nonetheless, it is interesting to us because the following much more trivial fact: any map of sets can be decomposed into an injective map followed by a surjective map.
\end{example}
\begin{definition}\label{definition:factorization-system}
	Let \(\calC\) be a category. A \emph{factorization system} on \(\calC\) is a pair \((S,T)\) of collections of maps satisfying the following properties.
	\begin{enumerate}[label=(\alph*)]
		\item The collections \(S\) and \(T\) are closed under retracts.
		\item For all \(f\!:x\to y\) in \(\calC\), there is a factorization \(f = p\circ i\) with \(i\in S\) and \(p\in T\).
		\item \(S\boxslash T\).
	\end{enumerate}
\end{definition}

\begin{lemma}[``The retract lemma'']\label{lemma:retract-lemma}
	Suppose we have a factorization
	\[
	\begin{tikzcd}[cramped, column sep=small]
		x\ar[rr,"f"]\ar[dr,"i"'] & & y \\
		& z\ar[ur,"p"'] &
	\end{tikzcd}.
	\]
	Then the following statements hold.
	\begin{enumerate}[label=(\arabic*)]
		\item If \(i\boxslash f\) then \(f\) is a retract of \(p\).
		\item If \(f\boxslash p\), then \(f\) is a retract of \(i\).
	\end{enumerate}
\end{lemma}
\begin{proof}
Statements (1) and (2) are dual, so we prove (1). The lifting problem below left
\[
	\begin{tikzcd}
		x\ar[r,equal]\ar[d,"i"'] & x \ar[d,"f"] \\
		z\ar[r,"p"]\ar[ur,dashed,"\exists"] & y
	\end{tikzcd}\quad\leadsto\quad
	\begin{tikzcd}
		x\ar[d,"f"]\ar[r,"i"] & z\ar[d,"p"]\ar[r,dashed,""] & x \ar[d,"f"] \\
		y \ar[r,equal] & y \ar[r,equal] & y
	\end{tikzcd}
\]
gives rise to the retraction as above right, as desired.
\end{proof}
\begin{proposition}\label{prop:factorization-system-equalities}
	Suppose we have a pair \((S,T)\) of collections of morphisms satisfying property (b) in Definition \ref{definition:factorization-system}. Then the following are equivalent.
	\begin{enumerate}[label=(\arabic*)]
		\item \((S,T)\) is a factorization system.
		\item \(S^\boxslash = T\) and \(S = \prescript{\boxslash}{}{T}\).
	\end{enumerate}
\end{proposition}
\begin{proof}
It is clear that (2) implies (1) (note Corollary \ref{corollary:llp-weakly-saturated}), so what remains is the other direction. We are given that \(S^\boxslash \supseteq T\) and \(S\subseteq \prescript{\boxslash}{}{T}\)
(which are equivalent). To get the other inclusions, suppose \(f\in S^\boxslash\). We may then write \(f = p\circ i\) for \(i\in S\) and \(p\in T\), but by Lemma \ref{lemma:retract-lemma} and the
fact that \(i\boxslash f\), we see that \(f\) is a retract of \(p\). Since \(T\) is closed under retracts, \(f \in T\). The other case is dual.
\end{proof}

\subsection{Quillen's small object argument}
A priori, it is very hard to produce factorization systems. In particular, while it is not at all hard to find nice pairs of collections of morphisms with some
lifting properties with respect to each other, manufacturing factorizations into these morphisms is tricky. Quillen provided a tool of great ingenuity for tackling this problem,
namely the \emph{small object argument.} There are a number of versions of it; we choose one which is fairly intelligible.

\begin{proposition}[Small object argument]\label{prop:small-object-argument}
	Let \(\calC\) be a locally small category admitting all small colimits, let \(\kappa\) be an infinite regular cardinal, and let \(M\) be some collection of morphisms
	in \(\calC\) with \(\kappa\)-compact domains. Then \((\prescript{\boxslash}{}{(M^\boxslash)}, M^\boxslash)\) is a factorization system.
\end{proposition}
\begin{proof}
By construction, conditions (a) and (b) of Definition \ref{definition:factorization-system} are already satisfied. In other words, the crux of this proposition is the existence,
for all \(f\!:x\to y\) in \(\calC\), of a factorization \(f = h\circ g\) where \(g\in \prescript{\boxslash}{}{(M^\boxslash)}\) and \(h\in M^\boxslash\). In fact, we will prove something
slightly stronger: we can choose \(g\in \overline{M}^\kappa\), which implies the former by Remark \ref{remark:saturation-inclusions}.

The strategy of the proof is to approximate \(f\) by a morphism in \(M\). To this end, for any map \(a\!: u\to v\), define the set
\[ I_M(a) := \left\{\begin{tikzcd}
	z\ar[d,"s"']\ar[r]  & u\ar[d,"a"] \\
	w\ar[r] & v
\end{tikzcd} : s \in M \right\}, \]
which consists of all morphisms \(s\To a\) in \(\Ar(\calC) \cong \Fun([1], \calC)\). We observe that this set is small since \(M\) is small and \(\calC\) is locally small.

For notational simplicity, given a commutative square \(i \in I_M(a)\), we write \(z_i\), \(w_i\), and \(s_i\) for the corresponding data as above. Note that we leave the horizontal maps
unnamed since we will not need them by name. Now, using \(I_M(a)\) we can associate to \(a\) an object \(u_M\) and a pair of maps \(g(a)\!:u\to u_M\) and \(h(a)\!:u_M\to v\) such that
\[ a = h(a)\circ g(a),\quad \text{and}\quad g(a)\in\overline{M}^\kappa. \]
In other words, a factorization almost like what we want. The procedure is the following: taking the coproduct over all \(i\in I_M(a)\), we get a diagram
\begin{diagram*}
	\coprod_{i\in I_M(a)} z_i\ar[r]\ar[d,"\coprod_i s_i"'] & u\ar[d,"a"] \\
	\coprod_{i\in I_M(a)} w_i\ar[r] & v
\end{diagram*}
by composing horizontally with the codiagonal. Taking the pushout of the morphism on the left, we get
\begin{diagram*}[row sep=large]
	\coprod_{i\in I_M(a)} z_i\ar[r]\ar[d,"\coprod_i s_i"'] & u\ar[ddr,bend left,"a"]\ar[d,dashed,"g(a)"] & \\
	\coprod_{i\in I_M(a)} w_i\ar[r,dashed]\ar[drr,bend right] & u_M \ar[dr,"h(a)"']\ar[ul,pushout] & \\
	& & v
\end{diagram*}
which defines our maps. We have that \(g(a)\in\overline{M}^\kappa\) since the latter is stable under small coproducts and pushouts.

If we apply the above to \(f\), we get a factorization \(f = h(f)\circ g(f)\) where \(g(f) \in \overline{M}^\kappa\). However, we do not know that \(h(f)\in M^\boxslash\),
so our approximation isn't good enough. To remedy this, we will approximate again and again, inductively, and in fact do it \(\kappa\) times.

Set \(x_0 = x\), \(h_0 = f\). We obtain \(g_{0,1} := g(h_0)\!:x_0 \to (x_0)_M =: x_1\) and \(h_1 := h(h_0)\). We continue inductively as follows: for any successor ordinal
\(\alpha+1\), supposing we have defined \(x_\alpha\), \(h_\alpha\!:x_{\alpha}\to y\), and \(g_{\beta,\alpha}\!:x_{\beta}\to x_{\alpha} \) for all \(\beta \leq \alpha\), we define
\[ x_{\alpha+1} = (x_\alpha)_M,\quad g_{\beta,\alpha+1} := g(h_\alpha)\circ g_{\beta,\alpha},\quad h_{\alpha+1} = h(h_\alpha). \]
We remark that \(g_{\alpha,\alpha} = \id_{x_\alpha}\), and therefore \(g_{\alpha,\alpha+1} = g(h_\alpha)\). For a limit ordinal \(\lambda \leq \kappa\), we define \(x_\lambda\) and \(h_\lambda\) by
\[ x_0 \overset{h_\lambda}\longto \injlim_{\beta < \lambda}x_\beta =: x_\lambda. \]
For any \(\beta < \lambda\) we obtain the canonical morphism \(g_{\beta,\lambda}\!:x_\beta \to x_\lambda\).

We get an object \(x_\kappa\) and a pair of maps
\[ x \overset{g_{0,\kappa}}\longto x_\kappa \overset{h_\kappa}\longto y, \]
so we set \(g = g_{0,\kappa }\) and \(h = h_\kappa\). These satisfy \(f = h\circ g\), and \(g\in\overline{M}^\kappa\) since the latter is closed under \(\kappa\)-transfinite compositions.
It remains to check that \(h\in M^\boxslash\).

Consider a lifting problem
\[
	\begin{tikzcd}
		z\ar[d,"M \ni s"']\ar[r] & x_\kappa\ar[d,"h"] \\
		w\ar[r]\ar[ur,dashed,"\exists?"] & y
	\end{tikzcd}
\]
to be solved. By definition, \(x_\kappa = \injlim_{\beta < \kappa}x_\beta\), which is a \(\kappa\)-filtered colimit. Hence, the \(\kappa\)-compactness of \(z\) means there is some
\(\alpha < \kappa\) and a factorization
\[
	\begin{tikzcd}
		z\ar[dd,"s"']\ar[rr]\ar[dr,dashed] & & x_\kappa\ar[dd,"h"] \\
		& x_\alpha\ar[dr,"h_\alpha"']\ar[ur,"g_{\alpha,\kappa}"' near start] & \\
		w\ar[rr] & & y
	\end{tikzcd}.
\]
Now comes the genius trick: the square defined by the bottom left morphisms
\[
	\begin{tikzcd}
		z\ar[d,"s"']\ar[r] & x_\alpha \ar[d,"h_\alpha"] \\
		w \ar[r] & y
	\end{tikzcd}
\]
is exactly a square used in defining \(x_{\alpha+1} = (x_{\alpha})_m\). This means precisely that we can factorize further into
\[
	\begin{tikzcd}[column sep=large]
		z\ar[ddd,"s"']\ar[rr]\ar[dr] & & x_\kappa\ar[ddd,"h"] \\
		& x_\alpha\ar[d,"g_{\alpha+1}"']\ar[ur,"g_{\alpha,\kappa}" near start] & \\
		& x_{\alpha+1}\ar[dr,"h_{\alpha+1}" near start]\ar[uur,"g_{\alpha+1,\kappa}"'] & \\
		w\ar[rr]\ar[ur] & & y
	\end{tikzcd}
\]
where we thus obtain our solution to the lifting problem.
\end{proof}
\begin{corollary}
	Let \(\calC\) be a locally small category admitting all small colimits, and suppose \(M\) is a collection of maps whose domains are all \(\kappa\)-compact for some fixed cardinal \(\kappa\).
	Then \(\overline{M} = \prescript{\boxslash}{}{(M^\boxslash)}\).
\end{corollary}
\begin{proof}
The proof of Proposition \ref{prop:small-object-argument} shows that \((\overline{M},M^\boxslash)\) is a factorization system. However, by Proposition \ref{prop:factorization-system-equalities},
one then has
\[ \overline{M} = \prescript{\boxslash}{}{(M^\boxslash)} \]
as desired.
\end{proof}
\begin{remark}
	In \cite[§12.2]{riehl-categorical-homotopy-theory}, there is a brief discussion of various kinds of assumptions one can take to get a small object argument. In the above, we've chosen
	a somewhat straightforward one, modulo the details imposed by allowing any cardinal \(\kappa\). One can weaken the assumptions drastically. For example, we do not need the domain
	of a morphism \(s\!:z\to w\) in \(M\) to be entirely \(\kappa\)-compact, as we only use that \(\calC(z,-)\) commutes with \(\kappa\)-filtered colimits of morphisms in \(\overline{M}^\kappa\).
	Furthermore, we only use one particular kind of \(\kappa\)-filtered colimit, namely a transfinite composition of \(\kappa\)-many arrows.
\end{remark}


\subsection{Appendix: Transposing lifting problems along adjunctions}
There are a few ways that lifting problems can be moved across adjunctions, and it is very useful to be aware of them, as a number of techniques in categorical homotopy
theory (for example, of simplicial sets) come down to convering a difficult lifting problem to an easy one through some adjunction you have available to you.

The following is the easiest possible case. It is taken from \cite[Lemma 11.1.5]{riehl-categorical-homotopy-theory}, though the overall argument is
of course standard.
\begin{proposition}\label{prop:adjunction-lifting-correspondence}
	Let \(
	\begin{tikzcd}[cramped]
		\calC\ar[r,bend left,"F",""{name=A,below}] & \calD \ar[l,bend left,"G",""{name=B,above}]\ar[from=A,to=B,symbol=\dashv]
	\end{tikzcd}
	\) be an adjunction, and suppose we have collections of morphisms \(S\subseteq\Mor(\calC)\) and \(T\subseteq\Mor(\calD)\). Then
	\( F(S) \boxslash T\) if and only if \( S \boxslash G(T) \).
\end{proposition}
\begin{proof}
Specializing the below Lemma \ref{lemma:functor-category-adjunction} to the case \(\calE=[1]\), we see that there is an induced adjunction
\[
	\begin{tikzcd}[cramped]
		\Fun([1],\calC)\ar[r,bend left,"F_*",""{name=A,below}] & \Fun([1],\calD) \ar[l,bend left,"G_*",""{name=B,above}]\ar[from=A,to=B,symbol=\dashv]
	\end{tikzcd}
\]
on the arrow categories, where we note that \(\Ar(-) \cong \Fun([1],-)\). The setup for a lifting problem is precisely a morphism in an arrow category,
so we get a bijection between a squares \(Fi \To f\) in \(\calD\) as below left, and solid squares \(i \To Gf\) in \(\calC\) as below right:
\[
	\begin{tikzcd}
		Fa\ar[d,"Fi"']\ar[r] & x\ar[d,"f"] \\
		Fb\ar[r]\ar[ur,dashed,"h"] & y
	\end{tikzcd} \quad \leftrightsquigarrow \quad
	\begin{tikzcd}
		a\ar[d,"i"']\ar[r] & Gx\ar[d,"Gf"] \\
		b\ar[r]\ar[ur,dashed,"h'"] & Gy
	\end{tikzcd}.
\]
Moreover, a solution \(h\) as above left is exactly the same as a morphism in \(\Fun([2],\calD)\) as below left, which by the same argument means we have a
correspondence with a solution \(h'\) as below right
\[
	\begin{tikzcd}
		Fa\ar[d,"Fi"']\ar[r] & x\ar[d,equal] \\
		Fb\ar[r,"h"]\ar[d,equal] & x\ar[d,"f"] \\
		Fb\ar[r] & y
	\end{tikzcd} \quad \leftrightsquigarrow \quad
	\begin{tikzcd}
		a\ar[d,"i"']\ar[r] & Gx\ar[d,equal] \\
		b\ar[r,"h'"]\ar[d, equal] & Gx\ar[d,"Gf"] \\
		b\ar[r] & Gy
	\end{tikzcd}
\]
which completes the argument.
\end{proof}
\begin{lemma}\label{lemma:functor-category-adjunction}
	Let \(
	\begin{tikzcd}[cramped]
		\calC\ar[r,bend left,"F",""{name=A,below}] & \calD \ar[l,bend left,"G",""{name=B,above}]\ar[from=A,to=B,symbol=\dashv]
	\end{tikzcd}
	\) be an adjunction. Then, for any category \(\calE\), there is an induced adjunction
	\[
		\begin{tikzcd}[cramped]
			\Fun(\calE,\calC)\ar[r,bend left,"F_*",""{name=A,below}] & \Fun(\calE,\calD) \ar[l,bend left,"G_*",""{name=B,above}]\ar[from=A,to=B,symbol=\dashv]
		\end{tikzcd}
	\]
\end{lemma}
\begin{proof}
The functors \(F_*\) and \(G_*\) are defined by postcomposition, so e.g.\ \(F_*\!:H\mapsto F H\), and \((\sigma\!:H\To H') \mapsto (F\sigma\!: FH\To FH')\).
Let \(\eta\) and \(\varepsilon\) be the unit and counit of the adjunction \(F\ladj G\). We then obtain natural transformations
\[ \eta'\!: \1 \To G_*F_*,\quad \varepsilon'\!: F_*G_* \To \1 \]
given by
\[ \eta'_H := \eta H,\quad \varepsilon'_K := \varepsilon K. \]
These satisfy the triangle identities. Indeed, it suffices to check at each component, at which point one is checking the commutativity of the diagrams
\[
	\begin{tikzcd}[cramped, column sep=small]
		 & FGFH\ar[dr,Rightarrow,"\varepsilon FH"] &\\
		FH \ar[ur,Rightarrow,"F\eta H"]\ar[rr,equal] & & FH
	\end{tikzcd}\quad\text{and}\quad
	\begin{tikzcd}[cramped, column sep=small]
		 & GFGK \ar[dr,Rightarrow,"G \varepsilon K"] &\\
		GK \ar[ur,Rightarrow,"\eta GK"]\ar[rr,equal] & & FK
	\end{tikzcd}
\]
which follows from the original triangle identities.
\end{proof}

The above is pleasingly simple, but unfortunately also fairly uncommon in practice. Usually, one is in a somewhat more complex situation, involving a \emph{two-variable adjunction}
instead. Those are perhaps \emph{surprisingly} common, so we will take the time to explain them in detail, following \cite{riehl-categorical-homotopy-theory}. First, we need the notion of what a two-variable
adjunction is to begin with.
\begin{definition}
	A \emph{two-variable adjunction} consists of functors
	\[ -\otimes-\!: \calC\times\calD\to\calE,\quad \{-,-\}\!:\calC^\op\times\calE\to\calD,\quad [-,-]\!:\calD^\op\times\calE\to\calC, \]
	equipped with natural isomorphisms
	\[ \calE(c\otimes d, e) \cong \calD(d, \{c,e\}) \cong \calC(c,[d,e]) \]
	of functors \(\calC^\op\times\calD^\op\times\calE\to\Set\).
\end{definition}
\begin{remark}
	Here, we should think of the \(\{-,-\}\) and \([-,-]\) functors as ``internal Hom's'' of a sort. Indeed, typical examples of two-variable adjunctions come
	exactly from such situations, in which case \(\{-,-\} = [-,-]\). The easiest example is then when \(\otimes = \times\) and
	\[ \{-,-\} = [-,-] = \Set(-,-). \]
\end{remark}

Before we do anything serious with these, let's prove a small but useful lemma, which will be extremely helpful to us very soon.

\begin{lemma}\label{lemma:two-variable-adjunction-transposition-formula}
	Consider a two-variable adjunction
	\[ -\otimes-\!: \calC\times\calD\to\calE,\quad \{-,-\}\!:\calC^\op\times\calE\to\calD,\quad [-,-]\!:\calD^\op\times\calE\to\calC, \]
	equipped with natural isomorphisms having components
	\[ \alpha_{cde}\!: \calE(c\otimes d, e) \cong \calD(d, \{c,e\}),\quad \beta_{cde}\!: \calE(c\otimes d, e) \cong \calC(c,[d,e]). \]
	Suppose we have morphisms
	\[ i\!:c\to c', \quad j\!:d\to d',\quad h\!:c'\otimes d \to e,\quad h'\!: c\otimes d'\to e. \]
	Then
	\begin{align*}
		\alpha( c\otimes d \overset{i\otimes d}\longto c'\otimes d \overset{h}\longto e ) &= d \overset{\alpha(h)}\longto \{c',e\} \overset{ \{i,e\} }\longto \{c,e\},\text{ and} \\
		\beta( c\otimes d \overset{c\otimes j}\longto c\otimes d' \overset{h'}\longto e ) &= c \overset{\beta(h')}\longto [d',e] \overset{[j,e]}\longto [d,e].
	\end{align*}
\end{lemma}
\begin{proof}
We will prove the first statement, as the second is basically dual. Naturality of \(\alpha\) means we have the commutative diagram
\[
	\begin{tikzcd}
		\calE(c'\otimes d,e)\ar[d,"{(i\otimes d)^*}"'] \ar[r,"\sim","\alpha_{c'de}"'] & \calD(d,\{c',e\})\ar[d,"{\{i,e\}_*}"] \\
		\calE(c\otimes d,e)\ar[r,"\sim","\alpha_{cde}"'] & \calD(d,\{c,e\})
	\end{tikzcd}
\]
which says exactly that
\[ \alpha_{cde}(h\circ (i\otimes d)) = \{i,e\}\circ\alpha_{c'de}(h) \]
as desired.
\end{proof}

\begin{construction}[The ``Leibniz construction'']\label{construction:two-variable-adjunction-leibniz-construction}
	Suppose we have a two-variable adjunction
	\[ -\otimes-\!: \calC\times\calD\to\calE,\quad \{-,-\}\!:\calC^\op\times\calE\to\calD,\quad [-,-]\!:\calD^\op\times\calE\to\calC, \]
	and that the following properties hold.
	\begin{enumerate}[label=(\roman*)]
		\item \(\calC\) and \(\calD\) have pullbacks.
		\item \(\calE\) has pushouts.
	\end{enumerate}
	For simplicity, let us write \(\calB^\calA := \Fun(\calA,\calB)\). Then there is an induced two-variable adjunction
	\[ -\hat\otimes-\!: \calC^{[1]}\times\calD^{[1]}\to\calE^{[1]},\quad \hat{\{-,-\}}\!:(\calC^{[1]})^\op\times\calE^{[1]}\to\calD^{[1]},\quad \hat{[-,-]}\!:(\calD^{[1]})^\op\times\calE^{[1]}\to\calC^{[1]} \]
	on the level of arrow categories.

	Here's how they're defined on objects: the ``left adjoint'' \(-\hat\otimes-\) is given, on \(i\!:c\to c'\) in \(\calC\) and \(j\!:d\to d'\) in \(\calD\), by
	\[
		\begin{tikzcd}
			c\otimes d \ar[r, "i\otimes d"]\ar[d,"c\otimes j"'] & c'\otimes d \ar[d]\ar[ddr,bend left,"c'\otimes j"] & \\
			c\otimes d'\ar[r]\ar[drr,bend right=16,"i\otimes d'"'] & (c\otimes d') \underset{c\otimes d}\amalg(c'\otimes d)\ar[ul,pushout]\ar[dr,dashed,"i\hat\otimes j" description] & \\
			& & c'\otimes d'
		\end{tikzcd}
	\]
	which one easily verifies is functorial. The other functors are similarly defined, except using pullbacks instead. In particular, given \(f\!:e\to e'\), we have
	\[
		\begin{tikzcd}[column sep=small]
			\{c',e\}\ar[dr,dashed,"\hat{ \{ i,f \} }" description]\ar[drr,bend left=15,"{ \{i,e\} }"]\ar[ddr,bend right,"{ \{c',f\} }"'] & & \\
			         & \{c,e\}\underset{\{c,e'\}}\times \{c,e\} \ar[r]\ar[d] \ar[dr,pullback] & \{c,e\}\ar[d,"\{c{,}f\}"] \\
			         & \{c',e'\}\ar[r,"\{i{,}e'\}"'] & \{c,e'\}
		\end{tikzcd}\quad
		\begin{tikzcd}[column sep=small]
			[d',e]\ar[dr,dashed,"\hat{ [ j,f ] }" description]\ar[drr,bend left=15,"{ [j,e] }"]\ar[ddr,bend right,"{ [d',f] }"'] & & \\
			         & {[d,e]} \underset{[d,e']}\times [d,e] \ar[r]\ar[d] \ar[dr,pullback] & {[d,e]}\ar[d,"{[d,f]}"] \\
			         & {[d',e']}\ar[r,"{[j,e']}"'] & {[d,e']}
		\end{tikzcd}
	\]
	which one also easily verifies are functorial.

	We must now verify that this is actually a two-variable adjunction, meaning we want to show that
	\[ \calE^{[1]}(i\hat\otimes j, f) \cong \calD^{[1]}(j, \hat{ \{i,f\} }) \cong \calC^{[1]}(i, \hat{ [j,f] }). \]
	We will neglect checking naturality carefully, as it is not particularly insightful. Suppose we are given a morphism \((a,b)\!:i\hat\otimes j \To f\) as displayed below left;
	then, by Lemma \ref{lemma:two-variable-adjunction-transposition-formula}, we have natural one-to-one correspondences
	\begin{align*}
		\begin{tikzcd}[ampersand replacement=\&]
			(c\otimes d') \underset{c\otimes d}\amalg(c'\otimes d) \ar[d,"i\hat\otimes j"']\ar[r,"a"] \& e\ar[d,"f"] \\
			c'\otimes d'\ar[r,"b"] \& e'
		\end{tikzcd} \quad &\leftrightsquigarrow \quad
		\begin{tikzcd}[ampersand replacement=\&]
			c\otimes d\ar[r]\ar[d] \& c'\otimes d \ar[d]\ar[r,"a'"] \& e\ar[d,"f"] \\
			c\otimes d'\ar[r] \& c'\otimes d'\ar[r,"b"] \& e'
		\end{tikzcd} \\
		&\leftrightsquigarrow \quad
		\begin{tikzcd}[ampersand replacement=\&]
			d\ar[r,"\overline{a'}"]\ar[d,"j"'] \& \{c',e\} \ar[d]\ar[r] \& \{c,e\} \ar[d,"f"] \\
			d'\ar[r,"\overline{b}"] \& \{c',e'\}\ar[r] \& \{c,e'\}
		\end{tikzcd} \\
		&\leftrightsquigarrow \quad
		\begin{tikzcd}[ampersand replacement=\&]
			d\ar[r,"\overline{a'}"]\ar[d,"j"'] \& \{c',e\} \ar[d,"\hat{ \{i,f\} }"] \\
			d'\ar[r,"\overline{b}'"] \& \{c,e\}\underset{\{c,e'\}}\times \{c,e\}
		\end{tikzcd}
	\end{align*}
	The crux of the calculation for \(\hat{[-,-]}\) is the same.
\end{construction}

\begin{proposition}\label{prop:two-variable-adjunction-lifting-correspondence}
	Consider a two-variable adjunction
	\[ -\otimes-\!: \calC\times\calD\to\calE,\quad \{-,-\}\!:\calC^\op\times\calE\to\calD,\quad [-,-]\!:\calD^\op\times\calE\to\calC, \]
	where \(\calC\) and \(\calD\) admit pullbacks, and \(\calE\) admits pushouts. Suppose we have collections of morphisms \(C\subseteq\Mor(\calC)\), \(D\subseteq\Mor(\calD)\), and
	\(E\subseteq\Mor(\calE)\). Then
	\[ C\hat\otimes D \boxslash E \iff D\boxslash \hat{\{C,E\}} \iff C \boxslash \hat{[D,E]}. \]
\end{proposition}
\begin{proof}
A lifting problem \(i\hat\otimes j \To f\) transposes to \(j \To \hat{\{i,f\}}\) and \(i \To \hat{ [j,f] }\), and one easily sees that solutions also
transport across these.
\end{proof}
\begin{remark}
	We were inexplicit in the above proof to demonstrate how little is really going on. However, it is still insightful to show the inner workings of the calculation,
	so we will be explicit, instead, in this remark. Take the calculation from Construction \ref{construction:two-variable-adjunction-leibniz-construction};
	filling in a diagonal arrow for the desired lifts, we see that all that's happening is
	\begin{align*}
		\begin{tikzcd}[ampersand replacement=\&]
			(c\otimes d') \underset{c\otimes d}\amalg(c'\otimes d) \ar[d,"i\hat\otimes j"']\ar[r,"a"] \& e\ar[d,"f"] \\
			c'\otimes d'\ar[r,"b"]\ar[ur,dashed] \& e'
		\end{tikzcd} \quad &\leftrightsquigarrow \quad
		\begin{tikzcd}[ampersand replacement=\&]
			c\otimes d\ar[r]\ar[d] \& c'\otimes d \ar[d]\ar[r,"a'"] \& e\ar[d,"f"] \\
			c\otimes d'\ar[r] \& c'\otimes d'\ar[r,"b"] \ar[ur,dashed] \& e'
		\end{tikzcd} \\
		&\leftrightsquigarrow \quad
		\begin{tikzcd}[ampersand replacement=\&]
			d\ar[r,"\overline{a'}"]\ar[d,"j"'] \& \{c',e\} \ar[d]\ar[r] \& \{c,e\} \ar[d,"f"] \\
			d'\ar[r,"\overline{b}"']\ar[ur,dashed] \& \{c',e'\}\ar[r] \& \{c,e'\}
		\end{tikzcd} \\
		&\leftrightsquigarrow \quad
		\begin{tikzcd}[ampersand replacement=\&]
			d\ar[r,"\overline{a'}"]\ar[d,"j"'] \& \{c',e\} \ar[d,"\hat{ \{i,f\} }"] \\
			d'\ar[r,"\overline{b}'"']\ar[ur,dashed] \& \{c,e\}\underset{\{c,e'\}}\times \{c,e\}
		\end{tikzcd}
	\end{align*}
	where the intermediate step is entirely about ordinary ``one-variable'' adjunctions, and follows purely by Proposition \ref{prop:adjunction-lifting-correspondence}.
\end{remark}


%!TEX root = ../lectures.tex

\newcommand{\Cof}{\operatorname{Cof}}
\newcommand{\Fib}{\operatorname{Fib}}
\newcommand{\tailto}{\rightarrowtail}
\renewcommand{\simh}{\overset{h}{\sim}}

\section{Homotopy theory in model categories (WIP)}\label{lecture:model categories}

Model categories are, in short, a systematic framework for doing homotopy theory, meaning it is a tool for controlling localizations of categories. Previously, when trying to understand
relative categories \((\calC,W)\), we have made use of the assumption that \(W\) is a (right or left) multiplicative system, which allows us to understand the morphisms a lot more
cleanly due to the calculus of fractions.

Model categories take a different approach. Essentially, they introduce what one might describe as \emph{phantom structure:} the data of \emph{fibrations} and \emph{cofibrations,}
``meaningless'' on their own, the (imposed) properties of which allow you to describe the localization in terms very similar to the homotopy theory of topological spaces.
Essentially, the (co)fibrations provide distinguished classes of ``nice'' objects and morphisms. Model categories also provide a very convenient setting in which to understand homotopical analogues
of limits and colimits, though in a sense this is hardly unique to them, and is really a consequence of a good theory of derived functors.

For the material here, we mainly follow \cite{riehl-categorical-homotopy-theory}, \cite{cisinski-book}, and \cite{may-ponto-more-concise-algebraic-topology}.

\subsection{Model categories}
\begin{definition}
	Let \(\calC\) be a category. A \emph{model structure} on \(\calC\) is a triple \((\Cof,W,\Fib)\) of sets of morphisms in \(\calC\) satisfying the following conditions.
	\begin{enumerate}[label=(\arabic*)]
		\item The set \(W\) has the 2-out-of-3 property.
		\item The pairs \((\Cof,\Fib\cap W)\) and \((\Cof\cap W,\Fib)\) are weak factorization systems on \(\calC\).
	\end{enumerate}
	A category \(\calC\) with a model structure is called a \emph{model category} if, in addition, \(\calC\) admits finite limits and finite colimits.
\end{definition}
\begin{terminology}
	Given a model structure \((\Cof,W,\Fib)\) on \(\calC\), one calls a morphism \(f\) in \(\calC\)
	\begin{itemize}
		\item a \emph{cofibration} if \(f\in\Cof\),
		\item a \emph{fibration} if \(f\in\Fib\),
		\item a \emph{weak equivalence} if \(f\in W\), or
		\item a \emph{trivial (co)fibration} if \(f\) is both a weak equivalence and (co)fibration.
	\end{itemize}
	Suppose \(\calC\) is a model category. One says an object \(x\in\calC\) is
	\begin{itemize}
		\item \emph{fibrant} if \(x\to *\) is a fibration, where \(*\in\calC\) is the terminal object, or
		\item \emph{cofibrant} if \(\varnothing\to x\) is a cofibration, where \(\varnothing\in\calC\) is the initial object.
	\end{itemize}
\end{terminology}
\begin{remark}
	There are a few different variations on the definition of a model category, and we use the one in \cite{cisinski-book}. It is fairly common to assume a model category admits 
	\emph{all} small limits and colimits, and further, that the factorization systems are \emph{functorial,} meaning that the factorizations can be chosen
	to determine functors \(\Fun([1],\calC)\to\Fun([2],\calC)\). As many examples of model categories are produced by using the small object argument, which
	produces functorial factorizations, this is often satisfied, but it is also not really necessary for the theory to a large extent. Note, however, that
	a big convenience in having functorial factorizations is that choosing a \emph{(co)fibrant replacement} (obtained by factorizing either \(\varnothing\to x\) or \(x\to *\))
	determines a deformation in the sense of Lecture \ref{lecture:homotopical-algebra-through-deformations}, and thus the theory of derived functors becomes very nice.
\end{remark}
\begin{notation}
	We write \(\calC_c\) for the full subcategory of cofibrant objects; dually, \(\calC_f\) for the full subcategory of fibrant objects. Combining the two,
	we write \(\calC_{cf}\) for the full subcategory of objects which are both fibrant and cofibrant.
\end{notation}
\begin{notation}
	We adopt the following notation, in situations where no ambiguity is expected: \(\iso\) denotes a weak equivalence, \(\sur\) denotes a fibration, and \(\tailto\)
	denotes a cofibration.
\end{notation}
\begin{remark}
	Observe that a model structure has a large amount of redundant information: if one exists, it is completely determined by its weak equivalences
	and either the trivial fibrations or trivial cofibrations. This is a corollary of Proposition \ref{prop:factorization-system-equalities}; suppose we know
	\(W\) and \(\Fib\cap W\). Then
	\[ \Cof = \prescript{\boxslash}{}{(\Fib\cap W)},\quad \leadsto\quad \Fib = (\Cof\cap W)^\boxslash, \]
	which recovers the triple \((\Cof,W,\Fib)\).

	This also means that \(W\) contains a weakly saturated class of morphisms, namely \(\Fib\cap W\), and therefore contains all isomorphisms. In particular, all identities,
	so \(W\) determines a wide subcategory with the 2-out-of-3 property. We conclude that \((\calC,W)\) is a pseudo-homotopical category, in the terminology
	of Lecture \ref{lecture:homotopical-algebra-through-deformations}. In fact, we will see that \(W\) necessarily also satisfies the 2-out-of-6 property,
	so that \((\calC,W)\) is moreover a homotopical category. The reader should therefore not spend any time seeking examples of model categories whose weak
	equivalences do not satisfy the 2-out-of-6 property.
\end{remark}
\begin{remark}
	In many cases, model structures are constructed using the small object argument, such as the one we give in Proposition \ref{prop:small-object-argument}. That is,
	one has some class of basic cofibrations \(J\) from which one forms the factorization systems \((\prescript{\boxslash}{}{(J^\boxslash)},J^\boxslash)\) using the small
	object argument. Choosing some basic class of trivial cofibrations \(I\), one can then form \((\prescript{\boxslash}{}{(I^\boxslash)},I^\boxslash)\). If \(J\)
	and \(I\) are chosen such that \(J^\boxslash \subseteq W\) and \(\prescript{\boxslash}{}{(I^\boxslash)} \subseteq W\), then this will yield a model structure given by
	\[ (\Cof,W,\Fib) = (\prescript{\boxslash}{}{(J^\boxslash)},W,I^\boxslash). \]
	In general, model structures for which one can find \emph{some} collections of maps \(I\) and \(J\) generating the (trivial) cofibrations are called \emph{cofibrantly generated,} and they have a
	number of benefits. For example, since the small object argument provides functorial factorizations, any such model structure of course admits this extra structure.
\end{remark}
\begin{construction}[Opposites]
	Let \((\calC, \Cof, W, \Fib)\) be a model category. Then the triple \((\Fib^\op, W^\op, \Cof^\op)\) is a model structure on \(\calC^\op\), so \(\calC^\op\) can be promoted to a model category
	as well. Furthermore, this demonstrates that fibrations and cofibrations are dual to each other, such that any result about cofibrations dualizes to a result about fibrations.
\end{construction}
\begin{construction}[Slices]
	Let \(\calC\) be a model category, and let \(x\in\calC\). Then \(\calC/x\) can be promoted to a model category. In particular, let a morphism in \(\calC/x\) be a weak equivalence (resp.\ fibration, cofibration)
	if it is a weak equivalence (resp.\ fibration, cofibration) in \(\calC\). Dualizing yields that \(x/\calC\) can be promoted to a model category in a similar way.
\end{construction}

\begin{proposition}\label{prop:model-category-cof-saturated-fib-cosaturated}
	Let \((\calC,\Cof,W,\Fib)\) be a model category. Then \(\Cof\) is weakly saturated and \(\Fib\) is weakly cosaturated. In particular, whenever it makes sense,
	\begin{enumerate}[label=(\arabic*)]
		\item \(\Cof\) is closed under retracts, compositions, small coproducts, and pushouts, and
		\item \(\Fib\) is closed under retracts, compositions, small products, and pullbacks.
	\end{enumerate}
\end{proposition}
\begin{proof}
Follows by Proposition \ref{corollary:llp-weakly-saturated} and Proposition \ref{prop:factorization-system-equalities}.
\end{proof}

Recall the importance of functors being \emph{homotopical} in the construction of derived functors; see Lecture \ref{lecture:homotopical-algebra-through-deformations}. In a model category,
the morphisms which are easily controlled are the (co)fibrations, and the objects with nice properties are the (co)fibrant ones. Consequently, it is often the case that one can
more easily say things about, say, trivial cofibrations between cofibrant objects than about weak equivalences in general. The below result is useful in these circumstances.
\begin{lemma}[Ken Brown's lemma]
	Let \(\calC\) be a model category, and let \(\calD\) be a pseudo-homotopical category. Suppose we have a functor \(F\!:\calC\to\calD\). Then the following are equivalent.
	\begin{enumerate}[label=(\arabic*)]
		\item \(F\) is homotopical on the full subcategory of cofibrant objects, i.e.\ \(F\) sends weak equivalences between cofibrant objects to weak equivalences.
		\item \(F\) sends trivial cofibrations between cofibrant objects to weak equivalences.
	\end{enumerate}
\end{lemma}
\begin{proof}
The implication (1) \(\Rightarrow\) (2) is clear. For the converse, the idea is that we may rewrite any weak equivalence between cofibrant objects in terms of trivial cofibrations.
Suppose we have a morphism \(f\!:x\to y\) where \(x,y\in\calC_c\). Form their coproduct \(x\amalg y\) and factor the map \(x\amalg y \to y\) given by
\(f\!:x\to y\) and \(\id_y\!:y\to y\) into a cofibration followed by a trivial fibration
\[
	\begin{tikzcd}
		\varnothing\ar[r,tail]\ar[d,tail] & x\ar[d,"\iota_x"] \\
		y\ar[r,"\iota_y"] & x\amalg y\ar[ul,pushout]
	\end{tikzcd} \quad \leadsto \quad
	\begin{tikzcd}
		x\ar[d,tail,"\iota_x"']\ar[drr,bend left=20,"\sim","f"'] \\
		x\amalg y \ar[r,tail] & z \ar[r,"\sim"] & y \\
		y\ar[u,tail,"\iota_y"]\ar[urr,bend right=20,equal]
	\end{tikzcd}
\]
and, using the left diagram, note that since cofibrations are closed under pushouts, \(\iota_x\) and \(\iota_y\) are cofibrations. In particular, the compositions
\[ i\!:x \overset{\iota_x}\tailto x\amalg y \tailto z,\quad j\!:y \overset{\iota_y}\tailto x\amalg y \tailto z \]
are both cofibrations (by composition) and weak equivalences (by 2-out-of-3), hence trivial cofibrations. By commutativity, we have
\[ j\circ f = i \implies F(j)\circ F(f) = F(i)  \]
which implies \(F(f)\) is a weak equivalence by the 2-out-of-3 property.
\end{proof}
\begin{remark}
	The above result is a simple but good example of how (co)fibrations and (co)fibrant objects can help us despite not being intrinsically meaningful.
\end{remark}

Before we move on to showing how homotopy theory may be done in the context of a model category, let us prove a nice property of the weak equivalences.
Recall that isomorphisms are closed under retracts, as was shown in Lemma \ref{lemma:isomorphisms-closed-under-retracts}. This also holds for the weak
equivalences in a model category. The proof is interesting, as it demonstrates a typical approach employed in the context of a model structure,
namely to prove something for the (co)fibrations, and then lift it to all morphisms by using the factorization properties.

\begin{lemma}\label{lemma:model-category-fibration-retract-of-weak-equivalence-is-trivial-fibration}
	Let \(\calC\) be a model category. If a fibration in \(\calC\) is the retract of a weak equivalence, then it is a trivial fibration.
\end{lemma}
\begin{proof}
Let \(f\!:x\tailto y\) be fibration which is the retract of a weak equivalence \(w\!:x' \to y'\). First, factor \(w\) into a composition \(v\circ u\) using either factorization system,
and observe that by the 2-out-of-3 property, both \(u\) and \(v\) are weak equivalences (and hence a trivial cofibration and trivial fibration, respectively), so we have
\[
	\begin{tikzcd}
		x\ar[d,tail,"f"']\ar[r] & x'\ar[d,"w"', "\sim" labl]\ar[r] & x\ar[d,tail,"f"] \\
		y\ar[r] & y'\ar[r] & y
	\end{tikzcd}\quad\leadsto\quad
	\begin{tikzcd}
		x\ar[dd,tail,"f"']\ar[r] & x'\ar[d,tail,"u"', "\sim" labl]\ar[r] & x\ar[dd,tail,"f"] \\
		 & z'\ar[d,two heads,"v"', "\sim" labl] & \\
		y\ar[r] & y'\ar[r] & y
	\end{tikzcd}.
\]
Composing in the obvious way, we get a morphism \(s\!:x\to z'\), and by considering the lifting problem as below left (obtained by the other obvious composition), we thus have the diagram below right:
\[
	\begin{tikzcd}
		x'\ar[d,two heads,"u"', "\sim" labl]\ar[r] & x\ar[d,tail,"f"] \\
		z'\ar[r]\ar[ur,dashed,"t"] & y
	\end{tikzcd}\quad\leadsto\quad
	\begin{tikzcd}
		x\ar[dd,tail,"f"']\ar[r]\ar[dr,"s"'] & x'\ar[d,tail,"u"', "\sim" labl]\ar[r] & x\ar[dd,tail,"f"] \\
		 & z'\ar[d,two heads,"v"', "\sim" labl]\ar[ur,"t"'] & \\
		y\ar[r] & y'\ar[r] & y
	\end{tikzcd}.
\]
By commutativity, we see that \(t\circ s = \id\), so that \(f\) is a retract of \(v\). Since trivial fibrations are weakly cosaturated (by Proposition \ref{prop:model-category-cof-saturated-fib-cosaturated}),
they are closed under retracts, and hence \(f\) is a trivial fibration.
\end{proof}
\begin{proposition}
	Let \(\calC\) be a model category. Then the weak equivalences are closed under retracts.
\end{proposition}
\begin{proof}
Let \(f\!:x\to y\) be a retract of a weak equivalence \(w\!:x' \to y'\),
\[
	\begin{tikzcd}
		x\ar[d,"f"']\ar[r] & x'\ar[r,"r"]\ar[d,"w"',"\sim" labl] & x\ar[d,"f"] \\
		y\ar[r,"s"] & y'\ar[r] & y
	\end{tikzcd}.
\]
Factor \(f = h\circ g\) as a trivial cofibration \(g\) followed by a fibration \(h\). Our goal is to show that \(h\) is the retract of a weak equivalence,
so that, applying Lemma \ref{lemma:model-category-fibration-retract-of-weak-equivalence-is-trivial-fibration}, \(f\) is a composition of weak equivalences. To this end,
consider the pushout below left
\[
	\begin{tikzcd}
		x\ar[r]\ar[d,tail,"g"',"\sim" labl] & x'\ar[d,dashed,"i"] \\
		z\ar[r,dashed,"k"] & z'\ar[ul,pushout]
	\end{tikzcd}\quad\leadsto\quad
	\begin{tikzcd}
		x\ar[d,tail,"g"',"\sim" labl]\ar[r] & x'\ar[dd,bend left,"w" near start,"\sim" labl]\ar[d,tail,"i"',"\sim" labl]\ar[r] & x\ar[d,tail,"g"',"\sim" labl] \\
		z\ar[d,two heads,"h"']\ar[r,"k"'] & z'\ar[ul,pushout]\ar[d,dashed,"j"'] & z \ar[d,two heads,"h"] \\
		y\ar[r] & y'\ar[r] & y
	\end{tikzcd}
\]
and note that trivial cofibrations, beeing weakly saturated, are closed under pushouts. Hence, \(i\) is a trivial cofibration. The morphism \(j\) in the diagram
above right is induced by universal property applied to the morphisms \(w\) and \(s\circ h\); since \(j\circ i = w\) and both \(i\) and \(w\) are weak equivalences,
we deduce that \(j\) is a weak equivalence.

For the final step, observe that the identity \(\id_z\!:z\to z\) together with the composition \(g\circ r\) induce, by universal property, a
map \(\ell\!:z'\to z\) for which \(\ell\circ k = \id_z\) and \(\ell\circ i = g\circ r\). By universal property, one can check that the remaining bottom right square in the diagram
\[
	\begin{tikzcd}
		x\ar[d,tail,"g"',"\sim" labl]\ar[r] & x'\ar[d,tail,"i"',"\sim" labl]\ar[r] & x\ar[d,tail,"g"',"\sim" labl] \\
		z\ar[d,two heads,"h"']\ar[r,"k"'] & z'\ar[ul,pushout]\ar[d,"j"']\ar[r,dashed,"\ell"] & z \ar[d,two heads,"h"] \\
		y\ar[r] & y'\ar[r] & y
	\end{tikzcd}
\]
commutes (by using that there is a unique map \(z'\to y\) which after composition with \(i\) or \(k\) yields the outer paths of the square), so the entire diagram commutes.
In particular, \(h\) is a retract of the weak equivalence \(j\).
\end{proof}

\subsection{Cylinders, paths, and homotopies}
Recall that in standard algebraic topology, one has the notion of a \emph{homotopy} between two continuous maps \(f,g\!:X\to Y\). Intuitively,
this is some kind of continuous family of maps \(X\to Y\) which explains how to ``deform'' \(f\) into \(g\) (or vice versa). Formally, this is usually captured by
having a continuous map \(H\!:[0,1]\times X\to Y\) such that \(H(0,x) = f(x)\) and \(H(1,x) = g(x)\) for all \(x\in X\).

On the other hand, whenever it exists, one can also encode the notion of a homotopy as a path in the \emph{function space} \(Y^X\). The issue with this is that
the function space \emph{doesn't} always exist, in the sense that one may fail to have a bijection
\[ \Top([0,1]\times X, Y) \cong \Top([0,1],Y^X). \]
Still, when it does happen to make sense, it's a very intuitive way to define homotopies. Slightly less intuitive is the expression obtained by considering
the other possible adjunction,
\[ \Top([0,1]\times X, Y) \cong \Top(X,Y^{[0,1]}), \]
passing through the \emph{path space,} where a homotopy from \(f\) to \(g\) is a continuous map \(h\!:X\to Y^{[0,1]}\) such that \(h(x)(0) = f(x)\) and \(h(x)(1) = g(x)\).

These two approaches can be partially employed in any relative category, and especially well in the context of a model structure.
\begin{definition}
	Let \((\calC,W)\) be a relative category, and let \(x\in \calC\) be an object.
	\begin{enumerate}[label=(\roman*)]
		\item A \emph{cylinder object} for \(x\) is an object \(Ix\in\calC\) together with maps
		\[
			\begin{tikzcd}
				x\amalg x \ar[r,"{ (i_0, i_1) }"] & Ix \ar[r,"p"] & x
			\end{tikzcd}
		\]
		such that \(p\circ i_k = \id_x\), \(k=0,1\), and \(p\in W\).
		\item A \emph{path object} for \(x\) is an object \(x^I\in\calC\) together with maps
		\[
			\begin{tikzcd}
				x\ar[r,"i"] & x^I \ar[r,"{ (p_0,p_1) }"] & x\times x
			\end{tikzcd}
		\]
		such that \(p_k\circ i = \id_x\), \(k=0,1\), and \(i\in W\).
	\end{enumerate}
\end{definition}
\begin{remark}
	Note that these are not defined by universal property, and so do not need to be unique in any way, a priori.
\end{remark}
\begin{remark}
	If \((\calC,W)\) is pseudo-homotopical in the sense of Lecture \ref{lecture:homotopical-algebra-through-deformations}, then the 2-out-of-3 property
	shows that the \(i_k\) and \(p_k\) are weak equivalences.
\end{remark}

\begin{definition}
	Let \(\calC,W\) be a relative category, and let \(f,g\!:x\to y\) be morphisms in \(\calC\).
	\begin{enumerate}[label=(\roman*)]
		\item Consider a cylinder object \(Ix\) for \(x\). A \emph{left homotopy} from \(f\) to \(g\) (with respect to \(Ix\)) is a morphism \(h\!:Ix \to y\)
		such that \(h\circ i_0 = f\) and \(h\circ i_1 = g\). We write \(\simh_\ell\) for the equivalence relation on \(\calC(x,y)\) generated by the existence of
		a cylinder object for \(x\) together with a left homotopy from one morphism to another, and set
		\[ [x,y]_\ell := \calC(x,y)/\simh_\ell. \]
		\item Consider a path object \(y^I\) for \(y\). A \emph{right homotopy} from \(f\) to \(g\) (with respect to \(y^I\)) is a morphism \(h'\!:x\to y^I\)
		such that \(p_0\circ h' = f\) and \(p_1\circ h' = g\). We write \(\simh_r\) for the equivalence relation on \(\calC(x,y)\) generated by the existence of
		a path object for \(y\) together with a right homotopy from one morphism to another, and set
		\[ [x,y]_r := \calC(x,y)/\simh_r. \]
	\end{enumerate}
\end{definition}

\begin{remark}
	A warning to observe is that \(f \simh_\ell g\) (and its dual) does \emph{not} merely ask that there is a left homotopy from \(f\) to \(g\), since that is not
	necessarily an equivalence relation. This is why we must ask for the equivalence relation \emph{generated} by that simpler relation. In general, \(f\simh_\ell g\)
	means that there is some chain of left homotopies starting at \(f\) and ending with \(g\). See the proof of Proposition \ref{prop:model-category-cofibrant-object-left-homotopy-equivalence-relation}
	for an explanation of what properties the relation of left homotopy satisfies, and when it forms an equivalence relation.
\end{remark}
\begin{remark}
	We want to emphasize that the relations \(\simh_\ell\) and \(\simh_r\) do not require a fixed choice of cylinder object or path object, and rather allow any
	choice for defining a homotopy.
\end{remark}

At this level of generality, there is not much one can say, but with a small added hypothesis, one can at least say the following:
\begin{lemma}
	Let \((\calC,W)\) be a pseudo-homotopical category, and suppose we have morphisms \(f,g\!:x\to y\) and a left homotopy \(h\) from \(f\) to \(g\). Then
	\(f\) is a weak equivalence if and only if \(g\) is a weak equivalence.
\end{lemma}
\begin{proof}
We have the diagram
\[
	\begin{tikzcd}
		x\ar[d,"i_0"',"\sim" labl]\ar[dr,"f"] & \\
		Ix \ar[r,"h"] & y \\
		x \ar[u,"i_1","\sim"' labl] \ar[ur,"g"'] &
	\end{tikzcd}
\]
and therefore \(f\in W\) if and only if \(h\in W\) if and only if \(g\in W\), by the 2-out-of-3 property.
\end{proof}
\begin{remark}
	By the description of the equivalence relation \(\simh_\ell\), it follows that if \(f\simh_\ell g\) then \(f\in W\) if and only if \(g\in W\).
\end{remark}

The above definitions can be made in any relative category, as we have done, but typically one will work within a model category, in which case one can add some more conditions
to make the definitions substantially more powerful.
\begin{definition}
	Let \(\calC\) be a model category, and let \(x\in\calC\) be an object.
	\begin{enumerate}[label=(\roman*)]
		\item Consider a cylinder object \((Ix,i_0,i_1,p)\) for \(x\). We say \(Ix\) is a \emph{good} cylinder object if \((i_0,i_1)\!:x\amalg x\to Ix\) is a cofibration,
		We say \(Ix\) is a \emph{very good} cylinder object if it is good, and \(p\) is a trivial fibration. A left homotopy is (very) good if the corresponding cylinder object is (very) good.
		\item Consider a path object \((x^I,i,p_0,p_1)\) for \(x\). We say \(x^I\) is a \emph{good} path object if \((p_0,p_1)\!:x^I\to x\times x\) is a fibration,
		We say \(x^I\) is a \emph{very good} path object if it is good, and \(i\) is a trivial cofibration. A left homotopy is (very) good if the corresponding path object is (very) good.
	\end{enumerate}
\end{definition}
\begin{lemma}\label{lemma:model-category-good-cylinder-of-cofibrant-object-inclusions-cofibrant}
	Let \(\calC\) be a model category, let \(x\in\calC_c\) be a cofibrant object, and let \((Ix,i_0,i_1,p)\) be a good cylinder object. Then \(i_0\) and \(i_1\)
	are trivial cofibrations.
\end{lemma}
\begin{proof}
We know that \(i_0\) and \(i_1\) are weak equivalences by the 2-out-of-3 property, so we need to show they are cofibrations. However,
the canonical inclusions \(\iota_k\!:x\to x\amalg x\) are cofibrations (being pushouts of the cofibration \(\varnothing\to x\)), and
\[ i_k = (i_0,i_1)\circ\iota_k \]
so that \(i_k\), \(k=0,1\), are cofibrations.
\end{proof}
\begin{proposition}\label{prop:model-category-good-homotopies-are-enough}
	Let \(\calC\) be a model category. Then the following statements hold.
	\begin{enumerate}[label=(\arabic*)]
		\item For any object \(x\in\calC\) there exists a very good cylinder object and a very good path object.
		\item For two morphisms \(f,g\!:x\to y\), there exists a left (resp.\ right) homotopy from \(f\) to \(g\) if and only if there exists a good left (resp.\ right) homotopy from \(f\) to \(g\).
	\end{enumerate}
\end{proposition}
\begin{proof}
(1) Factorize the codiagonal and diagonal maps
\[ \nabla_x\!: x\amalg x \to x,\quad \Delta_x\!:x\to x\times x \]
using the appropriate factorization system.

(2) Obviously, one direction is trivial. For other, let \(h\!:Ix\to y\) be a left homotopy from \(f\) to \(g\). The strategy is similar to (1), except that we factorize the
map \((i_0,i_1)\!:x \amalg x \to Ix\) to get
\[
	\begin{tikzcd}
		x\amalg x \ar[r,tail,"{(i'_0,i'_1)}"] & I'x \ar[r,two heads,"\sim" labl] & Ix
	\end{tikzcd}
\]
which now defines a good cylinder, where the map \(p'\!:I'x \to x\) is defined by composing \(I'x\overset{\sim}\sur Ix\overset{\sim}\to x\). One immediately
sees that the morphism \(h'\) given by the composition \(I'x \to Ix \overset{h}\to y\) defines a good homotopy from \(f\) to \(g\). The proof for right homotopies is similar.
\end{proof}

Below, we give trivial versions of the homotopy extension and covering homotopy properties from algebraic topology in the context of model categories.

\begin{lemma}[Homotopy extension property]
	Let \(\calC\) be a model category, \(\iota\!:a\to x\) a cofibration, and \(y\) a fibrant object. Let \(y^I\) be a good path object.
	Then the lifting problem
	\[
		\begin{tikzcd}
			a\ar[d,tail,"\iota"']\ar[r] & y^I\ar[d,"p_0"] \\
			x\ar[r]\ar[ur,dashed] & y
		\end{tikzcd}
	\]
	has a solution.
\end{lemma}
\begin{proof}
Since \(y^I\) is good, \(p_0\) is a trivial fibration, so we conclude by the factorization system \((\Cof,W\cap\Fib)\).
\end{proof}
Similarly, one proves
\begin{lemma}[Covering homotopy property]
	Let \(\calC\) be a model category, \(\pi\!:e\to b\) a fibration, and \(x\) a cofibrant object. Let \(Ix\) be a good cylinder object.
	Then the lifting problem
	\[
		\begin{tikzcd}
			x\ar[d,"i_0"']\ar[r] & e\ar[d,two heads,"\pi"] \\
			Ix\ar[r]\ar[ur,dashed] & b
		\end{tikzcd}
	\]
	has a solution.
\end{lemma}
\begin{remark}
	We note that the choice of using \(i_0\) and \(p_0\) is somewhat arbitrary, as the same argument applies to \(i_1\) and \(p_1\), or even
	\((i_0,i_1)\!:x\amalg x \to Ix\) and \((p_0,p_1)\!:y^I \to y\times y\). In this way, one can see that the definition of a good cylinder/path object
	is exactly such that the above properties hold.
\end{remark}

We remarked earlier that since the existence of a left homotopy doesn't form an equivalence relation, we are forced to consider the equivalence relation generated by the former
to get \(\simh_\ell\). Naturally, one wonders what the obstruction is, as in standard algebraic topology, homotopy is easily seen to be an equivalence relation.
The below proposition provides an answer.

\begin{proposition}\label{prop:model-category-cofibrant-object-left-homotopy-equivalence-relation}
	Let \(\calC\) be a model category, let \(x,y\in\calC\) and assume \(x\) is cofibrant. Then, for morphisms \(f,g\in\calC(x,y)\), we have \(f\simh_\ell g\) if and only if there exists a left homotopy
	from \(f\) to \(g\). That is, the existence of a left homotopy forms an equivalence relation.
\end{proposition}
\begin{proof}
We have three things to check: reflexivity, symmetry, and transitivity.
\begin{enumerate}[label=(\arabic*)]
	\item Reflexivity: let \(f\!:x\to y\) be a morphism. Note that \((x,\id_x,\id_x,\id_x)\) forms a (very good) cylinder object for \(x\), so \(f\) itself
	determines a homotopy from \(f\) to \(f\).
	\item Symmetry: let \(f,g\!:x\to y\) be morphisms, and \(h\!: Ix\to y\) a homotopy from \(f\) to \(g\). Note that given the cylinder \((Ix,i_0,i_1,p)\),
	the tuple \((Ix,i_1,i_0,p)\) also defines a cylinder object for \(x\). Then the homotopy \(h\), now thought of in terms of the latter cylinder, defines a homotopy
	from \(g\) to \(f\).
	\item Transitivity: this is the only step where we need to use that \(x\) is cofibrant. Assume that we have \(f,g,k\!:x\to y\) and left homotopies \(h\!:Ix\to y\) from \(f\) to \(g\) and \(h'\!:I'x\to y\) from
	\(g\) to \(k\); by Proposition \ref{prop:model-category-good-homotopies-are-enough}, we may assume that these left homotopies are good. We form a third cylinder \(Jx\)
	as the pushout below left
	\[
		\begin{tikzcd}
			& Ix\ar[dr,dashed]\ar[drr,bend left,"p"] & \\
			x\ar[ur,"i_1"]\ar[dr,"i'_0"'] & & Jx\ar[ll,pushout]\ar[r,dashed,"q"] & x \\
			& I'x\ar[ur,dashed]\ar[urr,bend right,"p'"] &
		\end{tikzcd}
		\quad\leadsto\quad
		\begin{tikzcd}
			& Ix\ar[dr]\ar[drr,bend left,"h"] & \\
			x\ar[ur,"i_1"]\ar[dr,"i'_0"'] & & Jx\ar[ll,pushout]\ar[r,dashed,"h''"] & y \\
			& I'x\ar[ur]\ar[urr,bend right,"h'"] &
		\end{tikzcd}
	\]
	where, to be explicit, we set \(j_0 = x \overset{i_0}\to Ix \to Jx\) and \(j_1 = x\overset{i'_1}\to I'x\to Jx\). The cylinder is then given by \((Jx,j_0,j_1,q)\),
	but we need to check that this is actually a cylinder object for \(x\), and this is where being cofibrant is essential. The only non-trivial aspect is checking that \(q\in W\). Here,
	Lemma \ref{lemma:model-category-good-cylinder-of-cofibrant-object-inclusions-cofibrant} tells us that \(i_1\) and \(i'_0\) are trivial cofibrations, and thus their pushouts are
	as well, so by the 2-out-of-3 property, \(q\) is a weak equivalence. We now get an induced left homotopy \(h''\!:Jx\to y\) from \(f\) to \(k\), as displayed above right.
\end{enumerate}
This concludes the proof, as \(\simh_\ell\) is the equivalence relation generated by the existence of a left homotopy.
\end{proof}

In the model categorical setting, we in principle allow any choice of cylinder object (resp.\ path object) for a left (resp.\ right) homotopy. This is
in contrast with the classical theory in algebraic topology, where one uses an explicit choice, namely the product with the interval \([0,1]\). However,
much like the above, where everything simplifies significantly with a (co)fibrancy assumption, the same is true with regards to this. Below, we see that
when the domain is cofibrant and the codomain is fibrant, homotopy theory looks exactly as it does classically.

\begin{propositiondef}
	Let \(\calC\) be a model category, and let \(f,g\!:x\to y\) be morphisms in \(\calC\).
	\begin{enumerate}[label=(\arabic*)]
		\item Suppose \(x\) is cofibrant. If there is a left homotopy from \(f\) to \(g\), then there is a right homotopy from \(f\) to \(g\).
		\item Suppose \(y\) is fibrant. If there is a right homotopy from \(f\) to \(g\), then there is a left homotopy from \(f\) to \(g\).
	\end{enumerate}
	In particular, if \(x\in\calC_c\) and \(y\in\calC_f\), then the equivalence relations \(\simh_\ell\) and \(\simh_r\) on \(\calC(x,y)\)
	agree, and give an equivalence relation \(f\simeq g\). Furthermore, for the equivalence relation \(\simeq\), one may use a fixed good cylinder
	object for \(x\) and a fixed good path object for \(y\).
\end{propositiondef}
\begin{proof}
Statements (1) and (2) are dual, so it suffices to prove (1). In the process, we will see that we can translate an arbitrary (good) left homotopy from \(f\) to \(g\)
to a good right homotopy from \(f\) to \(g\) with a fixed path object for \(y\). So, assume we have a left homotopy \(h\!:Ix\to y\) from \(f\) to \(g\), where by Proposition \ref{prop:model-category-good-homotopies-are-enough}
we may assume that \(Ix\) is good. Fix a good path object \(y^I\) for \(y\), and note that \(i_0\!:x\to Ix\) is a trivial cofibration and \((p_0,p_1)\!:y^I \to y\times y\)
is a fibration, we have  a solution to the lifting problem
\[
	\begin{tikzcd}
		x\ar[r,"f"]\ar[d,"i_0"'] & y \ar[r,"i"] & y^I\ar[d,"{(p_0,p_1)}"] \\
		Ix\ar[r,"{(p,\id_{Ix})}"']\ar[urr,dashed,"k"] & x\times Ix \ar[r,"{f\times h}"'] & y\times y
	\end{tikzcd}
\]
where we now observe that
\[ h'\!: x \overset{i_1}\to Ix \overset{k}\to y^I \]
is a right homotopy from \(f\) to \(g\). Indeed
\[ p_0\circ h' = p_0\circ k\circ i_1 = f\circ p\circ i_1 = p_0\circ f\quad\text{and}\quad p_1\circ h' = h\circ i_1 = g \]
as desired. The rest of the proposition follows.
\end{proof}

\begin{notation}
	Let \(\calC\) be a model category. For \(x\) cofibrant and \(y\) fibrant, we write
	\[ [x,y] = \calC(x,y)/\simeq \]
	for the set of morphisms up to homotopy.
\end{notation}

\subsection{The homotopy category of bifibrant objects (TBD)}

\begin{lemma}
	If \(y\in\calC_f\), then there is a good left homotopy from \(f\) to \(g\) if and only if there is a very good left homotopy from \(f\) to \(g\).
\end{lemma}
\begin{proposition}
	If \(y\in\calC_f\) and there is a left homotopy from \(f\) to \(g\), then there is a left homotopy from \(f\circ e\) to \(g\circ e\).
\end{proposition}
\begin{propositiondef}
	Let \(\calC\) be a model category. Then the following statements hold.
	\begin{enumerate}[label=(\arabic*)]
		\item If \(y\) is fibrant, then composition in \(\calC\) induceds a composition
			\[ [x,y]_\ell \times [w,x]_\ell \to [w,y]_\ell. \]
		\item If \(w\) is cofibrant, then composition in \(\calC\) induces a composition
			\[ [x,y]_r \times [w,x]_r \to [w,y]_r. \]
	\end{enumerate}
	In particular, there is a category \(\hh{\calC_{cf}}\), the objects of which are the objects of \(\calC\), and for which the morphisms are given by
	\[ \hh{\calC_{cf}}(x,y) = [x,y]. \]
\end{propositiondef}

\begin{terminology}
	Let \(x,y\in\calC_{cf}\). A morphism \(f\!:x\to y\) is a \emph{homotopy equivalence} if it has an inverse up to homotopy, i.e.\ if its image in \(\hh{\calC_{cf}}\)
	is an isomorphism.
\end{terminology}

\begin{theorem}
	Let \(\calC\) be a model category, let \(x,y\in\calC_{cf}\), and let \(f\!:x\to y\) be a morphism. Then the following are equivalent.
	\begin{enumerate}[label=(\arabic*)]
		\item \(f\) is a homotopy equivalence.
		\item \(f\) is a weak equivalence.
	\end{enumerate}
\end{theorem}

\begin{theorem}
	Let \(\calC\) be a model category. Then there is a canonical equivalence of categories \(\hh{\calC_{cf}} \iso \ho(\calC)\), where we recall that \(\ho(\calC) = \calC[W^{-1}]\).
\end{theorem}


% alternative crackpot approach, seems super sketchy

% \begin{lemma}
% 	Let \(\calC\) be a model category, and let \(W_{cf} = \calC_{cf}\cap W\). Then the canonical functor
% 	\[ \calC_{cf}[W_{cf}^{-1}] \to \calC[W^{-1}] \]
% 	is an equivalence.
% \end{lemma}
% \begin{proof}
% We show that it is fully faithful and essentially surjective. To see that it is full, note that if \(x\in\calC_{cf}\), then applying the \((\Cof,\Fib\cap W)\) factorization system to a morphism
% \(f\!:x\to y\), we get
% \[
% 	\begin{tikzcd}
% 		\varnothing\ar[r,"\in\Cof"] & x \ar[dr,"\Cof\ni"']\ar[rr,"f"] & & y \ar[r,"\in\Fib"] & * \\
% 		& & z\ar[ur,"\in\Fib\cap W"'] & &
% 	\end{tikzcd}
% \]
% so that \(f\) factors through some \(z\in\calC_{cf}\). Furthermore, if \(f\in W\), then so is \(x\to z\) by the 2-out-of-3 property. In particular, any
% zigzag starting and ending in \(\calC_{cf}\) can be replaced by one living entirely in \(\calC_{cf}\). This shows that the functor is full. To see
% that it is faithful, note that we may always assume two zigzags have the same length and are thus related by a diagram of the form
% \[
% 	\begin{tikzcd}[cramped, row sep=tiny]
% 		 & \bullet\ar[dd] & \bullet\ar[l,"\in W"']\ar[r]\ar[dd] & \cdots & \bullet\ar[l,"\in W"']\ar[dr]\ar[dd] & \\
% 		x \ar[ur]\ar[dr] & & & & & y. \\
% 		& \bullet & \bullet\ar[l,"\in W"']\ar[r] & \cdots & \bullet\ar[l,"\in W"']\ar[ur] &
% 	\end{tikzcd}
% \]
% potentially with some arrows going in the same directions. The point is thus that if we have multiple of these, where the beginning top row and ending bottom row
% live in \(\calC_{cf}\), then we just get a single one where every object is in \(\calC_{cf}\) by composing.

% To see that the functor is essentially surjective, note that any object \(x\in\calC\) has a weak equivalence \(x'\iso x\) where \(x'\) is cofibrant and
% a weak equivalence \(x\to x''\) where \(x''\) is fibrant.
% \end{proof}



%!TEX root = ../lectures.tex

\section{Model categories \& homotopy (co)limits (WIP)}\label{lecture:model-categories-and-homotopy-colimits}
In Lecture \ref{lecture:model-categories}, we laid out the basic theory of model categories, and indicated that their structure can be used to understand localizations
in a very convenient way. As part of the road to formalizing this, we showed that weak equivalences between \emph{bifibrant objects} are exactly given by
\emph{homotopy equivalences,} defined in terms of mapping cylinders (or path spaces, equivalently).

The purpose of this lecture is to continue down the path started in the last lecture. We will show that the localization \(\ho(\calC) = \calC[W^{-1}]\) of
a model category \(\calC\) at its weak equivalences \(W\) is equivalent to the homotopy category of bifibrant objects \(\hh{\calC_{cf}}\), whose objects
are the bifibrant objects, and whose morphisms are given by homotopy-equivalence classes of maps in \(\calC_{cf}\).

After doing the above, we will move to explaining some features of derived categories in the setting of model categories, and one particularly interesting example,
namely homotopy (co)limits. In Lectures \ref{lecture:triangulated-categories} through to \ref{lecture:gluing-t-structures}, we explored topics in triangulated categories,
and frequently made use of the intuition of \emph{cones} as ``homotopy cokernels.'' Homotopy colimits in model categories provide a refinement of this notion.

We will use the same notation as in Lecture \ref{lecture:model-categories}. The sources for the contents are primarily the same also, namely
\cite{cisinski-book,riehl-categorical-homotopy-theory,may-ponto-more-concise-algebraic-topology}.

\subsection{The homotopy category of a model category (WIP)}

We start with a construction which is integral to our interests of localization, and which are a big part of the reason model categories are useful.

\begin{construction}[(Co)fibrant replacement]\label{construction:model-category-(co)fibrant-replacement}
	Consider a model category \(\calC\), and some object \(x\in\calC\). There are two operations afforded to us for free by the factorization systems on \(\calC\),
	namely:
	\begin{itemize}
		\item We can produce a \emph{fibrant replacement} of \(x\). That is, a trivial cofibration \(x \iso Rx\) where \(Rx\) is fibrant. To do this,
		factor \(x \to *\) into a trivial cofibration followed by a fibration. Note that if \(x\) is cofibrant, then \(Rx\) is cofibrant.
		\item We can produce a \emph{cofibrant replacement} of \(x\). That is, a trivial fibration \(Qx \iso x\), where \(Qx\) is cofibrant. To do this,
		factor \(\varnothing\to x\) into a cofibration followed by a trivial fibration. Note that if \(x\) is fibrant, then \(Qx\) is fibrant.
	\end{itemize}
	When \(\calC\) has functorial factorizations, the above operations organize into deformations \(Q\To\1\) and \(\1\To R\). As we do not assume we have functorial factorizations,
	we will not make use of this fact. On the other hand, one can still get some amount of naturality for free.

	Suppose we have a morphism \(f\!:x\to x'\), and that we fix cofibrant replacements
	\[ q\!:Qx\iso x,\quad q'\!:Qx'\iso x'\]
	and fibrant replacements
	\[ r\!:x\to Rx,\quad r'\!:x'\to Rx'. \]
	By solving the below lifting problems
	\[
	\begin{tikzcd}
		\varnothing\ar[d,tail]\ar[rr] & & Qx'\ar[d,two heads,"q'"',"\sim" labl] \\
		Qx\ar[r,"q"']\ar[urr,dashed,"Qf"] & x\ar[r,"f"'] & x'
	\end{tikzcd}
	\quad
	\begin{tikzcd}
		x\ar[d,tail,"\sim" labl,"r"']\ar[r,"f"] & x'\ar[r,"r'"] & Rx'\ar[d,two heads] \\
		Rx\ar[rr]\ar[urr,dashed,"Rf"] & & *
	\end{tikzcd}
	\]
	we obtain morphisms \(Qf\!:Qx\to Qx'\) and \(Rf\!:Rx\to Rx'\). Observe that these satisfy \(q'\circ Qf = f\circ q\) and \(Rf \circ r = r'\circ f\).
\end{construction}
\begin{proposition}\label{prop:model-category-replacement-properties}
	Let \(\calC\) be a model category, and let \(f\!:x\to x'\) be a morphism in \(\calC\). Consider the morphism \(Qf\!:Qx\to Qx'\) of
	Construction \ref{construction:model-category-(co)fibrant-replacement}. Then the following statements hold.
	\begin{enumerate}[label=(\arabic*)]
		\item The morphism \(f\) is a weak equivalence if and only if \(Qf\) is a weak equivalence.
		\item The left and right homotopy classes of \(Qf\) depends only on the left homotopy class of \(f\circ q\).
		\item If \(x'\) is fibrant, then the right homotopy class of \(Qf\) depends only on the right homotopy class of \(f\).
	\end{enumerate}
\end{proposition}
\begin{proof}
(1) Apply the 2-out-of-3 property to the defining diagram of \(Qf\).

(2) By Theorem \ref{thm:model-category-dual-whitehead}, we have a bijection
\[ q'_*\!:[Qx,Qx']_\ell \iso [Qx,x']_\ell. \]
Furthermore, by Proposition \ref{prop:model-category-left-right-homotopy-interchange}, anything in the left homotopy class of \(Qf\) is also in the right homotopy class of \(Qf\).

(3) Note that \(Qx'\) is bifibrant. Applying Theorem \ref{thm:model-category-dual-whitehead} once again, along with Proposition Proposition \ref{prop:model-category-fibrant-codomain-precomposition}
and Proposition \ref{prop:model-category-left-right-homotopy-interchange}, we have
\[ [x,x']_r \to [x,x']_\ell \to [Qx,x'] \xleftarrow{\sim} [Qx,Qx'] \]
which implements \(Q\), where we note that since \(Qx\) is cofibrant and \(x'\) is fibrant, all homotopy mapping spaces agree, hence why we discard the subscript.
\end{proof}

\begin{construction}[Replacement functors]
	With the help of Proposition \ref{prop:model-category-replacement-properties}, we can construct replacement functors up to homotopy. In particular, we can construct
	functors
	\[ Q\!:\calC\to\hh{\calC_c},\quad R\!:\calC\to\hh{\calC_f}. \]
	We handle the former, since the latter is suitably dual. For all \(x\in\calC\), choose cofibrant replacements \(q_x\!:Qx\iso x\), and define \(Q\) on objects by
	\(x\mapsto Qx\). Using these cofibrant replacements, apply Construction \ref{construction:model-category-(co)fibrant-replacement} on morphisms to define \(Qf\) for all \(f\!:x\to y\).
	On morphisms, define \(Q\) by
	\[ \calC(x,y) \to \calC(Qx,Qy) \to [Qx,Qy]_r = \hh{\calC_c}(Qx,Qy). \]
	We need to show that this is a functor. First of all, when \(x=y\) and \(f=\id_x\), one observes that \(Q\id_x\) only depends on the left homotopy class of \(q_x\), which corresponds to \(\id_x\),
	so that \(Q\id_x = \id_{Qx}\).

	Suppose we have morphisms \(x\overset{f}\to y\overset{g}\to z\). We need to see that \(Q(g\circ f) = Q(g)\circ Q(f)\), but for this, we note that \(Q(g\circ f)\) depends only
	on the left homotopy class of \(g\circ f\circ q_x = g\circ q_y \circ Q(f) = q_z\circ Q(g)\circ Q(f)\) and this is uniquely determined by the left homotopy class
	of \(Q(g)\circ Q(f)\).

	Thus, we see that we have a functor \(Q\!:\calC\to\hh{\calC_{c}}\), and dually, a functor \(R\!:\calC\to\hh{\calC_{f}}\). When we restrict the former to consider
	only fibrant objects, we can use the construction in Proposition \ref{prop:model-category-replacement-properties}(3) to see that it induces a functor
	\[ \hh{Q}\!:\hh{\calC_f}\to\hh{\calC_{cf}}, \]
	on morphisms given by the composition
	\[ \hh{\calC_f}(x,y) = [x,y]_\ell \to [Qx,y] \cong [Qx,Qy] = \hh{\calC_{cf}}(x,y). \]
	Similarly, one induces a functor \(\hh{R}\!:\hh{\calC_c}\to\hh{\calC_{cf}}\), and thus two functors
	\[ \hh{R}\circ Q, \hh{Q}\circ R\!: \calC\to\hh{\calC_{cf}}. \]
\end{construction}

\begin{exercise}
	Given an object \(x\in\calC\) in a model category \(\calC\) with chosen fibrant replacement \(Rx\), cofibrant replacement \(Qx\), and furthermore
	additional replacements \(QRx\) and \(RQx\), show that there is a weak equivalence \(\xi_x\!:RQx\to QRx\) which assembles into a natural isomorphism
	\[ \xi\!: \hh{R}\circ Q \To \hh{Q}\circ R \]
	in the homotopy category.
\end{exercise}

% TODO: Finish the below. Just use the approach in [MP12] as it is written.

\begin{proposition}
	Let \(\calC\) be a model category, and write \(RQ\) for the functor \(\hh{R}\circ Q\) constructed above. Then the following statements hold.
	\begin{enumerate}[label=(\arabic*)]
		\item A morphism \(f\!:x\to y\) in \(\calC\) is a weak equivalence if and only if \(RQf\!:RQx\to RQy\) is an isomorphism.
		\item Any map in \(\hh{\calC_{cf}}\) is a composite of morphisms in \(RQ(\calC)\) and inverses of morphisms in \(RQ(W)\).
	\end{enumerate}
\end{proposition}
\begin{proof}
(1) The morphism \(f\) is a weak equivalence if and only if the corresponding map \(RQx\to RQy\) in \(\calC\) is a weak equivalence, if and only if it is a homotopy equivalence,
if and only if it is an isomorphism in \(\hh{\calC_{cf}}\).

(2) 
\end{proof}

\begin{theorem}
	Let \(\calC\) be a model category. Write \(RQ\) for the functor \(\hh{R}\circ Q\) constructed above. Then
	\[ RQ\!: \calC \to \hh{\calC_{cf}} \]
	exhibits \(\hh{\calC_{cf}}\) as a localization of \(\calC\) at the weak equivalences. In particular, there is a canonical equivalence of categories \(\hh{\calC_{cf}}\simeq\ho(\calC)\).
\end{theorem}
\begin{proof}
Let \(F\!:\calC\to\calD\) be a functor inverting the weak equivalences.
\end{proof}

\subsection{Derived functors for model categories (TBD)}
% Follow [Cis19] and [Rie13].

\subsection{Homotopy (co)limits (TBD)}
% Probably [Cis19] and [Rie13], but need to look at how [MP12] does it (if it does)


%!TEX root = ../lectures.tex

\section{Grothendieck categories \& the Freyd--Mitchell embedding theorem}
This lecture concerns properties of Grothendieck categories, with the motivating goal of sketching the famous result that any small Abelian category may be embedded inside a module category (over some typically noncommmutative ring).
The fundamental strategy is to apply a result of the kind found in Appendix \ref{appendix:abelian-categories-with-a-compact-projective-generator}, which we begin with below, and show
that a suitable situation can be constructed for any small Abelian category through the use of Pro-categories (which are the duals of Ind-categories). In particular, one considers a composition
\[ \calA\inj\Pro(\calA) \to \Mod_A \]
for a suitable ring \(A\).

\subsection{A criterion for embedding into a module category}
In Appendix \ref{appendix:abelian-categories-with-a-compact-projective-generator}, we provided a criterion for being equivalent to a particular module category.
However, with some care, one can change the assumptions by dropping compactness to give a mere embedding, at least for small subcategories, which is good enough.

The underlying reason we can drop a compactness condition is the following lemma, telling us that as long as we are only working with a small set of objects,
we can get by with a finite amount of data.
\begin{lemma}\label{lemma:small-set-quotients-of-projective-generator}
	Let \(\calA\) be a locally small Abelian category with a projective generator, and let \(\calA'\subseteq\Ob(\calA)\) be a small set.
	Then there is a projective generator \(p'\) of \(\calA\) such that each object in \(\calA'\) is a quotient of \(p'\).
\end{lemma}
\begin{proof}
For any \(x\in\calA\), there is an epimorphism
\[ \coprod_{p\to x}p\sur x. \]
We thus set
\[ p' := \coprod_{x\in\calA'}\coprod_{p\to x}p \]
which exists, as \(\calA'\) is small. From this, we can componentwise define an epimorphism of the form \(p'\sur x\) for any \(x\in\calA'\), so that \(p'\) is a generator.
Since coproducts of projectives are projective, \(p'\) is a projective generator.
\end{proof}

With this in place, we can basically run through a simplified version of the proof of Theorem \ref{thm:abelian-cat-with-compact-projective-generator}.

\begin{theorem}\label{thm:embedding-of-small-category-into-module-category}
	Let \(\calA\) be a locally small Abelian category with a projective generator, and let \(\calA'\subseteq\calA\) be a small Abelian subcategory.
	Then there is a ring \(A\) and a fully faithful exact embedding
	\[ \calA'\inj\Mod_A \]
	of \(\calA'\) into the category of right modules over \(A\).
\end{theorem}
\begin{proof}
By Lemma \ref{lemma:small-set-quotients-of-projective-generator}, we have a projective generator \(p\in\calA\) for which every \(x\in\calA'\) has a surjection
\[ p\sur x. \]
We let \(A = \End(p)\). By (1) in Theorem \ref{thm:abelian-cat-with-compact-projective-generator}, the functors
\[ \calA(p,-)\!:\calA\to\Mod_A,\quad \leadsto\quad \calA(p,-)\!:\calA'\to\Mod_A \]
are faithful. We need to show that the latter functor, i.e.\ the restriction of \(\calA(p,-)\) to \(\calA'\), is full. For this, we can essentially apply the same proof
as (2) in Theorem \ref{thm:abelian-cat-with-compact-projective-generator}, except that we no longer need compactness. Consider an \(A\)-linear homomorphism
\[ \varphi\!:\calA(p,x)\to\calA(p,y),\quad x,y\in\calA'. \]
Consider an exact sequence
\[ 0 \to \ker\pi\overset\kappa\inj p\overset\pi\sur x\to 0. \]
We get a morphism \(\varphi(\pi)\!:p\to y\), and so we consider the situation
\begin{diagram*}
	& & p\ar[d,"\kappa\circ h"]\ar[dl,"h"'] & & \\
	0\ar[r] & \ker\kappa\ar[r,hook,"\kappa"] & p\ar[r,two heads,"\pi"]\ar[d,"\varphi(\pi)"] & x\ar[r] & 0 \\
	& & y & &
\end{diagram*}
where we need to check that for all \(h\!:p\to\ker\pi\), the composition satisfies \(\varphi(\pi)\circ\kappa\circ h = 0\). For this,
\[ \varphi(\pi)\circ\kappa\circ h = \varphi(\pi\circ\kappa\circ h) = \varphi(0\circ h) = 0 \]
so that \(\varphi(\pi)\circ\kappa = 0\). Thus, we get an induced map \(f\!:x\to y\) such that \(f\circ\pi = \varphi(\pi)\). To see that \(f_* = \varphi\), applying \(\calA(p,-)\) to our starting
exact sequence yields that
\[ \calA(p,p)\overset{\pi_*}\sur\calA(p,x) \]
since \(p\) is projective, so it suffices to see that \(\varphi\circ\pi_* = f_*\circ \pi_*\). However, for any \(a\!:p\to p\), we have
\[ (f_*\circ\pi_*)(a) = (f\circ\pi)_*(a) = \varphi(\pi)\circ a = \varphi(\pi\circ a) = (\varphi\circ\pi_*)(a). \]
We conclude that \(\calA(p,-)\!:\calA'\to\Mod_{\End(p)}\) is fully faithful. It is, in addition, exact since \(p\) is projective.
\end{proof}
\begin{remark}
	Note that in the above, by an \emph{Abelian subcategory,} we mean a subcategory which is Abelian and for which the inclusion functor is exact.
\end{remark}

\subsection{Grothendieck categories}
\begin{definition}
	A \emph{Grothendieck category} is a locally small Abelian category admitting a generator and small colimits, for which small filtered colimits are exact.
\end{definition}

\begin{definition}
	Let \(\calC\) be a category, and let \(x\in\calC\). A \emph{subobject} of \(x\) is an equivalence class of monomorphisms \(y\inj x\), under the equivalence
	relation \((y\inj x)\sim (y'\inj x)\) if and only if there is a commutative triangle
	\begin{diagram*}[cramped, column sep=small]
		y \ar[rr,"\sim"]\ar[dr,hook] & & y'\ar[dl,hook] \\
		& x. &
	\end{diagram*}
	We write \(y\subseteq x\) to say that \(y\) is a subobject of \(x\), choosing a particular representative of the equivalence class. Dually, a \emph{quotient} of \(x\) is an equivalence class of epimorphisms \(x\sur y\).
\end{definition}

\begin{lemma}\label{lemma:faithful-functors-reflects-monomorphisms-epimorphisms}
	Let \(F\!:\calC\to\calD\) be a faithful functor. Then \(F\) reflects monomorphisms and epimorphisms.
\end{lemma}
\begin{proof}
We prove that it reflects monomorphisms, as the other case is dual. Let \(f\!:x\to y\) be a morphism in \(\calC\), and suppose that \(F(f)\) is monic. Note that
\[ z \underset{b}{\overset{a}{\rightrightarrows}} x \overset{f}\to y \quad\leadsto\quad Fz \underset{Fb}{\overset{Fa}{\rightrightarrows}} Fx \overset{Ff}\to Fy \]
so that \(Fa = Fb\), hence \(a=b\) since \(F\) is faithful.
\end{proof}

\begin{lemma}\label{lemma:generators-in-balanced-categories-detect-isomorphisms}
	Let \(\calC\) be a balanced category, and let \(\calU\) be a small set of objects. Consider the following conditions.
	\begin{enumerate}[label=(\arabic*)]
	\item \(\calU\) generate \(\calC\).
	\item Suppose we have a morphism \(f\!:x\to y\) in \(\calC\). If
	\[ \forall u\in\calU,\quad f_*\!:\calA(u,x)\iso\calA(u,y) \]
	then \(f\) is an isomorphism.
	\end{enumerate}
	Then (1) implies (2), and if \(\calC\) admits equalizers then (2) implies (1).
\end{lemma}
\begin{proof}
(1) implies (2): the statement that \(\calU\) is a small set of generators is the same as saying that the functor
\[ \prod_{u\in\calU}\calC(u,-)\!:\calC\to\Set \]
is faithful. By Lemma \ref{lemma:faithful-functors-reflects-monomorphisms-epimorphisms}, it therefore reflects epimorphisms and monomorphisms; since \(\calC\) and \(\Set\) are balanced,
this means it is conservative. The condition (2) is exactly that \(f\) is sent to an isomorphism under the above functor.

(2) implies (1): by assumption \(\prod_{u\in\calU}\calC(u,-)\) is a convervative functor; let \(f,g\!:x\to y\) be a pair of parallel arrows, and consider \(k\!:\eq(f,g)\inj x\).
Since \(\prod_{u\in\calU}\calC(u,-)\) commutes with limits, we have
\[ \prod_{u\in\calU}\calC(u,\eq(f,g)) \iso \eq((f_*)_{u\in\calU},(g_*)_{u\in\calU}). \]
If we assume that \(f\circ h = g\circ h\) for all \(h\!:u\to x\), then the latter set is just \(\prod_{u\in\calU}\calC(u,x)\), so by conservativity we get that
\(\eq(f,g)\iso x\), hence \(f=g\).
\end{proof}

% \begin{lemma}
% 	Let \(\calA\) be a Grothendieck category with generator \(u\), and let \(\kappa\) be an infinite regular cardinal. Consider a small \(\kappa\)-filtered diagram \(D\!:I\to\calA\),
% 	and an epimorphism \(\injlim{D}\sur y\). Assume that one of the following conditions hold:
% 	\begin{enumerate}[label=(\alph*)]
% 	\item \(\#\calA(u,y) < \kappa\), or
% 	\item \(y\in\calA^\kappa\), i.e.\ \(y\) is \(\kappa\)-compact.
% 	\end{enumerate}
% 	Then there is some \(i\in I\) such that the induced map \(D(i)\to y\) is an epimorphism.
% \end{lemma}
% \begin{proof}
% Set \(y_i := \img(D(i)\to y)\). This induces a new diagram \(D'\!:I\to\calA\). Since small filtered colimits are exact, they commute with taking images, so
% \[ \injlim{D'} = \injlim_{i\in I}\img(D(i)\to y) \cong \img(\injlim{D}\sur y) \cong y. \]

% (a) Let \(S = \injlim_{i\in I}\calA(y_i,u)\), viewed as a subset of \(\calA(u,y)\). By assumption, \(\#S \leq \#\calA(u,y) < \kappa\), so \(S\) is \(\kappa\)-compact
% by the computation in Example \ref{example:compact-objects-in-Set}. Therefore,
% \[ \injlim_{i\in I}\Set(S,\calA(u,y_i)) \cong \Set(S,\injlim_{i\in I}\calA(u,y_i)) = \Set(S,S). \]
% We thus find some \(i_0\in I\) and morphism \(S\to \calA(y_{i_0},u)\) such that
% \[ (S\to \calA(u,y_{i_0}) \inj S) = \id_S. \]
% However, then the inclusion on the left is also surjective, so \(S = \calA(y_{i_0},u)\), from which we deduce that for any \((i_0\to i)\in I\), the induced map
% \(\calA(u,y_{i_0})\to\calA(u,y_i)\) is bijective. Since \(u\) is a generator, we conclude by Lemma \ref{lemma:generators-in-balanced-categories-detect-isomorphisms} that \(y_{i_0}\iso y_i\), hence \(y_{i_0}\iso y\). But then
% \[ \img(D(i)\to y) \cong y, \]
% so \(D(i)\sur y\).

% (b) If \(y\) is \(\kappa\)-compact, then
% \[ \injlim_{i\in I}\calA(y,y_i) \iso \calA(y,y). \]
% But then, once again, the identity factors as
% \begin{diagram*}[cramped,column sep=small]
% 	y\ar[rr,equal]\ar[dr] & & y \\
% 	& y_i\ar[ur,hook] & 
% \end{diagram*}
% so the canonical inclusion \(y_i\inj y\) is an epimorphism. Since Abelian categories are balanced, this implies \(y_i\cong y\); as before, we conclude that \(D(i)\sur y\).
% \end{proof}


\begin{proposition}\label{prop:grothendieck-categories-small-set-of-subobjects}
	Let \(\calA\) be a Grothendieck category, and let \(x\in\calA\). Then the set of subobjects of \(x\) and the set of quotients of \(x\) are small.
\end{proposition}
\begin{proof}
Note that any Abelian category admits finite limits, since they admit finite products and equalizers. Now, denote by \(S(x)\) the set of subobjects of \(x\), and let \(u\in\calA\)
be a generator. The functor \(\calC(u,-)\) provides a map
\[ \rho\!: S(x) \to \calP(\calC(u,x)), \quad (x'\overset\iota\inj x)\mapsto\iota_*(\calC(u,x'))\subseteq\calC(u,x) \]
where we note that \(\iota_*\) is injective since \(\iota\) is a monomorphism. By assumption, \(\calA\) is locally small, hence \(\calP(\calC(u,x))\) is small, so it suffices
to see that the above map of sets \(\rho\) is injective.

First of all, we have that \(\rho(x')\cong\calC(u,x')\) for any \(x'\subseteq x\). Let \(x'\subseteq x\) and \(x''\subseteq x\). The pullback \(x'\times_x x''\) is then also a subobject of \(x\),
and if \(\rho(x') = \rho(x'')\) then
\[ \calC(u,x'\times_xx'') \cong \calC(u,x')\times_{\calC(u,x)}\calC(u,x'') \cong \rho(x')\cap\rho(x'') = \rho(x') = \rho(x''). \]
Applying Lemma \ref{lemma:generators-in-balanced-categories-detect-isomorphisms}, we see that \( x'\xleftarrow\sim x'\times_xx''\xrightarrow\sim x'' \) and in particular \(x'\cong x''\).

To see that the set of quotients is small, note that \(f\!:x\to y\) is an epimorphism if and only if
\[ 0\to \ker{f} \inj x \overset{f}\to y \to 0 \]
is exact, which uniquely identifies \(y\) as the cokernel of \(\ker{f}\inj x\). In other words, quotients of \(x\) correspond to subobjects, which form a small set.
\end{proof}

\begin{proposition}\label{prop:grothendieck-category-injectives-detected-by-generator}
	Let \(\calA\) be a Grothendieck category, and let \(\calU\) be some small set of generators. Let \(z\in\calA\). Then the following are equivalent:
	\begin{enumerate}[label=(\arabic*)]
	\item \(z\) is injective.
	\item For all \(u\in\calU\) and all subobjects \(v\subseteq u\), the restriction map
	\[ \calA(u,z)\to\calA(v,z) \]
	is surjective.
	\end{enumerate}
\end{proposition}
\begin{proof}
Certainly (1) implies (2). Conversely, assume (2). We want to show that \(z\) is injective, so consider the situation
\begin{diagram*}
	x'\ar[r,hook,"f"]\ar[d,"h"'] & x\ar[dl,dashed,"\exists?"] \\
	z
\end{diagram*}
where we want to find the dashed arrow. We consider the category \(\Delta(h)\) whose objects are diagrams \(D\) of the form
\begin{diagram*}
	x'\ar[r,hook,"k"]\ar[d,"h"']\ar[rr,bend left,"f"] & y\ar[dl,"g"]\ar[r,hook,"\ell"] & x \\
	z
\end{diagram*}
which we denote \(D = (y,k,g,\ell)\), and a morphism \(D\to D' = (y',k',g',\ell')\) is a morphism \(t\!:y\to y'\) making the diagram
\begin{diagram*}
	y\ar[r,"t"] \ar[rr,bend left,hook,"\ell"]\ar[d,"g"'] &  y'\ar[r,hook,"\ell'"]\ar[dl,"g'"] & x \\
	z
\end{diagram*}
commute. Note that \(t\) is automatically a monomorphism, and that since \(\ell'\) is a monomorphism, if a morphism \(D\to D'\) exists, then it is unique.
Now, let \(\Sigma(h)\) be the set of isomorphism classes in \(\Delta(h)\); this is a small set by Proposition \ref{prop:grothendieck-categories-small-set-of-subobjects}.
We turn it into a poset by letting \(D\leq D'\) if and only if there exists a morphism \(D\to D'\). One sees that \(\Delta(h)\simeq\Sigma(h)\).

There is a functor \(\Delta(h) \to \calA\) given by sending \((y,k,g,\ell)\) to \(y\), which induces a functor \(\Sigma(h)\to\calA\). We can use this to see that any chain in \(\Sigma(h)\) has an upper bound: given a chain
\[ \alpha\!:I\to\Sigma(h), \]
since small filtered colimits are exact in \(\calA\), we may take the colimit of the induced diagram \(\alpha'\!:I\to\calA\), which is filtered, and hence the induced map \(\injlim\alpha' \to x\)
is injective by exactness. Then \(\injlim\alpha'\) together with its canonically induced maps to \(x\) and \(z\) is an upper bound for the chain \(\alpha\).

By Zorn's lemma, \(\Sigma(h)\) has a maximal element, say \((y_0,k_0,g_0,\ell_0)\). By Lemma \ref{lemma:generators-in-balanced-categories-detect-isomorphisms}, to check that \(y_0 \iso x\),
it suffices to check that the injective maps
\[ \forall u\in\calU,\quad \ell_*\!:\calC(u,y_0)\inj\calC(u,x) \]
are also surjective. Thus, consider a morphism \(\varphi\!:u\to x\), and define an object \(y\in\calA\) by the pullback
\begin{diagram*}
	y\ar[dr,pullback]\ar[r,dashed]\ar[d,dashed] & y_0\ar[d,hook,"\ell"] \\
	u\ar[r,"\varphi"] & x.
\end{diagram*}
Since \(\ell\) is monic, so is the induced map \(y\to u\). Now define an object \(y_1\in\calA\) by the pushout
\begin{diagram*}
	y\ar[r]\ar[d,hook] & y_0\ar[d,dashed] \\
	u\ar[r,dashed] & y_1.\ar[ul,pushout]
\end{diagram*}
Expressing pullbacks in terms of equalizers, and that in terms of kernels, we have an exact sequence
\[ 0 \to y \to y_0\oplus u \overset{r}\to x \]
but doing the same for the definition of \(y_1\), one sees that \(y_1 \cong \img{r} \subseteq x\), and we deduce that the canonical map \(y_1\to x\) is monic. Now, \(y\) is also a
subobject of \(u\), so by (2), the composite \(y\to y_0 \to z\) factors as in the diagram
\begin{diagram*}
	y\ar[r, hook]\ar[d] & u\ar[dr,"\varphi"]\ar[ddr,dashed]\ar[d] & \\
	y_0\ar[r,hook]\ar[drr] & y_1\ar[dr,dashed] & x\ar[from=l,hook,crossing over] \\
	& & z
\end{diagram*}
which induces the dashed arrow \(y_1\to z\) by universal property. This endows \(y_1\) with the structure of an object in \(\Delta(x)\), so by maximality
of \(y_0\), we have that \(y_0\cong y_1\) so that \(\varphi\) factors through \(y_0\) as desired.
\end{proof}

The below theorem is the fruit of our labour, and it crucially relies essentially on the above proposition.

\begin{theorem}
	Let \(\calA\) be a Grothendieck category. Then the following statements hold.
	\begin{enumerate}[label=(\arabic*)]
	\item \(\calA\)	has enough injectives.
	\item \(\calA\) has an injective cogenerator, i.e.\ \(\calA^\op\) has a projective generator.
	\end{enumerate}
\end{theorem}
\begin{proof}
Let \(u\) be a generator of \(\calA\).

(1) Let \(x\in\calA\). For any ordinal \(\alpha\), we define an object \(M_\alpha(x)\in\calA\), and we will do this in such a way that we can check condition (2) in
Proposition \ref{prop:grothendieck-category-injectives-detected-by-generator}. We do this inductively; consider the set
\[ I(x) := \{ (v,f)\mid v\subseteq u,\, f\!:v\to x \}. \]
We define \(M_1(x)\) by the pushout
\begin{diagram*}
	\coprod_{(v,f)\in I(x)}v\ar[r]\ar[d,hook] & x\ar[d,dashed,hook] \\
	\coprod_{(v,f)\in I(x)}u\ar[r,dashed] & M_1(x)
\end{diagram*}
so that \(M_1\) is the cokernel of the map
\[ \coprod_{\mathclap{(v,f)\in I(x)}}v \to x\oplus \coprod_{(v,f)\in I}u \]
induced by the maps \(f\!:v\to x\) and the inclusions \(v\subseteq u\). If \(f\!:v\to x\) is a morphism from a subobject of \(u\) to \(x\), then it extends to a map \(u\to M_1(x)\) in the obvious way.
We have thus embedded \(x\) in some object \(M_1(x)\), but it is not good enough as it is not injective.

We now define \(M_\alpha(x)\) for an ordinal \(\alpha\) through transfinite induction in such a way that there are monomorphisms
\[ M_{\kappa'}(x) \inj M_{\kappa}(x) \]
whenever \(\kappa' < \kappa < \alpha\), such that these are compatible with the ordering on ordinals. Assume that \(M_\beta\) and the above monomorphisms are defined for all \(\beta < \alpha\). If \(\alpha = \beta+1\),
then set \(M_\alpha(x) := M_1(M_\beta(x))\); we get a monomorphism \(M_\beta(x)\inj M_{\alpha}(x)\) by the above. If \(\alpha\) is a limit ordinal, then \(\{ M_{\beta}(x)\mid \beta < \alpha \}\) defines a directed system
and we let \(M_\alpha(x)\) be its colimit. For any \(\beta < \alpha\), the canonical map \(M_\beta(x)\to M_\alpha(x)\) is then a monomorphism.

Now let \(\Omega\) be the smallest infinite ordinal whose cardinality is strictly greater than the cardinality of the set of subobjects of \(u\). We claim that \(M_\Omega(x)\) is injective.
Let \(v\subseteq u\) be a subobject, and let \(f\!:v\to M_\Omega(x)\) be a morphism. One obtains a system of subobjects \(v_\kappa\subseteq u\), given by the pullback
\begin{diagram*}
	v_\kappa\ar[dr,pullback]\ar[r,dashed]\ar[d,hook,dashed] & M_\kappa(x) \ar[d,hook] \\
	v\ar[r,"f"] & M_\Omega(x)
\end{diagram*}
and one sees that
\[ \injlim_{\kappa < \Omega}v_\kappa = \injlim_{\kappa < \Omega}(v\times_{M_\Omega(x)}M_\kappa(x)) \cong v\times_{M_\Omega(x)}\injlim_{\kappa < \Omega}(M_\kappa(x)) = v\times_{M\Omega(x)}M_\Omega(x) = v \]
since filtered colimits in \(\calA\) commute with finite limits. In particular, \(v_\Omega = v\). However, since \(\Omega\) is strictly larger than the number of subobjects of \(u\), there
must be some \(\kappa < \Omega\) for which \(v = v_\kappa\), so \(f\!:v\to M_\Omega(x)\) factors through some map \(v\to M_\kappa(x)\), which we can then extend:
\begin{diagram*}[cramped,sep=small]
	v\ar[dd,hook]\ar[rr] & & M_\kappa(x)\ar[dd,hook]\ar[dl,hook] \\
	& M_{\kappa+1}(x)\ar[dr,hook] & \\
	u\ar[ur,dashed]\ar[rr,dashed] & & M_\Omega(x)
\end{diagram*}
so we conclude that \(M_\Omega(x)\) is injective by Proposition \ref{prop:grothendieck-category-injectives-detected-by-generator}.

(2) By Proposition \ref{prop:grothendieck-categories-small-set-of-subobjects}, the generator \(u\) has a small set of quotients; hence there is a set \(Q\) of objects such that any quotient of \(u\) is isomorphic
to an object in \(Q\). Let \(q_0 = \bigoplus_{z\in Q}z\), and consider an embedding \(q_0\inj q\) into an injective \(q\in\calA\), which exists by (1). We aim to show that \(q\) is a cogenerator. We begin by showing that
\[ \calA(x,q)\cong 0\implies x\cong0. \]
Suppose the left condition holds. For any map \(h\!:u\to x\), the below right map
\[ \img{h} \inj x \quad\leadsto\quad \calA(x,q)\to\calA(\img{h},q) \]
is surjective since \(q\) is injective. Therefore, if \(\calA(x,q)\cong 0\), we see that \(\calA(\img{h},q) \cong 0\). Now, \(\img{h}\) is a quotient of \(u\), hence isomorphic to some \(z\in Q\) and
there is a monomorphism
\[ \img{h} \iso z\inj q_0 \inj q. \]
Therefore, \(\img{h} \cong 0\), so \(h=0\) and \(\calA(u,x)\cong 0 \cong \calA(u,0)\). We conclude that \(x\cong 0\).

Finally, consider a morphism \(f\!:x\to y\) for which \(f^*\!:\calA(y,q)\iso\calA(x,q)\); it suffices to show that \(f\) is an isomorphism. Since \(\calA(-,q)\) is exact, we have an exact sequence
\[ 0 \to \calA(\coker{f},q) \to \calA(y,q)\iso \calA(x,q)\to\calA(\ker{f},q)\to 0 \]
which shows that \(\calA(\coker{f},q)\cong\calA(\ker{f},q)\cong 0\), so by what we have shown, \(\coker{f}\cong\ker{f}\cong 0\), so \(f\) is an isomorphism.
\end{proof}
\begin{exercise}
	\begin{enumerate}[label=(\arabic*)]
	\item Prove that in an Abelian category, the pushout of a monomorphism is a monomorphism
	\begin{diagram*}
		x\ar[r]\ar[d,hook] & y\ar[d,dashed,hook] \\
		x'\ar[r,dashed] & y'\ar[ul,pushout]
	\end{diagram*}
	by proving that the above diagram is also a pullback diagram, and that in a pullback diagram the kernels of the vertical arrows agree.
	\item Prove that in a Grothendieck category, the canonical maps to the colimit of a directed system whose transition maps are monic, are monic:
	\begin{diagram*}[sep=small]
		\cdots\ar[r,hook] & x_i\ar[r,hook]\ar[dr,hook] & \cdots \ar[r,hook] & x_{i'}\ar[dl,hook] \ar[r,hook] & \cdots \\
		& & \injlim_{k\in I}x_k & &
	\end{diagram*}
	\end{enumerate}
\end{exercise}

\subsection{Closure properties of Ind-categories \& the Embedding Theorem (WIP)}
\begin{theorem}
	Let \(\calA\) be a small Abelian category. Then \(\Ind(\calA)\) is a Grothendieck category, and the inclusion \(\calA\inj\Ind(\calA)\) is exact.
\end{theorem}

Proving this is done by studying the closure properties of \(\Ind(\calA)\) under (co)limit constructions.
\begin{proposition}
	Let \(\calC\) be a locally small category, and assume that \(\calC\) admits finite limits. Then the following statements hold.
	\begin{enumerate}[label=(\arabic*)]
	\item \(\Ind(\calC)\) admits finite limits.
	\item \(\Ind(\calC)\inj\PSh(\calC)\) preserves finite limits.
	\item \(\calC\inj\Ind(\calC)\) preserves finite limits.
	\item Filtered colimits in \(\Ind(\calC)\) are exact, i.e.\ commute with finite limits.
	\end{enumerate}
\end{proposition}
\begin{proof}
We prove (1) and (2) simultaneously, afterwhich (3) follows by noting that the Yoneda embedding \(\calC\inj\PSh(\calC)\) preserves all small limits, and (4) follows
by filtered colimits in \(\PSh(\calC)\) being exact (inheriting this property from \(\Set\)).

We show that \(\Ind(\calC)\) admits binary products and equalizers. Thus, consider \(A,B\in\Ind(\calC)\), and write
\[ A \cong \injlim_{i\in I}h_{D(i)},\quad B\cong \injlim_{j\in J}h_{D'(j)}  \]
for some filtered diagrams \(D\!:I\to\calC\), \(D'\!:J\to\calC\). Then, in \(\PSh(\calC)\),
\[ A\times B \cong \injlim_{(i,j)\in I \times J}h_{D(i)\times D'(j)} \in \Ind(\calC) \]
so \(\Ind(\calC)\) admits binary products, computed in \(\PSh(\calC)\).

To see that we have equalizers, fix some \(x\in\calC\), and consider the full subcategory \(\calC'\) of all \(A\in\PSh(\calC)\) for which the equalizer in \(\PSh(\calC)\) of any pairs of maps
\(h_x \rightrightarrows A\) is in \(\Ind(\calC)\). Then \(\calC \inj \calC'\) since \(\calC\) admits finite limits, and \(\calC'\) is closed under filtered colimits. However, by universal property,
we must then have that \(\calC'\simeq\Ind(\calC)\). Now we reverse the roles, and consider
\end{proof}

\begin{theorem}[Freyd--Mitchell embedding theorem]\label{thm:freyd-mitchell-embedding-theorem}
	Let \(\calA\) be a small Abelian category. Then there is a ring \(A\) and a fully faithful exact functor
	\[ \calA\inj\Mod_A. \]
\end{theorem}
\begin{proof}
Note that \(\calA^\op\) is a small Abelian category, hence we can consider the embedding
\[ \calA^\op\inj\Ind(\calA^\op). \]
Taking the opposite, we get the embedding
\[ \calA\inj\Pro(\calA). \]
Now, \(\Ind(\calA^\op)\) is a Grothendieck category, hence has an injective cogenerator, so \(\Pro(\calA) := \Ind(\calA^\op)^\op\) is a locally small
Abelian category with a projective generator. Applying Theorem \ref{thm:embedding-of-small-category-into-module-category}, we are done.
\end{proof}

\begin{remark}
	While the embedding theorem itself only applies to small Abelian categories, it may still be utilized when working with potentially large ones. In practice,
	one applies the theorem when diagram-chasing; thus, given a small diagram \(D\!:I\to\calA\), one may consider the smallest subcategory \(\calA'\) of \(\calA\) which
	\begin{enumerate}[label=(\arabic*)]
	\item is small,
	\item is Abelian,
	\item contains the image of \(D\), and for which
	\item the inclusion \(\calA'\inj\calA\) is exact.
	\end{enumerate}
	Thus, one factors \(D\) as \(I\to\calA'\inj\calA\), and embeds \(\calA'\) in a module category wherein one can do as much chasing as one wants.
\end{remark}



\begin{comment}
%!TEX root = ../lectures.tex

\section{Metric \(\infty\)-categories (WIP)}
Recall that a metric \(d\!:X\times X\to \R_{\geq 0}\) on a set \(X\) is a function satisfying
\begin{enumerate}[label=(\arabic*)]
	\item \(\forall x,y\in X\), \(d(x,y) = 0\) if and only if \(x=y\),
	\item \(\forall x,y\in X\), \(d(x,y) = d(y,x)\), and
	\item \(\forall x,y,z\in X\), \(d(x,z) \leq d(x,y) + d(y,z)\).
\end{enumerate}
A pair \((X,d)\) of a set \(X\) and a metric \(d\) on \(X\) is called a metric space, as the reader is surely aware. These are ubiquitous in many areas, mainly in analysis.
There is a slightly weakened version one can consider, sometimes called a \emph{pseudoquasimetric,} or a \emph{hemimetric.} Here, instead of (1), one merely demands that \(d(x,x)=0\),
and one drops the symmetry requirement (2) entirely. That is, a hemimetric is a function \(d\!:X\times X \to \R_{\geq 0}\) such that
\begin{enumerate}[label=(\arabic*')]
	\item \(\forall x\in X\), \(d(x,x)=0\), and
	\item \(\forall x,y,z\in X\), \(d(x,z) \leq d(x,y) + d(y,z)\).
\end{enumerate}
The content of this lecture is that one may extend this notion to categories. This was originally studied by Lawvere \cite{lawvere73-metric-spaces}, and has been subsequently considered by many authors.
We are inspired primarily by the work of Neeman \cite{neeman2021metricstriangulatedcategories}. In \emph{loc. cit.,} the focus is on metrics on metrics on triangulated categories, and
we take the perspective that these should be interpreted as being homotopy categories of stable \(\infty\)-categories. Hence, we choose to begin by developing the theory in the
\(\infty\)-categorical context.

\subsection{Metrics}
\begin{notation}
	Let \(\calC\) be an \(\infty\)-category. We will write \(\Mor(\calC)\) for the set of morphisms in \(\calC\), i.e.\ the set of functors \(\Delta^1 \to \calC\), or alternatively
	the set of 1-simplices \(\calC_1\) of \(\calC\).
\end{notation}
\begin{definition}
	Let \(\calC\) be an \(\infty\)-category. A \emph{metric} on \(\calC\) is a function \(d\!:\Mor(\calC)\to\R_{\geq 0}\), satisfying the following properties.
	\begin{enumerate}[label=(\arabic*)]
		\item For all \(x\in\calC\), we have \(d(x \overset{\id_x}\longto x) = 0\).
		\item For any commutative triangle in \(\calC\) as below left, we have the triangle inequality as below right:
		\begin{center}
			\begin{tikzcd}[cramped, column sep=small]
				& y\ar[dr,"g"] & \\
				x\ar[ur,"f"]\ar[rr,"h"] & & z
			\end{tikzcd}\(\quad \leadsto \quad d(x\overset{h}\to z) \leq d(x\overset{f}\to y) + d(y\overset{g}\to z). \)
		\end{center}
	\end{enumerate}
	Two metrics \(d_1\) and \(d_2\) on \(\calC\) are \emph{equivalent} if for all \(\varepsilon > 0\), there is a \(\delta > 0\) such that
	\begin{align*}
		d_1(x\to y) < \delta &\implies d_2(x\to y) < \varepsilon, \\
		d_2(x\to y) < \delta &\implies d_1(x\to y) < \varepsilon.
	\end{align*}
	A \emph{metric} \(\infty\)\emph{-category} is a pair \((\calC,d)\) of an \(\infty\)-category \(\calC\) and a metric \(d\!:\Mor(\calC)\to\R_{\geq 0}\).
\end{definition}
\begin{definition}
	Let \((\calC,d)\) be a metric \(\infty\)-category. A sequence of morphisms
	\[ x_1 \to x_2 \to \cdots \to x_n \to \cdots \]
	in \(\calC\) is \emph{Cauchy} with respect to \(d\) if for all \(\varepsilon > 0\) there is some \(M > 0\) such that
	\[  M \leq i \leq j \implies d(x_{i} \to x_j) < \varepsilon. \]
\end{definition}
\begin{remark}
	For a 1-category \(\calC\), the triangle inequality requirement just says that for any composable pair of morphisms
	\[ x \overset{f}\to y \overset{g}\to z \]
	we have
	\[ d(x\overset{g\circ f}\to z) \leq d(x\overset{f}\to y) + d(y\overset{g}\to z). \]
\end{remark}
\begin{example}
	Let \(X\) be some set. Form the category \(\calX\), whose objects are the elements of \(X\), and where for each \(x,y\in\calX\) there is exactly
	one morphism \(x\to y\). Then a metric on \(\calX\) is exactly a hemimetric on \(X\). In particular, any (hemi)metric space \(X\) determines a
	metric category \(\calX\), and one observes that Cauchy sequences in \(X\) are the same as Cauchy sequences in \(\calX\).
\end{example}
\begin{example}
	Let \(\cat{Ban}\) be the category of Banach spaces. There is a metric \(d\) on \(\cat{Ban}\) given by
	\[ d(X \overset{T}\to Y) := \max\{0,\log\|T\|\}. \]
	To see that this is in fact a metric, note that
	\[ d(X\overset{\id_X}\to X) = \log(1) = 0, \]
	 and for \(X \overset{T}\to Y\overset{S}\to Z\), we have
	 \begin{align*}
	 	\log\|ST\| &\leq \log(\|S\|\cdot\|T\|) = \log\|S\| + \log\|T\|
	 \end{align*}
	 from which it follows that
	 \[ d(X\overset{ST}\longto Z) \leq \max\{0, \log\|S\| + \log\|T\|\} \leq d(X\overset{T}\to Y) + d(Y\overset{S}\to Z). \]
	 This example is discussed in \cite[§8]{clementino2024cauchyconvergencevnormedcategories}. This metric has a few trivial properties. For example, \(d(X\overset{T}\to Y)=0\)
	 if and only if \(\|T\|\leq 1\). It follows that one always has \(\|T\| \leq \exp{d(X\overset{T}\to Y)} \).
\end{example}

We now show that metrics do not see the higher-categorical data in an \(\infty\)-category, and depend only on the homotopy 1-category \(\ho(\calC)\).
\begin{proposition}\label{prop:metric-invariant-under-homotopy}
	Let \((\calC,d)\) be a metric \(\infty\)-category, and consider two morphisms \(f,g\!:x\to y\) such that \([f] = [g]\) in \(\ho(\calC)\).
	Then \(d(x\overset{f}\to y) = d(x\overset{g}\to y)\).
\end{proposition}
\begin{proof}
By assumption (see e.g.\ \cite[§1.6]{cisinski-book}), we have the following commutative triangles:
\begin{center}
	\begin{tikzcd}[cramped, column sep=small]
		& y\ar[dr,"\id_y"] & \\
		x\ar[ur,"f"]\ar[rr,"g"] & & y
	\end{tikzcd}\quad and \quad
	\begin{tikzcd}[cramped, column sep=small]
		& y\ar[dr,"\id_y"] & \\
		x\ar[ur,"g"]\ar[rr,"f"] & & y
	\end{tikzcd}
\end{center}
and therefore
\[ d(x\overset{g}\to y) \leq d(x\overset{f}\to y) + d(x\overset{\id_x}\to x), \quad d(x\overset{f}\to y) \leq d(x\overset{g}\to y) + d(x\overset{\id_x}\to x) \]
but since \(d(\id_x) = 0\), we thus have
\[ d(x\overset{g}\to y) \leq d(x\overset{f}\to y) \leq d(x\overset{g}\to y) \]
as desired.
\end{proof}

\begin{corollary}
	Let \(\calC\) be an \(\infty\)-category. Then there is a bijection
	\[ \{ \text{metrics on }\calC \}\cong\{ \text{metrics on }\ho(\calC) \}. \]
	Furthermore, the following statements hold about this bijection.
	\begin{enumerate}[label=(\arabic*)]
		\item For a metric \(d\) on \(\calC\), a sequence \(x_1 \to x_2 \to \cdots\) is Cauchy if and only if the sequence taken in \(\ho(\calC)\) is Cauchy with respect to the corresponding metric on \(\ho(\calC)\).
		\item Two metrics on \(\calC\) are equivalent if and only if the corresponding metrics on \(\ho(\calC)\) are equivalent.
	\end{enumerate}
\end{corollary}
\begin{proof}
Starting with a metric \(d\) on \(\calC\), one obtains a metric \(d'\) on \(\ho(\calC)\) given by \(d'([f]) = d(f)\). By Proposition \ref{prop:metric-invariant-under-homotopy}, this is well-defined,
and it is quite clear that one gets a metric. Conversely, starting with a metric \(d'\) on \(\ho(\calC)\), one obtains a metric \(d\) on \(\calC\) given by the same formula. These two operations
are inverse to each other, which provides the bijection. The remaining claims are obvious.
\end{proof}

\begin{remark}
	In particular, we could have originally defined metrics only for 1-categories, extending the notion to \(\infty\)-categories by saying that a metric on an \(\infty\)-category \(\calC\)
	is a metric on \(\ho(\calC)\).
\end{remark}

\subsection{Non-Archimedean metrics \& valued \(\infty\)-categories}
\begin{definition}
	Let \(\calC\) be an \(\infty\)-category. A metric \(d\) on \(\calC\) is \emph{non-Archimedean} (or an \emph{ultrametric}) if it satisfies the following stronger
	version of the triangle inequality, also known as the ultrametric inequality.
	\begin{itemize}[label=(2')]
		\item For any commutative triangle in \(\calC\) as below left, we have inequality as below right:
		\begin{center}
			\begin{tikzcd}[cramped, column sep=small]
				& y\ar[dr,"g"] & \\
				x\ar[ur,"f"]\ar[rr,"h"] & & z
			\end{tikzcd}\(\quad \leadsto \quad d(x\overset{h}\to z) \leq \max\{d(x\overset{f}\to y), d(y\overset{g}\to z)\}. \)
		\end{center}
	\end{itemize}
	A metric which is \emph{not} non-Archimedean is called \emph{Archimedean.}
\end{definition}

We will show that the data required for a non-Archimedean metric can be significantly simplified. The first step is the following, which should be familiar from topology.
\begin{lemma}\label{lemma:metric-category-composition-with-nice-function-yields-equivalent-metric}
	Let \(\calC\) be an \(\infty\)-category, and let \(d\) be a metric on \(\calC\). Suppose there is a weakly increasing function \(f\!:\R_{\geq 0}\to\R_{\geq 0}\) such
	\(\lim_{t\to 0^+}f(t) = 0\) and either one of the following conditions holds.
	\begin{enumerate}[label=(\alph*)]
		\item The function \(f\) is subadditive: \(f(a+b) \leq f(a) + f(b)\).
		\item The metric \(d\) is non-Archimedean.
	\end{enumerate}
	Then \(d' = f\circ d\) is a metric which is equivalent to \(d\). If (b) holds, then \(d'\) is also non-Archimedean.
\end{lemma}
\begin{proof}
Observe that since \(f\) is increasing and tends to zero as \(t\to 0\), we have \(f(0)=0\). First, we show that \(d'\) is a metric. For this, we check the axioms.
\begin{enumerate}[label=(\arabic*)]
	\item Let \(x\in\calC\). Then
	\[ d'(x\overset{\id_x}\to x) = f(d(x\overset{\id_x}\to x)) = f(0) = 0. \]
	\item Consider a commutative triangle \((f\!:x\to y,g\!:y\to z,h\!:x\to z)\). If we are in case (a), then
	\begin{align*}
		d'(x\overset{h}\to z) &= f(d(x\overset{h}\to z)) \\
		&\leq f(d(x\overset{f}\to y) + d(y\overset{g}\to z)) \\
		&\leq f(d(x\overset{f}\to y)) + f(d(y\overset{g}\to z))) = d'(x\overset{f}\to y) + d'(y\overset{g}\to z).
	\end{align*}
	On the other hand, if we are in case (b), then since \(f\) is order-preserving, we have \(f(\max\{a,b\}) = \max\{f(a),f(b)\}\), and hence the above computation
	follows through in an identical fashion.
\end{enumerate}
It follows easily that \(d'\) is non-Archimedean if \(d\) is non-Archimedean.

Now we must show that \(d'\) is equivalent to \(d\). Let \(\varepsilon > 0\), and set \(\delta_0\) such that \(f(\delta_0) < \varepsilon\). This is possible by the requirement that
\(\lim_{t\to 0^+}f(t) = 0\). Let us consider \(\delta = \min\{\delta_0, f(f(\delta_0))\}\). Then
\[ d(x\to y) < \delta \implies f(d(x\to y)) = d'(x\to y) \leq f(\delta) \leq f(\delta_0) < \varepsilon. \]
This proves one direction. For the other, note that
\[ d'(x\to y) < \delta \implies f(d(x\to y)) \leq f(f(\delta_0)) \implies d(x\to y) \leq f(\delta_0) < \varepsilon \]
as desired.
\end{proof}

\begin{lemma}
	Let \(\calC\) be an \(\infty\)-category, and let \(d\) be a non-Archimedean metric. Consider any strictly decreasing sequence \((r_n)_{n=1}^{\infty}\) in \(\R_{\geq 0}\) such that \(r_n\to 0\) as \(n\to\infty\).
	Then there is an equivalent non-Archimedean metric \(d'\) on \(\calC\) taking values in \(\{r_n \mid n \geq 1\}\).
\end{lemma}
\begin{proof}
We intend to apply Lemma \ref{lemma:metric-category-composition-with-nice-function-yields-equivalent-metric}. For simplicity, we assume \((r_n)_{n=1}^\infty\) is normalized such that \(r_1=1\).
First, we modify \(d\) to get a metric \(d_0\) that is bounded by \(1\). For this, apply the lemma to \(f_0(t) = t/(1+t)\) and set \(d_0 = f_0\circ d\). To get \(d'\), we define the function
\[ f\!: [0,1] \to \R_{\geq 0},\quad f(t) := \inf\{ r_n \mid n \geq 1,\, t \leq r_n \}. \]
In other words, \(f(t)\) is the smallest \(r_n\) larger than \(t\). Now, \(f\) is weakly increasing and clearly satisfies \(\lim_{t\to 0^+}f(t) = 0\). Since \(d\)
is assumed to be non-Archimedean, so is \(d_0\), and it follows that \(d' := f\circ d_0\) is a non-Archimedean metric. Note that, while the domain of \(f\) is not as
in the statement of the lemma, since the image of \(d_0\) is in \([0,1]\) there is no issue.
\end{proof}
\begin{remark}
	The above lemma is stated without proof in \cite{neeman2021metricstriangulatedcategories} in greater generality, namely for an arbitrary metric rather than a non-Archimedean one.
	However, this cannot hold with the same strategy we used: there are counter-examples using the above \(f\). The issue is that the function \(f\) is not subadditive,
	and there is no clear choice for one that is.
\end{remark}

\begin{definition}
	Let \(\calC\) be an \(\infty\)-category. A \emph{discrete valuation} on \(\calC\) is a map \(v\!:\Mor(\calC)\to\Z_{\geq 0}\cup\{\infty\}\) satisfying the following properties.
	\begin{enumerate}[label=(\arabic*)]
		\item For all \(x\in\calC\), we have \(v(x\overset{\id_x}\to x) = \infty\).
		\item For any commutative triangle as below left, we have the inequality as below right:
		\begin{center}
			\begin{tikzcd}[cramped, column sep=small]
				& y\ar[dr,"g"] & \\
				x\ar[ur,"f"]\ar[rr,"h"] & & z
			\end{tikzcd}\(\quad \leadsto \quad v(x\overset{h}\to z) \geq \min\{v(x\overset{f}\to y), v(y\overset{g}\to z)\}. \)
		\end{center}
	\end{enumerate}
\end{definition}

\begin{proposition}
	Let \((\calC,d)\) be a non-Archimedean metric \(\infty\)-category, and consider the full subcategories
	\[ \forall x\in\calC,\, \forall n\geq 1,\quad B_n(x) := \left\{ x\to y \mid d(x\to y) \leq \frac{1}{n} \right\} \subseteq x/\calC. \]
	Define the map
	\[ v\!: \Ob(x/\calC) \to \Z_{\geq 1}\cup\{\infty\},\quad v(x\to y) := \inf\{ n \mid n\geq 1,\, (x\to y) \not\in B_n(x) \}. \]
	Then the following statements hold.
	\begin{enumerate}[label=(\arabic*)]
		\item If \(m \geq n\) then \(B_m(x) \subseteq B_n(x)\).
		\item 
	\end{enumerate}
\end{proposition}

\begin{theorem}
	Let \(\calC\) be an \(\infty\)-category. Then there is a bijection
	\[ \left\{ \begin{array}{c}
		\text{equivalence classes of}\\
		\text{non-Archimedean metrics on } \calC
	\end{array} \right\} \cong  \]
\end{theorem}


%!TEX root = ../lectures.tex

\section{Derived categories (TBD)}
\subsection{Chain homotopies \& the homotopy category of chain complexes}
\subsection{The triangulated structure on \(\bfK(\calA)\)}
\subsection{The derived category \(\sfD(\calA)\)}
\subsection{Appendix: Agreement between different definitions}
\subsection{Appendix: The snake lemma}


%!TEX root = ../lectures.tex

\section{Sheaves on sites (TBD)}
\subsection{Sieves}
\subsection{Grothendieck topologies}
\subsection{Sites \& sheaves}
\subsection{Sheafification}
\end{comment}


\clearpage
\phantomsection
\addcontentsline{toc}{section}{References}
\printbibliography
\end{document}


% \begin{center}
% \begin{tikzcd}
% 	z\ar[dr,dashed]\ar[drr,bend left]\ar[ddr,bend right] & & \\
% 	& x\ar[r]\ar[d,hook] & * \ar[d,hook] \\
% 	& y\ar[r,"\varphi"] & \Omega
% \end{tikzcd}\(\quad \iff\quad\)
% \begin{tikzcd}
% 	 & z\ar[d,"\psi"]\ar[dl,dashed] & \\
% 	x\ar[r,hook]\ar[dr,hook] & y\times *\ar[d,equal,"\sim" labl] \ar[r,shift left]\ar[r,shift right,"\varphi"'] & \Omega \\
% 	& y &
% \end{tikzcd}
% \end{center}
