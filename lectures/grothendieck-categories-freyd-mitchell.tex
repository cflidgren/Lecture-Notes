%!TEX root = ../lectures.tex

\section{Grothendieck categories \& the Freyd--Mitchell embedding theorem}
This lecture concerns properties of Grothendieck categories, with the motivating goal of sketching the famous result that any small Abelian category may be embedded inside a module category (over some typically noncommmutative ring).
The fundamental strategy is to apply a result of the kind found in Appendix \ref{appendix:abelian-categories-with-a-compact-projective-generator}, which we begin with below, and show
that a suitable situation can be constructed for any small Abelian category through the use of Pro-categories (which are the duals of Ind-categories). In particular, one considers a composition
\[ \calA\inj\Pro(\calA) \to \Mod_A \]
for a suitable ring \(A\).

\subsection{A criterion for embedding into a module category}
In Appendix \ref{appendix:abelian-categories-with-a-compact-projective-generator}, we provided a criterion for being equivalent to a particular module category.
However, with some care, one can change the assumptions by dropping compactness to give a mere embedding, at least for small subcategories, which is good enough.

The underlying reason we can drop a compactness condition is the following lemma, telling us that as long as we are only working with a small set of objects,
we can get by with a finite amount of data.
\begin{lemma}\label{lemma:small-set-quotients-of-projective-generator}
	Let \(\calA\) be a locally small Abelian category with a projective generator, and let \(\calA'\subseteq\Ob(\calA)\) be a small set.
	Then there is a projective generator \(p'\) of \(\calA\) such that each object in \(\calA'\) is a quotient of \(p'\).
\end{lemma}
\begin{proof}
For any \(x\in\calA\), there is an epimorphism
\[ \coprod_{p\to x}p\sur x. \]
We thus set
\[ p' := \coprod_{x\in\calA'}\coprod_{p\to x}p \]
which exists, as \(\calA'\) is small. From this, we can componentwise define an epimorphism of the form \(p'\sur x\) for any \(x\in\calA'\), so that \(p'\) is a generator.
Since coproducts of projectives are projective, \(p'\) is a projective generator.
\end{proof}

With this in place, we can basically run through a simplified version of the proof of Theorem \ref{thm:abelian-cat-with-compact-projective-generator}.

\begin{theorem}\label{thm:embedding-of-small-category-into-module-category}
	Let \(\calA\) be a locally small Abelian category with a projective generator, and let \(\calA'\subseteq\calA\) be a small Abelian subcategory.
	Then there is a ring \(A\) and a fully faithful exact embedding
	\[ \calA'\inj\Mod_A \]
	of \(\calA'\) into the category of right modules over \(A\).
\end{theorem}
\begin{proof}
By Lemma \ref{lemma:small-set-quotients-of-projective-generator}, we have a projective generator \(p\in\calA\) for which every \(x\in\calA'\) has a surjection
\[ p\sur x. \]
We let \(A = \End(p)\). By (1) in Theorem \ref{thm:abelian-cat-with-compact-projective-generator}, the functors
\[ \calA(p,-)\!:\calA\to\Mod_A,\quad \leadsto\quad \calA(p,-)\!:\calA'\to\Mod_A \]
are faithful. We need to show that the latter functor, i.e.\ the restriction of \(\calA(p,-)\) to \(\calA'\), is full. For this, we can essentially apply the same proof
as (2) in Theorem \ref{thm:abelian-cat-with-compact-projective-generator}, except that we no longer need compactness. Consider an \(A\)-linear homomorphism
\[ \varphi\!:\calA(p,x)\to\calA(p,y),\quad x,y\in\calA'. \]
Consider an exact sequence
\[ 0 \to \ker\pi\overset\kappa\inj p\overset\pi\sur x\to 0. \]
We get a morphism \(\varphi(\pi)\!:p\to y\), and so we consider the situation
\begin{diagram*}
	& & p\ar[d,"\kappa\circ h"]\ar[dl,"h"'] & & \\
	0\ar[r] & \ker\kappa\ar[r,hook,"\kappa"] & p\ar[r,two heads,"\pi"]\ar[d,"\varphi(\pi)"] & x\ar[r] & 0 \\
	& & y & &
\end{diagram*}
where we need to check that for all \(h\!:p\to\ker\pi\), the composition satisfies \(\varphi(\pi)\circ\kappa\circ h = 0\). For this,
\[ \varphi(\pi)\circ\kappa\circ h = \varphi(\pi\circ\kappa\circ h) = \varphi(0\circ h) = 0 \]
so that \(\varphi(\pi)\circ\kappa = 0\). Thus, we get an induced map \(f\!:x\to y\) such that \(f\circ\pi = \varphi(\pi)\). To see that \(f_* = \varphi\), applying \(\calA(p,-)\) to our starting
exact sequence yields that
\[ \calA(p,p)\overset{\pi_*}\sur\calA(p,x) \]
since \(p\) is projective, so it suffices to see that \(\varphi\circ\pi_* = f_*\circ \pi_*\). However, for any \(a\!:p\to p\), we have
\[ (f_*\circ\pi_*)(a) = (f\circ\pi)_*(a) = \varphi(\pi)\circ a = \varphi(\pi\circ a) = (\varphi\circ\pi_*)(a). \]
We conclude that \(\calA(p,-)\!:\calA'\to\Mod_{\End(p)}\) is fully faithful. It is, in addition, exact since \(p\) is projective.
\end{proof}
\begin{remark}
	Note that in the above, by an \emph{Abelian subcategory,} we mean a subcategory which is Abelian and for which the inclusion functor is exact.
\end{remark}

\subsection{Grothendieck categories}
\begin{definition}
	A \emph{Grothendieck category} is a locally small Abelian category admitting a generator and small colimits, for which small filtered colimits are exact.
\end{definition}

\begin{definition}
	Let \(\calC\) be a category, and let \(x\in\calC\). A \emph{subobject} of \(x\) is an equivalence class of monomorphisms \(y\inj x\), under the equivalence
	relation \((y\inj x)\sim (y'\inj x)\) if and only if there is a commutative triangle
	\begin{diagram*}[cramped, column sep=small]
		y \ar[rr,"\sim"]\ar[dr,hook] & & y'\ar[dl,hook] \\
		& x. &
	\end{diagram*}
	We write \(y\subseteq x\) to say that \(y\) is a subobject of \(x\), choosing a particular representative of the equivalence class. Dually, a \emph{quotient} of \(x\) is an equivalence class of epimorphisms \(x\sur y\).
\end{definition}

\begin{lemma}\label{lemma:faithful-functors-reflects-monomorphisms-epimorphisms}
	Let \(F\!:\calC\to\calD\) be a faithful functor. Then \(F\) reflects monomorphisms and epimorphisms.
\end{lemma}
\begin{proof}
We prove that it reflects monomorphisms, as the other case is dual. Let \(f\!:x\to y\) be a morphism in \(\calC\), and suppose that \(F(f)\) is monic. Note that
\[ z \underset{b}{\overset{a}{\rightrightarrows}} x \overset{f}\to y \quad\leadsto\quad Fz \underset{Fb}{\overset{Fa}{\rightrightarrows}} Fx \overset{Ff}\to Fy \]
so that \(Fa = Fb\), hence \(a=b\) since \(F\) is faithful.
\end{proof}

\begin{lemma}\label{lemma:generators-in-balanced-categories-detect-isomorphisms}
	Let \(\calC\) be a balanced category, and let \(\calU\) be a small set of objects. Consider the following conditions.
	\begin{enumerate}[label=(\arabic*)]
	\item \(\calU\) generate \(\calC\).
	\item Suppose we have a morphism \(f\!:x\to y\) in \(\calC\). If
	\[ \forall u\in\calU,\quad f_*\!:\calA(u,x)\iso\calA(u,y) \]
	then \(f\) is an isomorphism.
	\end{enumerate}
	Then (1) implies (2), and if \(\calC\) admits equalizers then (2) implies (1).
\end{lemma}
\begin{proof}
(1) implies (2): the statement that \(\calU\) is a small set of generators is the same as saying that the functor
\[ \prod_{u\in\calU}\calC(u,-)\!:\calC\to\Set \]
is faithful. By Lemma \ref{lemma:faithful-functors-reflects-monomorphisms-epimorphisms}, it therefore reflects epimorphisms and monomorphisms; since \(\calC\) and \(\Set\) are balanced,
this means it is conservative. The condition (2) is exactly that \(f\) is sent to an isomorphism under the above functor.

(2) implies (1): by assumption \(\prod_{u\in\calU}\calC(u,-)\) is a convervative functor; let \(f,g\!:x\to y\) be a pair of parallel arrows, and consider \(k\!:\eq(f,g)\inj x\).
Since \(\prod_{u\in\calU}\calC(u,-)\) commutes with limits, we have
\[ \prod_{u\in\calU}\calC(u,\eq(f,g)) \iso \eq((f_*)_{u\in\calU},(g_*)_{u\in\calU}). \]
If we assume that \(f\circ h = g\circ h\) for all \(h\!:u\to x\), then the latter set is just \(\prod_{u\in\calU}\calC(u,x)\), so by conservativity we get that
\(\eq(f,g)\iso x\), hence \(f=g\).
\end{proof}

% \begin{lemma}
% 	Let \(\calA\) be a Grothendieck category with generator \(u\), and let \(\kappa\) be an infinite regular cardinal. Consider a small \(\kappa\)-filtered diagram \(D\!:I\to\calA\),
% 	and an epimorphism \(\injlim{D}\sur y\). Assume that one of the following conditions hold:
% 	\begin{enumerate}[label=(\alph*)]
% 	\item \(\#\calA(u,y) < \kappa\), or
% 	\item \(y\in\calA^\kappa\), i.e.\ \(y\) is \(\kappa\)-compact.
% 	\end{enumerate}
% 	Then there is some \(i\in I\) such that the induced map \(D(i)\to y\) is an epimorphism.
% \end{lemma}
% \begin{proof}
% Set \(y_i := \img(D(i)\to y)\). This induces a new diagram \(D'\!:I\to\calA\). Since small filtered colimits are exact, they commute with taking images, so
% \[ \injlim{D'} = \injlim_{i\in I}\img(D(i)\to y) \cong \img(\injlim{D}\sur y) \cong y. \]

% (a) Let \(S = \injlim_{i\in I}\calA(y_i,u)\), viewed as a subset of \(\calA(u,y)\). By assumption, \(\#S \leq \#\calA(u,y) < \kappa\), so \(S\) is \(\kappa\)-compact
% by the computation in Example \ref{example:compact-objects-in-Set}. Therefore,
% \[ \injlim_{i\in I}\Set(S,\calA(u,y_i)) \cong \Set(S,\injlim_{i\in I}\calA(u,y_i)) = \Set(S,S). \]
% We thus find some \(i_0\in I\) and morphism \(S\to \calA(y_{i_0},u)\) such that
% \[ (S\to \calA(u,y_{i_0}) \inj S) = \id_S. \]
% However, then the inclusion on the left is also surjective, so \(S = \calA(y_{i_0},u)\), from which we deduce that for any \((i_0\to i)\in I\), the induced map
% \(\calA(u,y_{i_0})\to\calA(u,y_i)\) is bijective. Since \(u\) is a generator, we conclude by Lemma \ref{lemma:generators-in-balanced-categories-detect-isomorphisms} that \(y_{i_0}\iso y_i\), hence \(y_{i_0}\iso y\). But then
% \[ \img(D(i)\to y) \cong y, \]
% so \(D(i)\sur y\).

% (b) If \(y\) is \(\kappa\)-compact, then
% \[ \injlim_{i\in I}\calA(y,y_i) \iso \calA(y,y). \]
% But then, once again, the identity factors as
% \begin{diagram*}[cramped,column sep=small]
% 	y\ar[rr,equal]\ar[dr] & & y \\
% 	& y_i\ar[ur,hook] & 
% \end{diagram*}
% so the canonical inclusion \(y_i\inj y\) is an epimorphism. Since Abelian categories are balanced, this implies \(y_i\cong y\); as before, we conclude that \(D(i)\sur y\).
% \end{proof}


\begin{proposition}\label{prop:grothendieck-categories-small-set-of-subobjects}
	Let \(\calA\) be a Grothendieck category, and let \(x\in\calA\). Then the set of subobjects of \(x\) and the set of quotients of \(x\) are small.
\end{proposition}
\begin{proof}
Note that any Abelian category admits finite limits, since they admit finite products and equalizers. Now, denote by \(S(x)\) the set of subobjects of \(x\), and let \(u\in\calA\)
be a generator. The functor \(\calC(u,-)\) provides a map
\[ \rho\!: S(x) \to \calP(\calC(u,x)), \quad (x'\overset\iota\inj x)\mapsto\iota_*(\calC(u,x'))\subseteq\calC(u,x) \]
where we note that \(\iota_*\) is injective since \(\iota\) is a monomorphism. By assumption, \(\calA\) is locally small, hence \(\calP(\calC(u,x))\) is small, so it suffices
to see that the above map of sets \(\rho\) is injective.

First of all, we have that \(\rho(x')\cong\calC(u,x')\) for any \(x'\subseteq x\). Let \(x'\subseteq x\) and \(x''\subseteq x\). The pullback \(x'\times_x x''\) is then also a subobject of \(x\),
and if \(\rho(x') = \rho(x'')\) then
\[ \calC(u,x'\times_xx'') \cong \calC(u,x')\times_{\calC(u,x)}\calC(u,x'') \cong \rho(x')\cap\rho(x'') = \rho(x') = \rho(x''). \]
Applying Lemma \ref{lemma:generators-in-balanced-categories-detect-isomorphisms}, we see that \( x'\xleftarrow\sim x'\times_xx''\xrightarrow\sim x'' \) and in particular \(x'\cong x''\).

To see that the set of quotients is small, note that \(f\!:x\to y\) is an epimorphism if and only if
\[ 0\to \ker{f} \inj x \overset{f}\to y \to 0 \]
is exact, which uniquely identifies \(y\) as the cokernel of \(\ker{f}\inj x\). In other words, quotients of \(x\) correspond to subobjects, which form a small set.
\end{proof}

\begin{proposition}\label{prop:grothendieck-category-injectives-detected-by-generator}
	Let \(\calA\) be a Grothendieck category, and let \(\calU\) be some small set of generators. Let \(z\in\calA\). Then the following are equivalent:
	\begin{enumerate}[label=(\arabic*)]
	\item \(z\) is injective.
	\item For all \(u\in\calU\) and all subobjects \(v\subseteq u\), the restriction map
	\[ \calA(u,z)\to\calA(v,z) \]
	is surjective.
	\end{enumerate}
\end{proposition}
\begin{proof}
Certainly (1) implies (2). Conversely, assume (2). We want to show that \(z\) is injective, so consider the situation
\begin{diagram*}
	x'\ar[r,hook,"f"]\ar[d,"h"'] & x\ar[dl,dashed,"\exists?"] \\
	z
\end{diagram*}
where we want to find the dashed arrow. We consider the category \(\Delta(h)\) whose objects are diagrams \(D\) of the form
\begin{diagram*}
	x'\ar[r,hook,"k"]\ar[d,"h"']\ar[rr,bend left,"f"] & y\ar[dl,"g"]\ar[r,hook,"\ell"] & x \\
	z
\end{diagram*}
which we denote \(D = (y,k,g,\ell)\), and a morphism \(D\to D' = (y',k',g',\ell')\) is a morphism \(t\!:y\to y'\) making the diagram
\begin{diagram*}
	y\ar[r,"t"] \ar[rr,bend left,hook,"\ell"]\ar[d,"g"'] &  y'\ar[r,hook,"\ell'"]\ar[dl,"g'"] & x \\
	z
\end{diagram*}
commute. Note that \(t\) is automatically a monomorphism, and that since \(\ell'\) is a monomorphism, if a morphism \(D\to D'\) exists, then it is unique.
Now, let \(\Sigma(h)\) be the set of isomorphism classes in \(\Delta(h)\); this is a small set by Proposition \ref{prop:grothendieck-categories-small-set-of-subobjects}.
We turn it into a poset by letting \(D\leq D'\) if and only if there exists a morphism \(D\to D'\). One sees that \(\Delta(h)\simeq\Sigma(h)\).

There is a functor \(\Delta(h) \to \calA\) given by sending \((y,k,g,\ell)\) to \(y\), which induces a functor \(\Sigma(h)\to\calA\). We can use this to see that any chain in \(\Sigma(h)\) has an upper bound: given a chain
\[ \alpha\!:I\to\Sigma(h), \]
since small filtered colimits are exact in \(\calA\), we may take the colimit of the induced diagram \(\alpha'\!:I\to\calA\), which is filtered, and hence the induced map \(\injlim\alpha' \to x\)
is injective by exactness. Then \(\injlim\alpha'\) together with its canonically induced maps to \(x\) and \(z\) is an upper bound for the chain \(\alpha\).

By Zorn's lemma, \(\Sigma(h)\) has a maximal element, say \((y_0,k_0,g_0,\ell_0)\). By Lemma \ref{lemma:generators-in-balanced-categories-detect-isomorphisms}, to check that \(y_0 \iso x\),
it suffices to check that the injective maps
\[ \forall u\in\calU,\quad \ell_*\!:\calC(u,y_0)\inj\calC(u,x) \]
are also surjective. Thus, consider a morphism \(\varphi\!:u\to x\), and define an object \(y\in\calA\) by the pullback
\begin{diagram*}
	y\ar[dr,pullback]\ar[r,dashed]\ar[d,dashed] & y_0\ar[d,hook,"\ell"] \\
	u\ar[r,"\varphi"] & x.
\end{diagram*}
Since \(\ell\) is monic, so is the induced map \(y\to u\). Now define an object \(y_1\in\calA\) by the pushout
\begin{diagram*}
	y\ar[r]\ar[d,hook] & y_0\ar[d,dashed] \\
	u\ar[r,dashed] & y_1.\ar[ul,pushout]
\end{diagram*}
Expressing pullbacks in terms of equalizers, and that in terms of kernels, we have an exact sequence
\[ 0 \to y \to y_0\oplus u \overset{r}\to x \]
but doing the same for the definition of \(y_1\), one sees that \(y_1 \cong \img{r} \subseteq x\), and we deduce that the canonical map \(y_1\to x\) is monic. Now, \(y\) is also a
subobject of \(u\), so by (2), the composite \(y\to y_0 \to z\) factors as in the diagram
\begin{diagram*}
	y\ar[r, hook]\ar[d] & u\ar[dr,"\varphi"]\ar[ddr,dashed]\ar[d] & \\
	y_0\ar[r,hook]\ar[drr] & y_1\ar[dr,dashed] & x\ar[from=l,hook,crossing over] \\
	& & z
\end{diagram*}
which induces the dashed arrow \(y_1\to z\) by universal property. This endows \(y_1\) with the structure of an object in \(\Delta(x)\), so by maximality
of \(y_0\), we have that \(y_0\cong y_1\) so that \(\varphi\) factors through \(y_0\) as desired.
\end{proof}

The below theorem is the fruit of our labour, and it crucially relies essentially on the above proposition.

\begin{theorem}
	Let \(\calA\) be a Grothendieck category. Then the following statements hold.
	\begin{enumerate}[label=(\arabic*)]
	\item \(\calA\)	has enough injectives.
	\item \(\calA\) has an injective cogenerator, i.e.\ \(\calA^\op\) has a projective generator.
	\end{enumerate}
\end{theorem}
\begin{proof}
Let \(u\) be a generator of \(\calA\).

(1) Let \(x\in\calA\). For any ordinal \(\alpha\), we define an object \(M_\alpha(x)\in\calA\), and we will do this in such a way that we can check condition (2) in
Proposition \ref{prop:grothendieck-category-injectives-detected-by-generator}. We do this inductively; consider the set
\[ I(x) := \{ (v,f)\mid v\subseteq u,\, f\!:v\to x \}. \]
We define \(M_1(x)\) by the pushout
\begin{diagram*}
	\coprod_{(v,f)\in I(x)}v\ar[r]\ar[d,hook] & x\ar[d,dashed,hook] \\
	\coprod_{(v,f)\in I(x)}u\ar[r,dashed] & M_1(x)
\end{diagram*}
so that \(M_1\) is the cokernel of the map
\[ \coprod_{\mathclap{(v,f)\in I(x)}}v \to x\oplus \coprod_{(v,f)\in I}u \]
induced by the maps \(f\!:v\to x\) and the inclusions \(v\subseteq u\). If \(f\!:v\to x\) is a morphism from a subobject of \(u\) to \(x\), then it extends to a map \(u\to M_1(x)\) in the obvious way.
We have thus embedded \(x\) in some object \(M_1(x)\), but it is not good enough as it is not injective.

We now define \(M_\alpha(x)\) for an ordinal \(\alpha\) through transfinite induction in such a way that there are monomorphisms
\[ M_{\kappa'}(x) \inj M_{\kappa}(x) \]
whenever \(\kappa' < \kappa < \alpha\), such that these are compatible with the ordering on ordinals. Assume that \(M_\beta\) and the above monomorphisms are defined for all \(\beta < \alpha\). If \(\alpha = \beta+1\),
then set \(M_\alpha(x) := M_1(M_\beta(x))\); we get a monomorphism \(M_\beta(x)\inj M_{\alpha}(x)\) by the above. If \(\alpha\) is a limit ordinal, then \(\{ M_{\beta}(x)\mid \beta < \alpha \}\) defines a directed system
and we let \(M_\alpha(x)\) be its colimit. For any \(\beta < \alpha\), the canonical map \(M_\beta(x)\to M_\alpha(x)\) is then a monomorphism.

Now let \(\Omega\) be the smallest infinite ordinal whose cardinality is strictly greater than the cardinality of the set of subobjects of \(u\). We claim that \(M_\Omega(x)\) is injective.
Let \(v\subseteq u\) be a subobject, and let \(f\!:v\to M_\Omega(x)\) be a morphism. One obtains a system of subobjects \(v_\kappa\subseteq u\), given by the pullback
\begin{diagram*}
	v_\kappa\ar[dr,pullback]\ar[r,dashed]\ar[d,hook,dashed] & M_\kappa(x) \ar[d,hook] \\
	v\ar[r,"f"] & M_\Omega(x)
\end{diagram*}
and one sees that
\[ \injlim_{\kappa < \Omega}v_\kappa = \injlim_{\kappa < \Omega}(v\times_{M_\Omega(x)}M_\kappa(x)) \cong v\times_{M_\Omega(x)}\injlim_{\kappa < \Omega}(M_\kappa(x)) = v\times_{M\Omega(x)}M_\Omega(x) = v \]
since filtered colimits in \(\calA\) commute with finite limits. In particular, \(v_\Omega = v\). However, since \(\Omega\) is strictly larger than the number of subobjects of \(u\), there
must be some \(\kappa < \Omega\) for which \(v = v_\kappa\), so \(f\!:v\to M_\Omega(x)\) factors through some map \(v\to M_\kappa(x)\), which we can then extend:
\begin{diagram*}[cramped,sep=small]
	v\ar[dd,hook]\ar[rr] & & M_\kappa(x)\ar[dd,hook]\ar[dl,hook] \\
	& M_{\kappa+1}(x)\ar[dr,hook] & \\
	u\ar[ur,dashed]\ar[rr,dashed] & & M_\Omega(x)
\end{diagram*}
so we conclude that \(M_\Omega(x)\) is injective by Proposition \ref{prop:grothendieck-category-injectives-detected-by-generator}.

(2) By Proposition \ref{prop:grothendieck-categories-small-set-of-subobjects}, the generator \(u\) has a small set of quotients; hence there is a set \(Q\) of objects such that any quotient of \(u\) is isomorphic
to an object in \(Q\). Let \(q_0 = \bigoplus_{z\in Q}z\), and consider an embedding \(q_0\inj q\) into an injective \(q\in\calA\), which exists by (1). We aim to show that \(q\) is a cogenerator. We begin by showing that
\[ \calA(x,q)\cong 0\implies x\cong0. \]
Suppose the left condition holds. For any map \(h\!:u\to x\), the below right map
\[ \img{h} \inj x \quad\leadsto\quad \calA(x,q)\to\calA(\img{h},q) \]
is surjective since \(q\) is injective. Therefore, if \(\calA(x,q)\cong 0\), we see that \(\calA(\img{h},q) \cong 0\). Now, \(\img{h}\) is a quotient of \(u\), hence isomorphic to some \(z\in Q\) and
there is a monomorphism
\[ \img{h} \iso z\inj q_0 \inj q. \]
Therefore, \(\img{h} \cong 0\), so \(h=0\) and \(\calA(u,x)\cong 0 \cong \calA(u,0)\). We conclude that \(x\cong 0\).

Finally, consider a morphism \(f\!:x\to y\) for which \(f^*\!:\calA(y,q)\iso\calA(x,q)\); it suffices to show that \(f\) is an isomorphism. Since \(\calA(-,q)\) is exact, we have an exact sequence
\[ 0 \to \calA(\coker{f},q) \to \calA(y,q)\iso \calA(x,q)\to\calA(\ker{f},q)\to 0 \]
which shows that \(\calA(\coker{f},q)\cong\calA(\ker{f},q)\cong 0\), so by what we have shown, \(\coker{f}\cong\ker{f}\cong 0\), so \(f\) is an isomorphism.
\end{proof}
\begin{exercise}
	\begin{enumerate}[label=(\arabic*)]
	\item Prove that in an Abelian category, the pushout of a monomorphism is a monomorphism
	\begin{diagram*}
		x\ar[r]\ar[d,hook] & y\ar[d,dashed,hook] \\
		x'\ar[r,dashed] & y'\ar[ul,pushout]
	\end{diagram*}
	by proving that the above diagram is also a pullback diagram, and that in a pullback diagram the kernels of the vertical arrows agree.
	\item Prove that in a Grothendieck category, the canonical maps to the colimit of a directed system whose transition maps are monic, are monic:
	\begin{diagram*}[sep=small]
		\cdots\ar[r,hook] & x_i\ar[r,hook]\ar[dr,hook] & \cdots \ar[r,hook] & x_{i'}\ar[dl,hook] \ar[r,hook] & \cdots \\
		& & \injlim_{k\in I}x_k & &
	\end{diagram*}
	\end{enumerate}
\end{exercise}

\subsection{Closure properties of Ind-categories \& the Embedding Theorem (WIP)}
\begin{theorem}
	Let \(\calA\) be a small Abelian category. Then \(\Ind(\calA)\) is a Grothendieck category, and the inclusion \(\calA\inj\Ind(\calA)\) is exact.
\end{theorem}

Proving this is done by studying the closure properties of \(\Ind(\calA)\) under (co)limit constructions.
\begin{proposition}
	Let \(\calC\) be a locally small category, and assume that \(\calC\) admits finite limits. Then the following statements hold.
	\begin{enumerate}[label=(\arabic*)]
	\item \(\Ind(\calC)\) admits finite limits.
	\item \(\Ind(\calC)\inj\PSh(\calC)\) preserves finite limits.
	\item \(\calC\inj\Ind(\calC)\) preserves finite limits.
	\item Filtered colimits in \(\Ind(\calC)\) are exact, i.e.\ commute with finite limits.
	\end{enumerate}
\end{proposition}
\begin{proof}
We prove (1) and (2) simultaneously, afterwhich (3) follows by noting that the Yoneda embedding \(\calC\inj\PSh(\calC)\) preserves all small limits, and (4) follows
by filtered colimits in \(\PSh(\calC)\) being exact (inheriting this property from \(\Set\)).

We show that \(\Ind(\calC)\) admits binary products and equalizers. Thus, consider \(A,B\in\Ind(\calC)\), and write
\[ A \cong \injlim_{i\in I}h_{D(i)},\quad B\cong \injlim_{j\in J}h_{D'(j)}  \]
for some filtered diagrams \(D\!:I\to\calC\), \(D'\!:J\to\calC\). Then, in \(\PSh(\calC)\),
\[ A\times B \cong \injlim_{(i,j)\in I \times J}h_{D(i)\times D'(j)} \in \Ind(\calC) \]
so \(\Ind(\calC)\) admits binary products, computed in \(\PSh(\calC)\).

To see that we have equalizers, fix some \(x\in\calC\), and consider the full subcategory \(\calC'\) of all \(A\in\PSh(\calC)\) for which the equalizer in \(\PSh(\calC)\) of any pairs of maps
\(h_x \rightrightarrows A\) is in \(\Ind(\calC)\). Then \(\calC \inj \calC'\) since \(\calC\) admits finite limits, and \(\calC'\) is closed under filtered colimits. However, by universal property,
we must then have that \(\calC'\simeq\Ind(\calC)\). Now we reverse the roles, and consider
\end{proof}

\begin{theorem}[Freyd--Mitchell embedding theorem]\label{thm:freyd-mitchell-embedding-theorem}
	Let \(\calA\) be a small Abelian category. Then there is a ring \(A\) and a fully faithful exact functor
	\[ \calA\inj\Mod_A. \]
\end{theorem}
\begin{proof}
Note that \(\calA^\op\) is a small Abelian category, hence we can consider the embedding
\[ \calA^\op\inj\Ind(\calA^\op). \]
Taking the opposite, we get the embedding
\[ \calA\inj\Pro(\calA). \]
Now, \(\Ind(\calA^\op)\) is a Grothendieck category, hence has an injective cogenerator, so \(\Pro(\calA) := \Ind(\calA^\op)^\op\) is a locally small
Abelian category with a projective generator. Applying Theorem \ref{thm:embedding-of-small-category-into-module-category}, we are done.
\end{proof}

\begin{remark}
	While the embedding theorem itself only applies to small Abelian categories, it may still be utilized when working with potentially large ones. In practice,
	one applies the theorem when diagram-chasing; thus, given a small diagram \(D\!:I\to\calA\), one may consider the smallest subcategory \(\calA'\) of \(\calA\) which
	\begin{enumerate}[label=(\arabic*)]
	\item is small,
	\item is Abelian,
	\item contains the image of \(D\), and for which
	\item the inclusion \(\calA'\inj\calA\) is exact.
	\end{enumerate}
	Thus, one factors \(D\) as \(I\to\calA'\inj\calA\), and embeds \(\calA'\) in a module category wherein one can do as much chasing as one wants.
\end{remark}

