%!TEX root = ../lectures.tex

\newcommand{\Cof}{\operatorname{Cof}}
\newcommand{\Fib}{\operatorname{Fib}}
\newcommand{\tailto}{\rightarrowtail}

\section{Homotopy theory in model categories (WIP)}\label{lecture:model categories}

Model categories are, in short, a systematic framework for doing homotopy theory, meaning it is a tool for controlling localizations of categories. Previously, when trying to understand
relative categories \((\calC,W)\), we have made use of the assumption that \(W\) is a (right or left) multiplicative system, which allows us to understand the morphisms a lot more
cleanly due to the calculus of fractions.

Model categories take a different approach. Essentially, they introduce what one might describe as \emph{phantom structure:} the data of \emph{fibrations} and \emph{cofibrations,}
``meaningless'' on their own, the (imposed) properties of which allow you to describe the localization in terms very similar to the homotopy theory of topological spaces.
Essentially, the (co)fibrations provide distinguished classes of ``nice'' objects and morphisms. Model categories also provide a very convenient setting in which to understand homotopical analogues
of limits and colimits, though in a sense this is hardly unique to them, and is really a consequence of a good theory of derived functors.

\subsection{Model categories}
\begin{definition}
	Let \(\calC\) be a category. A \emph{model structure} on \(\calC\) is a triple \((\Cof,W,\Fib)\) of sets of morphisms in \(\calC\) satisfying the following conditions.
	\begin{enumerate}[label=(\arabic*)]
		\item The set \(W\) has the 2-out-of-3 property.
		\item The pairs \((\Cof,\Fib\cap W)\) and \((\Cof\cap W,\Fib)\) are weak factorization systems on \(\calC\).
	\end{enumerate}
	A category \(\calC\) with a model structure is called a \emph{model category} if, in addition, \(\calC\) admits finite limits and finite colimits.
\end{definition}
\begin{terminology}
	Given a model structure \((\Cof,W,\Fib)\) on \(\calC\), one calls a morphism \(f\) in \(\calC\)
	\begin{itemize}
		\item a \emph{cofibration} if \(f\in\Cof\),
		\item a \emph{fibration} if \(f\in\Fib\),
		\item a \emph{weak equivalence} if \(f\in W\), or
		\item a \emph{trivial (co)fibration} if \(f\) is both a weak equivalence and (co)fibration.
	\end{itemize}
	Suppose \(\calC\) is a model category. One says an object \(x\in\calC\) is
	\begin{itemize}
		\item \emph{fibrant} if \(x\to *\) is a fibration, where \(*\in\calC\) is the terminal object, or
		\item \emph{cofibrant} if \(\varnothing\to x\) is a cofibration, where \(\varnothing\in\calC\) is the initial object.
	\end{itemize}
\end{terminology}
\begin{notation}
	We write \(\calC_c\) for the full subcategory of cofibrant objects; dually, \(\calC_f\) for the full subcategory of fibrant objects. Combining the two,
	we write \(\calC_{cf}\) for the full subcategory of objects which are both fibrant and cofibrant.
\end{notation}
\begin{notation}
	We adopt the following notation, in situations where no ambiguity is expected: \(\iso\) denotes a weak equivalence, \(\sur\) denotes a fibration, and \(\tailto\)
	denotes a cofibration.
\end{notation}
\begin{remark}
	Observe that a model structure has a large amount of redundant information: if one exists, it is completely determined by its weak equivalences
	and either the trivial fibrations or trivial cofibrations. This is a corollary of Proposition \ref{prop:factorization-system-equalities}; suppose we know
	\(W\) and \(\Fib\cap W\). Then
	\[ \Cof = \prescript{\boxslash}{}{(\Fib\cap W)},\quad \leadsto\quad \Fib = (\Cof\cap W)^\boxslash, \]
	which recovers the triple \((\Cof,W,\Fib)\).

	This also means that \(W\) contains a weakly saturated class of morphisms, namely \(\Fib\cap W\), and therefore contains all isomorphisms. In particular, all identities,
	so \(W\) determines a wide subcategory with the 2-out-of-3 property. We conclude that \((\calC,W)\) is a pseudo-homotopical category, in the terminology
	of Lecture \ref{lecture:homotopical-algebra-through-deformations}. In fact, we will see that \(W\) necessarily also satisfies the 2-out-of-6 property,
	so that \((\calC,W)\) is moreover a homotopical category. The reader should therefore not spend any time seeking examples of model categories whose weak
	equivalences do not satisfy the 2-out-of-6 property.
\end{remark}
\begin{remark}
	In many cases, model structures are constructed using the small object argument, such as the one we give in Proposition \ref{prop:small-object-argument}. That is,
	one has some class of basic cofibrations \(J\) from which one forms the factorization systems \((\prescript{\boxslash}{}{(J^\boxslash)},J^\boxslash)\) using the small
	object argument. Choosing some basic class of trivial cofibrations \(I\), one can then form \((\prescript{\boxslash}{}{(I^\boxslash)},I^\boxslash)\). If \(J\)
	and \(I\) are chosen such that \(J^\boxslash \subseteq W\) and \(\prescript{\boxslash}{}{(I^\boxslash)} \subseteq W\), then this will yield a model structure given by
	\[ (\Cof,W,\Fib) = (\prescript{\boxslash}{}{(J^\boxslash)},W,I^\boxslash). \]
	In general, model structures for which one can find \emph{some} collections of maps \(I\) and \(J\) generating the (trivial) cofibrations are called \emph{cofibrantly generated,} and they have a
	number of benefits. For example, since the small object argument provides functorial factorizations, any such model structure of course admits this extra structure.
\end{remark}
\begin{construction}[Opposites]
	Let \((\calC, \Cof, W, \Fib)\) be a model category. Then \((\Fib^\op, W^\op, \Cof^\op)\) is a model structure on \(\calC^\op\), so \(\calC^\op\) can be promoted to a model category
	as well. Furthermore, this demonstrates that fibrations and cofibrations are dual to each other, such that any result about cofibrations dualizes to a result about fibrations.
\end{construction}
\begin{construction}[Slices]
	Let \(\calC\) be a model category, and let \(x\in\calC\). Then \(\calC/x\) can be promoted to a model category. In particular, let a morphism in \(\calC/x\) be a weak equivalence (resp.\ fibration, cofibration)
	if it is a weak equivalence (resp.\ fibration, cofibration) in \(\calC\). Dualizing yields that \(x/\calC\) can be promoted to a model category in a similar way.
\end{construction}

\begin{proposition}
	Let \((\calC,\Cof,W,\Fib)\) be a model category. Then \(\Cof\) is weakly saturated and \(\Fib\) is weakly cosaturated. In particular,
	\begin{enumerate}[label=(\arabic*)]
		\item \(\Cof\) is closed under retracts, compositions, small coproducts, and pushouts, and
		\item \(\Fib\) is closed under retracts, compositions, small products, and pullbacks.
	\end{enumerate}
\end{proposition}
\begin{proof}
Follows by Proposition \ref{corollary:llp-weakly-saturated} and Proposition \ref{prop:factorization-system-equalities}.
\end{proof}

Recall the importance of functors being \emph{homotopical} in the construction of derived functors; see Lecture \ref{lecture:homotopical-algebra-through-deformations}. In a model category,
the morphisms which are easily controlled are the (co)fibrations, and the objects with nice properties are the (co)fibrant ones. Consequently, it is often the case that one can
more easily say things about, say, trivial cofibrations between cofibrant objects than about weak equivalences in general. The below result is useful in these circumstances.
\begin{lemma}[Ken Brown's lemma]
	Let \(\calC\) be a model category, and let \(\calD\) be a pseudo-homotopical category. Suppose we have a functor \(F\!:\calC\to\calD\). Then the following are equivalent.
	\begin{enumerate}[label=(\arabic*)]
		\item \(F\) is homotopical on the full subcategory of cofibrant objects, i.e.\ \(F\) sends weak equivalences between cofibrant objects to weak equivalences.
		\item \(F\) sends trivial cofibrations between cofibrant objects to weak equivalences.
	\end{enumerate}
\end{lemma}
\begin{proof}
The implication (1) \(\Rightarrow\) (2) is clear. For the converse, the idea is that we may rewrite any weak equivalence between cofibrant objects in terms of trivial cofibrations.
Suppose we have a morphism \(f\!:x\to y\) where \(x,y\in\calC_c\). Form their coproduct \(x\amalg y\) and factor the map \(x\amalg y \to y\) given by
\(f\!:x\to y\) and \(\id_y\!:y\to y\) into a cofibration followed by a trivial fibration
\[
	\begin{tikzcd}
		\varnothing\ar[r,tail]\ar[d,tail] & x\ar[d,"\iota_x"] \\
		y\ar[r,"\iota_y"] & x\amalg y\ar[ul,pushout]
	\end{tikzcd} \quad \leadsto \quad
	\begin{tikzcd}
		x\ar[d,tail,"\iota_x"']\ar[drr,bend left=20,"\sim","f"'] \\
		x\amalg y \ar[r,tail] & z \ar[r,"\sim"] & y \\
		y\ar[u,tail,"\iota_y"]\ar[urr,bend right=20,equal]
	\end{tikzcd}
\]
and, using the left diagram, note that since cofibrations are closed under pushouts, \(\iota_x\) and \(\iota_y\) are cofibrations. In particular, the compositions
\[ i\!:x \overset{\iota_x}\tailto x\amalg y \tailto z,\quad j\!:y \overset{\iota_y}\tailto x\amalg y \tailto z \]
are both cofibrations (by composition) and weak equivalences (by 2-out-of-3), hence trivial cofibrations. By commutativity, we have
\[ j\circ f = i \implies F(j)\circ F(f) = F(i)  \]
which implies \(F(f)\) is a weak equivalence by the 2-out-of-3 property.
\end{proof}
\begin{remark}
	The above result is a simple but good example of how (co)fibrations and (co)fibrant objects can help us despite not being intrinsically meaningful.
\end{remark}

\subsection{Cylinders, paths, and homotopies (TBD)}

\subsection{The homotopy category (TBD)}

\subsection{Derived functors for model categories (TBD)}

\subsection{Homotopy (co)limits (TBD)}
