%!TEX root = ../lectures.tex

\section{Metric \(\infty\)-categories (WIP)}
Recall that a metric \(d\!:X\times X\to \R_{\geq 0}\) on a set \(X\) is a function satisfying
\begin{enumerate}[label=(\arabic*)]
	\item \(\forall x,y\in X\), \(d(x,y) = 0\) if and only if \(x=y\),
	\item \(\forall x,y\in X\), \(d(x,y) = d(y,x)\), and
	\item \(\forall x,y,z\in X\), \(d(x,z) \leq d(x,y) + d(y,z)\).
\end{enumerate}
A pair \((X,d)\) of a set \(X\) and a metric \(d\) on \(X\) is called a metric space, as the reader is surely aware. These are ubiquitous in many areas, mainly in analysis.
There is a slightly weakened version one can consider, sometimes called a \emph{pseudoquasimetric,} or a \emph{hemimetric.} Here, instead of (1), one merely demands that \(d(x,x)=0\),
and one drops the symmetry requirement (2) entirely. That is, a hemimetric is a function \(d\!:X\times X \to \R_{\geq 0}\) such that
\begin{enumerate}[label=(\arabic*')]
	\item \(\forall x\in X\), \(d(x,x)=0\), and
	\item \(\forall x,y,z\in X\), \(d(x,z) \leq d(x,y) + d(y,z)\).
\end{enumerate}
The content of this lecture is that one may extend this notion to categories. This was originally studied by Lawvere \cite{lawvere73-metric-spaces}, and has been subsequently considered by many authors.
We are inspired primarily by the work of Neeman \cite{neeman2021metricstriangulatedcategories}. In \emph{loc. cit.,} the focus is on metrics on metrics on triangulated categories, and
we take the perspective that these should be interpreted as being homotopy categories of stable \(\infty\)-categories. Hence, we choose to begin by developing the theory in the
\(\infty\)-categorical context.

\subsection{Metrics}
\begin{notation}
	Let \(\calC\) be an \(\infty\)-category. We will write \(\Mor(\calC)\) for the set of morphisms in \(\calC\), i.e.\ the set of functors \(\Delta^1 \to \calC\), or alternatively
	the set of 1-simplices \(\calC_1\) of \(\calC\).
\end{notation}
\begin{definition}
	Let \(\calC\) be an \(\infty\)-category. A \emph{metric} on \(\calC\) is a function \(d\!:\Mor(\calC)\to\R_{\geq 0}\), satisfying the following properties.
	\begin{enumerate}[label=(\arabic*)]
		\item For all \(x\in\calC\), we have \(d(x \overset{\id_x}\longto x) = 0\).
		\item For any commutative triangle in \(\calC\) as below left, we have the triangle inequality as below right:
		\begin{center}
			\begin{tikzcd}[cramped, column sep=small]
				& y\ar[dr,"g"] & \\
				x\ar[ur,"f"]\ar[rr,"h"] & & z
			\end{tikzcd}\(\quad \leadsto \quad d(x\overset{h}\to z) \leq d(x\overset{f}\to y) + d(y\overset{g}\to z). \)
		\end{center}
	\end{enumerate}
	Two metrics \(d_1\) and \(d_2\) on \(\calC\) are \emph{equivalent} if for all \(\varepsilon > 0\), there is a \(\delta > 0\) such that
	\begin{align*}
		d_1(x\to y) < \delta &\implies d_2(x\to y) < \varepsilon, \\
		d_2(x\to y) < \delta &\implies d_1(x\to y) < \varepsilon.
	\end{align*}
	A \emph{metric} \(\infty\)\emph{-category} is a pair \((\calC,d)\) of an \(\infty\)-category \(\calC\) and a metric \(d\!:\Mor(\calC)\to\R_{\geq 0}\).
\end{definition}
\begin{definition}
	Let \((\calC,d)\) be a metric \(\infty\)-category. A sequence of morphisms
	\[ x_1 \to x_2 \to \cdots \to x_n \to \cdots \]
	in \(\calC\) is \emph{Cauchy} with respect to \(d\) if for all \(\varepsilon > 0\) there is some \(M > 0\) such that
	\[  M \leq i \leq j \implies d(x_{i} \to x_j) < \varepsilon. \]
\end{definition}
\begin{remark}
	For a 1-category \(\calC\), the triangle inequality requirement just says that for any composable pair of morphisms
	\[ x \overset{f}\to y \overset{g}\to z \]
	we have
	\[ d(x\overset{g\circ f}\to z) \leq d(x\overset{f}\to y) + d(y\overset{g}\to z). \]
\end{remark}
\begin{example}
	Let \(X\) be some set. Form the category \(\calX\), whose objects are the elements of \(X\), and where for each \(x,y\in\calX\) there is exactly
	one morphism \(x\to y\). Then a metric on \(\calX\) is exactly a hemimetric on \(X\). In particular, any (hemi)metric space \(X\) determines a
	metric category \(\calX\), and one observes that Cauchy sequences in \(X\) are the same as Cauchy sequences in \(\calX\).
\end{example}
\begin{example}
	Let \(\cat{Ban}\) be the category of Banach spaces. There is a metric \(d\) on \(\cat{Ban}\) given by
	\[ d(X \overset{T}\to Y) := \max\{0,\log\|T\|\}. \]
	To see that this is in fact a metric, note that
	\[ d(X\overset{\id_X}\to X) = \log(1) = 0, \]
	 and for \(X \overset{T}\to Y\overset{S}\to Z\), we have
	 \begin{align*}
	 	\log\|ST\| &\leq \log(\|S\|\cdot\|T\|) = \log\|S\| + \log\|T\|
	 \end{align*}
	 from which it follows that
	 \[ d(X\overset{ST}\longto Z) \leq \max\{0, \log\|S\| + \log\|T\|\} \leq d(X\overset{T}\to Y) + d(Y\overset{S}\to Z). \]
	 This example is discussed in \cite[§8]{clementino2024cauchyconvergencevnormedcategories}. This metric has a few trivial properties. For example, \(d(X\overset{T}\to Y)=0\)
	 if and only if \(\|T\|\leq 1\). It follows that one always has \(\|T\| \leq \exp{d(X\overset{T}\to Y)} \).
\end{example}

We now show that metrics do not see the higher-categorical data in an \(\infty\)-category, and depend only on the homotopy 1-category \(\ho(\calC)\).
\begin{proposition}\label{prop:metric-invariant-under-homotopy}
	Let \((\calC,d)\) be a metric \(\infty\)-category, and consider two morphisms \(f,g\!:x\to y\) such that \([f] = [g]\) in \(\ho(\calC)\).
	Then \(d(x\overset{f}\to y) = d(x\overset{g}\to y)\).
\end{proposition}
\begin{proof}
By assumption (see e.g.\ \cite[§1.6]{cisinski-book}), we have the following commutative triangles:
\begin{center}
	\begin{tikzcd}[cramped, column sep=small]
		& y\ar[dr,"\id_y"] & \\
		x\ar[ur,"f"]\ar[rr,"g"] & & y
	\end{tikzcd}\quad and \quad
	\begin{tikzcd}[cramped, column sep=small]
		& y\ar[dr,"\id_y"] & \\
		x\ar[ur,"g"]\ar[rr,"f"] & & y
	\end{tikzcd}
\end{center}
and therefore
\[ d(x\overset{g}\to y) \leq d(x\overset{f}\to y) + d(x\overset{\id_x}\to x), \quad d(x\overset{f}\to y) \leq d(x\overset{g}\to y) + d(x\overset{\id_x}\to x) \]
but since \(d(\id_x) = 0\), we thus have
\[ d(x\overset{g}\to y) \leq d(x\overset{f}\to y) \leq d(x\overset{g}\to y) \]
as desired.
\end{proof}

\begin{corollary}
	Let \(\calC\) be an \(\infty\)-category. Then there is a bijection
	\[ \{ \text{metrics on }\calC \}\cong\{ \text{metrics on }\ho(\calC) \}. \]
	Furthermore, the following statements hold about this bijection.
	\begin{enumerate}[label=(\arabic*)]
		\item For a metric \(d\) on \(\calC\), a sequence \(x_1 \to x_2 \to \cdots\) is Cauchy if and only if the sequence taken in \(\ho(\calC)\) is Cauchy with respect to the corresponding metric on \(\ho(\calC)\).
		\item Two metrics on \(\calC\) are equivalent if and only if the corresponding metrics on \(\ho(\calC)\) are equivalent.
	\end{enumerate}
\end{corollary}
\begin{proof}
Starting with a metric \(d\) on \(\calC\), one obtains a metric \(d'\) on \(\ho(\calC)\) given by \(d'([f]) = d(f)\). By Proposition \ref{prop:metric-invariant-under-homotopy}, this is well-defined,
and it is quite clear that one gets a metric. Conversely, starting with a metric \(d'\) on \(\ho(\calC)\), one obtains a metric \(d\) on \(\calC\) given by the same formula. These two operations
are inverse to each other, which provides the bijection. The remaining claims are obvious.
\end{proof}

\begin{remark}
	In particular, we could have originally defined metrics only for 1-categories, extending the notion to \(\infty\)-categories by saying that a metric on an \(\infty\)-category \(\calC\)
	is a metric on \(\ho(\calC)\).
\end{remark}

\subsection{Non-Archimedean metrics \& valued \(\infty\)-categories}
\begin{definition}
	Let \(\calC\) be an \(\infty\)-category. A metric \(d\) on \(\calC\) is \emph{non-Archimedean} (or an \emph{ultrametric}) if it satisfies the following stronger
	version of the triangle inequality, also known as the ultrametric inequality.
	\begin{itemize}[label=(2')]
		\item For any commutative triangle in \(\calC\) as below left, we have inequality as below right:
		\begin{center}
			\begin{tikzcd}[cramped, column sep=small]
				& y\ar[dr,"g"] & \\
				x\ar[ur,"f"]\ar[rr,"h"] & & z
			\end{tikzcd}\(\quad \leadsto \quad d(x\overset{h}\to z) \leq \max\{d(x\overset{f}\to y), d(y\overset{g}\to z)\}. \)
		\end{center}
	\end{itemize}
	A metric which is \emph{not} non-Archimedean is called \emph{Archimedean.}
\end{definition}

We will show that the data required for a non-Archimedean metric can be significantly simplified. The first step is the following, which should be familiar from topology.
\begin{lemma}\label{lemma:metric-category-composition-with-nice-function-yields-equivalent-metric}
	Let \(\calC\) be an \(\infty\)-category, and let \(d\) be a metric on \(\calC\). Suppose there is a weakly increasing function \(f\!:\R_{\geq 0}\to\R_{\geq 0}\) such
	\(\lim_{t\to 0^+}f(t) = 0\) and either one of the following conditions holds.
	\begin{enumerate}[label=(\alph*)]
		\item The function \(f\) is subadditive: \(f(a+b) \leq f(a) + f(b)\).
		\item The metric \(d\) is non-Archimedean.
	\end{enumerate}
	Then \(d' = f\circ d\) is a metric which is equivalent to \(d\). If (b) holds, then \(d'\) is also non-Archimedean.
\end{lemma}
\begin{proof}
Observe that since \(f\) is increasing and tends to zero as \(t\to 0\), we have \(f(0)=0\). First, we show that \(d'\) is a metric. For this, we check the axioms.
\begin{enumerate}[label=(\arabic*)]
	\item Let \(x\in\calC\). Then
	\[ d'(x\overset{\id_x}\to x) = f(d(x\overset{\id_x}\to x)) = f(0) = 0. \]
	\item Consider a commutative triangle \((f\!:x\to y,g\!:y\to z,h\!:x\to z)\). If we are in case (a), then
	\begin{align*}
		d'(x\overset{h}\to z) &= f(d(x\overset{h}\to z)) \\
		&\leq f(d(x\overset{f}\to y) + d(y\overset{g}\to z)) \\
		&\leq f(d(x\overset{f}\to y)) + f(d(y\overset{g}\to z))) = d'(x\overset{f}\to y) + d'(y\overset{g}\to z).
	\end{align*}
	On the other hand, if we are in case (b), then since \(f\) is order-preserving, we have \(f(\max\{a,b\}) = \max\{f(a),f(b)\}\), and hence the above computation
	follows through in an identical fashion.
\end{enumerate}
It follows easily that \(d'\) is non-Archimedean if \(d\) is non-Archimedean.

Now we must show that \(d'\) is equivalent to \(d\). Let \(\varepsilon > 0\), and set \(\delta_0\) such that \(f(\delta_0) < \varepsilon\). This is possible by the requirement that
\(\lim_{t\to 0^+}f(t) = 0\). Let us consider \(\delta = \min\{\delta_0, f(f(\delta_0))\}\). Then
\[ d(x\to y) < \delta \implies f(d(x\to y)) = d'(x\to y) \leq f(\delta) \leq f(\delta_0) < \varepsilon. \]
This proves one direction. For the other, note that
\[ d'(x\to y) < \delta \implies f(d(x\to y)) \leq f(f(\delta_0)) \implies d(x\to y) \leq f(\delta_0) < \varepsilon \]
as desired.
\end{proof}

\begin{lemma}
	Let \(\calC\) be an \(\infty\)-category, and let \(d\) be a non-Archimedean metric. Consider any strictly decreasing sequence \((r_n)_{n=1}^{\infty}\) in \(\R_{\geq 0}\) such that \(r_n\to 0\) as \(n\to\infty\).
	Then there is an equivalent non-Archimedean metric \(d'\) on \(\calC\) taking values in \(\{r_n \mid n \geq 1\}\).
\end{lemma}
\begin{proof}
We intend to apply Lemma \ref{lemma:metric-category-composition-with-nice-function-yields-equivalent-metric}. For simplicity, we assume \((r_n)_{n=1}^\infty\) is normalized such that \(r_1=1\).
First, we modify \(d\) to get a metric \(d_0\) that is bounded by \(1\). For this, apply the lemma to \(f_0(t) = t/(1+t)\) and set \(d_0 = f_0\circ d\). To get \(d'\), we define the function
\[ f\!: [0,1] \to \R_{\geq 0},\quad f(t) := \inf\{ r_n \mid n \geq 1,\, t \leq r_n \}. \]
In other words, \(f(t)\) is the smallest \(r_n\) larger than \(t\). Now, \(f\) is weakly increasing and clearly satisfies \(\lim_{t\to 0^+}f(t) = 0\). Since \(d\)
is assumed to be non-Archimedean, so is \(d_0\), and it follows that \(d' := f\circ d_0\) is a non-Archimedean metric. Note that, while the domain of \(f\) is not as
in the statement of the lemma, since the image of \(d_0\) is in \([0,1]\) there is no issue.
\end{proof}
\begin{remark}
	The above lemma is stated without proof in \cite{neeman2021metricstriangulatedcategories} in greater generality, namely for an arbitrary metric rather than a non-Archimedean one.
	However, this cannot hold with the same strategy we used: there are counter-examples using the above \(f\). The issue is that the function \(f\) is not subadditive,
	and there is no clear choice for one that is.
\end{remark}

\begin{definition}
	Let \(\calC\) be an \(\infty\)-category. A \emph{discrete valuation} on \(\calC\) is a map \(v\!:\Mor(\calC)\to\Z_{\geq 0}\cup\{\infty\}\) satisfying the following properties.
	\begin{enumerate}[label=(\arabic*)]
		\item For all \(x\in\calC\), we have \(v(x\overset{\id_x}\to x) = \infty\).
		\item For any commutative triangle as below left, we have the inequality as below right:
		\begin{center}
			\begin{tikzcd}[cramped, column sep=small]
				& y\ar[dr,"g"] & \\
				x\ar[ur,"f"]\ar[rr,"h"] & & z
			\end{tikzcd}\(\quad \leadsto \quad v(x\overset{h}\to z) \geq \min\{v(x\overset{f}\to y), v(y\overset{g}\to z)\}. \)
		\end{center}
	\end{enumerate}
\end{definition}

\begin{proposition}
	Let \((\calC,d)\) be a non-Archimedean metric \(\infty\)-category, and consider the full subcategories
	\[ \forall x\in\calC,\, \forall n\geq 1,\quad B_n(x) := \left\{ x\to y \mid d(x\to y) \leq \frac{1}{n} \right\} \subseteq x/\calC. \]
	Define the map
	\[ v\!: \Ob(x/\calC) \to \Z_{\geq 1}\cup\{\infty\},\quad v(x\to y) := \inf\{ n \mid n\geq 1,\, (x\to y) \not\in B_n(x) \}. \]
	Then the following statements hold.
	\begin{enumerate}[label=(\arabic*)]
		\item If \(m \geq n\) then \(B_m(x) \subseteq B_n(x)\).
		\item 
	\end{enumerate}
\end{proposition}

\begin{theorem}
	Let \(\calC\) be an \(\infty\)-category. Then there is a bijection
	\[ \left\{ \begin{array}{c}
		\text{equivalence classes of}\\
		\text{non-Archimedean metrics on } \calC
	\end{array} \right\} \cong  \]
\end{theorem}
