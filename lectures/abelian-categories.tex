%!TEX root = ../lectures.tex

\section{Abelian categories}
\subsection{(Pre-)additive categories \& additive functors}
\begin{definition}
	A \emph{pre-additive category} is a category enriched in Abelian groups. In particular, a pre-additive category \(\calC\) consists of
	\begin{enumerate}[label=(\arabic*)]
	\item a set of objects \(\Ob(\calC)\),
	\item for each pair of objects \((x,y)\in\Ob(\calC)\times\Ob(\calC)\), an Abelian group \(\Hom_{\calC}(x,y) = \calC(x,y)\), and
	\item for each triple of objects \((x,y,z)\in \Ob(\calC)\times \Ob(\calC)\times \Ob(\calC)\) a composition law
	\[ \circ\!:\calC(y,z)\otimes\calC(x,y)\to\calC(x,z) \]
	which is associative and unital.
	\end{enumerate}
	That is, it's just a category where each Hom-set has the structure of an Abelian group and composition is bilinear.
\end{definition}
\begin{remark}
	This implies that for each \(x,y\in\calC\), the set \(\calC(x,y)\) has a distinguished element \(0\) which is absorbative. This is a simple
	computation:
	\[ 0\circ f = (0+0)\circ f = 0\circ f + 0\circ f \implies 0 = 0\circ f. \]
	The other direction for composition is dual. Categories with this property (or more general ones) are sometimes called \emph{pointed,} though this terminology
	is also used for the stricter notion of a category with a zero object.
\end{remark}

\begin{exercise}
	Let \(\calC\) be a pre-additive category, and suppose that there is an initial object \(\varnothing\in\calC\). Show that \(\varnothing\) is terminal.
	Dually, show that any terminal object \(*\in\calC\) is also initial.
\end{exercise}

Our first goal is to show that the above exercise generalizes to arbitrary finite (co)products; note that initial/terminal objects are exactly \emph{empty} (co)products.

\begin{construction}\label{construction:pre-additive-product-inclusions}
	Let \(\calC\) be a pointed category (i.e. one that admits zero morphisms), let \(x_1,x_2\in\calC\), and suppose that the product \(x_1\times x_2\) exists. Denote by
	\[ p_k\!:x_1\times x_2 \to x_k,\quad k=1,2 \]
	the canonical projection maps. We can construct maps
	\[ i_k\!:x_k\to x_1\times x_2,\quad k=1,2 \]
	by applying the universal property of the product to \(\id_{x_k}\) and \(0\).
\end{construction}
\begin{proposition}
	Let \(x_k\), \(p_k\), and \(i_k\) be as above. Then
	\[ \id_{x_1\times x_2} = (i_1\circ p_1) + (i_2\circ p_2). \]
\end{proposition}
\begin{proof}
By universal property, the following computation suffices:
\[ p_k \circ ((i_1\circ p_1) + (i_2\circ p_2)) = p_k\circ i_k\circ p_k = p_k \]
where we use that \(p_k\circ i_k = \id\).
\end{proof}
\begin{remark}
	Intuitively, this is the statement that
	\[ (a,b) = (a,0) + (0,b). \]
\end{remark}
\begin{proposition}\label{prop:pre-additive-direct-sum-data}
	Let \(\calC\) be a pre-additive category, and let \(x_1,x_2,y\in\calC\). Suppose we have maps
	\[ p_k\!:y\to x_k,\quad i_k\!:x_k\to y \]
	such that
	\[ \id_{y} = (i_1\circ p_1) + (i_2\circ p_2),\quad p_j\circ i_k = \begin{cases} \id_{x_k} & \text{if }j=k,\\ 0 & \text{if }j\not=k. \end{cases} \]
	Then
	\begin{enumerate}[label=(\arabic*)]
	\item \((y,p_1,p_2)\) defines a product \(x_1\times x_2\), and
	\item \((y,i_1,i_2)\) defines a coproduct \(x_1\amalg x_2\).
	\end{enumerate}
\end{proposition}
\begin{proofsketch}
To prove (1), we must show that the provided data makes \(y\) represent the functor \(\calC(-,x_1)\times\calC(-,x_2)\). Let \(z\in\calC\). Then we see that
\[  p_{k,*}\!:\calC(z,y)\to \calC(z,x_k),\quad i_{k,*}\!:\calC(z,x_k)\to\calC(z,y) \]
are maps of Abelian groups satisfying the assumptions in the proposition. It is then a simple computation that this induces a functorial isomorphism
\[ \calC(z,y) \iso \calC(z,x_1)\times\calC(z,x_2). \]
Similarly, to prove (2), the same argument applies but after applying \(\calC(-,z)\) instead.
\end{proofsketch}
\begin{remark}
	Objects in pre-additive categories equipped with maps as above are sometimes called \emph{direct sums,} and are denoted \(x_1\oplus x_2\). The proposition
	shows that any a product of two objects and a coproduct of two objects define the direct sum of those objects.
\end{remark}
\begin{corollary}
	Let \(\calC\) be a pre-additive category and let \(x_1,x_2\in\calC\). The product \(x_1\times x_2\) exists if and only if the coproduct \(x_1\amalg x_2\) exists if and only
	if the direct sum \(x_1\oplus x_2\) exists. Furthermore, in the case that these exist, the canonical map
	\begin{diagram*}
		x_1\amalg x_2 \ar[r,dashed] & x_1\times x_2 \\
		x_k\ar[u] \ar[r,equal] & x_k \ar[u,"i_k"']
	\end{diagram*}
	is an isomorphism, where \(i_k\) are the maps in Construction \ref{construction:pre-additive-product-inclusions}.
\end{corollary}
\begin{proof}
Proposition \ref{prop:pre-additive-direct-sum-data} handles the first statement by a clear induction argument. For the second statement,
note that the maps \(i_k\) exhibit \(x_1\times x_2\) as a direct sum by Proposition \ref{prop:pre-additive-direct-sum-data}, and therefore the induced map is an isomorphism.
\end{proof}

We conclude that in an appropriate sense, finite products and finite coproducts agree in any pre-additive category. It turns out, rather profoundly, that the pre-additive
structure on a category is detected by these (co)products whenever they exist.
\begin{proposition}
	Let \(\calC\) be a pre-additive category, and let \(f,g\!:x\to y\) be morphisms in \(\calC\). If \(x\oplus x\) and \(y\oplus y\) exist, then
	the morphism \(f+g\) is equal to the composition
	\[ x\xrightarrow{\Delta_x} x\oplus x \xrightarrow{f\oplus g} y\oplus y\xrightarrow{\nabla_y} y. \]
\end{proposition}
\begin{proofsketch}
One can check by a computation that
\[ i_1^x + i_2^x = \Delta_x,\quad \pi_1^y + \pi_2^y = \nabla_y \]
and deduce that
\[ \nabla_y \circ (f\oplus g)\circ i_1^x = f,\quad \nabla_y \circ (f\oplus g)\circ i_2^x = g. \]
Therefore
\[ \nabla_y \circ (f\oplus g)\circ \Delta_x = f+g \]
as desired.
\end{proofsketch}

\begin{definition}
	An \emph{additive category} is a pre-additive category which admits finite products.
\end{definition}
\begin{remark}\label{remark:additive-internal-characterization}
	An additive category therefore admits all finite direct sums, and the pre-additive structure is entirely determined by the properties of the underlying unenriched category. In fact,
	there is a characterization of additive categories in terms of purely ordinary category theory. Specifically, a category \(\calC\) is additive if and only if
	\begin{enumerate}[label=(\arabic*)]
	\item it admits a zero object \(0\in\calC\),
	\item for any \(x,y\in\calC\), the product \(x\times y\) and coproduct \(x\amalg y\) exist,
	\item the canonical map \(r\!:x\amalg y \to x\times y\) induced by the maps from Construction \ref{construction:pre-additive-product-inclusions} is an isomorphism, and
	\item for all \(x\in\calC\), there is some \(a\in\calC(x,x)\) such that
	\[ x\xrightarrow{\Delta_x} x\times x\xrightarrow{(a,\id_x)}x\times x\xrightarrow{r^{-1}} x\amalg x\xrightarrow{\nabla_x} x \]
	is zero.
	\end{enumerate}
	Here, (4) is what guarantees the existence of additive inverses.
\end{remark}

\begin{definition}
	A functor \(F\!:\calC\to\calD\) between pre-additive categories is \emph{additive} if for all \(x,y\in\calC\), the induced maps
	\[ \calC(x,y)\to\calD(Fx,Fy) \]
	are group homomorphisms.
\end{definition}
\begin{proposition}
	A functor between additive categories is additive if and only if it preserves finite products.
\end{proposition}
\begin{proof}
See \cite[Prop.\ 8.2.15]{kashiwara-schapira-book}.
\end{proof}

\subsection{Abelian categories}
The category \(\Ab\) of Abelian groups has certain special properties. First of all, it is additive; this is clear. However, on top of that, given any morphism of Abelian groups,
we can form a kernel, cokernel, image, etc., and these behave well (e.g.\ in that they detect interesting properties of the morphism). We wish to capture these properties
in a general form.
\begin{definition}
	Let \(\calC\) be a category with a zero object, and let \(f\!:x\to y\) be a morphism in \(\calC\). The \emph{kernel} of \(f\) is the pullback
	\begin{diagram*}
		\ker{f}\ar[r,dashed]\ar[d,dashed]\ar[dr, pullback] & x\ar[d,"f"] \\
		0\ar[r] & y
	\end{diagram*}
	in \(\calC\). Dually, the \emph{cokernel} of \(f\) is the pushout
	\begin{diagram*}
		x\ar[r,"f"]\ar[d] & y \ar[d,dashed] \\
		0\ar[r,dashed] & \coker{f}\ar[ul,pushout]
	\end{diagram*}
	in \(\calC\).
\end{definition}
\begin{exercise}
	Express \(\ker{f}\) and \(\coker{f}\) in terms of an equalizer and coequalizer, respectively.
\end{exercise}
\begin{exercise}\label{exercise:kernel-map-monomorphism}
	Let \(\calC\) be a pre-additive category with a zero object.
	\begin{enumerate}[label=(\arabic*)]
	\item Show that a morphism \(f\!:x\to y\) is a monomorphism if and only if for any \(a\!:z\to x\), \(f\circ a = 0\) implies \(a = 0\).
	\item Show that whenever \(\ker{f}\) exists, the canonical map \(\ker{f}\to x\) is a monomorphism.
	\end{enumerate}
	Furthermore, dualize these statements to obtain the corresponding results for epimorphisms and \(\coker{f}\).
\end{exercise}

\begin{proposition}
	Let \(\calC\) be a pre-additive category with a zero object, and let \(f\!:x\to y\) be a morphism with a kernel and cokernel.
	\begin{enumerate}[label=(\arabic*)]
	\item The morphism \(f\) is monic if and only if \(\ker{f} = 0\).
	\item The morphism \(f\) is epic if and only if \(\coker{f} = 0\).
	\end{enumerate}
\end{proposition}
\begin{proof}
Statement (2) is just the dual of (1). To prove (1), consider the diagram
\begin{diagram*}
	& z\ar[d,"a"']\ar[dl,dashed]\ar[dr,"0"] & \\
	\ker{f} \ar[r,hook] & x\ar[r,"f"] & y
\end{diagram*}
for some \(a\!:z\to x\) satisfying \(f\circ a = 0\). If \(\ker{f}=0\), then \(a\) factors through the zero object so \(a=0\) and \(f\) is monic.
Conversely, if \(f\) is monic, then \(a=0\) and we see that \(0\) satisfies the universal property of \(\ker{f}\).
\end{proof}


\begin{definition}
	Let \(\calC\) be a category with a zero object. The \emph{image} of a morphism \(f\!:x\to y\) is
	\[ \img{f} := \ker(y\sur\coker{f}) \]
	and the \emph{coimage} of \(f\) is
	\[ \coimg{f} := \coker(\ker{f}\inj x). \]
\end{definition}
\begin{remark}
	By Exercise \ref{exercise:kernel-map-monomorphism}, the canonical maps
	\[ \img{f}\to y \quad\text{and}\quad x\to\coimg{f} \]
	are a monomorphism and epimorphism, respectively.
\end{remark}
\begin{remark}
	The intuition for the definition comes from viewing the cokernel operation as taking the quotient.
\end{remark}
\begin{remark}
	Concretely, the universal property of the image of \(f\!:x\to y\) is given by the diagram
	\begin{diagram*}
		& z\ar[d]\ar[dr,"0"]\ar[dl,dashed,"\exists!"'] & \\
		\img{f}\ar[r,hook] & y \ar[r,two heads] & \coker{f}
	\end{diagram*}
	while, dually, the universal property of the coimage is given by the diagram
	\begin{diagram*}
		\ker{f}\ar[r,hook]\ar[dr,"0"'] & x\ar[r,two heads]\ar[d] & \coimg{f}\ar[dl,dashed,"\exists!"] \\
		& z
	\end{diagram*}
\end{remark}
\begin{construction}\label{construction:coimg-to-img}
	Observe that since the composition \(\ker{f}\to x\to y\) is zero (and by the dual statement), we can build the diagram
	\begin{diagram*}
		\ker{f} \ar[r,hook] \ar[dr,"0"'] & x\ar[r,"f"]\ar[d,two heads]\ar[dr] & y\ar[from=dl,crossing over] \ar[r,two heads] & \coker{f} \\
		& \coimg{f} & \img{f}\ar[u,hook]\ar[ur,"0"'] &
	\end{diagram*}
	and easily see that the compositions
	\[ \coimg{f} \to y \sur\coker{f}\quad \ker{f}\to x\to\img{f} \]
	are zero. In particular, we have an induced map
	\[ \coimg{f} \to \img{f} \]
	fitting in the square
	\begin{diagram*}
		x\ar[r]\ar[d,two heads] & y\\ 
		\coimg{f}\ar[r] & \img{f}\ar[u,hook]
	\end{diagram*}
	and is unique.
\end{construction}
\begin{definition}
	Let \(\calA\) be an additive category. We say \(\calA\) is \emph{Abelian} if for every \(x,y\in\calA\) and \(f\in\calA(x,y)\),
	\begin{enumerate}[label=(\arabic*)]
	\item the kernel and cokernel of \(f\) exist, and
	\item the canonical map
	\[ \coimg{f}\to\img{f} \]
	from Construction \ref{construction:coimg-to-img} is an isomorphism.
	\end{enumerate}
\end{definition}
\begin{remark}
	Condition (2) above is effectively demanding that the first isomorphism theorem holds in \(\calA\). Indeed, we can think of \(\coimg{f}\) as the quotient \(x/\ker{f}\),
	and (2) thus says we have a canonical isomorphism \(x/\ker{f}\iso\img{f}\).
\end{remark}
\begin{remark}
	Limits can be built from products and equalizers. Furthermore, in an additive category, any equalizer can be phrased as a kernel, namely \(\eq(f,g)\) will be \(\ker(f-g)\).
	In particular, since any Abelian category thus admits both equalizers and finite products, it admits all finite limits. The dual argument shows that Abelian categories also
	admit all finite colimits.
\end{remark}
\begin{exercise}\label{exercise:monic-epic-image}
	Let \(f\!:x\to y\) be a morphism in an Abelian category.
	\begin{enumerate}[label=(\arabic*)]
	\item Show that if \(f\) is monic, then
	\[ x\iso\img{f}. \]
	\item Dually, show that if \(f\) is epic, then
	\[ \img{f}\iso y. \]
	\item Conclude that \(f\) is an isomorphism if and only if it is monic and epic.
	\end{enumerate}
\end{exercise}

\subsection{Appendix: Enriched category theory}
Pre-additive categories are a special case of a more general paradigm, \emph{enriched categories.} There are many different formalisms for this at many different levels
of generality; we present only one, roughly following \cite{riehl-categorical-homotopy-theory}. The general idea is that an enriched category is like a category, but where the set of morphisms between two objects is replaced by
an object in some category, called the \emph{enrichment base.} The first order of business is to specify what kind of category we wish to have as base for enrichments.
\begin{definition}
	A \emph{symmetric monoidal category} is triple \((\calV,\otimes,\1)\) where \(\calV\) is a category, \(\otimes\!:\calV\times\calV\to\calV\) is a functor,
	and \(\1\in\calV\) is an object, together with the data of specified natural isomorphisms
	\[ v\otimes w \cong w\otimes v,\quad u\otimes(v\otimes w)\cong (u\otimes v)\otimes w,\quad \1\otimes v\cong v \cong v\otimes \1. \]
	These natural isomorphisms are required to satisfy various coherences which ensure that any two bracketings of \(\otimes\)-products are naturally isomorphic.
\end{definition}
\begin{example}
	By considering the tensor product over \(\Z\), the category \(\Ab\) can be promoted to a symmetric monoidal category \((\Ab,\otimes_\Z,\Z)\). More generally, for
	any ring \(R\), the category \(\Mod_R\) forms a symmetric monoidal category \((\Mod_R,\otimes_R,R)\).
\end{example}
\begin{example}
	Any category \(\calC\) admitting finite products yields a symmetric monoidal category \((\calC,\times,*)\) where \(*\in\calC\) is the terminal object. In particular,
	\(\Set\) provides a symmetric monoidal category \((\Set,\times,*)\).
\end{example}
\begin{example}
	Let \(R\) be a commutative ring. The tensor product on \(\Mod_R\) can be lifted to the category \(\Ch(R)\) of chain complexes in \(R\), to be defined in Section \ref{subsection:chain-complexes-and-exact-sequences},
	to give a functor \(\otimes_R\!:\Ch(R)\times\Ch(R)\to\Ch(R)\). This provides a symmetric monoidal category \((\Ch(R),\otimes_R,R)\).
\end{example}
\begin{definition}
	Let \((\calV,\otimes,\1)\) be a symmetric monoidal category. A \(\calV\)\emph{-category} \(\underline\calC\) consists of
	\begin{enumerate}[label=(\arabic*)]
	\item a set of objects \(\Ob(\underline\calC)\),
	\item for every pair of objects \((x,y)\) in \(\underline\calC\), an object \(\underline\calC(x,y)\in\calV\),
	\item for every \(x\in\underline\calC\), a morphism \(1_x\!:\1\to\underline\calC(x,x)\), and
	\item for each triple \((x,y,z)\) of objects in \(\calV\), a morphism
	\[ \circ\!:\underline\calC(y,z)\otimes\underline\calC(x,y)\to\underline\calC(x,z) \]
	\end{enumerate}
	such that for all \(x,y,z,w\in\underline\calC\), the diagrams
	\begin{diagram*}
		\underline\calC(z,w)\otimes\underline\calC(y,z)\otimes\underline\calC(x,y)\ar[r,"1\otimes\circ"]\ar[d,"\circ\otimes 1"'] & \underline\calC(z,w)\otimes\underline\calC(x,z)\ar[d,"\circ"] \\
		\underline\calC(y,w)\otimes\underline\calC(x,y)\ar[r,"\circ"] & \underline\calC(x,w)
	\end{diagram*}
	\begin{center}
	\begin{tikzcd}
		\underline\calC(x,y)\otimes\1\ar[r,"\id\otimes1_x"]\ar[dr,"\cong"'] & \underline\calC(x,y)\otimes\underline\calC(x,x)\ar[d,"\circ"] \\
		& \underline\calC(x,y)
	\end{tikzcd}
	\quad
	\begin{tikzcd}
		\underline\calC(y,y)\otimes\underline\calC(x,y)\ar[d,"\circ"] & \1\otimes\underline\calC(x,y)\ar[l,"1_y\otimes\id"']\ar[dl,"\cong"]\\
		\underline\calC(x,y) &
	\end{tikzcd}
	\end{center}
	expressing associativity and unitality commute. One also says that \(\underline\calC\) is a \emph{category enriched over} \((\calV,\otimes,\1)\).
\end{definition}
\begin{example}
	An ordinary category is a category enriched over \((\Set,\times,*)\).
\end{example}
\begin{example}
	A 2-category is a category enriched over \((\Cat,\times,[0])\).
\end{example}
\begin{example}
	A pre-additive category is a category enriched over \((\Ab,\otimes_\Z,\Z)\).
\end{example}
\begin{example}
	A \emph{dg-category} over a commutative ring \(R\) is a category enriched over \((\Ch(R),\otimes_R,R)\). These play a particularly important role in homological algebra,
	as they are a computationally convenient presentation for \(R\)-linear stable \(\infty\)-categories.
\end{example}
\begin{example}
	A \emph{simplicial category} is a category enriched over \((\sSet,\times,\Delta^0)\).
\end{example}
\begin{remark}
	Any \(\calV\)-category \(\underline\calC\) induces an ordinary category \(\calC\) whose objects are just the objects of \(\underline\calC\), and whose morphisms are given by
	\[ \calC(x,y) := \calV(\1,\underline\calC(x,y)). \]
\end{remark}
\begin{definition}
	Let \((\calV,\otimes,\1)\) be a symmetric monoidal category. A \(\calV\)\emph{-enriched functor,} or \(\calV\)\emph{-functor} \(F\!:\underline\calC\to\underline\calD\) between
	\(\calV\)-categories consists of
	\begin{enumerate}[label=(\arabic*)]
	\item a map \(\Ob(\underline\calC)\to\Ob(\underline\calD)\), \(x\mapsto Fx\),
	\item for each pair of objects \((x,y)\) in \(\underline\calC\), a morphism
	\[ F\!:\underline\calC(x,y)\to\underline\calD(Fx,Fy), \]
	\end{enumerate}
	such that for all triples \((x,y,z)\) of objects in \(\underline\calC\), the diagrams
	\begin{center}
	\begin{tikzcd}
		\underline\calC(y,z)\otimes\underline\calC(x,y)\ar[r,"\circ"]\ar[d,"F\otimes F"'] & \underline\calC(x,z)\ar[d,"F"] \\
		\underline\calD(Fy,Fz)\otimes\underline\calD(Fx,Fy)\ar[r,"\circ"] & \underline\calD(Fx,Fz)
	\end{tikzcd}
	\quad
	\begin{tikzcd}
		\1 \ar[r,"1_x"]\ar[dr,"1_{Fx}"'] & \underline\calC(x,x)\ar[d,"F"] \\
		& \underline\calD(Fx,Fx)
	\end{tikzcd}
	\end{center}
	commute.
\end{definition}

Much of category theory can be developed in the enriched setting, including the Yoneda lemma, (co)limits of various sorts, and so on. This turns out to be very useful
in higher category theory, as many models of higher categories can be ``strictified'' by working in an enriched context. An example of this mentioned before is
given by dg-categories.

Categories enriched in Kan complexes give a model for \((\infty,1)\)-categories, and categories enriched in weak Kan complexes (a.k.a.\ quasicategories, or \(\infty\)-categories) give a model for \((\infty,2)\)-categories.
In practice, this means that doing higher category theory, homotopy theory, etc.\ can be simplified by knowing some enriched category theory. Indeed, this is the topic of
\cite{riehl-categorical-homotopy-theory} and partly \cite{riehl-verity-elements}.

\subsection{Appendix: (Co)limits in terms of (co)products \& (co)equalizers}
Consider a small diagram \(D\!:I\to\Set\) in the category of sets. It is a standard result that one can compute the limit and colimit of \(D\)
as
\[ \projlim{D} \cong \left\{ (d_i)_{i\in I}\in\prod_{i\in I}D(i)\bigmid \forall(\varphi\!:i\to j)\in I,\, D(\varphi)(d_i) = d_j \right\} \]
and
\[ \injlim{D} \cong \left(\coprod_{i\in I}D(i)\right) / \sim \]
where \(\sim\) is the equivalence relation generated by (!) \((i,d) \sim (i',d')\) if there exists \(\varphi\!:i\to i'\) such that \(D(\varphi)(d) = d'\).
Notably, these are compute in terms of the easier-to-understand products and coproducts in \(\Set\).

We wish to generalize this to any category in order to deduce that a category admitting (co)products and (co)equalizers admits all small (co)limits.
\begin{construction}\label{construction:diagram-product-equalizer-diagram}
	Let \(D\!:I\to\calC\) be a small diagram in a category \(\calC\) which admits products, and consider a map \(\varphi\!:i\to j\) in \(I\). From this, we obtain two maps
	\begin{diagram*}
		D(i)\times D(j) \ar[r,shift left,"D(\varphi)\circ\pi_1"]\ar[r,shift right,"\pi_2"'] & D(j)
	\end{diagram*}
	and these allow us to produce two maps
	\begin{diagram*}
		D(\cod\varphi)\ar[r,equal] & D(\cod\varphi) \\
		\displaystyle\prod_{i\in I}D(i)\ar[r,shift left,dashed,"a"]\ar[r,shift right,dashed,"b"']\ar[u,"\pi_{\cod\varphi}"]\ar[d,"\pi_{\dom\varphi}"'] & \displaystyle\prod_{\varphi\in\Ar(I)}D(\cod{\varphi})\ar[u,"\pi_\varphi"']\ar[d,"\pi_\varphi"] \\
		D(\dom\varphi)\ar[r,"D(\varphi)"] & D(\cod\varphi)
	\end{diagram*}
	defined on the \(\varphi\)th component by the indicated morphisms. Furthermore, there is a canonical map
	\[ c\!:\projlim{D} \to \prod_{i\in I}D(i) \]
	induced by the projections \(\projlim{D}\to D(i)\), \(i\in I\). One notes that \(a\circ c = b\circ c\) since this equality comes down to verifying it on components, where it
	holds by the commutativity of
	\begin{diagram*}
		\projlim{D} \ar[r]\ar[dr] & D(\dom{\varphi})\ar[d,"D(\varphi)"] \\
		& D(\cod\varphi).
	\end{diagram*}
	In particular, we have an induced map \(\projlim{D}\to\eq(a,b)\).
\end{construction}
\begin{proposition}\label{prop:limit-from-products-and-equalizers}
	Let \(\calC\) be a category admitting products and equalizers, and let \(D\!:I\to\calC\) be a small diagram. Consider the maps
	\begin{diagram*}
		\displaystyle\prod_{i\in I}D(i)\ar[r,shift left,"a"]\ar[r,shift right,"b"'] & \displaystyle\prod_{\varphi\in\Ar(I)}D(\cod{\varphi})
	\end{diagram*}
	of Construction \ref{construction:diagram-product-equalizer-diagram}. Then \(\projlim{D}\) exists and the canonical map satisfies
	\[ \projlim{D}\iso\eq(a,b). \]
\end{proposition}
\begin{proof}
In \(\calC=\Set\), this is a triviality: indeed, it is just the statement we made at the start. To lift it to an arbitrary category \(\calC\), pick \(x\in\calC\) and apply
\(\calC(x,-)\). We then have, by the \(\Set\) case, that
\begin{diagram*}
	 & \calC(x,\eq(a,b))\ar[r]\ar[d,"\sim" labl] & \calC(x,\prod_{i\in I}D(i))\ar[d, "\sim" labl] \ar[r,shift left,"a_*"]\ar[r,shift right,"b_*"'] & \calC(x,\prod_{\varphi\in\Ar(I)}D(\cod{\varphi}))\ar[d, "\sim" labl] \\
	\projlim\calC(x,D)\ar[r,"\sim"] & \eq(a_*,b_*) \ar[r] & \prod_{i\in I}\calC(x,D(i)) \ar[r,shift left,"a_*"]\ar[r,shift right,"b_*"'] & \prod_{\varphi\in\Ar(I)}\calC(x,D(\cod{\varphi}))
\end{diagram*}
and thus that we have an isomorphism
\[ \projlim\calC(x,D) \cong \calC(x,\eq(a,b)) \]
which is natural in \(x\). We conclude that \(\projlim{D}\) exists and is isomorphic to \(\eq(a,b)\).
\end{proof}
\begin{corollary}
	Let \(\calC\) be a category. Then the following statements are equivalent.
	\begin{enumerate}[label=(\arabic*)]
	\item \(\calC\) admits all small (resp.\ finite) limits.
	\item \(\calC\) admits equalizers and all small (resp.\ finite) products.
	\end{enumerate}
	Let \(F\!:\calC\to\calD\) be a functor. Then the following are equivalent.
	\begin{enumerate}[label=(\arabic*')]
	\item \(F\) preserves all small (resp.\ finite) limits.
	\item \(F\) preserves equalizers and all small (resp.\ finite) products.
	\end{enumerate}
\end{corollary}
\begin{proof}
Clearly, (1) implies (2) and (1') implies (2'). On the other hand, (2) implies (1) by Proposition \ref{prop:limit-from-products-and-equalizers} (noting that the products
are finite whenever \(I\) is finite), so only (2') implies (1') remains. However, if \(F\) preserves products and equalizers then
\begin{diagram*}
	F(\projlim{D}) \ar[d,"\sim" labl] \ar[r] & F(\prod_{i\in I}D(i))\ar[r,shift left,"Fa"]\ar[r,shift right,"Fb"']\ar[d,"\sim" labl] & F(\prod_{\varphi\in\Ar(I)}D(\cod{\varphi}))\ar[d,"\sim" labl] \\
	F(\eq(a,b)) \ar[d,"\sim" labl] \ar[r] & F(\prod_{i\in I}D(i))\ar[r,shift left,"Fa"]\ar[r,shift right,"Fb"']\ar[d,"\sim" labl] & F(\prod_{\varphi\in\Ar(I)}D(\cod{\varphi}))\ar[d,"\sim" labl] \\
	\eq(Fa,Fb) \ar[r] & \prod_{i\in I}F(D(i))\ar[r,shift left,"Fa"]\ar[r,shift right,"Fb"'] & \prod_{\varphi\in\Ar(I)}F(D(\cod{\varphi})) \\
\end{diagram*}
and since \(\projlim(F\circ D)\cong\eq(Fa,Fb)\), we are done.
\end{proof}