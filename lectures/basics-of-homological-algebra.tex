%!TEX root = ../lectures.tex

\section{Basics of homological algebra}
\subsection{Chain complexes \& exact sequences}\label{subsection:chain-complexes-and-exact-sequences}
\begin{definition}
	Let \(\calC\) be a pre-additive category. A \emph{chain complex} \(x^\bullet\) in \(\calC\) is a sequence of morphisms
	\[ \cdots \to x^{i-1}\xrightarrow{d^{i-1}} x^{i} \xrightarrow{d^{i}} x^{i+1} \to \cdots  \]
	such that \(d^{i+1}\circ d^i = 0\) for all \(i\in\Z\). We denote this by either \(x^\bullet\) or \((x^\bullet,d_x)\).
\end{definition}
\begin{remark}
	Let \(\calC\) be some category, and regard the set \(\Z\) as a discrete category. A \emph{graded object} in \(\calC\) is a functor
	\[ \Z\to\calC. \]
	Consider the functor \(s\!:\Z\to\Z\) given by sending \(n\) to \(n+1\). This induces a \emph{shift} functor
	\[ (1)\!:\Fun(\Z,\calC)\to\Fun(\Z,\calC),\quad x^\bullet \mapsto x^\bullet\circ s, \]
	which concretely sends a graded object \(x^\bullet\) in \(\calC\) to the graded object given by \(x(1)^i = x^{i+1}\).

	Now, suppose \(\calC\) is pre-additive. A \emph{chain complex} in \(\calC\) can be regarded as a graded object \(x^\bullet\!:\Z\to\calC\) together with a morphism
	\[ d_x\!:x^\bullet \to x(1)^\bullet \]
	satisfying \(d_x(1)\circ d_x = 0\).
\end{remark}

Our first goal is to define the \emph{cohomology} of a chain complex. There are two obvious ways of doing this (accepting that one is aware of the ``classical'' hands-on definition),
and we want to ensure the two are the same in our formalism.
\begin{construction}\label{construction:chain-complex-induced-maps}
	Consider a chain complex \(x^\bullet\), and in particular zoom in around a fixed index \(i\in\Z\),
	\[ x^{i-1}\xrightarrow{d^{i-1}} x^i \xrightarrow{d^i} x^{i+1}. \]
	Forming kernels, cokernels, and images, we can produce the diagram
	\begin{diagram*}
		 & \img{d^{i-1}}\ar[rr,dashed,"\phi^i"]\ar[dr,hook] & & \ker{d^i}\ar[dr,"0"]\ar[dl,hook] & \\
		x^{i-1}\ar[ur,two heads]\ar[rr,"d^{i-1}"]\ar[dr,"0"'] & & x^i\ar[rr,"d^i"]\ar[dr,two heads]\ar[dl,two heads] & & x^{i+1} \\
		 & \coker{d^{i-1}}\ar[rr,dashed,"\psi^i"] & & \img{d^i}\ar[ur,hook] & 
	\end{diagram*}
	where the induced maps \(\img{d^{i-1}}\to\ker{d^i}\) and \(\coker{d^{i-1}}\to\img{d^i}\) come from \(d^2=0\) and various arrows being monic/epic.
\end{construction}
\begin{proposition}\label{prop:chain-complex-induced-maps-monic-epic}
	Let \(\phi^i\) and \(\psi^i\) be as in Construction \ref{construction:chain-complex-induced-maps}. Then
	\begin{enumerate}[label=(\arabic*)]
	\item \(\phi^i\) is monic, and
	\item \(\psi^i\) is epic.
	\end{enumerate}
\end{proposition}
\begin{proof}
Statement (2) is dual to (1). To prove (1), note that
\[ (z\to\img{d^{i-1}}\xrightarrow{\phi^i}\ker{d^i}) = 0 \]
if and only if
\[ (z\to\img{d^{i-1}}\xrightarrow{\phi^i}\ker{d^i} \inj x^i) = 0 \]
if and only if
\[ (z\to\img{d^{i-1}} \inj x^i) = 0 \]
if and only if \(z\to\img{d^{i-1}}\) is zero.
\end{proof}
\begin{exercise}
	Spell out the proof of (2) in Proposition \ref{prop:chain-complex-induced-maps-monic-epic}.
\end{exercise}
\begin{construction}\label{construction:chain-complex-induced-maps-2}
	Let \(u^i\) be the composition \(\ker{d^i}\inj x^i\sur\coker{d^{i-1}}\). Forming the kernel and cokernel of \(u^i\), we can enlarge our previous diagram to
	\begin{diagram*}[row sep=small]
		& & \ker{u^i} \ar[dr,hook]\ar[dl,hook,dashed,shift left] & &\\
		 & \img{d^{i-1}}\ar[rr,hook,"\phi^i"]\ar[ur,hook,dashed,shift left]\ar[dr,hook] & & \ker{d^i}\ar[dr,"0"]\ar[dl,hook] & \\
		x^{i-1}\ar[ur,two heads]\ar[rr,"d^{i-1}"]\ar[dr,"0"'] & & x^i\ar[rr,"d^i"]\ar[dr,two heads]\ar[dl,two heads] & & x^{i+1} \\
		 & \coker{d^{i-1}}\ar[rr,two heads,"\psi^i"]\ar[dr,two heads] & & \img{d^i}\ar[ur,hook]\ar[dl,two heads,dashed, shift right] & \\
		& & \coker{u^i}\ar[ur,two heads,dashed,shift right] & &
	\end{diagram*}
	where the dashed maps are induced by the universal properties of the image, kernel, and cokernel.
\end{construction}
\begin{exercise}
	Show that the dashed arrows in Construction \ref{construction:chain-complex-induced-maps-2} define isomorphisms
	\[ \ker{u^i}\iso \img{d^{i-1}} \quad\text{and}\quad \img{d^i}\iso\coker{u^i}. \]
\end{exercise}
Thus, in particular, we finally have the diagram
\begin{diagram*}
	& & \ker{u^i} \ar[dr,hook]\ar[dl,"\sim"'] & &\\
	 & \img{d^{i-1}}\ar[rr,hook,"\phi^i"]\ar[dr,hook] & & \ker{d^i}\ar[dr,"0"]\ar[dl,hook] & \\
	x^{i-1}\ar[ur,two heads]\ar[rr,"d^{i-1}"]\ar[dr,"0"'] & & x^i\ar[rr,"d^i"]\ar[dr,two heads]\ar[dl,two heads] & & x^{i+1} \\
	 & \coker{d^{i-1}}\ar[rr,two heads,"\psi^i"]\ar[dr,two heads] & & \img{d^i}\ar[ur,hook]\ar[dl,"\sim"'] & \\
	& & \coker{u^i} & &
\end{diagram*}
and using this, we define cohomology. In particular, we compute the image
\begin{align*}
	\img{u^i} &\cong \ker(\coker{d^{i-1}}\sur\coker{u^i}) \\
	&\cong \ker(\coker{d^{i-1}}\sur\img{d^{i-1}})
\end{align*}
and the coimage
\begin{align*}
	\coimg{u^i} &\cong \coker(\ker{u^i}\inj\ker{d^i}) \\
	&\cong \coker(\img{d^{i-1}}\inj\ker{d^i}) \\
	&\cong \ker(\coker{d^{i-1}}\to x^{i+1}) \\
	&\cong \coker(x^{i-1}\to\ker{d^{i-1}}).
\end{align*}
\begin{definition}
	Let \((x^\bullet,d)\) be a chain complex in an Abelian category. The \emph{cohomology} of \(x^\bullet\) at \(i\in\Z\) is
	\[ \HH^i(x^\bullet) := \coker(\img{d^{i-1}}\inj\ker{d^{i}}) \]
	or any one of the other canonically isomorphic choices from above. We say \(x^\bullet\) is \emph{exact} at \(i\in\Z\) if \(\HH^i(x^\bullet) = 0\).
\end{definition}
\begin{remark}
	Explicitly, \(\HH^i(x^\bullet) = 0\) is the same as
	\[ u^i = 0 \iff \coker{d^{i-1}} \iso \img{d^i} \iff \img{d^{i-1}}\iso\ker{d^i} \]
	which are the conditions one expect from exactness.
\end{remark}
\begin{definition}
	Let \(\calA\) be an Abelian category. A \emph{short exact sequence} is an exact sequence of the form
	\[ 0 \to x \to y \to z \to 0. \]
\end{definition}
\begin{lemma}\label{lemma:left-exact-sequence-implies-monic}
	A sequence \(0 \to x \overset{f}\to y\) is exact if and only if \(f\) is a monomorphism.
\end{lemma}
\begin{proof}
The sequence is exact if and only if \(\ker{f} = 0\), since the image of \(0\to x\) is \(0\). Since \(\ker{f}=0\) if and only if \(f\) is
monic, this completes the proof.
\end{proof}
\begin{remark}
	It follows that in any short exact sequence
	\[ 0\to x\overset{f}\to y\overset{g}\to z\to 0 \]
	the morphism \(f\) is monic and the morphism \(g\) is epic (by the dual of Lemma \ref{lemma:left-exact-sequence-implies-monic}).
\end{remark}
\begin{theorem}[First isomorphism theorem]
	Consider a short exact sequence
	\[ 0\to x\overset{f}\inj y\overset{g}\sur z\to 0 \]
	in an Abelian category. Then
	\[ x\iso\img{f}\quad\text{and}\quad\coker{f}\iso z. \]
\end{theorem}
\begin{proof}
Lemma \ref{lemma:left-exact-sequence-implies-monic} tells us that \(f\) is a monomorphism, in which case Exercise \ref{exercise:monic-epic-image} tells us that
\(x\iso\img{f}\). For the second isomorphism, we use that Exercise \ref{exercise:monic-epic-image} provides \(\img{g}\cong z\), and that exactness provides us a canonical isomorphism \(\ker{g}\cong\img{f}\).
Thus, compute
\begin{align*}
	z &\cong \img{g} \\
	&\cong \coker(\ker{g}\inj y) \\
	&\cong \coker(\img{f}\inj y) \cong \coker{f}
\end{align*}
which completes the proof.
\end{proof}

\subsection{Exactness properties of functors}
\begin{definition}
	Let \(F\!:\calA\to\calB\) be an additive functor between Abelian categories. We say \(F\) is \emph{left exact} if for any short exact sequence
	\[ 0 \to x \to y \to z \to 0 \]
	in \(\calA\), the sequence
	\[ 0\to Fx \to Fy \to Fz \]
	in \(\calB\) is exact. We say \(F\) is \emph{right exact} if \(F^\op\!:\calA^\op\to\calB^\op\) is left exact. We say \(F\) is \emph{exact} if
	it is both left exact and right exact.
\end{definition}
\begin{theorem}
	Let \(\calA\) be an Abelian category, and let \(a\in\calA\). Then
	\[ \calA(a,-)\quad\text{and}\quad\calA(-,a) \]
	are left exact.
\end{theorem}
\begin{proof}
Consider an exact sequence
\[ 0 \to x \overset{i}\to y \overset{p}\to z \to 0 \]
in \(\calA\). Applying \(\calA(a,-)\), we must show that
\[ 0 \to \calA(a,x)\overset{i_*}\to \calA(a,y)\overset{p_*}\to\calA(a,z) \]
is exact. Since \(i\) is a monomorphism, it follows (by definition!) that \(i_*\) is a monomorphism. For exactness in the middle, consider
a morphism \(f\in\ker{p_*}\). Since \(x\cong\img{i}\cong\ker{p}\), we then have the diagram
\begin{diagram*}
	& & a\ar[d,"f"']\ar[dr,"0"]\ar[dl,dashed,"f'"'] & & \\
	0 \ar[r] & x \ar[r,"i"] & y\ar[r,"p"] & z\ar[r] & 0
\end{diagram*}
so that \(f = i\circ f'\) and \(\img{i_*} = \ker{p_*}\). The other functor is dual, since \(\calA(-,a) = \calA^\op(a,-)\).
\end{proof}
\begin{exercise}
	Prove that \(\calA(-,a)\) is left exact explicitly.
\end{exercise}
\begin{theorem}
	A functor \(F\!:\calA\to\calB\) between Abelian categories is left exact if and only if it preserves finite limits.
\end{theorem}
\begin{proof}
If \(F\) is left exact, then it is additive and preserves left exact sequences. In particular, it preserves finite products and kernels, and therefore
equalizers. Since finite limits are built out of these, it preserves finite limits. Conversely, if it preserves finite limits, then it is additive since it
preserves finite products, and left exact because it preserves kernels. In particular, we use that
\[ 0\to Fx \to Fy\to Fz \]
is exact if and only if \(Fx\to Fy\) is the kernel of \(Fy\to Fz\).
\end{proof}
\begin{corollary}
	A functor \(F\!:\calA\to\calB\) is right exact if and only if it preserves finite colimits, and it is exact if and only if it preserves both
	finite limits and finite colimits.
\end{corollary}

\begin{example}
	Let \(X\) be a topological space, and let \(x\in X\). The stalk functor
	\[ (-)_x\!:\Sh(X,\Ab)\to\Ab \]
	is exact. In particular, \((-)_x\) is a left adjoint and hence commutes with all colimits. Furthermore, it is defined by a filtered colimit, and since the
	forgetful functor \(\Ab\to\Set\) preserves filtered colimits, this means \((-)_x\) commutes with finite limits (since filtered colimits commute with finite limits in \(\Set\)).

	In fact, more is true: a sequence of sheaves
	\[ 0 \to\scrF\to\scrG\to\scrH\to 0 \]
	is exact if and only if for all \(x\in X\),
	\[ 0\to\scrF_x\to\scrG_x\to\scrH_x\to0 \]
	is exact.
\end{example}

\begin{example}
	Let \(X\) be a topological space. The global sections functor
	\[ \Gamma(X,-)\!:\Sh(X,\Ab)\to\Ab \]
	is left exact. This is because it is right adjoint to the constant sheaf functor, hence commutes with all limits.

	More generally, given a continuous map \(f\!:X\to Y\), the functor
	\[ f_*\!:\Sh(X,\Ab)\to\Sh(Y,\Ab) \]
	is left exact since it has a right adjoint given by \(f^{-1}\). This specializes to the previous case when one sets \(Y = *\). The observation
	that these functors are only left exact is the start of the cohomology theory of sheaves, wherein one studies e.g.\ the right derived functor
	\[ \sfR f_*\!:\sfD(X)\to\sfD(Y). \]
	Sheaf cohomology is defined by the right derived functor \(\sfR\Gamma(X,-)\).
\end{example}
\begin{example}
	Sheafification is an exact functor. This is because on one hand, it is a left adjoint functor, and on the other hand, it is actually defined by
	a filtered colimit (if handled appropriately).
\end{example}
\begin{remark}
	We saw that \(\calA(x,-)\) and \(\calA(-,x)\) are left exact. One says \(x\in\calA\) is \emph{projective} if \(\calA(x,-)\) is exact, and dually,
	that \(x\) is \emph{injective} if \(\calA(-,x)\) is exact.
\end{remark}
\begin{exercise}
	Let \(X\) be a topological space, let \(x\in X\), and consider the stalk functor
	\[ (-)_x\!:\Sh(X,\calD)\to\calD. \]
	Using these, we will show, by hand, that under certain conditions, the sheafification functor is exact.
	\begin{enumerate}[label=(\arabic*)]
	\item Suppose we have a functor \(F\!:\calD\to\Set\) which preserves finite limits. Show that the functor
	\[ F_*\!:\Sh(X,\calD)\to\Sh(X),\quad \scrF\mapsto F\circ\scrF \]
	preserves finite limits.
	\item Suppose that \(F\) in addition preserves filtered colimits and reflects isomorphisms. Show that the stalk functor commutes with finite limits.
	\item Show that the sheafification functor \((-)^\dagger\!:\PSh(X,\calD)\to\Sh(X,\calD)\) preserves finite limits. \emph{Hint:} it suffices to show
	that it commutes with kernels and binary products; show this by checking it on stalks.
	\item Check that \(\Mod_R\) satisfies the above conditions.
	\end{enumerate}
\end{exercise}


\subsection{Appendix: Quillen exact categories}\label{appendix:quillen-exact-categories}
There are interesting additive categories, particularly in areas like functional analysis, where only some kernels and cokernels make sense, but not all of them. In
situations like this, one would still like access to some kind of theory. One possible approach to situations of this type is given by Quillen's exact categories. Roughly speaking,
they are additive categories wherein certain distinguished maps lie in kernel/cokernel pairs.
\begin{definition}
	Let \(\calA\) be an additive category. A \emph{kernel/cokernel pair} in \(\calA\) is a pair of morphisms
	\[ x\overset{f}\to y\overset{g}\to z \]
	such that \(f\) defines a kernel of \(g\) and \(g\) a cokernel of \(f\).
\end{definition}
\begin{example}
	In an Abelian category \(\calA\), the short exact sequences
	\[ 0\to x\to y \to z \to 0 \]
	define all kernel/cokernel pairs.
\end{example}
\begin{definition}
	An \emph{exact category} is a pair \((\calC,\calE)\) where \(\calC\) is additive and \(\calE\) is a class of \emph{admissible} kernel/cokernel pairs in \(\calC\); we
	call \(f\) (resp.\ \(g\)) in an admissible kernel/cokernel pair
	\[ x\overset{f}\to y\overset{g}\to z \]
	an \emph{admissible monomorphism} (resp. \emph{admissible epimorphism}) or an \emph{inflation} (resp.\ \emph{deflation}). The pair \((\calC,\calE)\) is required
	to satisfy the following conditions:
	\begin{enumerate}[label=(\arabic*)]
	\item The class \(\calE\) is closed under isomorphism, meaning that if we have a diagram
	\begin{diagram*}
		x\ar[r]\ar[d,"\sim" labl] & y\ar[r]\ar[d, "\sim" labl] & z\ar[d, "\sim" labl] \\
		x'\ar[r] & y'\ar[r] & z'
	\end{diagram*}
	where one row is in \(\calE\) then the other row is in \(\calE\).
	\item For each \(x\in X\), the identity map \(\id_x\) is an inflation and a deflation.
	\item The class of inflations is closed under composition. Dually, the class of deflations is closed under composition.
	\item The class of inflations is stable under pushouts. Dually, the class of deflations is closed under pullbacks.
	\end{enumerate}
\end{definition}
We provide a few arguments to give a taste of the kinds of proofs that are involved when working with exact categories.
\begin{remark}
	The axioms imply that all isomorphisms \(f\!:x\to x\) are both inflations and deflations. In particular, we have isomorphisms
	\begin{diagram*}
		x\ar[r,"f"]\ar[d,equal] & x\ar[r]\ar[d,"f^{-1}"] & 0\ar[d] \\
		x\ar[r,equal] & x\ar[r] & 0
	\end{diagram*}
	and
	\begin{diagram*}
		0 \ar[r]\ar[d] & x\ar[r,"f"]\ar[d,equal] & x\ar[d,"f^{-1}"] & \\
		0 \ar[r] & x\ar[r,equal] & x
	\end{diagram*}
	and by the closure of \(\calE\) under isomorphism, we conclude.
\end{remark}
\begin{remark}
	It also follows that the zero map \(0\to x\) is an inflation, and the zero map \(x\to 0\) is a deflation.
\end{remark}
\begin{remark}
	The data of the inflations or the deflations both determine the class \(\calE\). Indeed, supposing
	we have
	\[ x\overset{i}\to y\overset{p}\to z \]
	where \(i\) is an inflation and \(p\) is a cokernel of \(i\), we may find a deflation \(p'\!:y\to z'\) so that we have an isomorphism
	\begin{diagram*}
		x\ar[r,"i"]\ar[d,"\sim" labl] & y\ar[r,"p"]\ar[d,"\sim" labl] & z\ar[d,"\sim" labl] \\
		x\ar[r,"i"] & y\ar[r,"p'"] & z'
	\end{diagram*}
	where the bottom row is in \(\calE\), and so the original pair of maps was already in \(\calE\). Notably, a morphism is a deflation if and only if it appears as the cokernel of an inflation. A
	dual argument shows that a morphism is an inflation if and only if it appears as the kernel of a deflation.
\end{remark}
\begin{proposition}
	Let \((\calC,\calE)\) be an exact category, and let \(x,y\in\calC\). Then
	\begin{diagram*}
		x\ar[r,"\iota"] & x \oplus y\ar[r,"\pi"] & y
	\end{diagram*}
	is an admissible kernel/cokernel pair.
\end{proposition}
\begin{proof}
We have a pushout square
\[
\begin{tikzcd}
	0 \ar[r]\ar[d] & y\ar[d] \\
	x \ar[r,"\iota"] & x\oplus y\ar[ul,pushout]
\end{tikzcd}
\]
and since \(0\to y\) is an inflation, it follows that \(\iota\) is an inflation. Since \(\pi\) is the cokernel of \(\iota\), it is a deflation and
\begin{diagram*}
	x\ar[r,"\iota"] & x \oplus y\ar[r,"\pi"] & y
\end{diagram*}
is an admissible kernel/cokernel pair, as desired.
\end{proof}
\begin{corollary}
	Let \((\calC,\calE)\) be an exact category. Then \(\calE\) contains all split exact sequences.
\end{corollary}

One can prove various additional results, but since we do not aim to give an actual introduction here, we leave it for the references. A possible resource is \cite{frerick2010exact}.



\subsection{Appendix: An aside on stable ∞-categories}
Abelian categories form a convenient setting to formalize the properties of categories like \(\Ab\), and with which to develop homological algebra. However,
it has certain deficiencies. The most natural way to formalize homological algebra is through the use of \emph{derived categories} \(\sfD(\calA)\) of Abelian
categories \(\calA\). The way to understand these is as follows: in homological algebra, one studies properties of chain complexes that only depend on the cohomologies
of the chain complexes. In particular, we would like to think of maps between chain complexes which induce isomorphisms on the level of cohomology as being
``genuine'' isomorphisms; we call such maps \emph{quasi-isomorphisms.} The derived category \(\sfD(\calA)\) is obtained by taking the category of chain
complexes \(\Ch(\calA)\) and formally inverting the quasi-isomorphisms; the process of doing this is called \emph{localization.}

Localizations of categories, which we will study later, are very hard to control. For example, given a small category \(\calC\) and some class of morphisms \(\calS\) in \(\calC\),
the localization \(\calC[\calS^{-1}]\) need not be locally small. This is already a sign that localizations are badly behaved in general. The issue that comes up in homological algebra
is that while the localization of an additive category does turn out to be additive, the localization of an \emph{Abelian} category, such as \(\Ch(\calA)\), need not be Abelian.
This is bad! After all, we would like to understand their behaviour in terms of things we already know.

Historically, a proposed solution came in the development of \emph{triangulated categories,} which we will also study. Triangulated categories were the invention of Alexander Grothendieck
together with his PhD student Jean-Louis Verdier, and they capture the notion of ``homotopy'' Abelian categories. They can be thought of as an analogue of Quillen exact categories, in that
one specifies as part of additional structure a collection of \emph{distinguished} sequences of maps to be considered the ``homotopy short exact sequences.''

A more systematic proposal came later in the form of stable model categories. However, the most recent and sophisticated upgrade to the concept is provided by \emph{stable} \(\infty\)\emph{-categories.}
We give a heuristic explanation of these now.
\begin{definition}
	A \emph{pointed} \(\infty\)-category is an \(\infty\)-category which admits a zero element, namely an element which is both initial and terminal.
\end{definition}
\begin{remark}
	As an aside, any pointed \(\infty\)-category \(A\) which admits finite limits and finite colimits also admits two functors \(\Sigma\!:A\to A\) and \(\Omega\!:A\to A\) given on objects by the diagrams
	\begin{center}
	\begin{tikzcd}
		x\ar[r]\ar[d] & *\ar[d] \\
		*\ar[r] & \Sigma x\ar[ul,pushout]
	\end{tikzcd}
	\quad
	\begin{tikzcd}
		\Omega x\ar[r]\ar[d]\ar[dr,pullback] & * \ar[d] \\
		*\ar[r] & x
	\end{tikzcd}
	\end{center}
	and furthermore, it turns out that \(\Sigma\ladj\Omega\). These functors themselves can be used, along with some other things, to define stability. This is how it is handled
	in the context of stable model categories.
\end{remark}

\begin{definition}
	Let \(A\) be a pointed \(\infty\)-category, and let \(f\!:x\to y\) be a morphism in \(A\). The \emph{fiber} and \emph{cofiber} of \(f\) are defined by
	the diagrams
	\begin{center}
	\begin{tikzcd}
		\fib(f)\ar[r]\ar[d]\ar[dr,pullback] & x\ar[d,"f"] \\
		*\ar[r] & y
	\end{tikzcd}
	\quad
	\begin{tikzcd}
		x\ar[r,"f"]\ar[d] & y\ar[d] \\
		*\ar[r] & \cofib(f)\ar[ul,pushout]
	\end{tikzcd}
	\end{center}
	in \(A\) whenever they exist. The square above left is called a \emph{fiber sequence,} and the square above right is called a \emph{cofiber sequence.}
\end{definition}

With the above definition available, we can make a very simple definition.
\begin{definition}
	Let \(A\) be a pointed \(\infty\)-category. We say \(A\) is \emph{stable} if
	\begin{enumerate}[label=(\arabic*)]
	\item every morphism in \(A\) has a fiber and a cofiber, and
	\item every fiber sequence is a cofiber sequence, and vice versa.
	\end{enumerate}
\end{definition}

To any \(\infty\)-category \(X\), one can associate a 1-category \(\ho(X)\), called its \emph{homotopy category.} The functor \(X\mapsto\ho(X)\) is left adjoint to the inclusion of 1-categories
into \(\infty\)-categories. Stable \(\infty\)-categories \(A\) have the property that their homotopy categories \(\ho(A)\) have a canonical triangulated structure, induced by the functor
\(\Sigma\!:A\to A\) and with distinguished triangles
\[ x\to y\to z \to \Sigma x \]
in \(\ho(A)\) defined by being those sequences of morphisms which fit into
\begin{diagram*}
	x \ar[r]\ar[d] & y\ar[d]\ar[r] & *\ar[d] \\
	* \ar[r] & z \ar[r] & \Sigma x
\end{diagram*}
where both squares are (co)fiber sequences in \(A\). In this way, stable \(\infty\)-categories are a natural enrichment of triangulated categories.

Triangulated categories have a deficiency in that their distinguished triangles do not yield any universal properties, so there is no good way to characterize the ``homotopy (co)kernels''
that arise from them. Stable \(\infty\)-categories fix this as the \(\infty\)-categorical framework allows a genuine ``homotopy-theoretic'' universal property. Due to the framework
being so well-behaved, they come along with a number of other theoretical benefits, such as cofibers of exact functors between stable \(\infty\)-categories being stable. Furthermore,
there is a natural way to promote derived categories into stable \(\infty\)-categories in a way which recovers the classical ones after applying \(\ho(-)\).
