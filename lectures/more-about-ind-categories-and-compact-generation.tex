%!TEX root = ../lectures.tex

\section{More about Ind-categories \& compact generation}
This lecture has a few goals. Previously, we gave a sort of ``extrinsic'' definition of Ind-categories, and relied upon Lemma \ref{lemma:ind-category-admits-filtered-colimits} in our dealings with it without
justification. We want to justfy this lemma, and also provide a more ``intrinsic'' perspective on Ind-categories. The first thing we want is a more canonical way to detect if a presheaf \(A\) lives in an Ind-category.
The definition we gave required that
\[ A \cong \injlim{D} \]
where \(D\) is some small (\(\kappa\)-)filtered diagram of representable functors. It turns out that there is always a canonical way to represent
a presheaf as a colimit, and that this representation can be used to characterize when \(A\in\Ind_\kappa(\calC)\).

Another major goal is to explain a phenomenon whereby a category may be \emph{generated} under filtered colimits by compact objects. Consider the following motivating sketches.
\begin{example}
	Trivially, \(\Set \cong \PSh(*)\), expressing the fact that the category of (small) sets is freely generated under (small) colimits by a point. Indeed, one only needs small coproducts.
	On the other hand, one also has \(\Set \cong \Ind(\cat{Fin})\), expressing that the category of sets is freely generated by finite sets under \emph{filtered colimits.} This was implicitly
	used in the proof computing the compact objects of \(\Set\); there, we noted that any set is the filtered colimit of its poset of finite subsets.
\end{example}
\begin{example}
	Let \(\Bbbk\) be a field. Then
	\[ \Vect_\Bbbk \cong \Ind(\Vect_\Bbbk^\fd). \]
	This follows by noting that any \(\Bbbk\)-vector space can be written as the union of its finite-dimensional subspaces.
\end{example}

In order to justify the above examples, our goal is to prove that any category admitting some small set of compact objects which generate the category under filtered colimits (in
the intuitive sense) can be written as the Ind-category of those compact objects.

\subsection{Category of elements}
\begin{definition}
	Let \(\calC\) be a locally small category, and let \(A\in\PSh(\calC)\). The \emph{category of elements} \(\calC/A\) of \(A\) is given by:
	\begin{itemize}[label=\(\star\)]
	\item An object of \(\calC/A\) is a tuple \((x,a)\) where \(x\in\calC\) and \(a\in Ax\).
	\item A morphism \((x,a)\to(x',a')\) is a morphism \(f\!:x\to x'\) such that \((Af)(a') = a\), i.e.\ \(f^*a' = a\).
	\end{itemize}
\end{definition}
\begin{remark}
	One easily sees that the category of elements \(\calC/A\) can also be described as follows: an object is a pair \((x,a)\) where \(x\in\calC\) and
	\(a\!:h_x\to A\); a morphism \((x,a)\to(x',a')\) is a morphism \(f\!:x\to x'\) such that the diagram
	\begin{diagram*}[cramped,column sep=small]
		h_x \ar[rr,"f_*"]\ar[dr,"a"'] & & h_{x'}\ar[dl,"a'"] \\
		& A & 
	\end{diagram*}
	commutes. This follows by the Yoneda lemma.
\end{remark}
\begin{remark}
	There is a ``projection'' functor, which we consider canonical, of type
	\[ \pi_A\!:\calC/A \to \calC, \]
	given by sending \((x,a)\) to \(x\). Observe that for each \(x\in\calC\), the ``fiber of \(\pi_A\) over \(x\)'' is exactly \(Ax\). More precisely, consider the subcategory
	of \(\calC/A\) spanned by those objects of the form \((x,a)\), and those arrows which map to \(\id_x\). This is a discrete category, whose underlying set of objects is just \(Ax\).
\end{remark}
\begin{notation}
	Let \(\calC\) be a category. We denote by
	\[ \yo = \yo_\calC\!:\calC\inj\PSh(\calC),\quad x\mapsto \yo(x) := h_x \]
	the Yoneda embedding.
\end{notation}
\begin{lemma}\label{lemma:presheaves-are-colimits-of-representable-functors}
	Let \(\calC\) be a category, and let \(A\) be a presheaf on \(\calC\). Then
	\[ A = \injlim(\yo_\calC\circ\pi_A). \]
	Colloquially,
	\[ A = \injlim_{h_x\to A}h_x. \]
\end{lemma}
\begin{proof}
Observe that we trivially have a collection of morphisms \(\{a\!:h_x\to A\}_{(x,a)\in\calC/A}\) compatible with the morphisms in \(\calC/A\). In other words,
we have a cone under \(\yo_\calC\circ\pi_A\) with tip \(A\). Suppose we have another such cone \(\{b_{x,a}\!:h_x\to B\}_{(x,a)\in\calC/A}\). By the Yoneda lemma,
this uniquely induces a morphism \(\alpha\!:A\to B\) for which the diagram
\begin{diagram*}[cramped, column sep=small]
	& h_x\ar[dl,"a"']\ar[dr,"b_{x,a}"] & \\
	A\ar[rr,"\alpha"] & & B
\end{diagram*}
commutes. In particular, the natural transformation \(\alpha\) is given by \(\alpha_x(a) := b_{x,a}\), where we identify the morphisms with their
corresponding elements under the Yoneda lemma. It follows that \(A\) forms a colimit of \(\yo_\calC\circ\pi_A\), as desired.
\end{proof}
\begin{corollary}
	Let \(\calC\) be a category, and let \(x\in\calC\). Then, for all \(z\in\calC\), we have a natural isomorphism
	\[ \calC(z,x) \cong \injlim_{(y\to x)\in\calC/x}\calC(z,y). \]
\end{corollary}
\begin{exercise}\label{exercise:presheaves-on-slice-is-slice-of-presheaves}
	Let \(\calC\) be a locally small category, and let \(A\in\PSh(\calC)\). Show that there is an equivalence \(\PSh(\calC)/A \simeq \PSh(\calC/A)\) such that the diagram
	\begin{diagram*}
		\calC/A\ar[r]\ar[dr,hook,"\yo"'] & \PSh(\calC)/A\ar[d,"\sim" labl] \\
		& \PSh(\calC/A)
	\end{diagram*}
	commutes, where the functor \(\calC/A\to\PSh(\calC)/A\) is the canonical inclusion induced by the Yoneda embedding \(\yo_\calC\!:\calC\inj\PSh(\calC)\).
\end{exercise}

\begin{lemma}\label{lemma:category-of-elements-induced-colimits}
	Let \(\calC\) be a category, and assume that \(\calC\) admits \(I\)-shaped colimits.
	Consider a presheaf \(A\in\PSh(\calC)\). If \(A\) commutes with \(I\)-shaped limits, so
	that \(A(\injlim{D}) \cong \projlim(A\circ D^\op)\) for all diagrams \(D\!:I\to\calC\), then \(\calC/A\) admits \(I\)-shaped colimits and \(\pi_A\!:\calC/A\to\calC\) preserves these colimits.
\end{lemma}
\begin{proof}
Consider a diagram \(D\!:I\to\calC/A\), \(i\mapsto (D(i), u_i\in A(D(i)))\). Since \(\calC\) admits \(I\)-shaped colimits, the colimit of \(\pi_A\circ D\) exists. Furthermore,
since \(A\) commutes with \(I\)-shaped limits, we have
\[ A(\injlim_{i\in I}D(i)) \cong \projlim_{i\in I}{A(D(i))}. \]
The elements \(u_i\) then provide an element of the latter, hence we get an element \(u\in A(\injlim(\pi_A\circ D))\). Thus, we can lift the colimit in \(\calC\) to an element
\[ (\injlim(\pi_A\circ D), u)\in\calC/A. \]
One easily sees that this is a colimit of \(D\).
\end{proof}

\begin{proposition}\label{prop:filtered-category-of-elements-iff-preserves-small-limits}
	Let \(\calC\) be a category admitting \(\kappa\)-small colimits, and let \(A\in\PSh(\calC)\). Then the following are equivalent:
	\begin{enumerate}
	\item \(\calC/A\) is \(\kappa\)-filtered.
	\item \(A\) preserves \(\kappa\)-small limits.
	\end{enumerate}
	In particular, \(A\) is left exact if and only if \(\calC/A\) is filtered.
\end{proposition}
\begin{proof}
Assume (1), and let \(D\!:I\to\calC\) be a \(\kappa\)-small diagram. Then
\begin{align*}
	A(\injlim{D}) &= \injlim(\yo_\calC\circ\pi_A)(\injlim{D})\\
	&= \injlim_{h_x\to A}\calC(\injlim_{i\in I}{D(i)},x)\\
	&= \injlim_{h_x\to A}\projlim_{i\in I}\calC(D(i),x)\\
	&\cong \projlim_{i\in I}\injlim_{h_x\to A}\calC(D(i),x) = \projlim_{i\in I}A(D(i))
\end{align*}
so that (2) holds. Here, we used that \(\kappa\)-filtered colimits commute with \(\kappa\)-small limits in \(\Set\).

Conversely, assume (2). Then \(\calC/A\) admits \(\kappa\)-small colimits by Lemma \ref{lemma:category-of-elements-induced-colimits}. But then \(\calC/A\) is \(\kappa\)-filtered: indeed, given a \(\kappa\)-small diagram \(I\to \calC/A\),
the colimit exists and hence gives rise to an extension \(I^\triangleright\to\calC/A\).
\end{proof}

\subsection{Cofinality \& an intrinsic characterization of Ind-categories}
\begin{definition}
	Let \(\varphi\!:I\to J\) be a functor. We say \(\varphi\) is \emph{cofinal} (or \emph{final}) if for any diagram \(D\!:J\to\calC\) in a category \(\calC\), the canonical comparison map
	\[ \injlim(D\circ\varphi)\to\injlim{D} \]
	is an isomorphism. We say \(J\) is cofinally small if there is a cofinal functor \(I\to J\) where \(I\) is small.
\end{definition}

While cofinality is an important concept, and widely used, we do not want to spend too much time on it here, as it can be rather technical.

\begin{proposition}\label{prop:cofinal-equivalent-conditions}
	Let \(\varphi\!:I\to J\) be a functor. The following are equivalent:
	\begin{enumerate}[label=(\arabic*)]
	\item \(\varphi\) is final.
	\item For all diagrams \(D\!:J\to\Set\), the canonical comparison map
	\[ \injlim(D\circ\varphi)\to\injlim{D} \]
	is an isomorphism.
	\item For all \(j\in J\), the comma category \(j/\varphi\) is connected, i.e.\ it is non-empty, and every two objects are connected by a zigzag of morphisms.
	\item For all \(j\in J\), we have \(\injlim_{i\in I}J(j,\varphi(i)) \cong *\).
	\end{enumerate}
\end{proposition}
\begin{proof}
See \cite[Prop.\ 2.5.2]{kashiwara-schapira-book}.
\end{proof}
\begin{remark}
	The comma category \(j/\varphi\) has as objects pairs \((i,t)\) where \(i\in I\) and \(t\!:j\to\varphi(i)\), and a morphism \((i,t)\to(i',t')\) is a morphism \(s\!:i\to i'\) such that \(t' = \varphi(s)\circ t\).
\end{remark}

\begin{construction}
	Consider a locally small category \(\calC\) and a small diagram \(D\!:I\to\calC\). We can form the formal colimit of \(D\) by taking the colimit
	\[ A := \injlim(\yo_\calC\circ D) \]
	in the category of presheaves. Let \(h_i\!:h_{D(i)}\to A\) be the canonical morphism. This induces a new diagram
	\[ \tilde{D}\!:I\to\calC/A \]
	given by \(i\mapsto (D(i), h_i)\) on objects, and \((t\!:i\to i')\mapsto Dt\) on morphisms.
\end{construction}

\begin{lemma}\label{lemma:induced-diagram-cofinal}
	Let \(\calC\) be a locally small category, and let \(D\!:I\to\calC\) be a small diagram. Then the induced diagram \(\tilde{D}\!:I\to\calC/A\) is a cofinal
	functor.
\end{lemma}
\begin{proof}
By (the same argument as) Lemma \ref{lemma:category-of-elements-induced-colimits}, the functor \(\yo/A\!:\PSh(\calC)/A\to\PSh(\calC)\) commutes with small colimits. Observe that we have
a commutative diagram
\begin{diagram*}
	I\ar[r,"\tilde{D}"]\ar[dr,"D"'] & \calC/A\ar[d,"\pi_A"]\ar[r,"\yo/A"] & \PSh(\calC)/A\ar[d] \\
	& \calC\ar[r,"\yo"] & \PSh(\calC)
\end{diagram*}
In particular, the underlying presheaf of \(\injlim(\yo/A\circ\tilde{D})\) is \(A\). One checks, using the equivalence \(\PSh(\calC/A)\simeq\PSh(\calC)/A\) of
Exercise \ref{exercise:presheaves-on-slice-is-slice-of-presheaves}, that the structure map is just the identity \(\id_A\!:A\to A\), so that \(\injlim(\yo/A\circ\tilde{D})\)
is the terminal object in \(\PSh(\calC)/A\). Using the characterization (4) in Proposition \ref{prop:cofinal-equivalent-conditions}, one sees that \(\tilde{D}\) is cofinal.
\end{proof}

% \begin{lemma}\label{lemma:cofinal-filtered-implies-filtered}
% 	Let \(\varphi\!:I\to J\) be a cofinal functor. If \(I\) is \(\kappa\)-filtered, then \(J\) is \(\kappa\)-filtered.
% \end{lemma}

\begin{theorem}\label{thm:ind-equivalent-characterizations}
	Let \(\calC\) be a locally small category, let \(\kappa\) be an infinite regular cardinal, and let \(A\in\PSh(\calC)\). Then the following are equivalent:
	\begin{enumerate}[label=(\arabic*)]
	\item \(A\in\Ind_\kappa(\calC)\).
	\item \(\calC/A\) is \(\kappa\)-filtered and cofinally small.
	\item \(A\) commutes with \(\kappa\)-small limits and \(\calC/A\) is cofinally small.
	\end{enumerate}
\end{theorem}
\begin{proof}
(3) implies (2) by Proposition \ref{prop:filtered-category-of-elements-iff-preserves-small-limits}. That (2) implies (1) is clear, by Lemma \ref{lemma:presheaves-are-colimits-of-representable-functors}.
Finally, (1) implies (3) by combining Lemma \ref{lemma:induced-diagram-cofinal} with the same argument as in Proposition \ref{prop:filtered-category-of-elements-iff-preserves-small-limits}.
\end{proof}

\begin{remark}
	If we drop the requirement that an Ind-object should be represented by a small diagram, then we can drop the cofinal smallness condition. Furthermore, if \(\calC\) is
	actually a \emph{small} category, rather than just locally small, then \(\calC/A\) too is small. Indeed, it is clearly locally small, and the objects can be realized
	as a disjoint union of small sets indexed by a small set. Thus, when \(\calC\) is small, the technicalities presented by cofinal smallness disappear.
\end{remark}

\begin{proof}[Proof of Lemma \ref{lemma:ind-category-admits-filtered-colimits}]
Let \(D\!:I\to\Ind_\kappa(\calC)\) be a small \(\kappa\)-filtered diagram. Let \(A\) be the colimit of \(D\) in \(\PSh(\calC)\).
We will use condition (2) in Theorem \ref{thm:ind-equivalent-characterizations} to show that \(A\in\Ind_\kappa(\calC)\); in doing this, we will neglect showing that \(\calC/A\) is
cofinally small, as it requires a very technical argument about the interplay between cofinality and \(\kappa\)-filteredness. A proof can be found in \cite[Thm.\ 6.1.8]{kashiwara-schapira-book}.

To see that \(\calC/A\) is \(\kappa\)-filtered, consider a diagram \(K\!:J\to\calC/A\) where \(J\) is \(\kappa\)-small. We need to find a cone under \(K\).
Let \(u_i\!:h_{D(i)}\to A\) be the canonical maps, and write \(\widehat\calC := \PSh(\calC)\). For any \((x,a)\in\calC/A\), we have
\begin{align*}
	(\widehat\calC/A)(h_x\to A,A\overset{\id}\to A)) &\cong \injlim_{i\in I}(\widehat\calC/A)(h_x\to A,h_{D(i)}\overset{u_i}\to A))\\
	&\cong \injlim_{i\in I}\injlim_{(y\to D(i))\in\calC/D(i)}\calC(x,y).
\end{align*}
Now, \(I\) is \(\kappa\)-filtered by assumption, and \(\calC/D(i)\) has a terminal object and in hence \(\kappa\)-filtered, and therefore
\begin{align*}
	\{*\} &\cong \projlim_{j\in J}(\widehat\calC/A)(K(j),A\overset{\id}\to A) \\
	&\cong \projlim_{j\in J}\injlim_{i\in I}\injlim_{(y\to D(i))\in\calC/D(i)}\calC(K(j),y) \\
	&\cong \injlim_{i\in I}\injlim_{(y\to D(i))\in\calC/D(i)}\projlim_{j\in J}\calC(K(j),y) \\
\end{align*}
so we find some \(i\in I\) and \(y_0\to D(i)\), which define an element \(y = (y_0, h_{y_0}\to h_{D(i)}\overset{u_i}\to A)\), for which
\[ \projlim_{j\in J}(\calC/A)(K(j),y) \not= \varnothing. \]
In other words, we can find a collection of morphisms forming a cone \(K\To y\) as desired.
\end{proof}

\subsection{Functoriality \& universal property of Ind}
Here, we should really be spelling out 2-categorical data. We will avoid doing this, however, and pretend that everything holds strictly for simplicity, as
otherwise the technical details riks obscuring the fundamental ideas, which are very simple.
\begin{proposition}
	Let \(\calC\) be a small category, \(\calD\) a locally small category, and let \(\kappa\) be an infinite regular cardinal. Consider a functor \(F\!:\calC\to\calD\). Then
	there is a unique functor \(\Ind_\kappa(F)\!:\Ind_\kappa(\calC)\to\Ind_\kappa(\calD)\) commuting with small \(\kappa\)-filtered colimits and for which the diagram
	\begin{diagram*}
		\calC\ar[r,"F"]\ar[d,hook,"\yo_\calC"'] & \calD \ar[d,hook,"\yo_\calD"] \\
		\Ind_\kappa(\calC)\ar[r,"\Ind_\kappa(F)"] & \Ind_\kappa(\calD)
	\end{diagram*}
	commutes.
\end{proposition}
\begin{proof}
Let \(A\in\Ind_\kappa(\calC)\). Then we have
\[  A = \injlim(\calC/A\overset{\pi_A}\longto\calC\overset{\yo_\calC}\longto\Ind_\kappa(\calC)). \]
Since \(\calC\) is small, \(\calC/A\) is also small. In particular, for \(\Ind_\kappa(F)\) to commute with small \(\kappa\)-filtered colimits and for the diagram to commute, we must have
\[ \Ind_\kappa(F)(A) = \injlim(\yo_\calD\circ F\circ\pi_A) \]
and for a morphism \(A\to B\) in \(\Ind_\kappa(\calC)\), the morphism \(\Ind_\kappa(F)(A)\to\Ind_\kappa(F)(B)\) must be the canonical morphism
\[ \injlim(\yo_\calD\circ F\circ\pi_A) \to \injlim(\yo_\calD\circ F\circ\pi_B). \]
So we define a unique functor by these requirements.
\end{proof}

The above functoriality statement makes it fairly easy to provide a universal property for \(\Ind_\kappa(\calC)\). It expresses that this category is the
formal cocompletion of \(\calC\) with respect to \(\kappa\)-filtered colimits.

\begin{lemma}
	Let \(\calC\) be a locally small category admitting small \(\kappa\)-filtered colimits for some infinite regular cardinal \(\kappa\).
	\begin{enumerate}[label=(\arabic*)]
	\item The functor \(\yo_\calC\!:\calC\to\Ind_\kappa(\calC)\) admits a left adjoint \(\sigma_\calC\!:\Ind_\kappa(\calC)\to\calC\), taking a ``formal'' \(\kappa\)-filtered
	colimit to its actual colimit object in \(\calC\).
	\item The functors compose to give \(\sigma_\calC\circ\yo_\calC \cong \1_\calC\).
	\end{enumerate}
\end{lemma}
\begin{proof}
Given the construction below (2) is clear, so we do not prove it explicitly. For (1), let \(A\in\Ind_\kappa(\calC)\), \(x\in\calC\). Then we have natural isomorphisms
\[ \Ind_\kappa(\calC)(A,\yo_\calC(x)) \cong \projlim\Ind_\kappa(\calC)(\yo_\calC\circ\pi_A,\yo_\calC(x)) \cong \projlim\calC(\pi_A,x) \cong \calC(\injlim\pi_A,x)  \]
so we are done, since the colimit \(\injlim\pi_A\) exists on account of \(\calC/A\) being cofinally small and \(\kappa\)-filtered.
\end{proof}
\begin{theorem}
	Let \(\calC\) be a small category, and let \(\kappa\) be an infinite regular cardinal. Then \(\Ind_\kappa(\calC)\) satisfies the following universal property:
	for any locally small category \(\calD\) admitting \(\kappa\)-filtered colimits, and any functor \(F\!:\calC\to\calD\), there is a functor \(F'\!:\Ind_\kappa(\calC)\to\calD\) unique
	up to unique isomorphism such that the diagram
	\begin{diagram*}
		\calC\ar[d,hook]\ar[r,"F"] & \calD \\
		\Ind_\kappa(\calC)\ar[ur,dashed,"F'"']
	\end{diagram*}
	commutes up to natural isomorphism and \(F'\) commutes with \(\kappa\)-filtered colimits.
\end{theorem}
\begin{proof}
The functor \(F'\) is given by the composition
\[ \Ind_\kappa(\calC)\overset{\Ind_\kappa(F)}\longto\Ind_\kappa(\calD)\overset{\sigma_\calD}\longto\calD. \]
One easily sees that this choice is unique from the uniqueness of \(\Ind_\kappa(F)\).
\end{proof}

\subsection{Accessibility}
\begin{lemma}\label{lemma:induced-ind-functor-fully-faithful}
	Let \(F\!:\calC\to\calD\) be a functor, where \(\calC\) is small and \(\calD\) is locally small. Assume further that the following conditions are satisfied:
	\begin{enumerate}[label=(\arabic*)]
	\item \(\calD\) admits small \(\kappa\)-filtered colimits.
	\item \(F\) is fully faithful.
	\item For all \(x\in\calC\), the object \(Fx\in\calD\) is \(\kappa\)-compact.
	\end{enumerate}
	Then the induced functor \(F'\!:\Ind_\kappa(\calC)\to\calD\) is fully faithful.
\end{lemma}
\begin{proof}
This is a computation. Let \(A,B\in\Ind_\kappa(\calC)\). Then
\begin{align*}
	\Ind_\kappa(\calC)(A,B) &\cong \projlim\injlim\Ind_\kappa(\calC)(\yo_\calC\circ\pi_A,\yo_\calC\circ\pi_B) \\
	&\cong \projlim\injlim\calC(\pi_A,\pi_B) \\
	&\cong \projlim\injlim\calD(F\circ\pi_A,F\circ\pi_B) \\
	&\cong \projlim\calD(F\circ\pi_A,\injlim(F\circ\pi_B)) \\
	&\cong \calD(\projlim(F\circ\pi_A),\injlim(F\circ\pi_B)) \cong \calD(F'(A),F'(B))
\end{align*}
as desired.
\end{proof}
\begin{theorem}
	Let \(\calC\) be a category, and let \(\kappa\) be an infinite regular cardinal. Then the following are equivalent:
	\begin{enumerate}[label=(\arabic*)]
	\item There is a small category \(\calC_0\) such that \(\calC\simeq\Ind_\kappa(\calC_0)\).
	\item The category \(\calC\) satisfies the following conditions:
		\begin{enumerate}[label=(\roman*)]
		\item \(\calC\) is locally small.
		\item \(\calC\) admits \(\kappa\)-filtered colimits.
		\item There is a small full subcategory \(\calC_0\) of \(\calC\) consisting of \(\kappa\)-compact objects for which every object of \(\calC\) can be written as the colimit of a \(\kappa\)-filtered diagram
		in \(\calC_0\).
		\end{enumerate}
	\end{enumerate}
\end{theorem}
\begin{proof}
It is clear that (1) implies (2), effectively by definition. To show that (2) implies (1), the idea is to show that \(\calC\simeq\Ind_\kappa(\calC_0)\) for the chosen full subcategory \(\calC_0\) of \(\kappa\)-compact
objects. To do this, consider the canonical functor \(\Ind_\kappa(\calC_0)\to\calC\) induced by universal property from the inclusion \(\calC_0\inj\calC\). By assumption, all the
conditions of Lemma \ref{lemma:induced-ind-functor-fully-faithful} are satisfied, so this functor is fully faithful. However, it is also essentially surjective by assumption, and therefore
an equivalence.
\end{proof}
\begin{exercise}
	Implicitly, in many places we have used that \(\Ind_\kappa(\calC)\) is locally small.
	\begin{enumerate}[label=(\arabic*)]
	\item Compute the Hom-sets in \(\Ind_\kappa(\calC)\) in terms of those in \(\calC\), using that representables are compact.
	\item Deduce that \(\Ind_\kappa(\calC)\) is locally small.
	\end{enumerate}
\end{exercise}
\begin{definition}
	Consider a regular infinite cardinal \(\kappa\). We say a category \(\calC\) is \(\kappa\)\emph{-accessible} if there is a small category \(\calC_0\) such that \(\calC\simeq\Ind_\kappa(\calC_0)\).
	We say that \(\calC\) is \emph{accessible} if it is \(\kappa\)-accessible for some \(\kappa\). A functor \(F\!:\calC\to\calD\) between accessible categories is called accessible
	if both \(\calC\) and \(\calD\) are \(\kappa\)-accesible for some common \(\kappa\), and \(F\) preserves \(\kappa\)-filtered colimits.
\end{definition}

Accessibility is \emph{very nearly} a purely set-theoretic condition.
\begin{proposition}
	Let \(\calC\) be a small category. Then \(\calC\) is accessible if and only if it is idempotent complete.
\end{proposition}
\begin{proof}
See \cite[Prop.\ 2.2.1 \& Thm.\ 2.2.2]{makkai-pare-accessible-categories}.
\end{proof}

\begin{definition}
	An accessible category is \emph{presentable} if it is cocomplete.
\end{definition}

Presentable categories are very useful in the context of adjoint functor theorems, because the automatically satisfy all the conditions for them to hold. In \(\infty\)-category theory,
this is often exploited by making use of the \(\infty\)-category \(\cat{Pr}^L\) of presentable \(\infty\)-categories and functors preserving small colimits. This \(\infty\)-category
is particularly nice, because it also has the structure of a symmetric monoidal \(\infty\)-category under the \emph{Lurie tensor product.}

Being presentable and \(\kappa\)-accessible (that is, \(\kappa\)-accessible and cocomplete) is sometimes refered to as being \(\kappa\)\emph{-compactly generated,} or \(\kappa\)-presentable.
