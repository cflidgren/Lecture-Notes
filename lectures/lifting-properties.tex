%!TEX root = ../lectures.tex

% copied from tex.stackexchange @ https://tex.stackexchange.com/questions/615542/boxtimes-without-a-diagonal
% don't need it, actually, but I'm keeping it around just in case
% \DeclareFontFamily{U}{mathb}{\hyphenchar\font45}
% \DeclareFontShape{U}{mathb}{m}{n}{
%       <5> <6> <7> <8> <9> <10> gen * mathb
%       <10.95> mathb10 <12> <14.4> <17.28> <20.74> <24.88> mathb12
%       }{}
% \DeclareSymbolFont{mathb}{U}{mathb}{m}{n}
% \DeclareMathSymbol{\boxslash}{2}{mathb}{"6D}%%%%%% imported only two symbol of mathabx; here there are the slot of this package: https://mirror.las.iastate.edu/tex-archive/fonts/mathabx/texinputs/mathabx.dcl
% \DeclareMathSymbol{\boxbackslash}{2}{mathb}{"6E}

\section{Lifting problems \& Quillen's small object argument}\label{lecture:lifting-problems}

In essence, homotopy theory is the study of localizations of categories. For example, classically one was interested in the localization of some nice category of topological spaces
at the (weak) homotopy equivalences. Studying these kinds of structures takes many forms, and we have seen a few already in Lectures
\ref{section:localization-of-categories} \& \ref{lecture:homotopical-algebra-through-deformations}. However, modern homotopy theory is, largely speaking, fundamentally phrased in terms of
so-called \emph{lifting problems.} The crux is that much homotopy theory can be done in the setting of a \emph{model category,} which makes use of factorizations of morphisms into pairs
(a trivial fibration and a cofibration, or a fibration and a trivial cofibration) having some lifting property with respect to each other. Pairs of classes of maps where one can make such
factorizations are called \emph{factorization systems.}

The other topic of this lecture, the small object argument (due to Quillen), is a method for producing factorization systems out of some specified class of maps whose domains
are sufficiently compact objects. This is the origin of the name, as compact objects have historically also been referred to as ``small'' objects.

\subsection{Lifting problems}

This is our basic notion for study.
\begin{definition}
	Let \(\calC\) be a category. We say a morphism \(f\!:x\to y\) has the \emph{left lifting property} with respect to a morphism \(g\!:x'\to y'\),
	or that \(g\) has the \emph{right lifting property} with respect to \(f\), if for any solid diagram
	\begin{diagram*}
		x\ar[d,"f"']\ar[r] & x'\ar[d,"g"] \\
		y\ar[r]\ar[ur,dashed] & y'
	\end{diagram*}
	a dashed arrow exists.

	Consider some set of maps \(S\) in \(\calC\). We say \(f\) has the left lifting property with respect to \(S\) if it has the left lifting property with respect to all elements of \(S\).
	The case for the right lifting property is similar.
\end{definition}
\begin{remark}
	A diagram of the form
	\begin{diagram*}
		x\ar[d,"f"']\ar[r] & x'\ar[d,"g"] \\
		y\ar[r]\ar[ur,dashed] & y'
	\end{diagram*}
	describes a \emph{lifting problem.} A dashed arrow, if one exists, is called a \emph{solution} to the lifting problem. In this way, \(f\) has the left lifting property with
	respect to \(g\) if we can solve any lifting problem with \(f\) on the left and \(g\) on the right.
\end{remark}
\begin{example}
	Let \(\calA\) be an Abelian category. Recall that an object \(p\in\calA\) is \emph{projective} if \(\calA(p,-)\) is exact. This essentially reduces to asking that
	for any epimorphism \(x \sur y\) and morphism \(p\to y\), there is a lift factoring this through a morphism \(p\to x\). Observe that we can phrase this as a lifting problem:
	\begin{diagram*}
		0\ar[d]\ar[r] & x\ar[d,two heads] \\
		p\ar[r]\ar[ur,dashed] & y
	\end{diagram*}
	so that \(p\) is projective if and only if the unique map \(0 \to p\) has the left lifting property with respect to the set of epimorphisms in \(\calA\). Dually, an object \(q\in\calA\)
	is \emph{injective} if and only if the unique map \(0\to q\) has the right lifting property with respect to the set of monomorphisms.
\end{example}
\begin{remark}
	In the situation above, where one of the morphisms is trivial, it is common to omit it entirely. For example, one might write
	\begin{diagram*}
		 & x\ar[d,two heads] \\
		p\ar[r]\ar[ur,dashed] & y
	\end{diagram*}
	for the lifting problem associated to a projective.
\end{remark}
\begin{example}\label{example:surjections-in-terms-of-rlp}
	A map of sets \(f\!:A\to B\) is surjective if and only if it has the right lifting property with respect to \(\varnothing \inj *\). This is easy to see; indeed, the diagram is
	\begin{diagram*}
		\varnothing\ar[d]\ar[r] & A\ar[d,"f"] \\
		*\ar[r]\ar[ur,dashed] & B
	\end{diagram*}
	which tells us that for any element \(b\in B\), there is some element \(a\in A\) such that \(f(a)=b\), i.e.\ \(f\) is surjective.
\end{example}

So, many interesting maps and objects arise in terms of lifting problems admitting a solution. Let us fix some notation for these things, taken from \cite{riehl-categorical-homotopy-theory}.
\begin{notation}
	Let \(\calC\) be a category, and let \(f,g\) be two morphisms in \(\calC\). We write \(f\boxslash g\) to say that \(f\) has the left lifting property with respect to \(g\) (or, equivalently, that \(g\) has the
	right lifting property with respect to \(f\)).

	If \(S\) is a set of morphisms in \(\calC\), we write \(S^\boxslash\) for those morphisms which have the right lifting property with respect to \(S\), and dually, \(\prescript{\boxslash}{}{S}\)
	for those morphisms having the left lifting property with respect to \(S\). If \(T\) is another set of morphisms, we write \(S\boxslash T\) to say that \(S\subseteq \prescript{\boxslash}{}{T}\),
	or equivalently that \(T\subseteq S^\boxslash\).
\end{notation}
\begin{remark}
	To spell it out, writing \(S\boxslash T\) means that we can solve any lifting problem with an \(S\) on the left and \(T\) on the right.
\end{remark}
\begin{example}
	We see from Example \ref{example:surjections-in-terms-of-rlp} that
	\[ \{ \text{surjections in }\Set \} = \{\varnothing\inj *\}^\boxslash. \]
\end{example}

\begin{example}
	Any isomorphism has the left and right lifting property with respect to all maps. Indeed, you just compose with the inverse of the isomorphism to produce the desired lift.
\end{example}

\begin{proposition}
	Let \(\calC\) be a category, and let \(S\) and \(T\) be sets of morphisms in \(\calC\). Then the following statements hold.
	\begin{enumerate}[label=(\arabic*)]
		\item If \(S\subseteq T\), then \(\prescript{\boxslash}{}{T}\subseteq\prescript{\boxslash}{}{S}\).
		\item If \(S\subseteq T\), then \(T^\boxslash\subseteq S^\boxslash\).
		\item \(S \subseteq \prescript{\boxslash}{}{(S^\boxslash)}\).
		\item \(S^\boxslash = (\prescript{\boxslash}{}{(S^\boxslash)})^\boxslash\).
		\item \(\prescript{\boxslash}{}{S} = \prescript{\boxslash}{}{((\prescript{\boxslash}{}{S})^\boxslash)}\).
	\end{enumerate}
\end{proposition}
\begin{proof}
Statements (1), (2), and (3) are trivial. Statements (4) and (5) are dual, so we prove (3). First, note that one inclusion follows by (2) and (3), i.e.
\[ S \subseteq \prescript{\boxslash}{}{(S^\boxslash)} \implies (\prescript{\boxslash}{}{(S^\boxslash)})^\boxslash \subseteq S^\boxslash. \]
On the other hand, the other inclusion is also trivial: let \(g\in S^\boxslash\); what we have to check is that \(\prescript{\boxslash}{}{(S^\boxslash)} \boxslash g\),
but to test this, we use a lifting problem against a morphism \(f\) with the left lifting property with respect to \(S^\boxslash\).
\end{proof}

\subsection{Saturated classes}

Let \(S\) be a class of morphisms in some category \(\calC\). The set of morphisms \(\prescript{\boxslash}{}{S}\) has some special properties. We will now explain them.
One has already been mentioned: \(\prescript{\boxslash}{}{S}\) contains all isomorphisms, which is easy to see.

Most of the properties that \(\prescript{\boxslash}{}{S}\) will satisfy are fairly self-explanatory. One, however, is a bit trickier to make sense of. It is not too hard to see
that it is closed under composition. However, something much stronger is true: it is closed under \emph{transfinite composition.} For the purposes of explaining this,
we will need to make use of ordinals. Rest assured, we do not need to know much about them.

Recall that a totally ordered set is a \emph{well order} if, in addition, every non-empty subset has a least element. These can be regarded as categories in an obvious way, and an
ordinal is merely an isomorphism class of well ordered sets. In particular, we can view any ordinal \(\alpha\) as a category. Now, the class of ordinals happens to be well-ordered (intuitively,
by inclusion), meaning that for any two ordinals \(\alpha\) and \(\beta\) we can also talk about whether \(\beta < \alpha\). In general, one can identify an ordinal \(\alpha\) with
the set \(\{ \beta \mid \beta < \alpha \}\).

Taking an ordinal \(\alpha\) and freely adjoining a terminal element, one obtains the \emph{succesor} \(\alpha+1\). An ordinal \(\lambda\) is a \emph{limit} ordinal if it is non-zero and
not the successor of any other ordinal. Observe that this happens if and only if \(\lambda\) is the \emph{colimit} of all ordinals strictly lesser than \(\lambda\). Indeed, if \(\lambda = \beta+1\), then
the colimit would instead be given by \(\beta\).

\begin{definition}
	Let \(\calC\) be a category, let \(S\) be a set of morphisms in \(\calC\), and let \(\alpha\) be some ordinal. An \(\alpha\)-composable sequence of morphisms in \(S\) is a
	diagram \(x_\bullet\!:\alpha \to \calC\) with the following properties.
	\begin{enumerate}[label=(\arabic*)]
		\item For any ordinal \(\beta < \alpha\) with a succesor \(\beta + 1 < \alpha\), the corresponding morphism \(x_{\beta} \to x_{\beta+1}\) is in \(S\).
		\item \(x_\bullet\) is ``continuous''. That is, for any limit ordinal \(\lambda < \alpha\), the canonical morphism
		\[ x_\lambda \to \injlim_{\beta < \lambda}x_\beta \]
		is an isomorphism.
	\end{enumerate}
	The \emph{composition} of an \(\alpha\)-composable sequence of morphisms \(x_\bullet\) is the canonical map
	\[ x_0 \to x_\alpha := \injlim x_\bullet. \]
	Let \(\kappa\) be a cardinal. We say that \(S\) is closed under \(\kappa\)-transfinite compositions if the composition of any \(\alpha\)-composable sequence of morphisms
	in \(S\) is in \(S\) for all ordinals \(\alpha\) with cardinality strictly less than \(\kappa\).

	We say that \(S\) is \emph{closed under transfinite compositions} if it is closed under \(\kappa\)-transfinite compositions for all \(\kappa\).
\end{definition}

\begin{remark}
	The easiest non-trivial picture here is \(\alpha = \omega\). Then we just have a sequence
	\[ x_0 \to x_1 \to \cdots \to x_i \to \cdots \]
	and we require that each \(x_{i} \to x_{i+1}\) is in \(S\). If \(S\) is now closed under transfinite composition, it means that the map
	\[ x_0 \to x_\omega := \injlim_i x_i \]
	is in \(S\) too.
\end{remark}

\begin{proposition}\label{prop:llp-closure-properties}
	The set \(\prescript{\boxslash}{}{S}\) is closed under the following constructs.
	\begin{enumerate}[label=(\arabic*)]
		\item Retracts.
		\item Small coproducts, meaning that for any small indexing set \(I\) and collection of morphisms \(\{f_i\!:x_i\to y_i\}_{i\in I}\subseteq S\), the morphism
		\[ \coprod_{i\in I}f_i \!: \coprod_{i\in I}x_i \to \coprod_{i\in I}y_i \]
		is in \(S\) whenever it exists.
		\item Pushouts, meaning:
		\[
		\begin{tikzcd}
			x \ar[d,"S\ni"']\ar[r] & z \\
			y
		\end{tikzcd}\quad\leadsto\quad
		\begin{tikzcd}
			x\ar[d,"S\ni"'] \ar[r] & z \ar[d,dashed,"\in S"] \\
			y\ar[r,dashed] & y\amalg_xz\ar[ul,pushout]
		\end{tikzcd}.
		\]
		\item Transfinite compositions.
	\end{enumerate}
\end{proposition}
\begin{proof}
The proofs of all of these facts are very similar. We prove (4) by transfinite induction. Let \(\alpha\) be some ordinal, and consider an \(\alpha\)-composable sequence of morphisms
\(x_\bullet\!:\alpha\to\calC\) in \(\prescript{\boxslash}{}{S}\). The base case when \(\alpha=1\) is trivial, so assume we know the result for all \(\beta < \alpha\). If \(\alpha = \alpha'+1\) is a successor ordinal, then
if \(f\in S\), we are in the following situation
\[
	\begin{tikzcd}
		x_0\ar[d,"\prescript{\boxslash}{}{S}\ni"']\ar[r] & x\ar[dd,"f"] \\
		x_{\alpha'}\ar[d,"\prescript{\boxslash}{}{S}\ni"'] & \\
		x_{\alpha'+1}\ar[r]\ar[uur,dashed] & y
	\end{tikzcd}\quad = \quad
	\begin{tikzcd}
		x_0 \ar[d,"\prescript{\boxslash}{}{S}\ni"']\ar[r] & x\ar[d,"f"] \\
		x_{\alpha'}\ar[r]\ar[ur,dashed,"\exists"] & y
	\end{tikzcd}\text{ and }
	\begin{tikzcd}
		x_{\alpha'} \ar[d,"\prescript{\boxslash}{}{S}\ni"']\ar[r] & x\ar[d,"f"] \\
		x_{\alpha'+1}\ar[r]\ar[ur,dashed,"\exists"] & y
	\end{tikzcd}
\]
giving the desired solution. If \(\alpha\) is a limit ordinal, note that a map \(x_\alpha \to x\) is given by a compatible collection of maps \(x_\beta\to x\) for all \(\beta < \alpha\). In
particular, by the induction hypothesis, we then have
\[
	\forall \beta < \alpha,\quad
	\begin{tikzcd}
		x_0\ar[d,"\prescript{\boxslash}{}{S}\ni"']\ar[r] & x\ar[dd,"f"] \\
		x_\beta\ar[d]\ar[ur,dashed,"\exists"] & \\
		x_\alpha\ar[r] & y
	\end{tikzcd}\quad\leadsto\quad
	\begin{tikzcd}
		x_0\ar[d]\ar[r] & x\ar[d,"f"] \\
		x_\alpha\ar[r]\ar[ur,dashed,"\exists"] & y
	\end{tikzcd}
\]
so we have our lift here too. This completes the argument.
\end{proof}
\begin{exercise}
	Complete the above proof.
\end{exercise}

\begin{definition}
	Let \(\calC\) be a category and let \(\kappa\) be a cardinal. A set of morphisms \(T\) in \(\calC\) is said to be \emph{weakly} \(\kappa\)\emph{-saturated} if it satisfies the following
	properties.
	\begin{enumerate}[label=(\arabic*)]
		\item \(T\) is closed under retracts.
		\item \(T\) is closed under small coproducts.
		\item \(T\) is closed under pushouts.
		\item \(T\) is closed under \(\kappa\)-transfinite compositions.
		\item \(T\) contains all isomorphisms.
	\end{enumerate}
	We say \(T\) is \emph{weakly saturated} if it is weakly \(\kappa\)-saturated for all \(\kappa\). Dually, we say \(T\) is \emph{weakly} (\(\kappa\)-)\emph{cosaturated} if \(T^\op\)
	is weakly (\(\kappa\)-)saturated in \(\calC^\op\).
\end{definition}
\begin{corollary}\label{corollary:llp-weakly-saturated}
	Let \(S\) be a set of morphisms in \(\calC\). Then \(\prescript{\boxslash}{}{S}\) is weakly saturated and \(S^\boxslash\) is weakly cosaturated.
\end{corollary}
\begin{remark}
	In \cite[Remark 2.1.3]{cisinski-book}, it is pointed out that the above list has a redundancy. In fact, being closed under transfinite composition and pushouts implies being
	closed under small coproducts.
\end{remark}

\begin{notation}
	Let \(S\) be some collection of morphisms in \(\calC\). We write \(\overline{S}^{(\kappa)}\) for the \(\kappa\)\emph{-saturation} of \(S\), i.e.\ the smallest
	weakly \(\kappa\)-saturated collection of morphisms containing \(S\). Similarly, we write \(\overline{S}\) for the smallest weakly saturated collection of morphisms containing \(S\).
\end{notation}
\begin{remark}\label{remark:saturation-inclusions}
	Observe that since \(\prescript{\boxslash}{}{(S^\boxslash)}\) is saturated and contains \(S\), we have \(\overline{S} \subseteq \prescript{\boxslash}{}{(S^\boxslash)}\). We will
	see that in favourable situations, this inclusion is an equality. This is one of the consequences of the small object argument.
	
	Additionally, we have that for all cardinals \(\kappa < \kappa'\), \(\overline{S}^{\kappa} \subseteq \overline{S}^{\kappa'} \subseteq \overline{S}\).
\end{remark}

\subsection{Factorization systems}
We begin with the following motivating example.
\begin{example}
	We saw that the set of surjections in \(\Set\) were characterized by having the right lifting property with respect to \(\varnothing\inj*\). One may ask: what are
	\[  \{\text{surjections}\}^\boxslash \quad\text{and}\quad\prescript{\boxslash}{}{\{\text{surjections}\}}? \]
	Curiously, the answer is the same in both cases, namely you recover the collection of injective maps. Proving this makes use of the axiom of choice, and we will not discuss it.
	Nonetheless, it is interesting to us because the following much more trivial fact: any map of sets can be decomposed into an injective map followed by a surjective map.
\end{example}
\begin{definition}\label{definition:factorization-system}
	Let \(\calC\) be a category. A \emph{factorization system} on \(\calC\) is a pair \((S,T)\) of collections of maps satisfying the following properties.
	\begin{enumerate}[label=(\alph*)]
		\item The collections \(S\) and \(T\) are closed under retracts.
		\item For all \(f\!:x\to y\) in \(\calC\), there is a factorization \(f = p\circ i\) with \(i\in S\) and \(p\in T\).
		\item \(S\boxslash T\).
	\end{enumerate}
\end{definition}

\begin{lemma}[``The retract lemma'']\label{lemma:retract-lemma}
	Suppose we have a factorization
	\[
	\begin{tikzcd}[cramped, column sep=small]
		x\ar[rr,"f"]\ar[dr,"i"'] & & y \\
		& z\ar[ur,"p"'] &
	\end{tikzcd}.
	\]
	Then the following statements hold.
	\begin{enumerate}[label=(\arabic*)]
		\item If \(i\boxslash f\) then \(f\) is a retract of \(p\).
		\item If \(f\boxslash p\), then \(f\) is a retract of \(i\).
	\end{enumerate}
\end{lemma}
\begin{proof}
Statements (1) and (2) are dual, so we prove (1). The lifting problem below left
\[
	\begin{tikzcd}
		x\ar[r,equal]\ar[d,"i"'] & x \ar[d,"f"] \\
		z\ar[r,"p"]\ar[ur,dashed,"\exists"] & y
	\end{tikzcd}\quad\leadsto\quad
	\begin{tikzcd}
		x\ar[d,"f"]\ar[r,"i"] & z\ar[d,"p"]\ar[r,dashed,""] & x \ar[d,"f"] \\
		y \ar[r,equal] & y \ar[r,equal] & y
	\end{tikzcd}
\]
gives rise to the retraction as above right, as desired.
\end{proof}
\begin{proposition}\label{prop:factorization-system-equalities}
	Suppose we have a pair \((S,T)\) of collections of morphisms satisfying property (b) in Definition \ref{definition:factorization-system}. Then the following are equivalent.
	\begin{enumerate}[label=(\arabic*)]
		\item \((S,T)\) is a factorization system.
		\item \(S^\boxslash = T\) and \(S = \prescript{\boxslash}{}{T}\).
	\end{enumerate}
\end{proposition}
\begin{proof}
It is clear that (2) implies (1) (note Corollary \ref{corollary:llp-weakly-saturated}), so what remains is the other direction. We are given that \(S^\boxslash \supseteq T\) and \(S\subseteq \prescript{\boxslash}{}{T}\)
(which are equivalent). To get the other inclusions, suppose \(f\in S^\boxslash\). We may then write \(f = p\circ i\) for \(i\in S\) and \(p\in T\), but by Lemma \ref{lemma:retract-lemma} and the
fact that \(i\boxslash f\), we see that \(f\) is a retract of \(p\). Since \(T\) is closed under retracts, \(f \in T\). The other case is dual.
\end{proof}

\subsection{Quillen's small object argument}
A priori, it is very hard to produce factorization systems. In particular, while it is not at all hard to find nice pairs of collections of morphisms with some
lifting properties with respect to each other, manufacturing factorizations into these morphisms is tricky. Quillen provided a tool of great ingenuity for tackling this problem,
namely the \emph{small object argument.} There are a number of versions of it; we choose one which is fairly intelligible.

\begin{proposition}[Small object argument]\label{prop:small-object-argument}
	Let \(\calC\) be a locally small category admitting all small colimits, let \(\kappa\) be an infinite regular cardinal, and let \(M\) be some collection of morphisms
	in \(\calC\) with \(\kappa\)-compact domains. Then \((\prescript{\boxslash}{}{(M^\boxslash)}, M^\boxslash)\) is a factorization system.
\end{proposition}
\begin{proof}
By construction, conditions (a) and (b) of Definition \ref{definition:factorization-system} are already satisfied. In other words, the crux of this proposition is the existence,
for all \(f\!:x\to y\) in \(\calC\), of a factorization \(f = h\circ g\) where \(g\in \prescript{\boxslash}{}{(M^\boxslash)}\) and \(h\in M^\boxslash\). In fact, we will prove something
slightly stronger: we can choose \(g\in \overline{M}^\kappa\), which implies the former by Remark \ref{remark:saturation-inclusions}.

The strategy of the proof is to approximate \(f\) by a morphism in \(M\). To this end, for any map \(a\!: u\to v\), define the set
\[ I_M(a) := \left\{\begin{tikzcd}
	z\ar[d,"s"']\ar[r]  & u\ar[d,"a"] \\
	w\ar[r] & v
\end{tikzcd} : s \in M \right\}, \]
which consists of all morphisms \(s\To a\) in \(\Ar(\calC) \cong \Fun([1], \calC)\). We observe that this set is small since \(M\) is small and \(\calC\) is locally small.

For notational simplicity, given a commutative square \(i \in I_M(a)\), we write \(z_i\), \(w_i\), and \(s_i\) for the corresponding data as above. Note that we leave the horizontal maps
unnamed since we will not need them by name. Now, using \(I_M(a)\) we can associate to \(a\) an object \(u_M\) and a pair of maps \(g(a)\!:u\to u_M\) and \(h(a)\!:u_M\to v\) such that
\[ a = h(a)\circ g(a),\quad \text{and}\quad g(a)\in\overline{M}^\kappa. \]
In other words, a factorization almost like what we want. The procedure is the following: taking the coproduct over all \(i\in I_M(a)\), we get a diagram
\begin{diagram*}
	\coprod_{i\in I_M(a)} z_i\ar[r]\ar[d,"\coprod_i s_i"'] & u\ar[d,"a"] \\
	\coprod_{i\in I_M(a)} w_i\ar[r] & v
\end{diagram*}
by composing horizontally with the codiagonal. Taking the pushout of the morphism on the left, we get
\begin{diagram*}[row sep=large]
	\coprod_{i\in I_M(a)} z_i\ar[r]\ar[d,"\coprod_i s_i"'] & u\ar[ddr,bend left,"a"]\ar[d,dashed,"g(a)"] & \\
	\coprod_{i\in I_M(a)} w_i\ar[r,dashed]\ar[drr,bend right] & u_M \ar[dr,"h(a)"']\ar[ul,pushout] & \\
	& & v
\end{diagram*}
which defines our maps. We have that \(g(a)\in\overline{M}^\kappa\) since the latter is stable under small coproducts and pushouts.

If we apply the above to \(f\), we get a factorization \(f = h(f)\circ g(f)\) where \(g(f) \in \overline{M}^\kappa\). However, we do not know that \(h(f)\in M^\boxslash\),
so our approximation isn't good enough. To remedy this, we will approximate again and again, inductively, and in fact do it \(\kappa\) times.

Set \(x_0 = x\), \(h_0 = f\). We obtain \(g_{0,1} := g(h_0)\!:x_0 \to (x_0)_M =: x_1\) and \(h_1 := h(h_0)\). We continue inductively as follows: for any successor ordinal
\(\alpha+1\), supposing we have defined \(x_\alpha\), \(h_\alpha\!:x_{\alpha}\to y\), and \(g_{\beta,\alpha}\!:x_{\beta}\to x_{\alpha} \) for all \(\beta \leq \alpha\), we define
\[ x_{\alpha+1} = (x_\alpha)_M,\quad g_{\beta,\alpha+1} := g(h_\alpha)\circ g_{\beta,\alpha},\quad h_{\alpha+1} = h(h_\alpha). \]
We remark that \(g_{\alpha,\alpha} = \id_{x_\alpha}\), and therefore \(g_{\alpha,\alpha+1} = g(h_\alpha)\). For a limit ordinal \(\lambda \leq \kappa\), we define \(x_\lambda\) and \(h_\lambda\) by
\[ x_0 \overset{h_\lambda}\longto \injlim_{\beta < \lambda}x_\beta =: x_\lambda. \]
For any \(\beta < \lambda\) we obtain the canonical morphism \(g_{\beta,\lambda}\!:x_\beta \to x_\lambda\).

We get an object \(x_\kappa\) and a pair of maps
\[ x \overset{g_{0,\kappa}}\longto x_\kappa \overset{h_\kappa}\longto y, \]
so we set \(g = g_{0,\kappa }\) and \(h = h_\kappa\). These satisfy \(f = h\circ g\), and \(g\in\overline{M}^\kappa\) since the latter is closed under \(\kappa\)-transfinite compositions.
It remains to check that \(h\in M^\boxslash\).

Consider a lifting problem
\[
	\begin{tikzcd}
		z\ar[d,"M \ni s"']\ar[r] & x_\kappa\ar[d,"h"] \\
		w\ar[r]\ar[ur,dashed,"\exists?"] & y
	\end{tikzcd}
\]
to be solved. By definition, \(x_\kappa = \injlim_{\beta < \kappa}x_\beta\), which is a \(\kappa\)-filtered colimit. Hence, the \(\kappa\)-compactness of \(z\) means there is some
\(\alpha < \kappa\) and a factorization
\[
	\begin{tikzcd}
		z\ar[dd,"s"']\ar[rr]\ar[dr,dashed] & & x_\kappa\ar[dd,"h"] \\
		& x_\alpha\ar[dr,"h_\alpha"']\ar[ur,"g_{\alpha,\kappa}"' near start] & \\
		w\ar[rr] & & y
	\end{tikzcd}.
\]
Now comes the genius trick: the square defined by the bottom left morphisms
\[
	\begin{tikzcd}
		z\ar[d,"s"']\ar[r] & x_\alpha \ar[d,"h_\alpha"] \\
		w \ar[r] & y
	\end{tikzcd}
\]
is exactly a square used in defining \(x_{\alpha+1} = (x_{\alpha})_m\). This means precisely that we can factorize further into
\[
	\begin{tikzcd}[column sep=large]
		z\ar[ddd,"s"']\ar[rr]\ar[dr] & & x_\kappa\ar[ddd,"h"] \\
		& x_\alpha\ar[d,"g_{\alpha+1}"']\ar[ur,"g_{\alpha,\kappa}" near start] & \\
		& x_{\alpha+1}\ar[dr,"h_{\alpha+1}" near start]\ar[uur,"g_{\alpha+1,\kappa}"'] & \\
		w\ar[rr]\ar[ur] & & y
	\end{tikzcd}
\]
where we thus obtain our solution to the lifting problem.
\end{proof}
\begin{corollary}
	Let \(\calC\) be a locally small category admitting all small colimits, and suppose \(M\) is a collection of maps whose domains are all \(\kappa\)-compact for some fixed cardinal \(\kappa\).
	Then \(\overline{M} = \prescript{\boxslash}{}{(M^\boxslash)}\).
\end{corollary}
\begin{proof}
The proof of Proposition \ref{prop:small-object-argument} shows that \((\overline{M},M^\boxslash)\) is a factorization system. However, by Proposition \ref{prop:factorization-system-equalities},
one then has
\[ \overline{M} = \prescript{\boxslash}{}{(M^\boxslash)} \]
as desired.
\end{proof}
\begin{remark}
	In \cite[§12.2]{riehl-categorical-homotopy-theory}, there is a brief discussion of various kinds of assumptions one can take to get a small object argument. In the above, we've chosen
	a somewhat straightforward one, modulo the details imposed by allowing any cardinal \(\kappa\). One can weaken the assumptions drastically. For example, we do not need the domain
	of a morphism \(s\!:z\to w\) in \(M\) to be entirely \(\kappa\)-compact, as we only use that \(\calC(z,-)\) commutes with \(\kappa\)-filtered colimits of morphisms in \(\overline{M}^\kappa\).
	Furthermore, we only use one particular kind of \(\kappa\)-filtered colimit, namely a transfinite composition of \(\kappa\)-many arrows.
\end{remark}


\subsection{Appendix: Transposing lifting problems along adjunctions}
There are a few ways that lifting problems can be moved across adjunctions, and it is very useful to be aware of them, as a number of techniques in categorical homotopy
theory (for example, of simplicial sets) come down to convering a difficult lifting problem to an easy one through some adjunction you have available to you.

The following is the easiest possible case. It is taken from \cite[Lemma 11.1.5]{riehl-categorical-homotopy-theory}, though the overall argument is
of course standard.
\begin{proposition}\label{prop:adjunction-lifting-correspondence}
	Let \(
	\begin{tikzcd}[cramped]
		\calC\ar[r,bend left,"F",""{name=A,below}] & \calD \ar[l,bend left,"G",""{name=B,above}]\ar[from=A,to=B,symbol=\dashv]
	\end{tikzcd}
	\) be an adjunction, and suppose we have collections of morphisms \(S\subseteq\Mor(\calC)\) and \(T\subseteq\Mor(\calD)\). Then
	\( F(S) \boxslash T\) if and only if \( S \boxslash G(T) \).
\end{proposition}
\begin{proof}
Specializing the below Lemma \ref{lemma:functor-category-adjunction} to the case \(\calE=[1]\), we see that there is an induced adjunction
\[
	\begin{tikzcd}[cramped]
		\Fun([1],\calC)\ar[r,bend left,"F_*",""{name=A,below}] & \Fun([1],\calD) \ar[l,bend left,"G_*",""{name=B,above}]\ar[from=A,to=B,symbol=\dashv]
	\end{tikzcd}
\]
on the arrow categories, where we note that \(\Ar(-) \cong \Fun([1],-)\). The setup for a lifting problem is precisely a morphism in an arrow category,
so we get a bijection between a squares \(Fi \To f\) in \(\calD\) as below left, and solid squares \(i \To Gf\) in \(\calC\) as below right:
\[
	\begin{tikzcd}
		Fa\ar[d,"Fi"']\ar[r] & x\ar[d,"f"] \\
		Fb\ar[r]\ar[ur,dashed,"h"] & y
	\end{tikzcd} \quad \leftrightsquigarrow \quad
	\begin{tikzcd}
		a\ar[d,"i"']\ar[r] & Gx\ar[d,"Gf"] \\
		b\ar[r]\ar[ur,dashed,"h'"] & Gy
	\end{tikzcd}.
\]
Moreover, a solution \(h\) as above left is exactly the same as a morphism in \(\Fun([2],\calD)\) as below left, which by the same argument means we have a
correspondence with a solution \(h'\) as below right
\[
	\begin{tikzcd}
		Fa\ar[d,"Fi"']\ar[r] & x\ar[d,equal] \\
		Fb\ar[r,"h"]\ar[d,equal] & x\ar[d,"f"] \\
		Fb\ar[r] & y
	\end{tikzcd} \quad \leftrightsquigarrow \quad
	\begin{tikzcd}
		a\ar[d,"i"']\ar[r] & Gx\ar[d,equal] \\
		b\ar[r,"h'"]\ar[d, equal] & Gx\ar[d,"Gf"] \\
		b\ar[r] & Gy
	\end{tikzcd}
\]
which completes the argument.
\end{proof}
\begin{lemma}\label{lemma:functor-category-adjunction}
	Let \(
	\begin{tikzcd}[cramped]
		\calC\ar[r,bend left,"F",""{name=A,below}] & \calD \ar[l,bend left,"G",""{name=B,above}]\ar[from=A,to=B,symbol=\dashv]
	\end{tikzcd}
	\) be an adjunction. Then, for any category \(\calE\), there is an induced adjunction
	\[
		\begin{tikzcd}[cramped]
			\Fun(\calE,\calC)\ar[r,bend left,"F_*",""{name=A,below}] & \Fun(\calE,\calD) \ar[l,bend left,"G_*",""{name=B,above}]\ar[from=A,to=B,symbol=\dashv]
		\end{tikzcd}
	\]
\end{lemma}
\begin{proof}
The functors \(F_*\) and \(G_*\) are defined by postcomposition, so e.g.\ \(F_*\!:H\mapsto F H\), and \((\sigma\!:H\To H') \mapsto (F\sigma\!: FH\To FH')\).
Let \(\eta\) and \(\varepsilon\) be the unit and counit of the adjunction \(F\ladj G\). We then obtain natural transformations
\[ \eta'\!: \1 \To G_*F_*,\quad \varepsilon'\!: F_*G_* \To \1 \]
given by
\[ \eta'_H := \eta H,\quad \varepsilon'_K := \varepsilon K. \]
These satisfy the triangle identities. Indeed, it suffices to check at each component, at which point one is checking the commutativity of the diagrams
\[
	\begin{tikzcd}[cramped, column sep=small]
		 & FGFH\ar[dr,Rightarrow,"\varepsilon FH"] &\\
		FH \ar[ur,Rightarrow,"F\eta H"]\ar[rr,equal] & & FH
	\end{tikzcd}\quad\text{and}\quad
	\begin{tikzcd}[cramped, column sep=small]
		 & GFGK \ar[dr,Rightarrow,"G \varepsilon K"] &\\
		GK \ar[ur,Rightarrow,"\eta GK"]\ar[rr,equal] & & FK
	\end{tikzcd}
\]
which follows from the original triangle identities.
\end{proof}

The above is pleasingly simple, but unfortunately also fairly uncommon in practice. Usually, one is in a somewhat more complex situation, involving a \emph{two-variable adjunction}
instead. Those are perhaps \emph{surprisingly} common, so we will take the time to explain them in detail, following \cite{riehl-categorical-homotopy-theory}. First, we need the notion of what a two-variable
adjunction is to begin with.
\begin{definition}
	A \emph{two-variable adjunction} consists of functors
	\[ -\otimes-\!: \calC\times\calD\to\calE,\quad \{-,-\}\!:\calC^\op\times\calE\to\calD,\quad [-,-]\!:\calD^\op\times\calE\to\calC, \]
	equipped with natural isomorphisms
	\[ \calE(c\otimes d, e) \cong \calD(d, \{c,e\}) \cong \calC(c,[d,e]) \]
	of functors \(\calC^\op\times\calD^\op\times\calE\to\Set\).
\end{definition}
\begin{remark}
	Here, we should think of the \(\{-,-\}\) and \([-,-]\) functors as ``internal Hom's'' of a sort. Indeed, typical examples of two-variable adjunctions come
	exactly from such situations, in which case \(\{-,-\} = [-,-]\). The easiest example is then when \(\otimes = \times\) and
	\[ \{-,-\} = [-,-] = \Set(-,-). \]
\end{remark}

Before we do anything serious with these, let's prove a small but useful lemma, which will be extremely helpful to us very soon.

\begin{lemma}\label{lemma:two-variable-adjunction-transposition-formula}
	Consider a two-variable adjunction
	\[ -\otimes-\!: \calC\times\calD\to\calE,\quad \{-,-\}\!:\calC^\op\times\calE\to\calD,\quad [-,-]\!:\calD^\op\times\calE\to\calC, \]
	equipped with natural isomorphisms having components
	\[ \alpha_{cde}\!: \calE(c\otimes d, e) \cong \calD(d, \{c,e\}),\quad \beta_{cde}\!: \calE(c\otimes d, e) \cong \calC(c,[d,e]). \]
	Suppose we have morphisms
	\[ i\!:c\to c', \quad j\!:d\to d',\quad h\!:c'\otimes d \to e,\quad h'\!: c\otimes d'\to e. \]
	Then
	\begin{align*}
		\alpha( c\otimes d \overset{i\otimes d}\longto c'\otimes d \overset{h}\longto e ) &= d \overset{\alpha(h)}\longto \{c',e\} \overset{ \{i,e\} }\longto \{c,e\},\text{ and} \\
		\beta( c\otimes d \overset{c\otimes j}\longto c\otimes d' \overset{h'}\longto e ) &= c \overset{\beta(h')}\longto [d',e] \overset{[j,e]}\longto [d,e].
	\end{align*}
\end{lemma}
\begin{proof}
We will prove the first statement, as the second is basically dual. Naturality of \(\alpha\) means we have the commutative diagram
\[
	\begin{tikzcd}
		\calE(c'\otimes d,e)\ar[d,"{(i\otimes d)^*}"'] \ar[r,"\sim","\alpha_{c'de}"'] & \calD(d,\{c',e\})\ar[d,"{\{i,e\}_*}"] \\
		\calE(c\otimes d,e)\ar[r,"\sim","\alpha_{cde}"'] & \calD(d,\{c,e\})
	\end{tikzcd}
\]
which says exactly that
\[ \alpha_{cde}(h\circ (i\otimes d)) = \{i,e\}\circ\alpha_{c'de}(h) \]
as desired.
\end{proof}

\begin{construction}[The ``Leibniz construction'']\label{construction:two-variable-adjunction-leibniz-construction}
	Suppose we have a two-variable adjunction
	\[ -\otimes-\!: \calC\times\calD\to\calE,\quad \{-,-\}\!:\calC^\op\times\calE\to\calD,\quad [-,-]\!:\calD^\op\times\calE\to\calC, \]
	and that the following properties hold.
	\begin{enumerate}[label=(\roman*)]
		\item \(\calC\) and \(\calD\) have pullbacks.
		\item \(\calE\) has pushouts.
	\end{enumerate}
	For simplicity, let us write \(\calB^\calA := \Fun(\calA,\calB)\). Then there is an induced two-variable adjunction
	\[ -\hat\otimes-\!: \calC^{[1]}\times\calD^{[1]}\to\calE^{[1]},\quad \hat{\{-,-\}}\!:(\calC^{[1]})^\op\times\calE^{[1]}\to\calD^{[1]},\quad \hat{[-,-]}\!:(\calD^{[1]})^\op\times\calE^{[1]}\to\calC^{[1]} \]
	on the level of arrow categories.

	Here's how they're defined on objects: the ``left adjoint'' \(-\hat\otimes-\) is given, on \(i\!:c\to c'\) in \(\calC\) and \(j\!:d\to d'\) in \(\calD\), by
	\[
		\begin{tikzcd}
			c\otimes d \ar[r, "i\otimes d"]\ar[d,"c\otimes j"'] & c'\otimes d \ar[d]\ar[ddr,bend left,"c'\otimes j"] & \\
			c\otimes d'\ar[r]\ar[drr,bend right=16,"i\otimes d'"'] & (c\otimes d') \underset{c\otimes d}\amalg(c'\otimes d)\ar[ul,pushout]\ar[dr,dashed,"i\hat\otimes j" description] & \\
			& & c'\otimes d'
		\end{tikzcd}
	\]
	which one easily verifies is functorial. The other functors are similarly defined, except using pullbacks instead. In particular, given \(f\!:e\to e'\), we have
	\[
		\begin{tikzcd}[column sep=small]
			\{c',e\}\ar[dr,dashed,"\hat{ \{ i,f \} }" description]\ar[drr,bend left=15,"{ \{i,e\} }"]\ar[ddr,bend right,"{ \{c',f\} }"'] & & \\
			         & \{c,e\}\underset{\{c,e'\}}\times \{c,e\} \ar[r]\ar[d] \ar[dr,pullback] & \{c,e\}\ar[d,"\{c{,}f\}"] \\
			         & \{c',e'\}\ar[r,"\{i{,}e'\}"'] & \{c,e'\}
		\end{tikzcd}\quad
		\begin{tikzcd}[column sep=small]
			[d',e]\ar[dr,dashed,"\hat{ [ j,f ] }" description]\ar[drr,bend left=15,"{ [j,e] }"]\ar[ddr,bend right,"{ [d',f] }"'] & & \\
			         & {[d,e]} \underset{[d,e']}\times [d,e] \ar[r]\ar[d] \ar[dr,pullback] & {[d,e]}\ar[d,"{[d,f]}"] \\
			         & {[d',e']}\ar[r,"{[j,e']}"'] & {[d,e']}
		\end{tikzcd}
	\]
	which one also easily verifies are functorial.

	We must now verify that this is actually a two-variable adjunction, meaning we want to show that
	\[ \calE^{[1]}(i\hat\otimes j, f) \cong \calD^{[1]}(j, \hat{ \{i,f\} }) \cong \calC^{[1]}(i, \hat{ [j,f] }). \]
	We will neglect checking naturality carefully, as it is not particularly insightful. Suppose we are given a morphism \((a,b)\!:i\hat\otimes j \To f\) as displayed below left;
	then, by Lemma \ref{lemma:two-variable-adjunction-transposition-formula}, we have natural one-to-one correspondences
	\begin{align*}
		\begin{tikzcd}[ampersand replacement=\&]
			(c\otimes d') \underset{c\otimes d}\amalg(c'\otimes d) \ar[d,"i\hat\otimes j"']\ar[r,"a"] \& e\ar[d,"f"] \\
			c'\otimes d'\ar[r,"b"] \& e'
		\end{tikzcd} \quad &\leftrightsquigarrow \quad
		\begin{tikzcd}[ampersand replacement=\&]
			c\otimes d\ar[r]\ar[d] \& c'\otimes d \ar[d]\ar[r,"a'"] \& e\ar[d,"f"] \\
			c\otimes d'\ar[r] \& c'\otimes d'\ar[r,"b"] \& e'
		\end{tikzcd} \\
		&\leftrightsquigarrow \quad
		\begin{tikzcd}[ampersand replacement=\&]
			d\ar[r,"\overline{a'}"]\ar[d,"j"'] \& \{c',e\} \ar[d]\ar[r] \& \{c,e\} \ar[d,"f"] \\
			d'\ar[r,"\overline{b}"] \& \{c',e'\}\ar[r] \& \{c,e'\}
		\end{tikzcd} \\
		&\leftrightsquigarrow \quad
		\begin{tikzcd}[ampersand replacement=\&]
			d\ar[r,"\overline{a'}"]\ar[d,"j"'] \& \{c',e\} \ar[d,"\hat{ \{i,f\} }"] \\
			d'\ar[r,"\overline{b}'"] \& \{c,e\}\underset{\{c,e'\}}\times \{c,e\}
		\end{tikzcd}
	\end{align*}
	The crux of the calculation for \(\hat{[-,-]}\) is the same.
\end{construction}

\begin{proposition}\label{prop:two-variable-adjunction-lifting-correspondence}
	Consider a two-variable adjunction
	\[ -\otimes-\!: \calC\times\calD\to\calE,\quad \{-,-\}\!:\calC^\op\times\calE\to\calD,\quad [-,-]\!:\calD^\op\times\calE\to\calC, \]
	where \(\calC\) and \(\calD\) admit pullbacks, and \(\calE\) admits pushouts. Suppose we have collections of morphisms \(C\subseteq\Mor(\calC)\), \(D\subseteq\Mor(\calD)\), and
	\(E\subseteq\Mor(\calE)\). Then
	\[ C\hat\otimes D \boxslash E \iff D\boxslash \hat{\{C,E\}} \iff C \boxslash \hat{[D,E]}. \]
\end{proposition}
\begin{proof}
A lifting problem \(i\hat\otimes j \To f\) transposes to \(j \To \hat{\{i,f\}}\) and \(i \To \hat{ [j,f] }\), and one easily sees that solutions also
transport across these.
\end{proof}
\begin{remark}
	We were inexplicit in the above proof to demonstrate how little is really going on. However, it is still insightful to show the inner workings of the calculation,
	so we will be explicit, instead, in this remark. Take the calculation from Construction \ref{construction:two-variable-adjunction-leibniz-construction};
	filling in a diagonal arrow for the desired lifts, we see that all that's happening is
	\begin{align*}
		\begin{tikzcd}[ampersand replacement=\&]
			(c\otimes d') \underset{c\otimes d}\amalg(c'\otimes d) \ar[d,"i\hat\otimes j"']\ar[r,"a"] \& e\ar[d,"f"] \\
			c'\otimes d'\ar[r,"b"]\ar[ur,dashed] \& e'
		\end{tikzcd} \quad &\leftrightsquigarrow \quad
		\begin{tikzcd}[ampersand replacement=\&]
			c\otimes d\ar[r]\ar[d] \& c'\otimes d \ar[d]\ar[r,"a'"] \& e\ar[d,"f"] \\
			c\otimes d'\ar[r] \& c'\otimes d'\ar[r,"b"] \ar[ur,dashed] \& e'
		\end{tikzcd} \\
		&\leftrightsquigarrow \quad
		\begin{tikzcd}[ampersand replacement=\&]
			d\ar[r,"\overline{a'}"]\ar[d,"j"'] \& \{c',e\} \ar[d]\ar[r] \& \{c,e\} \ar[d,"f"] \\
			d'\ar[r,"\overline{b}"']\ar[ur,dashed] \& \{c',e'\}\ar[r] \& \{c,e'\}
		\end{tikzcd} \\
		&\leftrightsquigarrow \quad
		\begin{tikzcd}[ampersand replacement=\&]
			d\ar[r,"\overline{a'}"]\ar[d,"j"'] \& \{c',e\} \ar[d,"\hat{ \{i,f\} }"] \\
			d'\ar[r,"\overline{b}'"']\ar[ur,dashed] \& \{c,e\}\underset{\{c,e'\}}\times \{c,e\}
		\end{tikzcd}
	\end{align*}
	where the intermediate step is entirely about ordinary ``one-variable'' adjunctions, and follows purely by Proposition \ref{prop:adjunction-lifting-correspondence}.
\end{remark}

There are two main examples of applications that will eventually come up for us, both of which are in the homotopy theory of simplicial sets.
In particular, in the study of \(\infty\)-categories. The most easy to understand out of them is the statement that for an \(\infty\)-category \(\calC\)
and a simplicial set \(K\), the simplicial set of ``functors'' \(\Fun(K,\calC)\) is an \(\infty\)-category. This turns out to be a statement one should deduce from a more general fact,
namely that the canonical map
\[ \Fun(L,X) \to \Fun(K,X)\times_{\Fun(L, Y)}(K, Y), \]
is a \emph{Joyal fibration,} where \(K\to L\) is a monomorphism of simplicial sets and \(X\to Y\) is a Joyal fibration.
