%!TEX root = ../lectures.tex

\section{Model categories \& homotopy (co)limits (WIP)}\label{lecture:model-categories-and-homotopy-colimits}
In Lecture \ref{lecture:model-categories}, we laid out the basic theory of model categories, and indicated that their structure can be used to understand localizations
in a very convenient way. As part of the road to formalizing this, we showed that weak equivalences between \emph{bifibrant objects} are exactly given by
\emph{homotopy equivalences,} defined in terms of mapping cylinders (or path spaces, equivalently).

The purpose of this lecture is to continue down the path started in the last lecture. We will show that the localization \(\ho(\calC) = \calC[W^{-1}]\) of
a model category \(\calC\) at its weak equivalences \(W\) is equivalent to the homotopy category of bifibrant objects \(\hh{\calC_{cf}}\), whose objects
are the bifibrant objects, and whose morphisms are given by homotopy-equivalence classes of maps in \(\calC_{cf}\).

After doing the above, we will move to explaining some features of derived categories in the setting of model categories, and one particularly interesting example,
namely homotopy (co)limits. In Lectures \ref{lecture:triangulated-categories} through to \ref{lecture:gluing-t-structures}, we explored topics in triangulated categories,
and frequently made use of the intuition of \emph{cones} as ``homotopy cokernels.'' Homotopy colimits in model categories provide a refinement of this notion.

We will use the same notation as in Lecture \ref{lecture:model-categories}. The sources for the contents are primarily the same also, namely
\cite{cisinski-book,riehl-categorical-homotopy-theory,may-ponto-more-concise-algebraic-topology}.

\subsection{The homotopy category of a model category (WIP)}

We start with a construction which is integral to our interests of localization, and which are a big part of the reason model categories are useful.

\begin{construction}[(Co)fibrant replacement]\label{construction:model-category-(co)fibrant-replacement}
	Consider a model category \(\calC\), and some object \(x\in\calC\). There are two operations afforded to us for free by the factorization systems on \(\calC\),
	namely:
	\begin{itemize}
		\item We can produce a \emph{fibrant replacement} of \(x\). That is, a trivial cofibration \(x \iso Rx\) where \(Rx\) is fibrant. To do this,
		factor \(x \to *\) into a trivial cofibration followed by a fibration. Note that if \(x\) is cofibrant, then \(Rx\) is cofibrant.
		\item We can produce a \emph{cofibrant replacement} of \(x\). That is, a trivial fibration \(Qx \iso x\), where \(Qx\) is cofibrant. To do this,
		factor \(\varnothing\to x\) into a cofibration followed by a trivial fibration. Note that if \(x\) is fibrant, then \(Qx\) is fibrant.
	\end{itemize}
	When \(\calC\) has functorial factorizations, the above operations organize into deformations \(Q\To\1\) and \(\1\To R\). As we do not assume we have functorial factorizations,
	we will not make use of this fact. On the other hand, one can still get some amount of naturality for free.

	Suppose we have a morphism \(f\!:x\to x'\), and that we fix cofibrant replacements
	\[ q\!:Qx\iso x,\quad q'\!:Qx'\iso x'\]
	and fibrant replacements
	\[ r\!:x\to Rx,\quad r'\!:x'\to Rx'. \]
	By solving the below lifting problems
	\[
	\begin{tikzcd}
		\varnothing\ar[d,tail]\ar[rr] & & Qx'\ar[d,two heads,"q'"',"\sim" labl] \\
		Qx\ar[r,"q"']\ar[urr,dashed,"Qf"] & x\ar[r,"f"'] & x'
	\end{tikzcd}
	\quad
	\begin{tikzcd}
		x\ar[d,tail,"\sim" labl,"r"']\ar[r,"f"] & x'\ar[r,"r'"] & Rx'\ar[d,two heads] \\
		Rx\ar[rr]\ar[urr,dashed,"Rf"] & & *
	\end{tikzcd}
	\]
	we obtain morphisms \(Qf\!:Qx\to Qx'\) and \(Rf\!:Rx\to Rx'\). Observe that these satisfy \(q'\circ Qf = f\circ q\) and \(Rf \circ r = r'\circ f\).
\end{construction}
\begin{proposition}\label{prop:model-category-replacement-properties}
	Let \(\calC\) be a model category, and let \(f\!:x\to x'\) be a morphism in \(\calC\). Consider the morphism \(Qf\!:Qx\to Qx'\) of
	Construction \ref{construction:model-category-(co)fibrant-replacement}. Then the following statements hold.
	\begin{enumerate}[label=(\arabic*)]
		\item The morphism \(f\) is a weak equivalence if and only if \(Qf\) is a weak equivalence.
		\item The left and right homotopy classes of \(Qf\) depends only on the left homotopy class of \(f\circ q\).
		\item If \(x'\) is fibrant, then the right homotopy class of \(Qf\) depends only on the right homotopy class of \(f\).
	\end{enumerate}
\end{proposition}
\begin{proof}
(1) Apply the 2-out-of-3 property to the defining diagram of \(Qf\).

(2) By Theorem \ref{thm:model-category-dual-whitehead}, we have a bijection
\[ q'_*\!:[Qx,Qx']_\ell \iso [Qx,x']_\ell. \]
Furthermore, by Proposition \ref{prop:model-category-left-right-homotopy-interchange}, anything in the left homotopy class of \(Qf\) is also in the right homotopy class of \(Qf\).

(3) Note that \(Qx'\) is bifibrant. Applying Theorem \ref{thm:model-category-dual-whitehead} once again, along with Proposition Proposition \ref{prop:model-category-fibrant-codomain-precomposition}
and Proposition \ref{prop:model-category-left-right-homotopy-interchange}, we have
\[ [x,x']_r \to [x,x']_\ell \to [Qx,x'] \xleftarrow{\sim} [Qx,Qx'] \]
which implements \(Q\), where we note that since \(Qx\) is cofibrant and \(x'\) is fibrant, all homotopy mapping spaces agree, hence why we discard the subscript.
\end{proof}

\begin{construction}[Replacement functors]
	With the help of Proposition \ref{prop:model-category-replacement-properties}, we can construct replacement functors up to homotopy. In particular, we can construct
	functors
	\[ Q\!:\calC\to\hh{\calC_c},\quad R\!:\calC\to\hh{\calC_f}. \]
	We handle the former, since the latter is suitably dual. For all \(x\in\calC\), choose cofibrant replacements \(q_x\!:Qx\iso x\), and define \(Q\) on objects by
	\(x\mapsto Qx\). Using these cofibrant replacements, apply Construction \ref{construction:model-category-(co)fibrant-replacement} on morphisms to define \(Qf\) for all \(f\!:x\to y\).
	On morphisms, define \(Q\) by
	\[ \calC(x,y) \to \calC(Qx,Qy) \to [Qx,Qy]_r = \hh{\calC_c}(Qx,Qy). \]
	We need to show that this is a functor. First of all, when \(x=y\) and \(f=\id_x\), one observes that \(Q\id_x\) only depends on the left homotopy class of \(q_x\), which corresponds to \(\id_x\),
	so that \(Q\id_x = \id_{Qx}\).

	Suppose we have morphisms \(x\overset{f}\to y\overset{g}\to z\). We need to see that \(Q(g\circ f) = Q(g)\circ Q(f)\), but for this, we note that \(Q(g\circ f)\) depends only
	on the left homotopy class of \(g\circ f\circ q_x = g\circ q_y \circ Q(f) = q_z\circ Q(g)\circ Q(f)\) and this is uniquely determined by the left homotopy class
	of \(Q(g)\circ Q(f)\).

	Thus, we see that we have a functor \(Q\!:\calC\to\hh{\calC_{c}}\), and dually, a functor \(R\!:\calC\to\hh{\calC_{f}}\). When we restrict the former to consider
	only fibrant objects, we can use the construction in Proposition \ref{prop:model-category-replacement-properties}(3) to see that it induces a functor
	\[ \hh{Q}\!:\hh{\calC_f}\to\hh{\calC_{cf}}, \]
	on morphisms given by the composition
	\[ \hh{\calC_f}(x,y) = [x,y]_\ell \to [Qx,y] \cong [Qx,Qy] = \hh{\calC_{cf}}(x,y). \]
	Similarly, one induces a functor \(\hh{R}\!:\hh{\calC_c}\to\hh{\calC_{cf}}\), and thus two functors
	\[ \hh{R}\circ Q, \hh{Q}\circ R\!: \calC\to\hh{\calC_{cf}}. \]
\end{construction}

\begin{exercise}
	Given an object \(x\in\calC\) in a model category \(\calC\) with chosen fibrant replacement \(Rx\), cofibrant replacement \(Qx\), and furthermore
	additional replacements \(QRx\) and \(RQx\), show that there is a weak equivalence \(\xi_x\!:RQx\to QRx\) which assembles into a natural isomorphism
	\[ \xi\!: \hh{R}\circ Q \To \hh{Q}\circ R \]
	in the homotopy category.
\end{exercise}

% TODO: Finish the below. Just use the approach in [MP12] as it is written.

\begin{proposition}
	Let \(\calC\) be a model category, and write \(RQ\) for the functor \(\hh{R}\circ Q\) constructed above. Then the following statements hold.
	\begin{enumerate}[label=(\arabic*)]
		\item A morphism \(f\!:x\to y\) in \(\calC\) is a weak equivalence if and only if \(RQf\!:RQx\to RQy\) is an isomorphism.
		\item Any map in \(\hh{\calC_{cf}}\) is a composite of morphisms in \(RQ(\calC)\) and inverses of morphisms in \(RQ(W)\).
	\end{enumerate}
\end{proposition}
\begin{proof}
(1) The morphism \(f\) is a weak equivalence if and only if the corresponding map \(RQx\to RQy\) in \(\calC\) is a weak equivalence, if and only if it is a homotopy equivalence,
if and only if it is an isomorphism in \(\hh{\calC_{cf}}\).

(2) 
\end{proof}

\begin{theorem}
	Let \(\calC\) be a model category. Write \(RQ\) for the functor \(\hh{R}\circ Q\) constructed above. Then
	\[ RQ\!: \calC \to \hh{\calC_{cf}} \]
	exhibits \(\hh{\calC_{cf}}\) as a localization of \(\calC\) at the weak equivalences. In particular, there is a canonical equivalence of categories \(\hh{\calC_{cf}}\simeq\ho(\calC)\).
\end{theorem}
\begin{proof}
Let \(F\!:\calC\to\calD\) be a functor inverting the weak equivalences.
\end{proof}

\subsection{Derived functors for model categories (TBD)}
% Follow [Cis19] and [Rie13].

\subsection{Homotopy (co)limits (TBD)}
% Probably [Cis19] and [Rie13], but need to look at how [MP12] does it (if it does)
