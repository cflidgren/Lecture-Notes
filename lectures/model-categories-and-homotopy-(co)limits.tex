%!TEX root = ../lectures.tex

\section{Model categories \& homotopy (co)limits (TBD)}\label{lecture:model-categories-and-homotopy-colimits}
In Lecture \ref{lecture:model categories}, we laid out the basic theory of model categories, and indicated that their structure can be used to understand localizations
in a very convenient way. As part of the road to formalizing this, we showed that weak equivalences between \emph{bifibrant objects} are exactly given by
\emph{homotopy equivalences,} defined in terms of mapping cylinders (or path spaces, equivalently).

The purpose of this lecture is to continue down the path started in the last lecture. We will show that the localization \(\ho(\calC) = \calC[W^{-1}]\) of
a model category \(\calC\) at its weak equivalences \(W\) is equivalent to the homotopy category of bifibrant objects \(\hh{\calC_{cf}}\), whose objects
are the bifibrant objects, and whose morphisms are given by homotopy-equivalence classes of maps in \(\calC_{cf}\).

After doing the above, we will move to explaining some features of derived categories in the setting of model categories, and one particularly interesting example,
namely homotopy (co)limits. In Lectures \ref{lecture:triangulated-categories} through to \ref{lecture:gluing-t-structures}, we explored topics in triangulated categories,
and frequently made use of the intuition of \emph{cones} as ``homotopy cokernels.'' Homotopy colimits in model categories provide a refinement of this notion.

\subsection{The homotopy category of a model category}
\begin{theorem}
	Let \(\calC\) be a model category. Then there is a canonical equivalence of categories \(\hh{\calC_{cf}} \iso \ho(\calC)\).
\end{theorem}

\subsection{Derived functors for model categories}
\subsection{Homotopy (co)limits}
