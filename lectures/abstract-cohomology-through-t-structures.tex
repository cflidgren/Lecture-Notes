%!TEX root = ../lectures.tex

\section{Abstract cohomology through t-structures}\label{lecture:abstract-cohomology-through-t-structures}
An important topic of homological algebra is \emph{(co)homology.} Indeed, the entire subject is named after it. Typically, cohomology is handled (one way or another) through
derived categories, by using that the cohomology functors on chain complexes descend to cohomology functors on the derived category. One can then study various properties in a nice and systematic way.
However, there is also an abstract formulation of cohomology on the level of triangulated categories, given by the theory of \emph{t-structures.} These were initially
invented in \cite{faisceaux-pervers}, where they acted as a way to find Abelian categories living inside triangulated categories (in particular, Abelian categories of so-called \emph{perverse sheaves}).

While the formalism of t-structures should really be motivated by seeing the concrete example of derived categories, we choose to reverse this
and instead view t-structures as a formal setting in which cohomology arises, later showing that this may be employed in the standard cases.
As a result, some of the axioms may at first seem a bit unmotivated, but the hope is that they are still convincing enough to be considered reasonable. We essentially follow
\cite{kashiwara-schapira-book-2}.

\subsection{Truncation functors \& t-structures}
In essence, the cohomology functors on chain complexes come about due to the natural \emph{grading} that chain complexes have: indeed, given a chain complex \(x^\bullet\in\sfC(\calA)\), we can consider
the piece \(x^i\) living in degree \(i\). Cohomology comes out of the relation between this piece and its neighbouring pieces. Accordingly, t-structures start with the perspective that this
should be doable in more generality by \emph{specifying} this grading as data.

\begin{definition}
	Let \(\calT\) be a triangulated category. A \emph{t-structure} on \(\calT\) is a pair \((\calT^{\leq 0}, \calT^{\geq 0})\) of replete full subcategories of \(\calT\) (the \emph{aisle} and \emph{co-aisle,} respectively;
	alternatively, the \emph{coconnective piece} and the \emph{connective piece}) satisfying the below conditions. For any \(n\in\Z\), we set \(\calT^{\leq n} := \Sigma^{-n}\calT^{\leq 0}\) and \(\calT^{\geq n} := \Sigma^{-n}\calT^{\geq 0}\).
	\begin{enumerate}[label=(T\arabic*)]
		\item \((\calT^{\leq 0})^\perp \supseteq \calT^{\geq 1}\). That is, if \(x\in\calT^{\leq 0}\) and \(y\in\calT^{\geq 1}\), then \(\calT(x,y) = 0\).
		\item \(\calT^{\leq -1}\subseteq\calT^{\leq 0}\) and \(\calT^{\geq 1}\subseteq\calT^{\geq 0}\).
		\item For all \(x\in\calT\), there is a distinguished triangle
		\[ x' \to x \to x'' \to \Sigma x' \]
		such that \(x'\in\calT^{\leq 0}\) and \(x''\in\calT^{\geq 1}\).
	\end{enumerate}
\end{definition}
\begin{remark}
	Requiring the subcategories in a t-structure to be replete is not \emph{strictly} necessary for all purposes, but it is a nice simplifying assumption, and furthermore, it is harmless to reduce to this case
	since one can always just enlarge one's subcategories to be replete.
\end{remark}

\begin{example}
	Let \(\calT\) be a triangulated category. There are two trivial t-structures one may put on \(\calT\), namely \((\calT,0)\) and \((0,\calT)\).
\end{example}
\begin{example}
	When we get around to the derived category in later lectures, we will see that for an Abelian category \(\calA\), the pair
	\[ (\sfD^{\leq 0}(\calA),\sfD^{\geq 0}(\calA)) \]
	given by the subcategories spanned by those complexes concentrated in non-positive or non-negative cohomological degrees determines a t-structure. This is called the \emph{standard t-structure}
	on \(\sfD(\calA)\).
\end{example}
\begin{exercise}
	Let \((\calT^{\leq0},\calT^{\geq 0})\) be a t-structure. Show that, for all \(n\in\Z\), the pair \((\calT^{\leq n},\calT^{\geq n})\) is a t-structure.
\end{exercise}

We may employ the results developed in Lecture \ref{lecture:localization-sequences-of-triangulated-categories} immediately. In particular, we can make use of
Proposition \ref{prop:adjoint-from-orthogonal-triangle} (and its dual); inspection of the assumptions reveal that one demands almost exactly half the conditions of a t-structure.
We deduce that the inclusions
\[ \iota^{\leq n}\!:\calT^{\leq n}\inj\calT,\quad \iota^{\geq n}\!:\calT^{\geq n}\inj\calT \]
admit a right (resp.\ left) adjoint.
\begin{definition}
	Let \((\calT^{\leq 0},\calT^{\geq 0})\) be a t-structure on \(\calT\). The right adjoint of \(\iota^{\leq n}\), respectively the left adjoint of \(\iota^{\geq n}\),
	are denoted
	\[ \tau^{\leq n}\!:\calT\to\calT^{\leq n},\quad \text{resp.\ } \tau^{\geq n}\!:\calT \to \calT^{\geq n} \]
	and are called the \emph{truncation functors.}
\end{definition}
By construction, the counit \(\varepsilon\) of the adjunction \(\iota^{\leq n}\ladj\tau^{\leq n}\) and the unit \(\eta\) of the adjunction \(\tau^{\geq n+1}\ladj\iota^{\geq n+1}\)
lie in distinguished triangles
\[ \tau^{\leq n}x \overset{\varepsilon_x}\longto x \overset{\eta_x}\longto \tau^{\geq n+1}x \longto \Sigma \tau^{\leq n}x. \]
Furthermore, any other distinguished triangle
\[ x' \to x \to x'' \to \Sigma x', \]
such that \(x'\in\calT^{\leq n}\) and \(x''\in\calT^{\geq n+1}\), is canonically isomorphic to the first one. The details of this are in Proposition \ref{prop:adjoint-from-orthogonal-triangle}.

\begin{proposition}
	For all \(n\in\Z\) and \(x\in\calT\), there is a \emph{unique} morphism \(d^n_x\!:\tau^{\geq n+1}x\to \Sigma\tau^{\leq n}x\) such that
	\[ \tau^{\leq n}x \overset{\varepsilon_x}\longto x \overset{\eta_x}\longto \tau^{\geq n+1}x \overset{d^n_x}\longto \Sigma \tau^{\leq n}x \]
	is a distinguished triangle. Moreover, the morphisms \(d^n_x\) assemble into a natural transformation \(d^n\!:\tau^{\geq n+1}\To \Sigma\tau^{\leq n}\).
\end{proposition}
\begin{proof}
We use Lemma \ref{lemma:simple-uniqueness-of-cone-shift-map}. All we have to do is check that
\[ \calT(\Sigma \tau^{\leq n}x, \tau^{\geq n+1}x) = 0 \]
but \(\Sigma \tau^{\leq n}x\in \Sigma \calT^{\leq n} = \calT^{\leq n-1} \subseteq \calT^{\leq n}\), and \((\calT^{\leq n})^\perp\supseteq\calT^{\geq n+1}\), so this is clear.

To see that the (now unique) morphisms \(d^n_x\) assemble into a natural transformation \(d^n\!:\tau^{\geq n+1}\To \Sigma\circ \tau^{\leq n}\), suppose we have a
map \(f\!:x\to y\). Applying \(\tau^{\leq n}\) to \(f\) and using adjointness, we get a morphism of distinguished triangles
\begin{diagram*}
	\tau^{\leq n}x\ar[d,"\tau^{\leq n}f"']\ar[r] & x\ar[r]\ar[d,"f"] & \tau^{\geq n+1}x\ar[d,dashed]\ar[r,"d^n_x"] & \Sigma\tau^{\leq n}x\ar[d,"\Sigma\tau^{\leq n}f"] \\
	\tau^{\leq n}y\ar[r] & y\ar[r] & \tau^{\geq n+1}y\ar[r,"d^n_y"] & \Sigma\tau^{\leq n}y
\end{diagram*}
which by uniqueness (see Lemma \ref{lemma:simple-tr3-uniqueness}) implies that the dashed arrow is \(\tau^{\geq n+1}f\).
\end{proof}

\begin{exercise}
	Prove that for all \(n\in\Z\),
	\[ \tau^{\leq n} = \Sigma^{-n}\circ\tau^{\leq 0}\circ\Sigma^{n},\quad \tau^{\geq n} = \Sigma^{-n}\circ\tau^{\leq 0}\circ\Sigma^{n}. \]
\end{exercise}

\subsection{Properties of the aisle and co-aisle}
The truncation functors can be used to detect if something is in the aisle or the co-aisle.
\begin{proposition}
	Let \((\calT^{\leq0},\calT^{\geq0})\) be a t-structure. Then \(x\in\calD^{\leq n}\) if and only if \(\tau^{\geq n+1}x \cong 0\).
\end{proposition}
\begin{proof}
Since \(\tau^{\leq n}\) is right adjoint to a fully faithful functor, we have that \(x\in\calD^{\leq n}\) if and only if then the counit component \(\tau^{\leq n}x\to x\) is an isomorphism.
This, in turn, is equivalent to the cone (which is given by \(\tau^{\geq n+1}x\)) being zero.
\end{proof}

\begin{exercise}\label{exercise:t-structure-zero-is-in-aisle-and-coaisle}
	Let \((\calT^{\leq 0}, \calT^{\geq 0})\) be a t-structure. Show that \(0\in\calT^{\leq 0}\cap\calT^{\geq 0}\).
\end{exercise}

\begin{corollary}\label{corollary:t-structure-orthogonality}
	Let \((\calT^{\leq 0},\calT^{\geq 0})\) be a t-structure. Then the following statements hold.
	\begin{enumerate}[label=(\arabic*)]
		\item \(y\in\calT^{\geq 1}\) if and only if \(\calT(x,y) = 0\) for all \(x\in\calT^{\leq 0}\).
		\item \(x\in\calT^{\leq 0}\) if and only if \(\calT(x,y) = 0\) for all \(y\in\calT^{\geq 1}\).
	\end{enumerate}
	In particular, \(\calT^{\geq 1} = (\calT^{\geq 0})^\perp\) and \(\calT^{\leq 0} = \bperp{(\calT^{\geq 1})}\). Furthermore, \(\calT^{\leq 0}\) and \(\calT^{\geq 0}\) are
	closed under direct summands.
\end{corollary}
\begin{proof}
We prove (1), since (2) follows dually. Since \(\calT^{\geq 1}\subseteq (\calT^{\geq0})^\perp\), one direction is clear. For the other, let \(y\in\calT\) and
assume that \(\calT(x,y) = 0\) for all \(x\in \calT^{\geq 0}\). We now note that
\[ 0 \cong \calT(x,0) \cong \calT(x,y) \cong \calT(x,\tau^{\leq 0}y) \]
so that \(\tau^{\leq 0}y \cong 0\), hence \(y\in\calT^{\geq 1}\).

To see that \(\calT^{\leq 0}\) is closed under direct summands, let \(x \cong x'\oplus x'' \in \calT^{\leq 0}\). Then for all \(y\in\calT^{\geq 1}\) we have
\[ 0\cong \calT(x,y) \cong \calT(x',y)\oplus \calT(x'',y) \implies \calT(x',y) \cong \calT(x'',y)\cong 0 \]
so that \(x',x''\in\calT^{\leq 0}\) by (2). The other case is dual.
\end{proof}

We thus see that the aisle and co-aisle in a t-structure are almost thick subcategories, missing only the property of being closed under a particular shift.
We can say more: these subcategories are also closed under extensions. To prove this, we need a lemma.
\begin{lemma}\label{lemma:t-structure-aisle-from-trivial-mapping-to-truncation}
	Let \((\calT^{\leq 0},\calT^{\geq0})\) be a t-structure, and let \(x\in\calT\). If \(\calT(x,\tau^{\geq 1}x)=0\), then \(x\in\calT^{\leq 0}\).
\end{lemma}
\begin{proof}
By assumption, we have a distinguished triangle
\[ \tau^{\leq 0}x\to x \overset{0}\to \tau^{\geq 1}x \to \Sigma\tau^{\leq 0}x. \]
We recognize that any distinguished triangle involving a zero map induces a direct sum decomposition by Corollary \ref{corollary:direct-sum-triangle}, and in particular, we see that
\[ \tau^{\leq 0}x\cong \Sigma^{-1}\tau^{\geq 1}x \oplus x. \]
Since \(\calT^{\leq 0}\) is closed under direct summands, we see that \(x\in\calT^{\leq 0}\).
\end{proof}
\begin{proposition}\label{prop:t-structure-closure-under-extension}
	Let \((\calT^{\leq 0},\calT^{\geq0})\) be a t-structure. Then \(\calT^{\leq 0}\) and \(\calT^{\geq 0}\) are closed under extensions.
\end{proposition}
\begin{proof}
We prove the first statement, as the other follows by duality. Consider a distinguished triangle
\[ x' \to x \to x'' \to \Sigma x' \]
where \(x',x''\in\calT^{\leq 0}\). We want to show that \(x\in\calT^{\leq 0}\). We apply the cohomological functor \(\calT(-,\tau^{\geq 1}x)\) to see that
\[ 0 = \calT(x'',\tau^{\geq 1}x) \to \calT(x,\tau^{\geq 1}x) \to \calT(x',\tau^{\geq 1}x) = 0 \]
is exact, so \(\calT(x,\tau^{\geq 1}x) = 0\). By the preceeding lemma, we conclude that \(x\in\calT^{\leq 0}\).
\end{proof}
\begin{exercise}\label{exercise:t-structures-cones-of-morphisms-in-aisle}
	Let \((\calT^{\leq 0},\calT^{\geq0})\) be a t-structure, and let \(f\!:x\to y\) be a morphism in \(\calT\). Fix some \(n\in\Z\). Prove the following statements.
	\begin{enumerate}[label=(\arabic*)]
		\item If \(x,y\in\calT^{\leq n}\), then any cone of \(f\) is in \(\calT^{\leq n}\) and any cocone of \(f\) is in \(\calT^{\leq n+1}\).
		\item If \(x,y\in\calT^{\geq n}\), then any cone of \(f\) is in \(\calT^{\geq n-1}\) and any cocone of \(f\) is in \(\calT^{\geq n}\).
	\end{enumerate}
\end{exercise}

\subsection{The Abelian heart of a t-structure}
One of the original motivations for t-structures is the following construction.

\begin{definition}
	Let \((\calT^{\leq 0},\calT^{\geq 0})\) be a t-structure. The \emph{heart} of \(\calT\) with respect to \((\calT^{\leq 0},\calT^{\geq 0})\) is
	\[ \calT^\heartsuit := \calT^{\leq 0}\cap\calT^{\geq 0}. \]
\end{definition}

The heart consists of objects ``concentrated in degree zero'' and so, from the example of the derived category, we expect this to be an Abelian category. Proving that is our next aim.
An easy step towards this is showing that \(\calT^\heartsuit\) is additive.

\begin{proposition}\label{prop:t-structure-heart-is-additive}
	Let \((\calT^{\leq0},\calT^{\geq0})\) be a t-structure. Then \(\calT^\heartsuit\) is additive.
\end{proposition}
\begin{proof}
By Exercise \ref{exercise:t-structure-zero-is-in-aisle-and-coaisle}, \(0\in\calT^\heartsuit\). Furthermore, it is clearly pre-additive, since it is a full subcategory of a pre-additive category.
Therefore, we need only show it admits finite direct sums. However, if \(x,x'\in\calT^{\leq0}\) then for all \(y\in\calT^{\geq 1}\), we have
\[ 0\cong\calT(x\oplus x',y) \cong \calT(x,y)\oplus\calT(x',y) \]
so \(x\oplus x'\in\calT^{\leq 0}\) by Corollary \ref{corollary:t-structure-orthogonality}. The same statement holds for \(\calT^{\geq 0}\) by a similar argument. Therefore, \(\calT^\heartsuit\) is
also closed under finite direct sums.
\end{proof}

To prove that \(\calT^\heartsuit\) is Abelian, we will make use of the uniqueness properties coming from the orthogonality assumptions in a t-structure, along
with the intuition that cones in triangulated categories are like ``homotopy cokernels''. Actually, the cone is a more subtle construction: the cocone, which should be a ``homotopy kernel'',
is just a shift of the cone. In other words, the cone contains both information about the kernel and cokernel of a morphism. One way in which this appears in other results
is that it suffices to know a morphism has trivial cone to know that it is an isomorphism, in contrast to an Abelian category where one needs to check both the kernel and cokernel.

\begin{theorem}\label{thm:t-structure-heart-is-abelian}
	Let \((\calT^{\leq 0},\calT^{\geq0})\) be a t-structure. Then \(\calT^\heartsuit\) is an Abelian category.
\end{theorem}
\begin{proof}
Since we already know \(\calT^\heartsuit\) is additive, what remains is to check that it admits kernels and cokernels, and that the image coincides with the coimage.
Let \(f\!:x\to y\) be a morphism in \(\calT^\heartsuit\), and take the cone to get a distinguished triangle
\[ x\overset{f}\to y \to z \to \Sigma x. \]
Inspecting the triangle and using closure under extension, one sees that \(z\in\calT^{\leq 0}\cap\calT^{\geq -1}\).
\begin{enumerate}[label=(\arabic*)]
	\item The kernel of \(f\) is given by \(\tau^{\leq0}\Sigma^{-1} z \to x\). To see this, let \(w\in\calT^\heartsuit\). Applying \(\calT(w,-)\), we see that we have an exact sequence
	\[  0 = \calT(w,\Sigma^{-1}y) \to \calT(w,\Sigma^{-1}z) \to \calT(w,x)\overset{f_*}\to\calT(w,y). \]
	Therefore, we see that \(\calT(w,\Sigma^{-1}z)\) is the kernel of \(f_*\!:\calT(w,x)\to\calT(w,y)\). Using the canonical isomorphism \(\calT(w,\Sigma^{-1}z) \cong \calT(w,\tau^{\leq 0}\Sigma^{-1}z)\),
	we are done.
	\item The cokernel of \(f\) is given by \(y\to \tau^{\geq 0}z\). This is identical to (1).
	\item The image and coimage agree. By (1) and (2), we obtain the image of \(f\) by truncating the cocone of the composition \(y\to z \to \tau^{\geq 0}z\). Let us consider the distinguished triangle
	\[ e \to y \to \tau^{\geq 0}z \to \Sigma e, \]
	so that \(\img{f} = \tau^{\geq 0}e\). We observe that \(e\in\calT^{\leq1}\cap\calT^{\geq0}\). Applying (TR4) to the composition \(y \to z \to \tau^{\geq 0}z\), we obtain a distinguished triangle
	\[ \Sigma x \to \Sigma e \to \Sigma\tau^{\leq-1}z \to \Sigma^2 x \quad\leadsto\quad x \to e \to \tau^{\leq-1}z \to \Sigma x.\]
	By closure under extension, we see that \(e\in\calT^{\leq 0}\), so in fact \(e\in\calT^\heartsuit\). Shifting to the left, we have the distinguished triangle
	\[ \Sigma^{-1}\tau^{\leq-1}z \to x \to e \to \tau^{\leq-1}z \]
	where we note that \(\Sigma^{-1}\tau^{\leq-1}z = \tau^{\leq 0}\Sigma^{-1} z = \ker{f}\), so really we have
	\[ \ker{f} \to x \to e \to \Sigma(\ker{f}). \]
	Since \(e \cong \tau^{\geq 0}e\), this exhibits \(e\) also as the cokernel of the kernel map, i.e.\ \(e \cong \coimg{f} \). On the other hand,
	\(e\cong\img{f}\), so we are done.
\end{enumerate}
We conclude that \(\calT^\heartsuit\) is Abelian.
\end{proof}

Remarkably, this means that whenever we have a t-structure on \(\calT\), we may find an Abelian subcategory where the kernel and cokernel are, in a sense, given by the
triangulated structure on \(\calT\).

\subsection{Cohomology functors}
By truncating to the left and right at position \(n\), we are left with something in \(\calT^{\leq n}\cap\calT^{\geq n} \simeq \Sigma^{-n}\calT^\heartsuit\). This
indudces the all-important \emph{cohomology functors} intrinsic to any t-structure.
\begin{definition}
	Let \((\calT^{\leq0},\calT^{\geq0})\) be a t-structure. The \emph{zeroth cohomology functor} with respect to the t-structure is
	\[ \HH^0\!:\calT\to\calT^\heartsuit,\quad \HH^0 := \tau^{\geq 0}\circ\tau^{\leq 0}. \]
	More generally, we define the \(n\)th cohomology functor, for any \(n\in\Z\), by
	\[ \HH^n\!:\calT\to\calT^\heartsuit,\quad \HH^n := \HH^0\circ\Sigma^n. \]
\end{definition}
\begin{exercise}
	Show that \(\HH^n = \Sigma^n\circ \tau^{\geq n}\circ\tau^{\leq n}\).
\end{exercise}

Now, we made a choice in the above definition of what order to do the truncations. However, clearly there should be no difference between truncating on the left then the right, or vice versa.
We verify this now, in (3) below.
\begin{proposition}\label{prop:t-structure-truncation-relations}
	Let \((\calT^{\leq 0},\calT^{\geq 0})\) be a t-structure. Then the following statements hold.
	\begin{enumerate}[label=(\arabic*)]
		\item If \(m\leq n\), then
		\[ \tau^{\leq n}\circ\tau^{\leq m} \cong \tau^{\leq m}\circ\tau^{\leq n} \cong \tau^{\leq m},\quad \text{and}\quad \tau^{\geq n}\circ\tau^{\geq m} \cong \tau^{\geq m}\circ\tau^{\geq n} \cong \tau^{\geq n}. \]
		\item If \(m > n\), then
		\[ \tau^{\geq m}\circ\tau^{\leq n} \cong 0 \cong \tau^{\leq n}\circ\tau^{\geq m}. \]
		\item For all \(m,n\in\Z\), there is a unique natural isomorphism \(\beta\!:\tau^{\geq m}\circ\tau^{\leq n}\cong\tau^{\leq n}\circ\tau^{\geq m}\) such that the diagram
		of natural transformations
		\begin{diagram*}
			\tau^{\leq n}\ar[r,Rightarrow,"\varepsilon"]\ar[d,Rightarrow,"\eta\tau^{\leq n}"'] & \1 \ar[r,Rightarrow,"\eta"] & \tau^{\geq m} \\
			\tau^{\geq m}\tau^{\leq n} \ar[rr,Rightarrow,"\sim","\beta"'] & & \tau^{\leq n}\tau^{\geq m}\ar[u,Rightarrow,"\varepsilon\tau^{\geq m}"']
		\end{diagram*}
		commutes, where \(\varepsilon\) is the counit of \(\iota^{\leq n}\ladj\tau^{\leq n}\) and \(\eta\) is the unit of \(\tau^{\geq m}\ladj\iota^{\geq m}\).
	\end{enumerate}
\end{proposition}
\begin{proof}
(1) We prove the leftmost statement, as the other one is dual. Observe that we have a natural transformation \(\varepsilon\tau^{\leq m}\!: \tau^{\leq n}\circ\tau^{\leq m} \To \tau^{\leq m}\).
Since \(\calT^{\geq n}\) is a reflective subcategory and \(\tau^{\leq m}x\in\calT^{\leq m}\subseteq\calT^{\leq n}\) for all \(x\in\calT\), we see that this natural transformation
must be a natural isomorphism by Lemma \ref{lemma:reflective-subcategory-essential-image}, showing that \(\tau^{\leq n}\circ\tau^{\leq m} \cong \tau^{\leq m}\). For the
other isomorphism, note that we have natural isomorphisms
\[ \calT^{\leq m}(-, (\tau^{\leq m}\circ\tau^{\leq n})(-)) \cong \calT^{\leq n}(\iota^{\leq m}(-),\tau^{\leq n}(-)) \cong \calT(\iota^{\leq m}(-), -) \]
so by uniqueness of adjoints, we get a natural isomorphism \( \tau^{\leq m}\circ\tau^{\leq n}\cong\tau^{\leq m} \).

(2) The natural transformation \( \tau^{\leq m-1}\tau^{\leq n} \To \tau^{\leq n} \) is a natural isomorphism since \(n < m\), by (1). In particular, the cone of each component is zero.
However, the cone here is canonically given by \(\tau^{\geq m}\circ \tau^{\leq n}\), hence we get one of the claimed isomorphisms. The other is dual.

(3) By (2), the claim already holds when \(m > n\). Hence, we may assume that \(m \leq n\). We have the distinguished triangles
\begin{align*}
	\tau^{\leq m-1}\tau^{\leq n}x \longto \tau^{\leq n}x \overset{(\eta\tau^{\leq n})_x}\longto \tau^{\geq m}\tau^{\leq n}x \longto \Sigma\tau^{\leq m-1}\tau^{\leq n}x, \\
	\tau^{\leq n}\tau^{\geq m}x \overset{(\varepsilon\tau^{\geq m})_x}\longto \tau^{\geq m}x \longto \tau^{\geq n+1}\tau^{\geq m}x \longto \Sigma\tau^{\leq n}\tau^{\geq m}x.
\end{align*}
Applying Exercise \ref{exercise:t-structures-cones-of-morphisms-in-aisle}, we see that \(\tau^{\leq n}\tau^{\geq m}x, \tau^{\geq m }\tau^{\leq n}x \in \calT^{\leq n}\cap\calT^{\geq m}\).
Now we use the universal properties of the truncations (being adjoints):
\begin{center}
	\begin{tikzcd}[row sep=small]
		\tau^{\leq n}x \ar[r,"(\eta\tau^{\leq n})_x"]\ar[d,"\varepsilon_x"'] & \tau^{\geq m}\tau^{\leq n}x\ar[ddl,dashed,"\exists!"]  \\
		x\ar[d,"\eta_x"'] & \\
		\tau^{\geq m}x & \tau^{\leq n}\tau^{\geq m}x \ar[l, "(\varepsilon\tau^{\geq m})_x"]
	\end{tikzcd}
	\(\quad\leadsto\quad\)
	\begin{tikzcd}[row sep=small]
		\tau^{\leq n}x \ar[r,"(\eta\tau^{\leq n})_x"]\ar[d,"\varepsilon_x"'] & \tau^{\geq m}\tau^{\leq n}x\ar[dd,dashed,"\exists!"]  \\
		x\ar[d,"\eta_x"'] & \\
		\tau^{\geq m}x & \tau^{\leq n}\tau^{\geq m}x \ar[l, "(\varepsilon\tau^{\geq m})_x"]
	\end{tikzcd}
\end{center}
where we use that \(\tau^{\geq m}\tau^{\leq n}x\in\calT^{\leq n}\). We see that there is a unique natural transformation \(\beta\!:\tau^{\geq m}\circ\tau^{\leq n}\To\tau^{\leq n}\tau^{\geq m}\)
making the provided diagram of natural transformations commute. We must now show it is an isomorphism.

By part (1), we may rewrite the triangles as
\begin{align*}
	\tau^{\leq m-1}x \longto \tau^{\leq n}x \overset{(\eta\tau^{\leq n})_x}\longto \tau^{\geq m}\tau^{\leq n}x \longto \Sigma\tau^{\leq m-1}x, \\
	\tau^{\leq n}\tau^{\geq m}x \overset{(\varepsilon\tau^{\geq m})_x}\longto \tau^{\geq m}x \longto \tau^{\geq n+1}x \longto \Sigma\tau^{\leq n}\tau^{\geq m}x.
\end{align*}
We apply (TR4) to the composition \(\tau^{\leq m-1}x \to \tau^{\leq n}x \to x\). This yields a distinguished triangle
\[ \tau^{\geq m}\tau^{\leq n}x \to \tau^{\geq m}x \to \tau^{\geq n+1}x \to \Sigma\tau^{\geq m}\tau^{\leq n}x. \]
However, we now have a distinguished triangle around \(\tau^{\geq m}x\) whose left term is in \(\calT^{\leq n}\) and right term is in \(\calT^{\geq n+1}\), which implies (by construction
of the truncation adjunction) that the canonical map \(\beta_x\!:\tau^{\geq m}\tau^{\leq n}x \to \tau^{\leq n}\tau^{\geq m}x \) is an isomorphism.
\end{proof}

\begin{corollary}\label{corollary:t-structure-kernel-and-cokernel-in-heart-cohomology}
	Let \((\calT^{\leq 0},\calT^{\geq 0})\) be a t-structure, and consider a morphism \(f\!:x\to y\) in \(\calT^\heartsuit\). Given a distinguished triangle
	\[ x \to y \to z \to \Sigma x, \]
	we have canonical isomorphisms
	\[ \ker{f} \cong \HH^{-1}(z),\quad \text{and}\quad \coker{f} \cong \HH^0(z).  \]
\end{corollary}
\begin{proof}
Recall from the proof of Theorem \ref{thm:t-structure-heart-is-abelian} that we have canonical isomorphisms
\[ \ker{f} \cong \tau^{\leq -1}z,\quad \text{and}\quad \coker{f} \cong \tau^{\geq 0}z  \]
and that \(z\in\calT^{\geq -1}\cap\calT^{\leq 0}\). In particular, we have canonical isomorphisms
\[ \tau^{\geq -1}z \cong z \cong \tau^{\leq 0}z. \]
Thus,
\[ \ker{f} \cong \tau^{\leq -1}\tau^{\geq -1}z \cong \HH^{-1}(z),\quad \text{and}\quad \coker{f} \cong \tau^{\geq 0}\tau^{\leq 0}z \cong \HH^0(z) \]
where, for the left computation, we use Proposition \ref{prop:t-structure-truncation-relations}.
\end{proof}

\begin{corollary}\label{corollary:t-structure-exact-sequence-in-heart-induces-unique-distinguished-triangle}
	Let \((\calT^{\leq 0},\calT^{\geq 0})\) be a t-structure, and consider a short exact sequence
	\[ 0 \to x \inj y \sur z \to 0 \]
	in \(\calT^\heartsuit\). Then there is a unique morphism \(z \to \Sigma x\) for which
	\[ x \to y \to z \to \Sigma x \]
	is a distinguished triangle in \(\calT\).
\end{corollary}
\begin{proof}
If such a morphism exists, it is unique by Lemma \ref{lemma:simple-uniqueness-of-cone-shift-map}, since there are no non-zero maps from \(\calT^{\leq -1}\) to \(\calT^{\geq 0}\). To see that one exists,
pick a distinguished triangle
\[ x \to y \to z' \to \Sigma x. \]
We aim to show that \(z' \cong z\). We have that \(z'\in\calT^{\leq 0}\calT^{\geq -1}\), and therefore the distinguished triangle
\[ \tau^{\leq -1}z' \to z' \to \tau^{\geq 0}z' \to \Sigma\tau^{\leq -1}z' \]
turns into
\[ \HH^{-1}(z') \to z' \to \HH^0(z') \to \Sigma \HH^{-1}(z'). \]
However, by Corollary \ref{corollary:t-structure-kernel-and-cokernel-in-heart-cohomology}, we have
\[ \HH^{-1}(z') \cong \ker(x\inj y) \cong 0,\quad \HH^0(z')\cong \img(y\sur z) \cong z. \]
Thus, \(z' \cong \HH^0(z') \cong z\).
\end{proof}

\subsection{Appendix: Cohomology is cohomological}
Of central importance about the cohomology functors induced by a t-structure is the fact that they are actually genuine cohomological functors, so
that a t-structure gives rise to some new \emph{non-trivial} and \emph{interesting} constructions. We now prove this.
\begin{exercise}\label{exercise:abelian-category-kernel-invariant-under-composition-with-monic}
	Let \(\calA\) be an Abelian category, and let \(f\!:x\to y\) be a morphism in \(\calA\). Suppose that we have a factorization
	\[ f = (x \overset{g}\to z \overset{h}\inj y). \]
	Show that there is a canonical isomorphism \(\ker{f} \cong \ker{g}\).
\end{exercise}
\begin{exercise}\label{exercise:t-structure-cohomology-of-object-in-aisle}
	Let \((\calT^{\leq 0},\calT^{\geq 0})\) be a t-structure, and let \(x\in\calT^{\leq n}\). Show that \(\HH^i(x) \cong 0\) for all \(i > n\). Prove the dual statement also.
\end{exercise}
\begin{theorem}\label{thm:t-structure-cohomology-is-cohomological}
	Let \((\calT^{\leq 0},\calT^{\geq 0})\) be a t-structure. Then the cohomology functors \(\HH^n\!:\calT\to\calT^{\heartsuit}\) are cohomological.
\end{theorem}
\begin{proof}
Note that \(\calT^\heartsuit\) is Abelian by Theorem \ref{thm:t-structure-heart-is-abelian}, so the proposition is a valid one to consider. The functors are clearly additive, so we must check that
they send distinguished triangles to exact sequences. Consider a distinguished triangle
\[ x \to y \to z \to \Sigma x \]
in \(\calT\). It suffices to show that \(\HH^0\) sends this to an exact sequence
\[ \HH^0(x) \to \HH^0(y) \to \HH^0(z). \]
The proof proceeds in three steps, increasing in generality.
\begin{enumerate}[label=(\arabic*)]
	\item The case when \(x,y,z\in\calT^{\leq 0}\). We show that, in fact,
	\[ \HH^0(x) \to \HH^0(y) \to \HH^0(z) \to 0 \]
	is exact. To do this, note that for all \(w\in\calT^\heartsuit\) and \(u\in\calT^{\leq 0}\), we have natural isomorphisms
	\[ \calT^\heartsuit(\HH^0(u),w) = \calT(\tau^{\geq 0}\tau^{\leq 0}u, w) \cong \calT(\tau^{\geq 0}u, w) \cong \calT(u,w) \]
	since \(\tau^{\leq 0}u \iso u\). Furthermore, \(\calT(\Sigma u,w) \cong 0\) since \(\Sigma u \in \calT^{\leq -1}\). Therefore, applying \(\calT(-,w)\) to our original distinguished triangle yields
	exact sequences
	\begin{diagram*}[cramped]
		\calT(\Sigma x,w) \ar[r]\ar[d,"\sim" labl] & \calT(z,w)\ar[r]\ar[d,"\sim" labl] & \calT(y,w)\ar[d,"\sim" labl]\ar[r] & \calT(x,w)\ar[d,"\sim" labl] \\
		0 \ar[r] & \calT(\HH^0(z),w)\ar[r] & \calT(\HH^0(y),w)\ar[r] & \calT(\HH^0(x),w)
	\end{diagram*}
	for all \(w\in\calT^\heartsuit\), from which we deduce that the claimed sequence is exact.
	\item The case when \(x\in\calT^{\leq 0}\). We again show that
	\[ \HH^0(x) \to \HH^0(y) \to \HH^0(z) \to 0 \]
	is exact, by reduction to (1). Let \(w\in\calT^{\geq 1}\). We see that \(\calT(x,w) \cong \calT(\Sigma x,w) \cong 0\). In particular, applying \(\calT(-,w)\) to our distinguished triangle yields an
	exact sequence
	\[ 0 \to \calT(z,w) \to \calT(y,w) \to 0 \]
	so that we have a natural isomorphism \(\calT(z,w)\iso\calT(y,w)\) when \(w\in\calT^{\geq 1}\). Therefore, we have natural isomorphisms
	\[ \calT(\tau^{\geq 1}z,w) \cong \calT(z,w) \iso \calT(y,w) \cong \calT(\tau^{\geq 1}y,w) \]
	from which we conclude that \(\tau^{\geq 1}y \iso \tau^{\geq 1}z\). To finally reduce to (1), we apply (TR4) to the composition \(y \to z \to \tau^{\geq 1}z\), yielding a distinguished triangle
	\[ \Sigma x \to \Sigma\tau^{\leq 0}y \to \Sigma\tau^{\leq 0}z \to \Sigma^2x \quad \leadsto \quad x \to \tau^{\leq 0}y \to \tau^{\leq 0}z \to \Sigma x \]
	where we are now in the situation of (1). We have a natural isomorphism \(\HH^0\cong\HH^0\circ\tau^{\leq 0}\), so in the end, applying \(\HH^0\) to the above yields the desired exact sequence.
	\item The case when \(x,y,z\) are all arbitrary. Consider the composition \(\tau^{\leq 0}x \to x \to y\), and take the cone so we have a distinguished triangle
	\[ \tau^{\leq 0}x \to y \to e \to \Sigma\tau^{\leq 0}x. \]
	Applying (TR4) to the composition, we have the commutative diagram
	\begin{diagram*}[cramped]
		\tau^{\leq 0}x\ar[r]\ar[d,equal] & x\ar[d]\ar[r] & \tau^{\geq 1}x \ar[r]\ar[d,dashed] & \Sigma \tau^{\leq 0}x \ar[d,equal] \\
		\tau^{\leq 0}x\ar[r]\ar[d] & y\ar[r]\ar[d,equal] & e\ar[r]\ar[d,dashed] & \Sigma\tau^{\leq 0}x \ar[d] \\
		x\ar[r]\ar[d] & y\ar[d]\ar[r] & z \ar[r]\ar[d,equal] & \Sigma x\ar[d] \\
		\tau^{\geq 1}x \ar[r,dashed] & e\ar[r,dashed] & z\ar[r,dashed] & \Sigma \tau^{\geq 1}x
	\end{diagram*}
	with rows distinguished triangles. Applying \(\HH^0\) to the second row, by (2), gives the exact sequence
	\[ \HH^0(x) \to \HH^0(y) \to \HH^0(e). \]
	By the dual of (2), shifting the bottom row and applying \(\HH^0\) yields the exact sequence
	\[ 0 \to \HH^0(e) \to \HH^0(z) \to \HH^1(x) \]
	and in particular, \(\HH^0(e) \to \HH^0(z)\) is a monomorphism. Noting that \(\HH^0(y)\to\HH^0(z)\) factors as \(\HH^0(y)\to\HH^0(e)\inj\HH^0(z)\) by the commutativity
	of the big diagram, we deduce by Exercise \ref{exercise:abelian-category-kernel-invariant-under-composition-with-monic} that
	\[ \img(\HH^0(x)\to\HH^0(y)) \cong \ker(\HH^0(y)\to\HH^0(e)) \cong \ker(\HH^0(y)\to\HH^0(z)) \]
	showing that
	\[ \HH^0(x) \to \HH^0(y) \to \HH^0(z) \]
	is exact.
\end{enumerate}
This completes the proof.
\end{proof}

\begin{corollary}
	Let \((\calT^{\leq 0},\calT^{\geq 0})\) be a t-structure, and suppose we have a distinguished triangle
	\[ x \to y \to z \to \Sigma x,\quad x,y,z\in\calT^\heartsuit. \]
	Then the sequence
	\[ 0 \to x \to y \to z \to 0 \]
	is exact. In particular, there is a bijection between exact sequences in \(\calT^\heartsuit\) and distinguished triangles whose terms lie in \(\calT^\heartsuit\).
\end{corollary}
\begin{proof}
We have a natural isomorphism \(\HH^0|_{\calT^\heartsuit} \cong \1_{\calT^\heartsuit}\), since on objects of the heart, the components of the (co)unit for the truncation at zero are isomorphisms.
In particular, Theorem \ref{thm:t-structure-cohomology-is-cohomological} specializes to say that given a distinguished triangle
\[ x \to y \to z \to \Sigma x,\quad x,y,z\in\calT^\heartsuit, \]
applying \(\HH^0\) yields exact sequences
\begin{diagram*}[cramped]
	0 \ar[r]\ar[d,equal] & \HH^0(x) \ar[r]\ar[d,"\sim" labl] & \HH^0(y) \ar[d,"\sim" labl]\ar[r] & \HH^0(y)\ar[d,"\sim" labl]\ar[r] & 0\ar[d,equal] \\
	0 \ar[r] & x \ar[r] & y \ar[r] & z \ar[r] & 0
\end{diagram*}
as desired. The final assertion follows by combining the above with Corollary \ref{corollary:t-structure-exact-sequence-in-heart-induces-unique-distinguished-triangle}.
\end{proof}

\subsection{Appendix: Bounded and non-degenerate t-structures}
Consider a complex \(x^\bullet\) in some Abelian category category \(\calA\):
\[ x^\bullet\!:\quad \cdots \to x^{i-1} \to x^i \to x^{i+1} \to \cdots. \]
The t-structure on \(\sfD(\calA)\) is such that \(x^\bullet\in\sfD(\calA)^{\leq 0}\) if and only if \(x^\bullet \in \sfD^{\leq 0}(\calA)\), i.e.\ if \(\HH^i(x^\bullet) \cong 0\) for all \(i > 0\).
In other words, one can use the cohomology functors to completely characterize the aisle (and co-aisle) of the t-structure.

Now consider some arbitrary triangulated category \(\calT\) with a t-structure \((\calT^{\leq 0},\calT^{\geq 0})\). The above statement no longer holds, as, in fact, there may be non-zero \(x\in\calT\)
for which \(\HH^i(x)\cong 0\) \emph{for all} \(i\in\Z\). Indeed, consider the trivial t-structure \((\calT,0)\). The left truncation functor does nothing, while the right truncation functor
is the functor sending everything to zero. In this t-structure, all cohomology functors are just the zero functor \(\calT \to 0\), and so in fact every object \(x\in\calT\) forms a counter-example.
We would like to find critera which prevent this kind of pathology.

\begin{proposition}\label{prop:t-structure-aisle-from-cohomology}
	Let \((\calT^{\leq 0},\calT^{\geq 0})\) be a t-structure, and let \(x\in\calT\). Assume one of the following conditions hold.
	\begin{enumerate}[label=(\roman*)]
		\item There is some \(n\in\Z\) such that \(x\in\calT^{\leq n}\).
		\item If \(\tau^{\geq 1}x \in \calT^{\geq n} \) for all \(n\geq 1\), then \(\tau^{\geq 1}x\cong 0\).
	\end{enumerate}
	Then \(x\in\calT^{\leq 0}\) if and only if \(\HH^i(x)\cong 0\) for all \(i > 0\).
\end{proposition}
\begin{proof}
If \(x\in\calT^{\leq 0}\), then \(\HH^i(x)\cong 0\) for all \(i>0\); this is Exercise \ref{exercise:t-structure-cohomology-of-object-in-aisle}. For the converse, we split into the cases presented.
\begin{enumerate}[label=(\roman*)]
	\item We proceed by induction. If \(n\leq 0\), we are done. Thus, suppose \(n > 0\), and that we know the result for all \(0 \leq m < n\). By assumption, \(\HH^n(x) \cong 0\), so the distinguished triangle
	\[ \tau^{\leq n-1}\tau^{\leq n}x \to \tau^{\leq n}x \to \tau^{\geq n}\tau^{\leq n}x \to \Sigma \tau^{\leq n-1}\tau^{\leq n}x \]
	becomes
	\[ \tau^{\leq n-1}x \to x \to 0 \to \Sigma\tau^{\leq n-1}x \]
	showing that \(\tau^{\leq -1}x\iso x\). In particular, \(x\in\calT^{\leq n-1}\), so by the induction hypothesis \(x\in\calT^{\leq 0}\).
	\item We wish to show that \(\tau^{\leq 0}x \to x\) is an isomorphism, to which it suffices to prove \(\tau^{\geq 1}x\cong 0\). To this end, let us show that for all \(n \geq 1\),
	the morphism \(\tau^{\geq n}x \to \tau^{\geq n+1}x\) is an isomorphism. Consider the distinguished triangle
	\[ \tau^{\leq n}\tau^{\geq n}x \to \tau^{\geq n}x \to \tau^{\geq n+1}\tau^{\geq n}x \to \Sigma\tau^{\leq n}\tau^{\geq n}x \]
	which we compute as
	\[ \Sigma^{-n}\HH^n(x) \to \tau^{\geq n}x \to \tau^{\geq n+1}x \to \Sigma^{-n+1}\HH^n(x). \]
	Since \(n>0\), we see that \(\HH^n(x)\cong 0\) and therefore \(\tau^{\geq n}x\iso\tau^{\geq n+1}x\). We see that \(\tau^{\geq 1}x\cong \tau^{\geq n}x\) for all \(n\geq 1\),
	so that \(\tau^{\geq 1}x \in \bigcap_{n\geq 1}\calT^{\geq n}\). By assumption, \(\tau^{\geq 1}x\) is such that if it satisfies this, it is zero.
\end{enumerate}
This completes the proof.
\end{proof}

This proposition hopefully motivates the following definitions being interesting.

\begin{definition}
	Let \((\calT^{\leq 0},\calT^{\geq 0})\) be a t-structure. We say it is \emph{left bounded,} resp.\ right bounded, if
	\[ \calT = \bigcup_{n\in\Z}\calT^{\geq n},\quad \text{resp.\ } \calT = \bigcup_{n\in\Z}\calT^{\leq n}. \]
	If the t-structure is both left and right bounded, we say is is bounded.

	We say the t-structure is \emph{left non-degenerate,} resp.\ \emph{right non-degenerate} if
	\[ \{0\} \simeq \bigcap_{n\in\Z}\calT^{\leq n},\quad \text{resp.\ } \{0\} \simeq \bigcap_{n\in\Z}\calT^{\geq n}. \]
	If the t-structure is both left and right non-degenerate, we say it is non-degenerate.
\end{definition}

\begin{proposition}
	Let \((\calT^{\leq 0},\calT^{\geq 0})\) be a t-structure. If it is right bounded, then it is right non-degenerate.
\end{proposition}
\begin{proof}
Suppose that \(x\in\bigcap_{k\in\Z}\calT^{\geq k}\). Since the t-structure is right bounded, we also have that \(x\in\calT^{\leq n}\) for some \(n\in\Z\).
In particular, \(x\in\calT^{\leq n}\cap\calT^{\geq n+1}\), so
\[ x \cong \tau^{\geq n+1}\tau^{\leq n}x \cong 0 \]
by Proposition \ref{prop:t-structure-truncation-relations}. Therefore, the t-structure is non-degenerate.
\end{proof}

\begin{corollary}\label{corollary:t-structure-right-non-degenerate-aisle-from-cohomology}
	Let \((\calT^{\leq 0},\calT^{\geq 0})\) be a right non-degenerate t-structure. Then
	\[ \calT^{\leq 0} = \{ x\in\calT \mid \forall i>0,\, \HH^i(x) \cong 0 \}. \]
\end{corollary}
\begin{proof}
Clearly, we have
\[ \calT^{\leq 0} \subseteq \{ x\in\calT \mid \forall i>0,\, \HH^i(x) \cong 0 \}. \]
For the converse, we apply Proposition \ref{prop:t-structure-aisle-from-cohomology}. In applying (ii), note that since \(\calT^{\geq m}\subseteq\calT^{\geq n}\) for all \(m\geq n\), we have
\[ \bigcap_{n\in\Z}\calT^{\geq n} = \bigcap_{n\geq 1}\calT^{\geq n}. \]
By assumption, the right term is thus zero, which means (ii) in the proposition is satisfied for all \(x\in\calT\).
\end{proof}

By appropriately dualizing what we have proven, we see that in a non-degenerate t-structure (for example, one which is bounded) the aisle and co-aisle are completely
determined by the cohomology functors \(\HH^n\!:\calT^\to\calT^\heartsuit\). In this case, the cohomology functors act nicely as a family.
\begin{proposition}
	Let \((\calT^{\leq 0},\calT^{\geq 0})\) be a non-degenerate t-structure. Then \(\{\HH^n\}_{n\in\Z}\) is a conservative family of functors. That is, a morphsism \(f\!:x\to y\)
	is an isomorphism if and only if the morphisms \(\HH^n(f)\!:\HH^n(x)\to\HH^n(y)\) are isomorphisms for all \(n\in\Z\).
\end{proposition}
\begin{proof}
One direction is easy: if \(f\) is already an isomorphism, the latter statement is clear. Conversely, assume \(\HH^n(f)\) is an isomorphism for all \(n\in\Z\). Taking the cone of \(f\),
we have a distinguished triangle
\[ x \overset{f}\to y \to z \to \Sigma x. \]
From Theorem \ref{thm:t-structure-cohomology-is-cohomological}, the functors \(\HH^n\) are cohomological, so we have the exact sequence
\begin{diagram*}
	\HH^n(x) \ar[r,"\sim","\HH^n(f)"'] & \HH^n(y) \ar[r] & \HH^n(z) \ar[r] & \HH^{n+1}(x) \ar[r,"\sim","\HH^{n+1}(f)"'] & \HH^{n+1}(y)
\end{diagram*}
which implies that \(\HH^n(z)\cong 0\) for all \(n\in\Z\). In particular, since the t-structure is right non-degenerate, Corollary \ref{corollary:t-structure-right-non-degenerate-aisle-from-cohomology} tells us that
\[ z \in \bigcap_{n\in\Z}\calT^{\leq n}. \]
However, since the t-structure is also left non-degenerate, this means that \(z\cong 0\), so \(f\) is an isomorphism.
\end{proof}

In fact, this property completely characterizes non-degenerate t-structures.
\begin{theorem}
	Let \((\calT^{\leq 0},\calT^{\geq 0})\) be a t-structure. Then the following are equivalent.
	\begin{enumerate}[label=(\arabic*)]
		\item The t-structure is non-degenerate.
		\item The family of cohomological functors \(\{\HH^n\!:\calT\to\calT^{\heartsuit}\}_{n\in\Z}\) is conservative.
	\end{enumerate}
\end{theorem}
\begin{proof}
The implication \((1)\Rightarrow(2)\) has already been proven. For the converse, let us first assume that
\[ x \in\bigcap_{n\in\Z}\calT^{\leq n}. \]
Then \(\HH^n(x)\cong 0\) for all \(n\in\Z\) by Exercise \ref{exercise:t-structure-cohomology-of-object-in-aisle}. In particular, \(x \to 0\) is an isomorphism, by conservativity. This shows
that the t-structure is left non-degenerate. Showing that it is right non-degenerate is dual.
\end{proof}

\begin{remark}
	The above characterization of non-degenerate t-structures suggests that perhaps one ought to be able to go the other way: starting from a cohomological functor \(H\!:\calT\to\calA\)
	such that the family \(\{H\circ\Sigma^n\}_{n\in\Z}\) is conservative, construct a t-structure on \(\calT\) such that \(\calT^\heartsuit\simeq\calA\) and the cohomology functors
	are given by \(H\circ\Sigma^n\).

	In general, one cannot hope for this to be possible, but for particular kinds of functors it actually is. A prominent example is the theorem
	of Hoshino--Kato--Miyachi \cite{hkm02}, previously discussed briefly in Appendix \ref{appendix:abelian-categories-with-a-compact-projective-generator}, which proves that
	this works for the functor \(\calT(s,-)\!:\calT\to\Mod_{\End(s)}\) as long as \(s\in\calT\) is a compact generator and satisfies
	\[ \forall i > 0,\quad \calT(s,s[i]) \cong 0. \]
	Proving the theorem takes some work, but involves many interesting ingredients which we will cover later.
\end{remark}

\subsection{Appendix: Stable t-structures}
Let \(\calT\) be a triangulated category. The trivial t-structure \((\calT,0)\) has a special property.
\begin{definition}
	A t-structure \((\calT^{\leq 0},\calT^{\geq 0})\) is called \emph{stable} if \(\Sigma\calT^{\leq 0} = \calT^{\leq 0}\).
\end{definition}
We dedicate this short appendix to a simple characterization of such t-structures, which also demonstrates that they are somewhat pathological. Everything is taken
from \cite{chen2022extensionststructures}, where a little more can be found too.

\begin{proposition}
	Let \((\calT^{\leq 0},\calT^{\geq 0})\) be a t-structure. The following statements are equivalent.
	\begin{enumerate}[label=(\arabic*)]
		\item The t-structure \((\calT^{\leq 0},\calT^{\geq 0})\) is stable.
		\item Both \(\calT^{\leq0}\) and \(\calT^{\geq0}\) are triangulated subcategories of \(\calT\).
		\item The heart of the t-structure \((\calT^{\leq 0},\calT^{\geq 0})\) satisfies \(\calT^\heartsuit\simeq\{0\}\).
	\end{enumerate}
\end{proposition}
\begin{proof}
That (1) and (2) are equivalent is pretty easy. In particular, (2) trivially implies (1), while to see that (1) implies (2), use that by definition of being
stable, \(\calT^{\leq 0} = \calT^{\leq -1}\) is a triangulated subcategory and that \(\calT^{\geq 0} = (\calT^{\leq -1})^\perp\).

To see that (1) \(\Rightarrow\) (3), we compute
\[ \calT^\heartsuit = \calT^{\leq 0}\cap\calT^{\geq0} = \calT^{\leq -1}\calT^{\geq 0} = \calT^{\leq -1} \cap (\calT^{\leq -1})^\perp \simeq \{0\}. \]
For the converse implication (3) \(\Rightarrow\) (1), it suffices to prove that for all \(x\in\calT^{\leq 0}\), we have \(\tau^{\leq -1}x \iso x\).
To this end, consider the distinguished triangle
\[ \tau^{\leq -1}x \to x \to \tau^{\geq 0}x \to \Sigma\tau^{\leq -1}x. \]
Since \(x\in\calT^{\leq 0}\), we have \(\tau^{\geq 0}x \cong \HH^0(x)\cong 0\) since \(\calT^\heartsuit\simeq\{0\}\), and therefore
the map on the left is an isomorphism.
\end{proof}

\subsection{Appendix: t-structures on stable \(\infty\)-categories (TBD)}

