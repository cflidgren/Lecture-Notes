%!TEX root = ../lectures.tex

\section{Graded modules and algebras}\label{lecture:graded-algebra}
In areas incorporating homological algebra, including homological algebra itself, one is commonly concerned with chain complexes of modules over a ring.
However, often these chain complexes admit a bit more structure than merely being chain complexes: they can in many cases admit a kind of multiplication.

\subsection{Graded objects}
Let us fix a commutative ring \(\Bbbk\). Here, we use \(\Bbbk\) to denote a \emph{ring} instead of a \emph{field} for historical reasons.
Commonly, a (\(\Z\)-)\emph{graded} \(\Bbbk\)\emph{-module} is said to be a \(\Bbbk\)-module \(M\) together with a decomposition
\[ M \cong \coprod_{i\in\Z}M^i \]
of \(M\) as a \(\Z\)-indexed coproduct of \(\Bbbk\)-modules---the \emph{grading} on \(M\). It is not hard to see that this datum is really equivalent to \(M\), and
in particular, it makes sense to make the following definition.
\begin{definition}
	Let \(\calC\) be a category. The category of \(\Z\)-graded objects in \(\calC\) is the functor category
	\[ \Fun(\Z,\calC) \cong \prod_{\Z}\calC, \]
	where \(\Z\) is regarded as a discrete category. In particular, we obtain the category \(\gMod_\Bbbk\) of \emph{graded} \(\Bbbk\)\emph{-modules}
	\[ \gMod_{\Bbbk} \coloneq \prod_{\Z}\Mod_{\Bbbk}. \]
	For \(x\in\Fun(\Z,\calC)\), we call \(x^i\) the \emph{degree \(i\) piece of \(x\).} For \(M\in\gMod_\Bbbk\), we say that an element
	of \(M^i\) is \emph{homogeneous of degree \(i\).}
\end{definition}
\begin{remark}
	We will essentially always be working with \(\Z\)-graded objects, so henceforth we will drop bothering to specify that.
\end{remark}
\begin{remark}
	One could have defined graded \(\Bbbk\)-modules as being pairs \((M,(M^i)_{i\in\Z})\), with morphisms being the \(\Bbbk\)-linear maps
	which preserve the grading. It is easily seen that this results in an equivalent category.
\end{remark}

\begin{terminology}
	Let \(M,N\in\gMod_\Bbbk\). A morphism \(M\to N\) in \(\gMod_\Bbbk\) is said to be a \emph{graded morphism of degree \(0\),}
	or otherwise a \emph{strict morphism.}
\end{terminology}

Let us translate some of the terminology standard to graded objects to our setting. In particular, let us detail how one encodes morphisms
of non-zero degree in this formalism.

\begin{definition}
	Let \(\calC\) be a category. The automorphism \(k\mapsto k+1\) of \(\Z\) induces the automorphism
	\[ (1)\!:\Fun(\Z,\calC)\to\Fun(\Z,\calC),\quad x = (x^i)_{i\in\Z} \mapsto (x^{i+1})_{i\in\Z} \eqcolon x(1). \]
	of categories of graded objects in \(\calC\). We denote by \((-1)\) the inverse of this functor, and thus produce functors
	\((i)\) for all \(i\in\Z\) by repeated application of either \((1)\) or \((-1)\).
\end{definition}
\begin{terminology}
	Let \(M,N\in\gMod_\Bbbk\), and let \(i\in\Z\). A \emph{morphism of degree \(i\)} from \(M\) to \(N\) is a strict morphism
	\(M \to N(i)\).
\end{terminology}
\begin{remark}
	This definition makes sense: a morphism of degree \(i\) should send a degree \(n\) homogeneous element of \(M\) to a degree \(n+i\) homogeneous element
	of \(N\), and indeed, the above definition yields
	\[ M^n \to N^{n+i}. \]
	Given a morphism \(M\to N(i)\) and a morphism \(N\to L(j)\), one can ``compose'' these by forming
	\[ M \to N(i) \to L(i+j), \]
	thus obtaining a morphism from \(M\) to \(L\) of degree \(i+j\).
\end{remark}
\begin{remark}
	Note that since \((i)\) is an automorphism for all \(i\in\Z\), we have that
	\[ \gMod_\Bbbk(M,N) \cong \gMod_{\Bbbk}(M(i),N(i)). \]
	In particular, we get no value from considering maps of the form \(M(i)\to N(j)\), as we can always assume \(i=0\).
\end{remark}

\subsection{Tensor products and Hom in the graded world}
We will now turn to discussing a little bit of actual algebra, in contrast to the above entirely pure definitions. We will define the \emph{tensor product}
of two graded modules in such a way that it is a graded module itself, and similarly produce a graded version of \(\Hom\). For the purposes of the former,
it is useful to introduce \emph{bigraded} modules. These are just modules graded over \(\Z\times\Z\), defined just as one does for \(\Z\).
\begin{definition}
	Let \(\calC\) be a category. The category of \emph{bigraded objects} in \(\calC\) is the functor category
	\[ \Fun(\Z\times\Z,\calC) \cong \prod_{\Z\times\Z}\calC. \]
	In particular, we obtain the category
	\[ \bgMod_{\Bbbk} \coloneq \prod_{\Z\times\Z}\Mod_{\Bbbk}. \]
\end{definition}
The reason we introduce this is because given two graded modules \(M,N\), the most natural way to form a tensor product is to form the
\emph{bigraded} module
\[ (M^i\otimes_\Bbbk N^j), \quad {(i,j)\in\Z\times\Z}. \]
A problem with this, of course, is that we are interested in the world of \emph{graded} modules, not \emph{bigraded} modules. The solution is:
\begin{definition}
	Let \(\calC\) be a category. We define the functor
	\[ \Tot\!:\Fun(\Z\times\Z,\calC)\to\Fun(\Z,\calC),\quad (x^{i,j})_{(i,j)\in\Z\times\Z} \mapsto \left(\coprod_{i+j=n}x^{i,j}\right)_{n\in\Z}.  \]
\end{definition}
The way to visualize this functor is by picturing the bigraded object as lying in a grid, then taking the coproduct of the diagonal lines:
\[
	\begin{tikzcd}
		x^{-1,2}\ar[dr,no head] & x^{0,2}\ar[dr,no head] & x^{1,2}\ar[dr,no head] & x^{2,2} \\
		x^{-1,1}\ar[dr,no head] & x^{0,1}\ar[dr,no head] & x^{1,1}\ar[dr,no head] & x^{2,1} \\
		x^{-1,0}\ar[dr,no head] & x^{0,0}\ar[dr,no head] & x^{1,0}\ar[dr,no head] & x^{2,0} \\
		x^{-1,-1} & x^{0,-1} & x^{1,-1} & x^{2,-1}
	\end{tikzcd}
\]
\begin{proposition}\label{prop:graded-Tot-adjunction}
	Let \(\calC\) be a category admitting countable coproducts. Consider the functor
	\[ G\!: \Fun(\Z,\calC) \to \Fun(\Z\times\Z,\calC),\quad (x^i)_{i\in\Z} \mapsto (x^{i+j})_{(i,j)\in\Z\times\Z}. \]
	Then \(\Tot\ladj G\).
\end{proposition}
\begin{proof}
For \(x\in\Fun(\Z\times\Z,\calC)\) and \(y\in\Fun(\Z,\calC)\), we have
\begin{align*}
	\Hom(\Tot(x),y) &= \prod_{n\in\Z}\Hom(\Tot(x)^n,y^n) \\
	&= \prod_{n\in\Z}\Hom(\coprod_{i+j=n}x^{i,j},y^n) \\
	&\cong \prod_{n\in\Z}\prod_{i+j=n}\Hom(x^{i,j},y^n) \\
	&= \prod_{n\in\Z}\prod_{i+j=n}\Hom(x^{i,j},G(y)^{i,j}) \\
	&= \prod_{(i,j)\in\Z\times\Z}\Hom(x^{i,j},G(y)^{i,j}) = \Hom(x,G(y))
\end{align*}
as desired.
\end{proof}

\begin{definition}
	Let \(M,N\in\gMod_\Bbbk\). Temporarily, let us denote the bigraded \(\Bbbk\)-module \((M^i\otimes_\Bbbk N^j)_{(i,j)\in\Z\times\Z}\) by \(M\otimes_\Bbbk^b N\).
	We define the graded modules
	\begin{align*}
		M\otimes_\Bbbk N &\coloneq \Tot(M\otimes_\Bbbk^b N), \\
		\ugMod_\Bbbk(M,N) &\coloneq (\gMod_\Bbbk(M,N(i)))_{i\in \Z}.
	\end{align*}
	We will also write \(\iHom(M,N) \coloneq \ugMod_\Bbbk(M,N)\) if the context makes it clear what this means. These clearly organize into functors.
\end{definition}

\begin{proposition}
	Let \(M,N,L\in\gMod_\Bbbk\). There is a natural isomorphism
	\[ \gMod_\Bbbk(M\otimes_\Bbbk N, L) \cong \gMod_\Bbbk(M,\ugMod_\Bbbk(N,L)). \]
	In particular, we have an adjunction \(-\otimes_\Bbbk N \ladj \ugMod_\Bbbk(N,-)\).
\end{proposition}
\begin{proof}
Recall the adjunction \(\Tot\ladj G\) from Proposition \ref{prop:graded-Tot-adjunction}. Then
\begin{align*}
	\gMod_\Bbbk(M\otimes_\Bbbk N, L) &= \gMod_{\Bbbk}(\Tot(M\otimes_\Bbbk^b N),L) \\
	&\cong \bgMod_{\Bbbk}(M\otimes_\Bbbk^b N,G(L)) \\
	&= \prod_{(i,j)\in\Z\times\Z}\Mod_{\Bbbk}(M^i\otimes_\Bbbk N^j,L^{i+j}) \\
	&\cong \prod_{(i,j)\in\Z\times\Z}\Mod_{\Bbbk}(M^i,\Mod_\Bbbk(N^j,L^{i+j})) \\
	&= \prod_{i\in\Z}\prod_{j\in\Z}\Mod_{\Bbbk}(M^i,\Mod_\Bbbk(N^j,L^{i+j})) \\
	&\cong \prod_{i\in\Z}\Mod_{\Bbbk}(M^i, \prod_{j\in\Z}\Mod_\Bbbk(N^j,L^{i+j})) \\
	&= \prod_{i\in\Z}\Mod_{\Bbbk}(M^i, \gMod_\Bbbk(N,L(i))) = \gMod_\Bbbk(M,\ugMod_\Bbbk(N,L))
\end{align*}
as desired.
\end{proof}
\begin{remark}
	Note that the foregoing proof and definition goes through in the generality of a closed symmetric monoidal Abelian category. That is,
	an Abelian category \(\calA\) admitting a symmetric monoidal structure \(\otimes\) such that \(-\otimes x\) has a right adjoint for all \(x\in\calA\).
\end{remark}

\begin{remark}
	It is perhaps useful, for purposes of intuition, to observe that the elements homogeneous of degree \(i\) in \(\ugMod_\Bbbk(M,N)\) are exactly the degree \(i\) maps from \(M\) to \(N\).
\end{remark}

\begin{remark}
	By Proposition \ref{prop:graded-Tot-adjunction}, to describe a map \(f\!:M\otimes_\Bbbk N \to L\), it suffices to say how \(f\) acts on elementary tensors
	\[ x\otimes y,\quad x\in M^i,\, y\in N^j, \]
	where one must ensure that \(f(x\otimes y) \in L^{i+j}\).
\end{remark}

The tensor product \(-\otimes_\Bbbk-\!:\gMod_\Bbbk\times\gMod_\Bbbk\to\gMod_\Bbbk\) endows \(\gMod_\Bbbk\) with the structure
of a (closed) symmetric monoidal category. In particular, it is clear that one has natural isomorphisms \(M\otimes_\Bbbk \Bbbk \cong M\),
and that one has associativity isomorphisms
\[ (M\otimes_\Bbbk N)\otimes_\Bbbk L \cong M\otimes_\Bbbk (N\otimes_\Bbbk L). \]
It is maybe a \emph{little} less clear that these satisfy the various coherences required of a monoidal category, but the idea is that
they are inherited from the ones on \(\Mod_\Bbbk\). This justifies the assertion that we have a \emph{monoidal} structure.

There is a slight complication in how we make the monoidal structure \emph{symmetric.} In particular, the natural isomorphisms
\[ \tau_{M,N}\!: M\otimes_\Bbbk N \iso N\otimes_\Bbbk M \]
are chosen to act by the \emph{Koszul sign rule,}
\[ \tau(x\otimes y) = (-1)^{ij} y\otimes x, \quad x\in M^i,\, y\in N^j. \]
This sign convention is the fundamental reason that graded commutativity is different from ordinary commutativity. The reason one is
interested in this form of commutativity is that one would like objects like the exterior algebra \(\bigwedge^\bullet M\) of a module
\(M\) to be examples of \emph{graded (commutative) algebras,} since these arise often in e.g.\ algebraic geometry and topology. These
exterior algebras satisfy \(x\wedge y = (-1)^{ij}y\wedge x\).

\begin{exercise}
	Show that there is a natural isomorphism
	\[ \ugMod_\Bbbk(M\otimes N,L) \cong \ugMod_\Bbbk(M,\ugMod_\Bbbk(N,L)). \]
	Hint: this is purely formal, and holds in any closed monoidal category.
\end{exercise}

\subsection{Graded algebras}
In the context of any symmetric monoidal category, one can define \emph{monoid} objects. Applying that to our situation, we end up with
\begin{definition}
	A \emph{graded \(\Bbbk\)-algebra} is a pair \((A,\mu)\) of a graded \(\Bbbk\)-module \(A\) and a multiplication map
	\[ \mu\!: A\otimes_\Bbbk A\to A \]
	which is associative and unital. That is, there is an element \(1 = 1_A \in A^0\) such that for all \(x\in A^i\), we have \(\mu(1\otimes x) = x = \mu(x\otimes 1)\),
	and one should have \(\mu(\mu(x\otimes y)\otimes z) = \mu(x\otimes\mu(y\otimes z))\). We write \(xy \coloneq \mu(x\otimes y)\).
\end{definition}
\begin{remark}
	As before, it is helpful to note that a map \(A\otimes_\Bbbk A \to A\) consists of the data of maps
	\[ A^i\otimes_\Bbbk A^j \to A^{i+j},\quad \forall i,j\in\Z. \]
	The conditions we require on \(\mu\) are just such that
	\[ 1\cdot x = x = x\cdot 1,\quad (xy)z = x(yz). \]
	That is, the expected axioms.
\end{remark}
\begin{definition}
	We say that a graded algebra \(A\) is \emph{(graded) commutative} if the diagram
	\[
	\begin{tikzcd}
		A\otimes A \ar[r,"\mu"]\ar[d,"\tau","\cong"'] & A \ar[d,equal] \\
		A\otimes A \ar[r,"\mu"] & A
	\end{tikzcd}
	\]
	commutes. Concretely, this means that
	\[ xy = (-1)^{ij}yx,\quad x\in A^i,\, y\in A^j. \]
\end{definition}
\begin{remark}
	Note that the Koszul sign rule means that commutativity, in characteristic not equal to \(2\), necessarily means that
	\[ y^2 = 0 \]
	whenever \(y\) is homogeneous of odd degree. Indeed,
	\[ y^2 = (-1)^{\deg(y)\deg(y)}y^2 = -y^2 \implies 2y^2 = 0. \]
	For this reason, one often \emph{requires} that this holds as part of the definition in the even characteristic case. We will not make
	this distinction here, because we will often implicitly assume we are not working in a context where there will be problems.
\end{remark}
\begin{example}
	We now make precise the example from earlier. Let \(M\) be a \(\Bbbk\)-module. We define \(M\wedge_\Bbbk M\) to be the
	\(\Bbbk\)-module quotient
	\[ M\wedge_\Bbbk M \coloneq (M\otimes_\Bbbk M)/(x\otimes y + y\otimes x)_{x,y\in M}. \]
	We denote the image of \(x\otimes y\) in \(M\wedge_\Bbbk M\) by \(x\wedge y\). The definition is such that \(x\wedge y = -y\wedge x\).

	Now, for all \(k\geq 0\), we define
	\[ \bigwedge^kM \coloneq \underbrace{M\wedge_\Bbbk \cdots \wedge_\Bbbk M}_{k\text{ copies}}, \]
	and set \(\bigwedge^0M \coloneq \Bbbk\). We then have a graded \(\Bbbk\)-module defined by
	\[ \left(\bigwedge M\right)^i = \begin{cases}
		\bigwedge^i M & \text{if }i\geq 0, \\
		0 & \text{if }i \leq 0.
	\end{cases} \]
	This graded module can be endowed with the structure of a graded \(\Bbbk\)-algebra, which is furthermore commutative. In particular, given
	elementary generators \(x = x_1\wedge \cdots \wedge x_i\) and \(y = y_1\wedge \cdots \wedge y_j\), one defines
	\[ x\wedge y = x_1\wedge \cdots \wedge x_i \wedge y_1 \wedge \cdots \wedge y_j \]
	and one easily sees that this extends to maps
	\[ \wedge\!: \left(\bigwedge M\right)^i \otimes_\Bbbk \left(\bigwedge M\right)^j \to \left(\bigwedge M\right)^{i+j} \]
	and thus to yields a map
	\[ \bigwedge M \otimes_\Bbbk \bigwedge M \to \bigwedge M. \]
	It is left as an exercise to check that this is associative and unital. That it is graded commutative comes from the definition of the wedge.
\end{example}
