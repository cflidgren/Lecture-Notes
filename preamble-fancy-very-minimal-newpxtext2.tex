\usepackage[utf8]{inputenc}
\tracinglostchars=3
\usepackage{amsmath, amsfonts, amssymb, bm}

\usepackage[dvipsnames]{xcolor}
\usepackage{listings}
\usepackage{verbatim,mathtools,graphicx}

\usepackage[T1]{fontenc}
\usepackage[scaled=.97,helvratio=.93,p,theoremfont]{newpxtext}
\usepackage[varbb,smallerops,bigdelims]{newpxmath}
\usepackage[scaled=.85]{beramono}
\usepackage[scr=rsfso,cal=euler]{mathalfa}

\usepackage{tikz, tikz-cd}
\usepackage{enumitem,xparse,url}
\usepackage[most]{tcolorbox}
\usepackage{multirow,array}
\usepackage{hyperref,bookmark}
\usepackage{stmaryrd}
\usepackage{fontspec}
\usepackage{microtype}

% Renewed commands
\renewcommand{\injlim}{\varinjlim}
\renewcommand{\projlim}{\varprojlim}

% Column stuff
\newcolumntype{x}[1]{>{\centering\let\newline\\\arraybackslash\hspace{0pt}}p{#1}}

% TikZ stuff
\tikzset{
    labl/.style={anchor=north, rotate=90},
    pullback/.style={commutative diagrams/phantom, "\usebox\pullback" , very near start},
  pushout/.style={commutative diagrams/phantom, "\usebox\pushout" , very near start},
  symbol/.style={%
    draw=none,
    every to/.append style={edge node={node [sloped, allow upside down, auto=false]{$#1$}}}
  }
}

\newsavebox{\pullback}
\sbox\pullback{%
\begin{tikzpicture}%
\draw (0,0) -- (1ex,0ex);%
\draw (1ex,0ex) -- (1ex,1ex);%
\end{tikzpicture}}

\newsavebox{\pushout}
\sbox\pushout{%
\begin{tikzpicture}%
\draw (0,0) -- (0ex,1ex);%
\draw (0ex,1ex) -- (1ex,1ex);%
\end{tikzpicture}}

% Environments
% warning box
\newenvironment{warningbox}{\begin{tcolorbox}[colback=red!5!white, colframe=red!80!black, title=\textbf{Warning!}]}{\end{tcolorbox}}
% note box
\newenvironment{notebox}{\begin{tcolorbox}[colback=blue!5!white, colframe=blue!80!black, title=\textbf{Note}]}{\end{tcolorbox}}
% todo box
\newenvironment{todobox}{\begin{tcolorbox}[colback=green!5!white, colframe=green!80!black, title=\textbf{TODO}]}{\end{tcolorbox}}
% tikzcd diagram
\newenvironment{diagram*}{\begin{center}\begin{tikzcd}}{\end{tikzcd}\end{center}}
\newenvironment{diagram}{\begin{equation}\begin{tikzcd}}{\end{tikzcd}\end{equation}\ignorespaces}

\newenvironment{graph*}{\vspace{0.1cm}\begin{center}\begin{tikzcd}[every arrow/.append style={dash,thick}]}{\end{tikzcd}\end{center}\vspace{0.1cm}}
\newenvironment{graph}{\begin{equation}\begin{tikzcd}[every arrow/.append style={dash,thick}]}{\end{tikzcd}\end{equation}\ignorespaces}


% Fancy theorem, proof, example, etc. environments.
\tcbuselibrary{theorems}
\tcbuselibrary{skins}
\tcbuselibrary{breakable}
\tcbuselibrary{vignette}

% styling
\tcbset{
  plategeneric/.style={
    enhanced, breakable,
    borderline west={2pt}{0pt}{white!75!black},
    colback=white,
    top=1.3pt,
    bottom=1.3pt,
    left=7pt,
    right=7pt,
    grow to left by=10pt,
    grow to right by=10pt,
    boxrule=0pt,
    sharp corners
  },
  plate/.style={
    plategeneric,
    frame hidden,
    beforeafter skip balanced=0.22\baselineskip plus 1pt,
    height fixed for=first and middle,
    lines before break=1,
    pad at break*=1.1mm,
    detach title,
    coltitle=black
  },
  fancythm/.style={
    plate,
    top=4pt,
    bottom=4pt,
    borderline west={2pt}{0pt}{Goldenrod!70!black},
    colback=Goldenrod!5!white,
    before upper={\parindent15pt\noindent\textup{\textbf{\tcbtitle.}}\,\,},
    fontupper=\itshape
  },
  fancyconj/.style={
    plate,
    borderline west={2pt}{0pt}{Goldenrod!80!black},
    %colback=Goldenrod!10!white,
    before upper={\parindent15pt\noindent\textup{\textbf{\tcbtitle.}}\,\,},
    fontupper=\itshape
  },
  fancyproof/.style={
    plate,
    borderline west={2pt}{0pt}{Peach!75!black},
    colback=Peach!1!white,
    fonttitle=\itshape,
    before upper={\parindent15pt\noindent\tcbtitle\textit{.}\,\,\,},
    fontupper=\upshape
  },
  fancydef/.style={
    plate,
    borderline west={2pt}{0pt}{NavyBlue!85!black},
    colback=white,
    fonttitle=\bfseries,
    before upper={\parindent15pt\noindent\tcbtitle\textbf{.}\,\,\,}
  },
  fancynotation/.style={
    plate,
    borderline west={2pt}{0pt}{Cyan!65!black},
    colback=white,
    fonttitle=\bfseries,
    before upper={\parindent15pt\noindent\tcbtitle\textbf{.}\,\,\,}
  },
  fancyexample/.style={
    plate,
    borderline west={2pt}{0pt}{Plum!75!black},
    colback=white,
    fonttitle=\bfseries,
    before upper={\parindent15pt\noindent\tcbtitle\textbf{.}\,\,\,}
  },
  fancyremark/.style={
    plate,
    borderline west={2pt}{0pt}{Emerald!75!black},
    colback=white,
    fonttitle=\itshape,
    before upper={\parindent15pt\noindent\tcbtitle\textit{.}\,\,\,}
  },
  fancyexercise/.style={
    plate,
    borderline west={2pt}{0pt}{WildStrawberry!85!black},
    colback=WildStrawberry!10!white,
    fonttitle=\bfseries,
    before upper={\parindent15pt\noindent\tcbtitle\textbf{.}\,\,\,}
  },
  fancysoln/.style={
    plate,
    borderline west={2pt}{0pt}{YellowGreen!85!black},
    colback=white,
    fonttitle=\bfseries,
    before upper={\parindent15pt\noindent\tcbtitle\textbf{.}\,\,\,}
  },
  fancywarning/.style={
    plate,
    borderline west={2pt}{0pt}{Red!75!black},
    colback=Red!15!white,
    before upper={\parindent15pt\noindent\tcbtitle\textbf{.}\,\,\,},
    fonttitle=\bfseries,
    fontupper=\upshape
  },
  fancynote/.style={
    plategeneric,
    grow to right by=4pt,
    borderline west={2pt}{0pt}{ProcessBlue!75!black},
    colback=ProcessBlue!15!white,
    colbacktitle=ProcessBlue!50!white,
    coltitle={ProcessBlue!30!black},
    title={Note},
    fonttitle=\bfseries,
    fontupper=\upshape,
    before upper={\parindent15pt\noindent}
  },
  fancytodo/.style={
    plategeneric,
    grow to right by=4pt,
    borderline west={2pt}{0pt}{Green!75!black},
    colback=Green!15!white,
    colbacktitle=Green!50!white,
    coltitle={Green!40!black},
    title={TO-DO},
    fonttitle=\bfseries,
    fontupper=\upshape,
    before upper={\parindent15pt\noindent}
  }
}

% fix the newtcbtheorem command to be more useful (broken, don't know how to fix this)
\ExplSyntaxOn

\NewDocumentCommand{\betternewtcbtheorem}{O{}mmmm}
 {
  \newtcbtheorem[#1]{#2inner}{#3}{#4}{#5}
  \NewDocumentEnvironment{#2}{O{}}
   {
    \keys_set:nn { hushus/tcb } { ##1 }
    \hushus_tcb_begin:nVV {#2inner} \l__hushus_tcb_title_tl \l__hushus_tcb_label_tl
   }
   {
    \end{#2inner}
   }
  \cs_if_exist:cF { c@#5} { \newcounter{#5} }
 }

\cs_new_protected:Nn \hushus_tcb_begin:nnn
 {
  \begin{#1}{#2}{#3}
 }
\cs_generate_variant:Nn \hushus_tcb_begin:nnn { nVV }
\keys_define:nn { hushus/tcb }
 {
  title .tl_set:N = \l__hushus_tcb_title_tl,
  label .tl_set:N = \l__hushus_tcb_label_tl,
 }

\ExplSyntaxOff

% Command for creating fancy theorem-like environments
% Creates the title format for some parts.
% args: display name, thesection, number (& added text for optional)
\newcommand{\ThmTitleNoOptional}[3]{#1 #2.#3}
\newcommand{\ThmTitleOptional}[4]{#1 #2.#3: #4}
% args: counter prefix (optional), name, display name, style, counter    \ThmTitleOptional{#3}{\thesection}{\arabic{#5}}{##1}
\NewDocumentCommand\goodnewtcbtheorem{ommmm}{%
  \IfNoValueTF{#1}{%
  \newenvironment{#2}[1][]{\refstepcounter{#5}\begin{tcolorbox}[title={\if\relax\detokenize{##1}\relax{\ThmTitleNoOptional{#3}{\thesection}{\arabic{#5}}}\else{\ThmTitleOptional{#3}{\thesection}{\arabic{#5}}{##1}}\fi},#4]}{\end{tcolorbox}}%
  \newenvironment{#2*}[1][]{\begin{tcolorbox}[title={\if\relax\detokenize{##1}\relax{#3}\else{#3: ##1}\fi},#4]}{\end{tcolorbox}}%
  }{%
  \newenvironment{#2}[1][]{\refstepcounter{#5}\begin{tcolorbox}[title={\if\relax\detokenize{##1}\relax{#3 #1\arabic{#5}}\else{#3 #1\arabic{#5}: ##1}\fi},#4]}{\end{tcolorbox}}%
  \newenvironment{#2*}[1][]{\begin{tcolorbox}[title={\if\relax\detokenize{##1}\relax{#3}\else{#3: ##1}\fi},#4]}{\end{tcolorbox}}%
  }
}
\NewDocumentCommand\goodrenewtcbtheorem{ommmm}{%
  \IfNoValueTF{#1}{%
  \renewenvironment{#2}[1][]{\refstepcounter{#5}\begin{tcolorbox}[title={\if\relax\detokenize{##1}\relax{#3 \thesection.\arabic{#5}}\else{#3 \thesection.\arabic{#5}: ##1}\fi},#4]}{\end{tcolorbox}}%
  \renewenvironment{#2*}[1][]{\begin{tcolorbox}[title={\if{#3}\else{#3: ##1}\fi},#4]}{\end{tcolorbox}}%
  }{%
  \renewenvironment{#2}[1][]{\refstepcounter{#5}\begin{tcolorbox}[title={\if\relax\detokenize{##1}\relax{#3 #1\arabic{#5}}\else{#3 #1\arabic{#5}: ##1}\fi},#4]}{\end{tcolorbox}}%
  \renewenvironment{#2*}[1][]{\begin{tcolorbox}[title={\if{#3}\else{#3: ##1}\fi},#4]}{\end{tcolorbox}}%
  }
}

% Counters
\newcounter{thmcounter}
\numberwithin{thmcounter}{section}

\newcounter{excounter}
\newcounter{probcounter}
\newcounter{solcounter}
\newcounter{taskcounter}

% Theorems, Lemmas, etc.
% Theorem style
\goodnewtcbtheorem{theorem}{Theorem}{fancythm}{thmcounter}
\goodnewtcbtheorem{theoremdef}{Theorem-Definition}{fancythm}{thmcounter}
\goodnewtcbtheorem{proposition}{Proposition}{fancythm}{thmcounter}
\goodnewtcbtheorem{propositiondef}{Proposition-Definition}{fancythm}{thmcounter}
\goodnewtcbtheorem{corollary}{Corollary}{fancythm}{thmcounter}
\goodnewtcbtheorem{lemma}{Lemma}{fancythm}{thmcounter}
\goodnewtcbtheorem{fact}{Fact}{fancythm}{thmcounter}

\goodnewtcbtheorem{conjecture}{Conjecture}{fancyconj, fontupper=\upshape}{thmcounter}

\goodnewtcbtheorem{goal}{Goal}{fancyconj, fontupper=\upshape}{thmcounter}

% Definition style
\goodnewtcbtheorem{definition}{Definition}{fancydef}{thmcounter}

% Notation style
\goodnewtcbtheorem{notation}{Notation}{fancynotation}{thmcounter}
\goodnewtcbtheorem{terminology}{Terminology}{fancynotation}{thmcounter}
\goodnewtcbtheorem{convention}{Convention}{fancynotation}{thmcounter}
\goodnewtcbtheorem{question}{Question}{fancynotation}{thmcounter}
\goodnewtcbtheorem{axiom}{Axiom}{fancynotation}{thmcounter}

% Example style
\goodnewtcbtheorem{example}{Example}{fancyexample}{thmcounter}
\goodnewtcbtheorem{construction}{Construction}{fancyexample}{thmcounter}
\goodnewtcbtheorem{sketch}{Sketch}{fancyexample}{thmcounter}

% Remark style
\goodnewtcbtheorem{remark}{Remark}{fancyremark}{thmcounter}
\goodnewtcbtheorem{discussion}{Discussion}{fancyremark}{thmcounter}

% Exercise style, empty [] necessary.
\goodnewtcbtheorem[]{exercise}{Exercise}{fancyexercise}{excounter}
\goodnewtcbtheorem[]{task}{Task}{fancyexercise}{taskcounter}
\goodnewtcbtheorem[]{problem}{Problem}{fancyexercise}{probcounter}

% Solution style, empty [] necessary.
\goodnewtcbtheorem[]{solution}{Solution}{fancysoln}{solcounter}

% Warning style
\goodnewtcbtheorem{warning}{Warning}{fancywarning}{thmcounter}

% fancy note and todo
\newtcolorbox{note}{fancynote}
\newtcolorbox{todo}{fancytodo}

% Proof environment.
\newcommand{\qedsymbol}{$\blacksquare$}
\newtcolorbox{proof}[1][Proof]{fancyproof, title={#1}}
\AtEndEnvironment{proof}{\null\hfill\qedsymbol}
% "Proof Sketch" environment
\newenvironment{proofsketch}{\begin{proof}[Proof sketch]\renewcommand\qedsymbol{$\square$}}{\end{proof}\renewcommand\qedsymbol{$\blacksquare$}}

% QED symbol should be black tombstone.
\renewcommand\qedsymbol{$\blacksquare$}
\makeatletter
\newcommand{\oset}[3][0ex]{%
  \mathrel{\mathop{#3}\limits^{
    \vbox to#1{\kern-2\ex@
    \hbox{$\scriptstyle#2$}\vss}}}}
\makeatother

% utility "such that" command.
\newcommand{\st}{
  \text{ s.t. }
}

\newcommand{\bigmid}{\biggm|}
\newcommand{\ldsq}{\llbracket}
\newcommand{\rdsq}{\rrbracket}
\newcommand{\dashto}{\dashrightarrow}

\newcommand{\vecb}{\mathbf}
\newcommand{\lcm}{\operatorname{lcm}}
\newcommand{\divides}{\mid}
\newcommand{\longto}{\longrightarrow}
\newcommand{\ot}{\leftarrow}
\newcommand{\xto}{\xrightarrow}
\newcommand{\xot}{\xleftarrow}
\newcommand{\longot}{\longleftarrow}
\newcommand{\Longto}{\Longrightarrow}
\newcommand{\oT}{\Leftarrow}
\newcommand{\To}{\Rightarrow}
\newcommand{\longoT}{\Longleftarrow}

\newcommand{\fd}{\textup{f.d.}}
\newcommand{\ev}{\operatorname{ev}}

% mathbf
\newcommand{\bfA}{\ensuremath{\mathbf{A}}}
\newcommand{\bfB}{\ensuremath{\mathbf{B}}}
\newcommand{\bfC}{\ensuremath{\mathbf{C}}}
\newcommand{\bfD}{\ensuremath{\mathbf{D}}}
\newcommand{\bfE}{\ensuremath{\mathbf{E}}}
\newcommand{\bfF}{\ensuremath{\mathbf{F}}}
\newcommand{\bfG}{\ensuremath{\mathbf{G}}}
\newcommand{\bfH}{\ensuremath{\mathbf{H}}}
\newcommand{\bfI}{\ensuremath{\mathbf{I}}}
\newcommand{\bfJ}{\ensuremath{\mathbf{J}}}
\newcommand{\bfK}{\ensuremath{\mathbf{K}}}
\newcommand{\bfL}{\ensuremath{\mathbf{L}}}
\newcommand{\bfM}{\ensuremath{\mathbf{M}}}
\newcommand{\bfN}{\ensuremath{\mathbf{N}}}
\newcommand{\bfO}{\ensuremath{\mathbf{O}}}
\newcommand{\bfP}{\ensuremath{\mathbf{P}}}
\newcommand{\bfQ}{\ensuremath{\mathbf{Q}}}
\newcommand{\bfR}{\ensuremath{\mathbf{R}}}
\newcommand{\bfS}{\ensuremath{\mathbf{S}}}
\newcommand{\bfT}{\ensuremath{\mathbf{T}}}
\newcommand{\bfU}{\ensuremath{\mathbf{U}}}
\newcommand{\bfV}{\ensuremath{\mathbf{V}}}
\newcommand{\bfW}{\ensuremath{\mathbf{W}}}
\newcommand{\bfX}{\ensuremath{\mathbf{X}}}
\newcommand{\bfY}{\ensuremath{\mathbf{Y}}}
\newcommand{\bfZ}{\ensuremath{\mathbf{Z}}}
\newcommand{\bfDelta}{\ensuremath{\mathbf{\Delta}}}

% mathsf
\newcommand{\sfA}{\ensuremath{\mathsf{A}}}
\newcommand{\sfB}{\ensuremath{\mathsf{B}}}
\newcommand{\sfC}{\ensuremath{\mathsf{C}}}
\newcommand{\sfD}{\ensuremath{\mathsf{D}}}
\newcommand{\sfE}{\ensuremath{\mathsf{E}}}
\newcommand{\sfF}{\ensuremath{\mathsf{F}}}
\newcommand{\sfG}{\ensuremath{\mathsf{G}}}
\newcommand{\sfH}{\ensuremath{\mathsf{H}}}
\newcommand{\sfI}{\ensuremath{\mathsf{I}}}
\newcommand{\sfJ}{\ensuremath{\mathsf{J}}}
\newcommand{\sfK}{\ensuremath{\mathsf{K}}}
\newcommand{\sfL}{\ensuremath{\mathsf{L}}}
\newcommand{\sfM}{\ensuremath{\mathsf{M}}}
\newcommand{\sfN}{\ensuremath{\mathsf{N}}}
\newcommand{\sfO}{\ensuremath{\mathsf{O}}}
\newcommand{\sfP}{\ensuremath{\mathsf{P}}}
\newcommand{\sfQ}{\ensuremath{\mathsf{Q}}}
\newcommand{\sfR}{\ensuremath{\mathsf{R}}}
\newcommand{\sfS}{\ensuremath{\mathsf{S}}}
\newcommand{\sfT}{\ensuremath{\mathsf{T}}}
\newcommand{\sfU}{\ensuremath{\mathsf{U}}}
\newcommand{\sfV}{\ensuremath{\mathsf{V}}}
\newcommand{\sfW}{\ensuremath{\mathsf{W}}}
\newcommand{\sfX}{\ensuremath{\mathsf{X}}}
\newcommand{\sfY}{\ensuremath{\mathsf{Y}}}
\newcommand{\sfZ}{\ensuremath{\mathsf{Z}}}

% mathbb
\newcommand{\bbA}{\ensuremath{\mathbb{A}}}
\newcommand{\bbB}{\ensuremath{\mathbb{B}}}
\newcommand{\bbC}{\ensuremath{\mathbb{C}}}
\newcommand{\bbD}{\ensuremath{\mathbb{D}}}
\newcommand{\bbE}{\ensuremath{\mathbb{E}}}
\newcommand{\bbF}{\ensuremath{\mathbb{F}}}
\newcommand{\bbG}{\ensuremath{\mathbb{G}}}
\newcommand{\bbH}{\ensuremath{\mathbb{H}}}
\newcommand{\bbI}{\ensuremath{\mathbb{I}}}
\newcommand{\bbJ}{\ensuremath{\mathbb{J}}}
\newcommand{\bbK}{\ensuremath{\mathbb{K}}}
\newcommand{\bbL}{\ensuremath{\mathbb{L}}}
\newcommand{\bbM}{\ensuremath{\mathbb{M}}}
\newcommand{\bbN}{\ensuremath{\mathbb{N}}}
\newcommand{\bbO}{\ensuremath{\mathbb{O}}}
\newcommand{\bbP}{\ensuremath{\mathbb{P}}}
\newcommand{\bbQ}{\ensuremath{\mathbb{Q}}}
\newcommand{\bbR}{\ensuremath{\mathbb{R}}}
\newcommand{\bbS}{\ensuremath{\mathbb{S}}}
\newcommand{\bbT}{\ensuremath{\mathbb{T}}}
\newcommand{\bbU}{\ensuremath{\mathbb{U}}}
\newcommand{\bbV}{\ensuremath{\mathbb{V}}}
\newcommand{\bbW}{\ensuremath{\mathbb{W}}}
\newcommand{\bbX}{\ensuremath{\mathbb{X}}}
\newcommand{\bbY}{\ensuremath{\mathbb{Y}}}
\newcommand{\bbZ}{\ensuremath{\mathbb{Z}}}

% mathcal
\newcommand{\calA}{\ensuremath{\mathcal{A}}}
\newcommand{\calB}{\ensuremath{\mathcal{B}}}
\newcommand{\calC}{\ensuremath{\mathcal{C}}}
\newcommand{\calD}{\ensuremath{\mathcal{D}}}
\newcommand{\calE}{\ensuremath{\mathcal{E}}}
\newcommand{\calF}{\ensuremath{\mathcal{F}}}
\newcommand{\calG}{\ensuremath{\mathcal{G}}}
\newcommand{\calH}{\ensuremath{\mathcal{H}}}
\newcommand{\calI}{\ensuremath{\mathcal{I}}}
\newcommand{\calJ}{\ensuremath{\mathcal{J}}}
\newcommand{\calK}{\ensuremath{\mathcal{K}}}
\newcommand{\calL}{\ensuremath{\mathcal{L}}}
\newcommand{\calM}{\ensuremath{\mathcal{M}}}
\newcommand{\calN}{\ensuremath{\mathcal{N}}}
\newcommand{\calO}{\ensuremath{\mathcal{O}}}
\newcommand{\calP}{\ensuremath{\mathcal{P}}}
\newcommand{\calQ}{\ensuremath{\mathcal{Q}}}
\newcommand{\calR}{\ensuremath{\mathcal{R}}}
\newcommand{\calS}{\ensuremath{\mathcal{S}}}
\newcommand{\calT}{\ensuremath{\mathcal{T}}}
\newcommand{\calU}{\ensuremath{\mathcal{U}}}
\newcommand{\calV}{\ensuremath{\mathcal{V}}}
\newcommand{\calW}{\ensuremath{\mathcal{W}}}
\newcommand{\calX}{\ensuremath{\mathcal{X}}}
\newcommand{\calY}{\ensuremath{\mathcal{Y}}}
\newcommand{\calZ}{\ensuremath{\mathcal{Z}}}

% mathscr
\newcommand{\scrA}{\ensuremath{\mathscr{A}}}
\newcommand{\scrB}{\ensuremath{\mathscr{B}}}
\newcommand{\scrC}{\ensuremath{\mathscr{C}}}
\newcommand{\scrD}{\ensuremath{\mathscr{D}}}
\newcommand{\scrE}{\ensuremath{\mathscr{E}}}
\newcommand{\scrF}{\ensuremath{\mathscr{F}}}
\newcommand{\scrG}{\ensuremath{\mathscr{G}}}
\newcommand{\scrH}{\ensuremath{\mathscr{H}}}
\newcommand{\scrI}{\ensuremath{\mathscr{I}}}
\newcommand{\scrJ}{\ensuremath{\mathscr{J}}}
\newcommand{\scrK}{\ensuremath{\mathscr{K}}}
\newcommand{\scrL}{\ensuremath{\mathscr{L}}}
\newcommand{\scrM}{\ensuremath{\mathscr{M}}}
\newcommand{\scrN}{\ensuremath{\mathscr{N}}}
\newcommand{\scrO}{\ensuremath{\mathscr{O}}}
\newcommand{\scrP}{\ensuremath{\mathscr{P}}}
\newcommand{\scrQ}{\ensuremath{\mathscr{Q}}}
\newcommand{\scrR}{\ensuremath{\mathscr{R}}}
\newcommand{\scrS}{\ensuremath{\mathscr{S}}}
\newcommand{\scrT}{\ensuremath{\mathscr{T}}}
\newcommand{\scrU}{\ensuremath{\mathscr{U}}}
\newcommand{\scrV}{\ensuremath{\mathscr{V}}}
\newcommand{\scrW}{\ensuremath{\mathscr{W}}}
\newcommand{\scrX}{\ensuremath{\mathscr{X}}}
\newcommand{\scrY}{\ensuremath{\mathscr{Y}}}
\newcommand{\scrZ}{\ensuremath{\mathscr{Z}}}

% mathfrak
\newcommand{\frakA}{\ensuremath{\mathfrak{A}}}
\newcommand{\frakB}{\ensuremath{\mathfrak{B}}}
\newcommand{\frakC}{\ensuremath{\mathfrak{C}}}
\newcommand{\frakD}{\ensuremath{\mathfrak{D}}}
\newcommand{\frakE}{\ensuremath{\mathfrak{E}}}
\newcommand{\frakF}{\ensuremath{\mathfrak{F}}}
\newcommand{\frakG}{\ensuremath{\mathfrak{G}}}
\newcommand{\frakH}{\ensuremath{\mathfrak{H}}}
\newcommand{\frakI}{\ensuremath{\mathfrak{I}}}
\newcommand{\frakJ}{\ensuremath{\mathfrak{J}}}
\newcommand{\frakK}{\ensuremath{\mathfrak{K}}}
\newcommand{\frakL}{\ensuremath{\mathfrak{L}}}
\newcommand{\frakM}{\ensuremath{\mathfrak{M}}}
\newcommand{\frakN}{\ensuremath{\mathfrak{N}}}
\newcommand{\frakO}{\ensuremath{\mathfrak{O}}}
\newcommand{\frakP}{\ensuremath{\mathfrak{P}}}
\newcommand{\frakQ}{\ensuremath{\mathfrak{Q}}}
\newcommand{\frakR}{\ensuremath{\mathfrak{R}}}
\newcommand{\frakS}{\ensuremath{\mathfrak{S}}}
\newcommand{\frakT}{\ensuremath{\mathfrak{T}}}
\newcommand{\frakU}{\ensuremath{\mathfrak{U}}}
\newcommand{\frakV}{\ensuremath{\mathfrak{V}}}
\newcommand{\frakW}{\ensuremath{\mathfrak{W}}}
\newcommand{\frakX}{\ensuremath{\mathfrak{X}}}
\newcommand{\frakY}{\ensuremath{\mathfrak{Y}}}
\newcommand{\frakZ}{\ensuremath{\mathfrak{Z}}}


% Calculus, transforms, and other stuff
\newcommand*{\diff}{\mathop{}\!\mathrm{d}}
\newcommand{\Res}{\operatorname{Res}}
\newcommand{\p}{\partial}
\newcommand{\diam}{\mathrm{diam}\,}
\renewcommand{\Im}{\operatorname{Im}}
\renewcommand{\Re}{\operatorname{Re}}
\newcommand{\sinc}{\mathrm{sinc}}

% Common Sets
\newcommand{\Z}{\mathbb{Z}}
\newcommand{\Q}{\mathbb{Q}}
\newcommand{\R}{\mathbb{R}}
\newcommand{\C}{\mathbb{C}}
\newcommand{\D}{\mathbb{D}}
\newcommand{\N}{\mathbb{N}}
\newcommand{\E}{\mathbb{E}}
\newcommand{\F}{\mathbb{F}}
\newcommand{\G}{\mathbb{G}}
\newcommand{\A}{\mathbb{A}}
\newcommand{\TT}{\mathbb{T}}
\newcommand{\PP}{\mathbb{P}}
\newcommand{\pow}{
  \mathcal{P}
}

% Algebra
\newcommand{\adj}{\operatorname{adj}}
\newcommand{\ind}{\operatorname{ind}}
\newcommand{\height}{\operatorname{ht}}
\newcommand{\dual}{\vee}
\newcommand{\sep}{\text{sep}}
\newcommand{\nor}{\text{nor}}
\newcommand{\red}{\mathrm{red}}
\newcommand{\gal}{\text{gal}}
\newcommand{\ideal}{\mathfrak}
\newcommand{\GL}{\operatorname{GL}}
\newcommand{\SL}{\operatorname{SL}}
\newcommand{\M}{\operatorname{M}}
\newcommand{\nrmleq}{\trianglelefteq}
\newcommand{\nrmgeq}{\trianglerighteq}
\newcommand{\cyclic}[1]{\langle #1 \rangle}
\newcommand{\coimg}{\operatorname{coim}}
\newcommand{\img}{\operatorname{im}}
\newcommand{\Frac}{\operatorname{Frac}}
\newcommand{\aut}{\operatorname{Aut}}
\newcommand{\Aut}{\operatorname{Aut}}
\newcommand{\End}{\operatorname{End}}
\newcommand{\Ann}{\operatorname{Ann}}
\newcommand{\Gal}{\operatorname{Gal}}
\newcommand{\Tor}{\operatorname{Tor}}
\newcommand{\coker}{\operatorname{coker}}
\newcommand{\eq}{\operatorname{eq}}
\newcommand{\coeq}{\operatorname{coeq}}
\newcommand{\spec}{\operatorname{Spec}}
\newcommand{\Spec}{\operatorname{Spec}}
\newcommand{\Proj}{\operatorname{Proj}}
\newcommand{\Ox}{\mathcal{O}}
\newcommand{\Open}{\operatorname{Open}}
\newcommand{\Char}{\operatorname{char}}
\newcommand{\Span}{\operatorname{Span}}
\newcommand{\intO}{\mathscr{O}}
\newcommand{\Pic}{\operatorname{Pic}}
\newcommand{\Ext}{\operatorname{Ext}}
\newcommand{\KProj}{\operatorname{KProj}}


\newcommand{\hooktwoheadrightarrow}{\hookrightarrow\mathrel{\mspace{-15mu}}\rightarrow}
\makeatletter
\newcommand{\xhooktwoheadrightarrow}[2][]{%
  \lhook\joinrel
  \ext@arrow 0359\rightarrowfill@ {#1}{#2}%
  \mathrel{\mspace{-15mu}}\rightarrow
}
\makeatother
\newcommand{\iso}{\xrightarrow{\sim}}
\newcommand{\isom}{\hooktwoheadrightarrow}
\newcommand{\inj}{\hookrightarrow}
\newcommand{\sur}{\twoheadrightarrow}
\newcommand{\orbits}{\mathrm{O}}
\newcommand{\rank}{\operatorname{rank}}
\newcommand{\nullity}{\operatorname{nullity}}
\newcommand{\tr}{\operatorname{tr}}

% Geometry
\newcommand{\simh}{\sim_\textup{h}}
\newcommand{\Supp}{\operatorname{Supp}}
\newcommand{\spa}{\operatorname{sp}}
\newcommand{\HH}{\operatorname{H}}
\newcommand{\hh}{\operatorname{h}}
\newcommand{\mult}{\operatorname{mult}}
\newcommand{\codim}{\operatorname{codim}}
\newcommand{\model}{\mathcal}
\newcommand{\Gm}[1][]{\mathbf{G}_{\mathrm{m} #1}}
\newcommand{\Ga}[1][]{\mathbf{G}_{\mathrm{a} #1}}
\newcommand{\Sp}{\mathrm{Sp}}
\newcommand{\Sk}{\operatorname{Sk}}
\newcommand{\Sing}{\operatorname{Sing}}
\newcommand{\ho}{\operatorname{ho}}

% Category Theory
\newcommand{\Qis}{\operatorname{Qis}}
\newcommand{\thick}{\operatorname{thick}}
\newcommand{\Lan}{\operatorname{Lan}}
\newcommand{\Ran}{\operatorname{Ran}}
\newcommand{\Com}{\operatorname{Com}}
\newcommand{\Ind}{\operatorname{Ind}}
\newcommand{\Pro}{\operatorname{Pro}}
\newcommand{\pre}{\mathrm{pre}}
\newcommand{\op}{\mathrm{op}}
\newcommand{\Hom}{\operatorname{Hom}}
\newcommand{\Map}{\operatorname{Map}}
\newcommand{\iHom}{\underline{\Hom}}
\newcommand{\SHom}{\mathcal{H}\hspace{-0.0833em}om}
\newcommand{\Mor}{\operatorname{Mor}}
\newcommand{\Fun}{\operatorname{Fun}}
\newcommand{\Nat}{\operatorname{Nat}}
\newcommand{\Der}{\operatorname{Der}}
\newcommand{\hocolim}{\operatorname{hocolim}}
\newcommand{\sheaf}{\mathscr}
\newcommand{\sh}{\text{sh}}
\newcommand{\cat}{\mathbf}
\newcommand{\natcat}[1]{ \mathbf{[#1]} }
\newcommand{\dom}{\operatorname{dom}}
\newcommand{\cod}{\operatorname{cod}}
\newcommand{\id}{\mathrm{id}}
\newcommand{\Cones}{\operatorname{Cones}}
\newcommand{\ob}{\operatorname{Ob}}
\newcommand{\Ob}{\ob}
\newcommand{\Ar}{\operatorname{Ar}}
\newcommand{\natto}{%
  \mathrel{\vbox{\offinterlineskip
    \mathsurround=0pt
    \ialign{\hfil##\hfil\cr
      \normalfont\scalebox{1.2}{.}\cr
%      \noalign{\kern-.05ex}
      $\rightarrow$\cr}
  }}%
}

% Some Categories
\newcommand{\Cat}{\cat{Cat}}
\newcommand{\inftyCat}{\infty\text{-}\cat{Cat}}
\newcommand{\Set}{\cat{Set}}
\newcommand{\PSet}{\cat{Set_*}}
\newcommand{\Ens}[1]{\cat{Ens}_{#1}}
\newcommand{\Mon}{\cat{Mon}}
\newcommand{\CMon}{\cat{CMon}}
\newcommand{\Grp}{\cat{Grp}}
\newcommand{\Gpd}{\cat{Gpd}}
\newcommand{\Ab}{\cat{Ab}}
\newcommand{\Rng}{\cat{Rng}}
\newcommand{\CRng}{\cat{CRng}}
\newcommand{\Mod}{\cat{Mod}}
\newcommand{\Alg}{\cat{Alg}}
\newcommand{\Rep}{\cat{Rep}}
\newcommand{\RMod}[1][R]{#1\text{-}\cat{Mod}}
\newcommand{\ModR}[1][R]{\cat{Mod}\text{-}#1}
\newcommand{\RAlg}[1][R]{#1\text{-}\cat{Alg}}
\renewcommand{\Top}{\cat{Top}}
%\@ifundefined{Top}{\newcommand{\Top}{\cat{Top}}}{\renewcommand{\Top}{\cat{Top}}}
\newcommand{\SMan}{\cat{SMan}}
\newcommand{\SmMan}{\cat{SmMan}}
\newcommand{\Sm}{\cat{Sm}}
\newcommand{\PTop}{\cat{Top_*}}
\newcommand{\hTop}{\cat{hTop}}
\newcommand{\Vect}{\cat{Vect}}
\newcommand{\PShf}{\operatorname{\cat{PShf}}}
\newcommand{\Shf}{\operatorname{\cat{Shf}}}
\newcommand{\PSh}{\operatorname{\cat{PSh}}}
\newcommand{\Sh}{\operatorname{\cat{Sh}}}
\newcommand{\RSp}{\cat{RSp}}
\newcommand{\Sch}{\cat{Sch}}
\newcommand{\AffSch}{\cat{AffSch}}
\newcommand{\sSet}{\cat{sSet}}
\newcommand{\Ch}{\cat{Ch}}

% Python Code Highlighting Style
\definecolor{codebg}{rgb}{0.9,0.9,0.9}
\lstdefinestyle{pythoncode}{
  backgroundcolor=\color{codebg},
  commentstyle=\color{Green},
  keywordstyle=\color{Magenta},
  numberstyle=\tiny\color{Gray},
  stringstyle=\color{ForestGreen},
  basicstyle=\footnotesize,
  breakatwhitespace=false,
  breaklines=true,
  captionpos=b,
  keepspaces=true,
  numbers=left,
  numbersep=5pt,
  showspaces=false,
  showstringspaces=false,
  showtabs=false,
  tabsize=3
}
\lstset{style=pythoncode}
